% !TEX encoding = UTF-8 Unicode
% лекции 11-12, 12 апреля 2016
% конец первой части
% вопросы 21-26
% 21. Теоремы о среднем для гармонических функций
% 22. Сильный принцип максимума для гармонических функций, его следствия. Внутренняя задача Дирихле для уравнения Пуассона. Единственность классического решения
% 23. Обратная теорема о среднем
% 24. Бесконечная дифференцируемость гармонических функций
% 25. Оценки производных гармонических функций. Функции, гармонические во всем пространстве. Теорема Лиувилля.
% 26. Функция Грина внутренней задачи Дирихле для уравнения Пуассона. Функция Грина для шара

\section{Гармонические функции}
\begin{definition} Пусть $\Omega \subset \real^n$ - область. Функция называется гармонической в $\Omega$, если она является классическим решением уравнения Лапласа:
$$-\Delta u = 0, \quad u \in C^2(\Omega)$$
\end{definition}
\begin{note}
Очевидно, в $\real$ функция будет гармонической тогда и только тогда, когда она линейна.
\end{note}

\subsection{Свойства гармонических функций}
\begin{note}
Пусть $\Omega \subset \real^n$ - ограниченная область, а $u \in C^2(\Omega)\cap C(\overline{\Omega})$ - гармоническая. Тогда 
$$0 = \int \limits_\Omega \Delta \, u dx = -\int \limits_\Omega \underbrace {\nabla 1}_{0} \cdot \nabla u dx + \int \limits_{\partial\Omega} 1 \cdot \dfrac{\partial u}{\partial \nu}d\sigma\,\Rightarrow\, \int\limits_{\partial\Omega}\dfrac{\partial u}{\partial \nu}d\sigma = 0.$$
Дадим физическую интерпретацию этому замечанию. Уравнение Лапласа задаёт стационарное температурное поле без источников и стоков теплоты. Чтобы такое поле существовало, необходимо, чтобы суммарный поток тепла через границу области должен быть равен нулю.
\end{note}

\begin{theorem}[о среднем для гармонических функций]
Пусть $\Omega \subset \real^n$, $n > 2$, $u$ - гармоническая в $\Omega$, тогда для любого $x_0 \in \Omega$ верно следующее:
$$u(x_0) = \fint \limits_{\partial B_\eps(x_0)} u(x) d\sigma(x) = \fint \limits_{B_\eps(x_0)} u(x)dx, \quad \forall \eps > 0 \,:\, \overline{B}_\eps(x_0)\subset\Omega.$$
То есть, значение гармонической функции в точке равно как среднему по границе шара, так и среднему по шару с центром в этой точке для любого шара в $\Omega$.
\end{theorem}

\begin{proof}Для начала докажем первое равенство. Пользуясь третьей формулой Грина, получаем

\begin{align*}
u(x_0) & = \int \limits_{\partial B_{\eps}(x_0)} \Phi(x-x_0) \pder[u]{\nu} d\sigma - \int \limits_{\partial B_{\eps}(x_0)} u \pder[\Phi]{\nu} d\sigma - \underbrace {\int \limits_{B_{\eps}(x_0)} \Delta u \Phi (x-x_0) dx}_{=0} = \\
& = \frac {1} {n(n-2)\omega_n}\left(\int \limits_{\partial B_{\eps}(x_0)} \frac {1} {\abs{x - x_0}}^{n-2} \pder[u]{\nu} d \sigma - \int \limits_{\partial B_{\eps}(x_0)} u \pder[]{\nu}\left(\frac {1} {\abs{x-x_0}^{n-2}}\right) d \sigma \right) = \\
& = \frac {1} {n(n-2)\omega_n} \left( \frac {1} {\eps^{n-2}} \underbrace {\int \limits_{\partial B_{\eps}(x_0)} \pder[u]{\nu} d \sigma}_{=0 \text{ по замечанию}} - \int \limits_{\partial B_{\eps}(x_0)} u \frac {n-2} {\eps^{n-1}} d \sigma  \right) = \\
& = \frac {1} {n \omega_n \eps^{n-1}} \int \limits_{\partial B_{\eps} (x_0)} u d \sigma = \fint \limits_{\partial B_{\eps} (x_0)} u d \sigma.
\end{align*}
Первое равенство доказано. Докажем второе. Перепишем интеграл по шару в сферической системе координат:
\begin{align*}
\int \limits_{B_{\eps} (x_0)} u dx &= \int \limits_0^\eps r^{n-1} dr \int \limits_{\partial B_1 (0)} u(\underbrace {x_0 + rs}_{=z}) d \sigma(s) = \int \limits_0^\eps dr \underbrace {\int \limits_{\partial B_r (x_0)} u(z) d \sigma(z)}_{\text{площадь поверхности } \cdot u(x_0)} = \\
& = \int \limits_0^\eps u(x_0) \sigma(\partial B_r(x_0)) dr = u(x_0) \abs{B_\eps (x_0)}.
\end{align*}
Итого,
$$u(x_0) = \fint \limits_{B_{\eps} (x_0)} u dx.$$

\end{proof}

\begin{note}
При $n = 2$ счёт происходит аналогично.
\end{note}

\begin{note}
Пусть $u \in C(\overline{\Omega})$ (непрерывна вплоть до границы). Тогда если есть шар $B_{\eps_0}$, касающийся границы $\Omega$, то утверждение теоремы верно для $\eps \leq \eps_0$. 
\end{note}
\begin{proof}
Пользуясь формулой (*) (?????) и непрерывностью $u$, переходим к пределу.

\end{proof}

\begin{theorem}[Обратная теорема о среднем]
Пусть $\Omega \subset \real^n$ - область, $u \in C(\Omega)$. Если для любого шара $B_\rho (x_0)$, замыкание которого лежит в $\Omega$, выполняется
$$ u(x_0) = \fint \limits_{B_\rho(x_0)} u(x) dx,$$
то $u$ - гармоническая. 
\end{theorem}
\begin{note}
Пусть выполняются условия теоремы, тогда $u$ удовлетворяет соотношению
$$ u(x_0) = \fint \limits_{\partial B_\rho(x_0)} u(x) dx$$
\end{note}
\begin{proof}[Доказательство первого замечания]
Снова воспользуемся сферическими координатами:
$$u(x_0) \cdot \abs{B_\rho (x_0)} = \int \limits_{B_\rho (x_0)} u(x) dx = \int \limits_0^\rho dr \int \limits_{\partial B_r (x_0)} u(z) d \sigma(z).$$
В то же время
$$ u(x_0) \cdot \abs{B_\rho (x_0)} = u(x_0) \cdot \omega_n \rho^n.$$
Приравняем и продифференцируем по $\rho$:
$$ u(x_0) \cdot \underbrace {\omega_n n \rho^{n-1}}_{= \sigma(\partial B_\rho (x_0))} = \int \limits_{\partial B_\rho (x_0)} u(z) d\sigma(z).$$
Итого
$$ u(x_0) = \fint \limits_{\partial B_\rho (x_0)} u(z) d\sigma(z).$$

\end{proof}

\begin{note}
Теорема верна в предположении, что $u \in C^2 (\Omega)$.
\end{note}
\begin{proof}[Доказательство второго замечания]
Дифференцируем соотношение из условия теоремы по $rho$:
\begin{align*}
0 &= \frac {d} {d\rho} \left( \frac {1} {n \omega_n \rho^{n-1}} \int \limits_{\partial B_\rho (x_0)} u(x) d\sigma(x) \right) = 
 \frac {d} {d\rho} \left( \frac {1} {n \omega_n} \int \limits_{\partial B_1(0)} u(x_0 + \rho s) d\sigma(s) \right) = \\
&= \frac {1} {n \omega_n} \int \limits_{\partial B_1 (0)} \nabla u(x_0 + \rho s) \cdot s d\sigma(s) = 
 \frac {1} {n \omega_n} \int \limits_{\partial B_1 (0)} \pder[u]{s}(x_0 +\rho_s) d\sigma(s) = \\
&= \frac {1} {n \omega_n \rho^{n-1}} \int \limits_{\partial B_\rho (x_0)} \pder[u]{n}(z) d\sigma(z) = 
 \frac {1} {n \omega_n \rho^{n-1}} \int \limits_{B_\rho (x_0)} \Delta u (z) dz = 0, \quad \forall \rho
\end{align*}
Таким образом,
$$\Delta u(x_0) = 0, \quad \forall x_0 \in \Omega$$

\end{proof}

\begin{proof}[Доказательство теоремы]
TODO
\end{proof}
\subsection{Функция Грина}
TODO