% !TEX encoding = UTF-8 Unicode
% лекции 11-12, 19 марта 2016
% конец первой части
% вопросы 21-26
% 21. Теоремы о среднем для гармонических функций
% 22. Сильный принцип максимума для гармонических функций, его следствия. Внутренняя задача Дирихле для уравнения Пуассона. Единственность классического решения
% 23. Обратная теорема о среднем
% 24. Бесконечная дифференцируемость гармонических функций
% 25. Оценки производных гармонических функций. Функции, гармонические во всем пространстве. Теорема Лиувилля.
% 26. Функция Грина внутренней задачи Дирихле для уравнения Пуассона. Функция Грина для шара

\section{Гармонические функции}
\begin{definition} Пусть $\Omega \subset \real^n$ - область. Функция называется гармонической в $\Omega$, если она является классическим решением уравнения Лапласа:
$$-\Delta u = 0, \quad u \in C^2(\Omega)$$
\end{definition}
\begin{note}
Очевидно, в $\real$ функция будет гармонической тогда и только тогда, когда она линейна.
\end{note}

\subsection{Свойства гармонических функций}
\begin{note}
Пусть $\Omega \subset \real^n$ - ограниченная область, а $u \in C^2(\Omega)\cap C(\overline{\Omega})$ - гармоническая. Тогда 
$$0 = \int \limits_\Omega \Delta \, u dx = -\int \limits_\Omega \underbrace {\nabla 1}_{0} \cdot \nabla u dx + \int \limits_{\partial\Omega} 1 \cdot \dfrac{\partial u}{\partial \nu}d\sigma\,\Rightarrow\, \int\limits_{\partial\Omega}\dfrac{\partial u}{\partial \nu}d\sigma = 0.$$
Дадим физическую интерпретацию этому замечанию. Уравнение Лапласа задаёт стационарное температурное поле без источников и стоков теплоты. Чтобы такое поле существовало, необходимо, чтобы суммарный поток тепла через границу области должен быть равен нулю.
\end{note}

\begin{theorem}[о среднем для гармонических функций]
Пусть $\Omega \subset \real^n$, $n > 2$, $u$ - гармоническая в $\Omega$, тогда для любого $x_0 \in \Omega$ верно следующее:
$$u(x_0) = \fint \limits_{\partial B_\eps(x_0)} u(x) d\sigma(x) = \fint \limits_{B_\eps(x_0)} u(x)dx, \quad \forall \eps > 0 \,:\, \overline{B}_\eps(x_0)\subset\Omega.$$
То есть, значение гармонической функции в точке равно как среднему по границе шара, так и среднему по шару с центром в этой точке для любого шара в $\Omega$.
\end{theorem}

\begin{proof}Для начала докажем первое равенство. Пользуясь третьей формулой Грина, получаем

\begin{align*}
u(x_0) & = \int \limits_{\partial B_{\eps}(x_0)} \Phi(x-x_0) \pder[u]{\nu} d\sigma - \int \limits_{\partial B_{\eps}(x_0)} u \pder[\Phi]{\nu} d\sigma - \underbrace {\int \limits_{B_{\eps}(x_0)} \Delta u \Phi (x-x_0) dx}_{=0} = \\
& = \frac {1} {n(n-2)\omega_n}\left(\int \limits_{\partial B_{\eps}(x_0)} \frac {1} {\abs{x - x_0}}^{n-2} \pder[u]{\nu} d \sigma - \int \limits_{\partial B_{\eps}(x_0)} u \pder[]{\nu}\left(\frac {1} {\abs{x-x_0}^{n-2}}\right) d \sigma \right) = \\
& = \frac {1} {n(n-2)\omega_n} \left( \frac {1} {\eps^{n-2}} \underbrace {\int \limits_{\partial B_{\eps}(x_0)} \pder[u]{\nu} d \sigma}_{=0 \text{ по замечанию}} - \int \limits_{\partial B_{\eps}(x_0)} u \frac {n-2} {\eps^{n-1}} d \sigma  \right) = \\
& = \frac {1} {n \omega_n \eps^{n-1}} \int \limits_{\partial B_{\eps} (x_0)} u d \sigma = \fint \limits_{\partial B_{\eps} (x_0)} u d \sigma.
\end{align*}
Первое равенство доказано. Докажем второе. Перепишем интеграл по шару в сферической системе координат:
\begin{align*}
\int \limits_{B_{\eps} (x_0)} u dx &= \int \limits_0^\eps r^{n-1} dr \int \limits_{\partial B_1 (0)} u(\underbrace {x_0 + rs}_{=z}) d \sigma(s) = \int \limits_0^\eps dr \underbrace {\int \limits_{\partial B_r (x_0)} u(z) d \sigma(z)}_{\text{площадь поверхности } \cdot u(x_0)} = \\
& = \int \limits_0^\eps u(x_0) \sigma(\partial B_r(x_0)) dr = u(x_0) \abs{B_\eps (x_0)}.
\end{align*}
Итого,
$$u(x_0) = \fint \limits_{B_{\eps} (x_0)} u dx.$$

\end{proof}

\begin{note}
При $n = 2$ счёт происходит аналогично.
\end{note}

\begin{note}
Пусть $u \in C(\overline{\Omega})$ (непрерывна вплоть до границы). Тогда если есть шар $B_{\eps_0}$, касающийся границы $\Omega$, то утверждение теоремы верно для $\eps \leq \eps_0$. 
\end{note}
\begin{proof}
Пользуясь непрерывностью $u$, переходим к пределу.

\end{proof}

\begin{theorem}[Обратная теорема о среднем]
Пусть $\Omega \subset \real^n$ - область, $u \in C(\Omega)$. Если для любого шара $B_\rho (x_0)$, замыкание которого лежит в $\Omega$, выполняется
$$ u(x_0) = \fint \limits_{B_\rho(x_0)} u(x) dx,$$
то $u$ - гармоническая. 
\end{theorem}
\begin{note}
Пусть выполняются условия теоремы, тогда $u$ удовлетворяет соотношению
$$ u(x_0) = \fint \limits_{\partial B_\rho(x_0)} u(x) dx$$
\end{note}
\begin{proof}[Доказательство первого замечания]
Снова воспользуемся сферическими координатами:
$$u(x_0) \cdot \abs{B_\rho (x_0)} = \int \limits_{B_\rho (x_0)} u(x) dx = \int \limits_0^\rho dr \int \limits_{\partial B_r (x_0)} u(z) d \sigma(z).$$
В то же время
$$ u(x_0) \cdot \abs{B_\rho (x_0)} = u(x_0) \cdot \omega_n \rho^n.$$
Приравняем и продифференцируем по $\rho$:
$$ u(x_0) \cdot \underbrace {\omega_n n \rho^{n-1}}_{= \sigma(\partial B_\rho (x_0))} = \int \limits_{\partial B_\rho (x_0)} u(z) d\sigma(z).$$
Итого
$$ u(x_0) = \fint \limits_{\partial B_\rho (x_0)} u(z) d\sigma(z).$$

\end{proof}

\begin{note}
Теорема верна в предположении, что $u \in C^2 (\Omega)$.
\end{note}
\begin{proof}[Доказательство второго замечания]
Дифференцируем соотношение из условия теоремы по $\rho$:
\begin{align*}
0 &= \frac {d} {d\rho} \left( \frac {1} {n \omega_n \rho^{n-1}} \int \limits_{\partial B_\rho (x_0)} u(x) d\sigma(x) \right) = 
 \frac {d} {d\rho} \left( \frac {1} {n \omega_n} \int \limits_{\partial B_1(0)} u(x_0 + \rho s) d\sigma(s) \right) = \\
&= \frac {1} {n \omega_n} \int \limits_{\partial B_1 (0)} \nabla u(x_0 + \rho s) \cdot s d\sigma(s) = 
 \frac {1} {n \omega_n} \int \limits_{\partial B_1 (0)} \pder[u]{s}(x_0 +\rho_s) d\sigma(s) = \\
&= \frac {1} {n \omega_n \rho^{n-1}} \int \limits_{\partial B_\rho (x_0)} \pder[u]{n}(z) d\sigma(z) = 
 \frac {1} {n \omega_n \rho^{n-1}} \int \limits_{B_\rho (x_0)} \Delta u (z) dz = 0, \quad \forall \rho
\end{align*}
Таким образом,
$$\Delta u(x_0) = 0, \quad \forall x_0 \in \Omega$$

\end{proof}

\begin{proof}[Доказательство теоремы]
Имеем функцию $u \in C(\Omega)$. Сгладим её. Для начала рассмотрим функцию
$$ \psi \in C_0^{\infty} (\real), \quad \psi(-x) = \psi(x), \quad \int \limits_{\real} \psi = 1, \quad \supp \psi \subset B_1(0).$$
Теперь введём аппроксимативную единицу $\varphi_\eps$:
$$ \eps >0, \quad \varphi_\eps(x) = \frac {1} {\eps^n} \psi (\frac {\abs{x}} {\eps}). $$
Определим $u_\eps$:
$$ u_\eps = u * \varphi_\eps$$
Заметим, что $u_\eps$ определена не на всём $\Omega$, а только на его подмножестве $\Omega_\eps$, расстояние от которого до границы $\Omega$ равно $\eps$. % здесь можно вставить рисунок

Распишем $u_\eps(x)$:
\begin{align*}
	u_\eps (x) &= \int \limits_{\real^n} u(y) \varphi_\eps (x-y) \, dy = \int \limits_{B_\eps (x)} u(y) \varphi_\eps (x-y) \, dy = \\
	&= \int \limits_0^\eps \, dr \int \limits_{\partial B_r (x)} u(y) \varphi_\eps (x-y) \, d\sigma(y) = \int \limits_0^\eps \, dr \int \limits_{\partial B_1 (0)} u(x+rs) \varphi_\eps (rs) r^{n-1} \, d\sigma(s)
\end{align*}
Функция $\varphi_\eps$ радиальна, значит, можем вынести её за интеграл по $d\sigma(s)$:
\begin{align*}
	u_\eps (x) &= \int \limits_0^\eps r^{n-1} \frac {\psi (\frac {r} {\eps})} {\eps^n} \, dr \int \limits_{\partial B_1 (0)} u(x+rs) \, d\sigma(s) = \int \limits_0^\eps \frac {\psi(\frac {r} {\eps})} {\eps^n} \, dr \int \limits_{\partial B_r(x)} u(z) \, d\sigma(z) =  \\% по первому замечанию 
	&= \int \limits_0^\eps \frac {\psi (\frac {r} {\eps})} {\eps^n} u(x) \omega_n n r^{n-1} \, dr = u(x) \underbrace {\int \limits_0^\eps \frac {r^{n-1}} {\eps^n} \psi \left( \frac {r} {\eps} \right) \omega_n n \, dr}_{\int_{\real^n} \varphi_\eps}
\end{align*}
Получили
\begin{align*}
u_\eps (x) &= u(x) \int \limits_{\real^n} \varphi_\eps = u(x) \quad \text{на } \Omega_\eps \, \forall \eps 
\end{align*}
% дорасписать про \Omega_\eps
\end{proof}

\begin{corollary}
Если $u$ - гармоническая в $\Omega$, то $u \in C^{\infty} (\Omega)$.
\end{corollary}

\begin{corollary}
Любая производная гармонической функции - тоже гармоническая функция.
\end{corollary}

\begin{corollary} Пусть $u$ - гармоническая, тогда
$$\abs{D^{\alpha} u(x)} \leq \frac {C_n^{\abs{\alpha}}} {r^{\abs{\alpha}}} \max_{B_r (x)} u(x).$$
\end{corollary}
\begin{proof}
По теореме о среднем:
$$ u_{x_i} (x) = \frac {1} {\omega_n r^n} \int \limits_{B_r (x)} u_{x_i} (x) = \frac {1} {\omega_n r^n} \int \limits_{\partial B_r (x)} u n_i \, d\sigma.$$
Тогда
$$ \abs{u_{x_i} (x)} \leq \frac {1} {\omega_n r^n} n \omega_n r^{n-1} \max_{B_r (x)} u(x) = \frac {n} {r} \max_{B_r (x)} u(x).$$
Для произвольного мультииндекса $\alpha$ оценивание можно проделать соответствующее количество раз, тогда:
$$ \abs{D^{\alpha} u(x)} \leq \frac {C_n^{\abs{\alpha}}} {r^{\abs{\alpha}}} \max_{B_r (x)} u(x).$$

\end{proof}

\begin{note}
В частности, если $u$ ограничена в $\overline \Omega$, то
$$ \abs{D^{\alpha} u(x)} \leq \frac {C_n^{\abs{\alpha}}} {d(x, \partial \Omega)^{\abs{\alpha}}} \max_{\overline{\Omega}} u.$$
\end{note}

\begin{definition}
Функция называется вещественно аналитической, когда она представима в виде своего ряда Тейлора в окрестности всякой точки. При этом в каждой точке ряд равномерно сходится.
\end{definition}
\begin{corollary}
Если $u$ - гармоническая, то она вещественно аналитическая.
\end{corollary}
\begin{proof}
Запишем ряд Тейлора:
$$ \sum \limits_\alpha \frac {D^{\alpha} u(x_0)} {\abs{\alpha}!} (x - x_0)^{\alpha}, \quad (x-x_0)^{\alpha} = \prod \limits_{i \in 1:n} (x^{(i)} - x_0^{(i)})^{\alpha_i}.$$
Будет ли этот ряд сходиться? Возьмём $x$:
$$ x \in B_r (x_0), \quad \abs{(x-x_0)^{\alpha}} \leq r^{\abs{\alpha}},$$
тогда каждая производная оценивается как
$$ \frac {\abs{D^\alpha u(x_0) \cdot (x-x_0)^\alpha}} {\abs{\alpha}!} \leq \frac {C_n^{\abs{\alpha}}} {\abs{\alpha}!} \max_{\overline{B}_r (x_0)} u(x),$$
а весь ряд можно оценить таким образом:
$$ \sum \limits_\alpha \frac {D^{\alpha} u(x_0)} {\abs{\alpha}!} (x - x_0)^{\alpha} \leq \max_{\overline{B}_r (x_0)} u(x) \cdot \underbrace {\sum_\alpha \frac {C_n^{\abs{\alpha}}} {\abs{\alpha}!}}_{\text{сходится}}.$$

\end{proof}

\begin{note}
Класс вещественно аналитических функций меньше класса бесконечно дифференцируемых функций.
\end{note}

\begin{note}
Если вещественно аналитическая функция равна 0 на каком-то интервале, то она равна нулю всюду.
\end{note}

\begin{corollary}[Теорема Лиувилля]
Пусть $\Omega = \real^n$, а функция $u$ - гармоническая и ограниченная. Тогда $u \equiv \const$.
\end{corollary}
\begin{proof}
Воспользуемся предыдущим следствием:
$$ \abs{u_{x_i} (x_0)} \leq \frac {C_n} {r} \sup_{\real^n} u(x) \quad \forall r$$
Устремив $r$ к бесконечности, получаем
$$ \abs{u_{x_i} (x_0)} = 0 \quad \forall x_0 \, \forall i$$
Значит, $\nabla u = 0$, и, как следствие, $u \equiv \const$.

\end{proof}

\begin{theorem}[Принцип максимума]
Пусть $\Omega$ - связная область, а $u$ - гармоническая в ней. Если
$$ \exists x_0 \in \Omega: \, u(x_0) = \sup_\Omega u(x),$$
 то $$u \equiv u(x_0).$$
\end{theorem}
\begin{proof}
Пусть такая точка $x_0$ существует. Представим $\Omega$ в виде дизъюнктного объединения $A_1$ и $A_2$:
\begin{align*}
	A_1 &:= \left\{ x \in \Omega: \, u(x) = u(x_0) \right\} \neq \varnothing, \\
	A_2 &:= \left\{ x \in \Omega: \, u(x) < u(x_0) \right\} = \Omega \setminus A_1.
\end{align*}
Пусть $A_2$ тоже непусто (иначе утверждение теоремы тривиально). Вспомним, что прообраз замкнутого множества замкнут. Заметим, что $A_1$ относительно замкнуто\footnote{ непрерывный образ замкнутого множества замкнут} в $\Omega$, откуда следует, что $A_2$ открыто.

Теперь покажем, что $A_1$ открыто. Возьмём некоторую точку $z \in A_1$ и шар $B_r (z)$ такой, что он не касается границы $\Omega$. Тогда весь шар находится внутри $A_1$. Допустим, это не так. Тогда в какой-то точке $y \in B_r (z) $ значение $u(y) < u(x_0)$ и среднее по этому шару будет меньше, чем $u(x_0)$. Но по теореме о среднем оно должно быть равно $u(z)$. Значит, любой шар $B_r (z)$ лежит в $A_1$. Следовательно, $A_1$ - открытое.
% TODO: написать получше, сейчас рассуждения просто сливаются

Мы пришли к тому, что $\Omega$ разбивается на два непересекающихся открытых множества, что противоречит связности. Значит, $A_2$ - пусто, и $A_1 = \Omega$.

\end{proof}

\begin{note} Заменой $u$ на $-u$ и супремума на инфимум получается принцип минимума.
\end{note}

\begin{corollary} Пусть $\Omega$ - ограниченная область в $\real^n$ и поставлена задача Дирихле для уравнения Пуассона:
\begin{align*}
	\begin{cases*}
		- \Delta u = f,\\
		u\Big\rvert_{\partial \Omega} = u_0.
	\end{cases*}
\end{align*}
Тогда если классическое решение
$$ u \in C^2(\Omega) \cap C(\overline{\Omega})$$
существует, то оно единственно.
\end{corollary}
\begin{proof}
Следует из принципа максимума. Пусть есть два решения $u_1$ и $u_2$, тогда $u = u_1 - u_2$ - решение задачи
\begin{align*}
	\begin{cases*}
		- \Delta u = 0,\\
		u\Big\rvert_{\partial \Omega} = 0.
	\end{cases*}
\end{align*}
Получается, что $u$ - непрерывная вплоть до границы гармоническая функция. Значит, она имеет и максимум, и минимум. Далее возможны два случая:
\begin{enumerate}
\item Максимум достигается внутри области. Тогда внутри каждой компоненты связности $u \equiv 0$, но при этом имеем $0$ на границе. Значит, $u \equiv 0$.
\item Максимум и минимум достигаются на границе. Там она $0$. Значит, $u \equiv 0$.
\end{enumerate}
\end{proof}

\subsection{Функция Грина}
TODO