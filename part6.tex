% !TEX encoding = UTF-8 Unicode
% лекции 11-12, 12 апреля 2016
% конец первой части
% вопросы 21-26
% 21. Теоремы о среднем для гармонических функций
% 22. Сильный принцип максимума для гармонических функций, его следствия. Внутренняя задача Дирихле для уравнения Пуассона. Единственность классического решения
% 23. Обратная теорема о среднем
% 24. Бесконечная дифференцируемость гармонических функций
% 25. Оценки производных гармонических функций. Функции, гармонические во всем пространстве. Теорема Лиувилля.
% 26. Функция Грина внутренней задачи Дирихле для уравнения Пуассона. Функция Грина для шара

\section{Гармонические функции}
\subsection{Свойства гармонических функций}
\begin{definition}
$u$ --- гармоническая функция в $\Omega$, если является решением уравнения Лапласа $-\Delta u = 0.$
\end{definition}
\begin{note}
$u\in C^2(\Omega)\cap C(\overline{\Omega}), \Omega$ --- ограничена, тогда 
$$0 = \int\limits_\Omega \Delta u dx = \int\limits_\Omega 1\cdot \Delta udx = -\int\limits_\Omega\underbrace{\nabla 1}_{0}\nabla udx+\int\limits_{\partial\Omega}1\cdot\dfrac{\partial u}{\partial \nu}d\sigma\,\Rightarrow\, \int\limits_{\partial\Omega}\dfrac{\partial u}{\partial \nu}d\sigma = 0.$$
\end{note}

\begin{theorem}[Теорема о среднем для гармонических функций]
Пусть $u\in C^2(\Omega), -\Delta u = 0,$ тогда $$u(x_0)=\fint\limits_{\partial B_\eps(x_0)}u(x)d\sigma(x)=\fint\limits_{B_\eps(x_0)}u(x)dx\quad \forall\eps\,:\,\overline{B}_\eps(x_0)\subset\Omega.$$
\begin{note}
Если $u\in C(\overline{\Omega}), \, \eps_0\,:\,B_\eps(x_0)$ касается границы, то утверждение верно $\forall \eps\leq\eps_0.$ 
\end{note}
\end{theorem}
\begin{proof}
$u(x_0)\stackrel{\text{по } \MakeUppercase{\romannumeral3}\text{ ф. Грина}}= $
\end{proof}

\subsection{Функция Грина}
TODO