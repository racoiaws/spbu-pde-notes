% !TEX encoding = UTF-8 Unicode
% лекция 19, 16 апреля 2016, конец курса
% 19. Множество собственных чисел самосопряженных компактных положительных операторов.
% 20. Собственные векторы самосопряженных компактных положительных операторов и полные ортогональные системы векторов.
% 21. Собственные функции лапласиана и полные ортогональные системы векторов в $L^2$ и в $H_0^1$.
% 22. Собственные числа и собственные функции лапласиана. Первое собственное число.
% 23. Разрешимость уравнения $− \Delta u + \lambda u = f$.
% 24. Свойства первого собственнного числа лапласиана в $H_0^1$.

% 19. Множество собственных чисел самосопряженных компактных положительных операторов.
\subsection{Собственные числа самосопряженных компактных положительных операторов.}
\begin{prop} Пусть $H$ --- гильбертово пространство, оператор $T \in \linop(H)$ --- самосопряжённый, $\lambda$ и $\mu$ --- его собственные значения. Тогда
$$ Tu = \lambda u, \quad Tv = \mu v \quad \Rightarrow \quad \scalprod*{H}{u}{v} = 0.$$
То есть, собственные векторы самосопряжённого линейного оператора, соответствующие разным собственным числам, ортогональны.
\end{prop}
\begin{proof}

$$\scalprod{}{u}{v} = \scalprod{}{\frac{1}{\lambda} T u}{v} = \frac{1}{\lambda} \scalprod{}{u}{Tv} = \frac {\mu}{\lambda} \scalprod{}{u}{v}.$$
Значит,
$$ \scalprod{}{u}{v} \left( 1 - \frac{\mu}{\lambda} \right) = 0, \quad \mu \neq \lambda \quad \Rightarrow \quad \scalprod{}{u}{v} = 0.$$

\end{proof}

\begin{prop} Пусть $H$ --- гильбертово пространство, оператор $T \in \linop(H)$ --- самосопряжённый компактный, тогда
$$ \forall C > 0 \quad \# \left\{ \mu : | \mu | \geq C, \quad \mu \in \sigma(T) \right\} < \infty.$$
То есть, у самосопряжённого оператора собственных значений не более чем счётное число.
\end{prop}
\begin{proof}[Доказательство предложения]
Пусть это не так. Тогда
$$ \exists C > 0 : \quad \#\left\{ \mu : | \mu | \geq C, \quad \mu \in \sigma(T) \right\} = \infty,$$
и существует последовательность из положительных чисел $\{ \mu_k \} \subset \sigma(T)$, которые являются собственными значениями оператора $T$. То есть,
$$ \exists u_k \neq 0 : \quad T u_k = \mu_k u_k, \quad \scalprod{}{u_n}{u_m} = 0 \quad \forall n,m. $$ Отнормируем $u_k$:
$$ u_k := \frac {u_k}{\norm{}{u_k}} \quad \Rightarrow \quad \scalprod{}{u_k}{u_j} = \delta_{kj} \quad \forall k \neq j.$$
Оценим:
\begin{align*}
\norm{}{Tu_k - Tu_j}^2 &= \norm{}{\mu_k u_k - \mu_j u_j}^2 = \scalprod{}{\mu_k u_k - \mu_j u_j}{\mu_k u_k - \mu_j u_j} \\
&= \mu_k^2 \scalprod{}{u_k}{u_k} + \mu_j^2 \scalprod{}{u_j}{u_j} - 2 \mu_k \mu_j \scalprod{}{u_k}{u_j} = \mu_k^2 + \mu_j^2 \geq 2C^2.
\end{align*}
Заметим, что $u_k$ лежат на единичной сфере в гильбертовом пространстве. Их образы $Tu_k$ лежат в компактном множестве, значит, эта последовательность должна иметь сходящуюся подпоследовательность. Но расстояние между любыми двумя элементами этой последовательности оценивается снизу константой. Получили противоречие.

\end{proof}

\begin{prop}
Пусть $H$ --- гильбертово пространство, оператор $T \in \linop(H)$ --- самосопряжённый, тогда у него есть хотя бы одно собственное число. Если $T$ к тому же компактен и положителен, то норма оператора является собственным числом:
$$ \lambda = \sup_{\norm{}{u} = 1} \scalprod{}{Tu}{u} = \norm{}{T}.$$
\end{prop}
\begin{proof} Пусть $\lambda = \norm{}{T}$, тогда
$$ \forall k \, \exists u_k : \norm{}{u_k} = 1, \quad \lambda \geq \scalprod{}{Tu_k}{u_k} \geq \lambda - \frac {1} {k}.$$
Посчитаем
\begin{align*}
\norm{}{T u_k - \lambda u_k}^2 &= \scalprod{}{Tu_k - \lambda u_k}{Tu_k - \lambda u_k} \\
& = \norm{}{Tu_k}^2 + \lambda^2 \norm{}{u_k}^2 - 2 \lambda \scalprod{}{Tu_k}{u_k} \\
&\leq \norm{}{T}^2 \cdot \norm{}{u_k}^2 + \lambda^2 \norm{}{u_k} - 2 \lambda \scalprod{}{Tu_k}{u_k} \\
&\leq 2 \lambda^2 - 2 \lambda \scalprod{}{Tu_k}{u_k} = 2\lambda \left( \lambda - \scalprod{}{Tu_k}{u_k} \right) \leq \frac {2\lambda}{k} \conv{}{k \to \infty} 0.
\end{align*}
Таким образом,
$$ T u_k \conv{}{k \to \infty} \lambda u_k \quad \Rightarrow \quad \norm{}{\left( T - \lambda E \right) u_k} \conv{}{k \to \infty} 0.$$
Если $\lambda$ --- не собственное число, то
$$ 1 = \norm{}{u_k} = \norm{}{\left( T - \lambda E\right)^{-1} \left( T - \lambda E \right) u_k } \leq C \norm{}{\left( T - \lambda E \right) u_k} \conv{}{k \to \infty} 0,$$
но это невозможно. Значит, $\lambda$ --- собственное число $T$.
% TODO: написать явно, почему невозможно

\end{proof} % и где конкретно мы пользуемся компактностью и положительностью?
\begin{note}
Вообще говоря, предложение верно для просто самосопряжённого оператора.
\end{note}

Таким образом, у компактного самосопряжённого линейного оператора в гильбертовом пространстве
\begin{enumerate}
\item весь спектр лежит на вещественной оси,
\item спектр состоит из собственных значений, и, возможно, нуля,
\item разным собственным значениям соответствуют взаимно ортогональные собственные подпространства,
\item множество собственных чисел не более чем счётно,
\item есть хотя бы одно собственное число, равное операторной норме.
\end{enumerate}

% 20. Собственные векторы самосопряженных компактных положительных операторов и полные ортогональные системы векторов.
\subsection{Теорема Гильберта-Шмидта}
% TODO: написать пояснее
\begin{theorem}[Гильберт-Шмидт]
Пусть $H$ --- гильбертово пространство, оператор $T \in \linop(H)$ --- самосопряжённый, компактный. Тогда существует не более чем счётная полная ортогональная система, состоящая исключительно из собственных векторов $\{ u_k \}$ оператора $T$.
\end{theorem}
\begin{proof} Обозначим $\mu_0 = 0$, $\mu_i$ --- остальные собственные числа, а $H_i$ --- соответствующие собственные подпространства. В силу теоремы Фредгольма $H_0$ может быть любой размерности, остальные конечномерны.
$$ \dim H_i < \infty \quad \forall i.$$
Обозначим за $\widetilde{H}$ все конечные линейные комбинации собственных векторов:
$$ \widetilde{H} = \Span \bigcup_{i=0}^\infty H_i.$$
Рассмотрим произвольный $u \in \widetilde{H}$. Его можно разложить по базису:
$$ u = \sum_{k=1}^n c_k u_k, \quad u_k \in H_i.$$ 
Пространство $\widetilde{H}$ состоит из собственных подпространств оператора $T$, значит,
$$ T \widetilde{H} \subset \widetilde{H}.$$
Рассмотрим ортогональное дополнение к $\widetilde{H}$. Заметим, что $T \widetilde{H}^\bot \subset \widetilde{H}^\bot$:
$$ \forall u \in \widetilde{H}^\bot \quad \forall v \in \widetilde{H} : \quad \scalprod{}{Tu}{v} = \scalprod{}{u}{Tv} = 0 \quad \Rightarrow \quad Tu \in \widetilde{H}^\bot. $$
Рассмотрим сужение $T$ на $\widetilde{H}^\bot$:
$$ \widetilde{T} = T\Big\rvert_{\widetilde{H}^\bot} : \widetilde{H}^\bot \longrightarrow \widetilde{H}^\bot.$$
У оператора $\widetilde{T}$ единственный возможный элемент спектра это $0$, так как все остальные --- элементы спектра сужения $T$ на $\widetilde{H}$. Отсюда
$$ \sup_{u \in \widetilde{H}} \scalprod{}{\widetilde{T} u} {u } = 0 \quad \Rightarrow \quad \widetilde{T} = 0,$$ потому что
$$ \forall u,v \in \widetilde{H} : \scalprod{}{\widetilde{T} u} {v} = \frac {1}{2} \left( \scalprod{}{\widetilde{T}(u+v)}{u+v} - \scalprod{}{\widetilde{T}v}{v} - \scalprod{}{\widetilde{Tu}}{u} \right) = 0.$$
Таким образом, $$\widetilde{H}^\bot = \Ker T \subset \widetilde{H} \quad \Rightarrow \quad \widetilde{H}^\bot = \{ 0 \} \quad \Rightarrow \quad \widetilde{H} \text{ плотно в } H.$$
Значит, из объединения ортогональных базисов каждого собственного подпространства получается полная ортогональная система, что и требовалось доказать. 

\end{proof}

% 21. Собственные функции лапласиана и полные ортогональные системы векторов в $L^2$ и в $H_0^1$.
\subsection{Собственные функции лапласиана и полные ортогональные системы векторов в $\m{L2}$ и в $\m{H01}$}
\begin{note}
Если $\{ \mu_k \}$ --- собственные числа оператора $K$, то собственные числа лапласиана это в точности $\left\{ \mu_k^{-1} \right\} = \{ \lambda_k \}$:
$$ u = \lambda_j K u \quad \Leftrightarrow \quad - \Delta u = \lambda_j u, \quad u \in \m{H01Om}$$
\end{note}
\begin{note}
Собственные числа оператора $K$ сгущаются у нуля.
\end{note}
\begin{note}
Ноль не является собственным значением лапласиана.
\end{note}
\begin{proof}
Пусть это не так. Тогда для некоторого $u \neq 0$ верно
$$ K u = \mu_0 u = \dot\iota \circ ( - \Delta)^{-1} u = 0 \quad \Rightarrow \quad (-\Delta)^{-1} u = 0 \quad \Rightarrow \quad u = 0.$$

\end{proof}

\begin{theorem}
Множество собственных чисел лапласиана $\{\lambda_j\}$ счётно и существует последовательность собственных функций $\{ u_k \} \subset \m{H01Om}$, являющаяся полной ортогональной системой в $\m{L2Om}$.
\end{theorem}
\begin{proof}
По теореме Гильберта-Шмидта.

\end{proof}

Таким образом, мы нашли полную ортогональную систему в $\m{L2Om}$, но ничего не знаем про полную ортогональную систему в $\m{H01Om}$.

\begin{reminder}В гильбертовом пространстве $H$ система функций $\{ w_j \}$ полна если и только
$$ \forall j \, \, \forall f \in H \quad \scalprod*{H}{f}{w_j} = 0 \quad \Rightarrow \quad f = 0.$$
\end{reminder}

\begin{theorem} Полная ортонормированная система в $\m{L2Om}$ из собственных функций $\{u_k\}$ лапласиана является полной ортогональной системой в $\m{H01Om}$.
\end{theorem}
\begin{proof} Семейство $\{ u_k \}$ таково, что
$$ - \Delta u_k = \lambda_k u_k,$$
и любую $f \in \m{L2Om}$ можно единственным образом разложить в ряд, сходящийся в смысле $\m{L2}$:
$$ f = \sum_{k=1}^\infty c_k u_k.$$

Введём энергетическое скалярное произведение в $\m{H01Om}$:
$$ \scalprod*{E}{u}{v} := \int \limits_\Omega \nabla u \cdot \nabla v \, dx.$$
Это скалярное произведение порождает энергетическую норму, которую мы ранее обозначали через $\norm*{\widehat{H}}{x}$. Мы уже доказывали, что она эквивалентна стандартной норме в $\m{H01Om}$. Значит, и скалярное произведение тоже эквивалентно стандартному.

Посчитаем энергетическое скалярное произведение от двух собственных функций:
\begin{align*}
\scalprod*{E}{u_k}{u_l} &= \int \limits_\Omega \nabla u \cdot \nabla v \, dx = \scalprod*{2}{-\Delta u_k}{u_l} = \lambda_k \scalprod*{2}{u_k}{u_l} = \lambda_k \delta_{kl}. 
\end{align*}
То есть, в энергетической норме в $\m{H01Om}$ собственные функции лапласиана не ортонормированы, но всё же ортогональны. Таким образом, система $\{ u_k \}$ ортогональна и в $\m{H01Om}$. Проверим её полноту:
\begin{align*}
\scalprod*{E}{f}{u_k} = \int \limits_\Omega \nabla f \cdot \nabla u_k \, dx = \scalprod*{2}{f}{- \Delta u_k} = \lambda_k \scalprod*{2}{f} {u_k} = 0 \quad \forall k
\end{align*}
Итого, полная в $\m{L2Om}$ ортонормированная система из собственных функций лапласиана полна и ортогональна в $\m{H01Om}$. Для превращения её в ортонормированную достаточно разделить каждую собственную функцию на корень из  соответствующего ей собственного числа.

\end{proof}

\begin{corollary} Пусть $f \in \m{L2Om}$, тогда её можно представить в виде сходящегося в смысле $\m{L2Om}$ ряда:
$$ f = \sum_{k=1}^\infty c_k u_k.$$ Но если при этом $f \in \m{H01Om}$, то ряд будет сходиться и в смысле $\m{H01Om}$.
\end{corollary}

% 22. Собственные числа и собственные функции лапласиана. Первое собственное число.
\subsection{Собственные числа и собственные функции лапласиана. Первое собственное число}
\subsubsection*{Как услышать собственное число}
Пусть $\Omega \subset \real^2$ - открытая связная область. Представим, что эта область --- закреплённая по краям мембрана барабана. Колебания мембраны превращаются в колебания воздуха, которые воспринимаются нашими ушами. Частота колебания мембраны  воспринимаемая нами частота звука.

Пусть $f \in \m{L2}((0,T) \times \Omega)$. Рассмотрим в $\Omega$ волновое уравнение:
\begin{align*}
	\begin{cases*}
		u_{tt} - a^2 \Delta u = f,\\
		u \big\rvert_{t=0} = u_0,\\
		u \big\rvert_{t=0} = v_0,\\
		u \big\rvert_{\partial \Omega} = 0.
	\end{cases*}
\end{align*}
Если $f = 0$, то уравнение описывает свободные колебания, иначе $f$ описывает постоянное возбуждение мембраны. Опыт подсказывает, что в случае свободных колебаний колебания быстро затухнут.

Решим это уравнение. Так как $f$ из $\m{L2}$, то о классическом решении думать не приходится. Заметим, что $f$ лежит в пространстве
$$ \m{L2}((0,T) \times \Omega) = \m{L2}((0,T) , \m{L2Om}).$$
векторнозначных функций. Это функции времени, которые в ответ на момент времени возвращают функцию от $x$. Будем искать решения в виде ряда
$$ u(t,x) = \sum_{j=1}^\infty c_j (t) u_j(x),$$
где $\{ u_j \}$ - полная ортогональная система, состоящая из собственных функций лапласиана:
$$ - \Delta u_j = \lambda_j u_j, \quad u_j \in \m{H01Om}.$$
Здесь ${\lambda_j}$ --- упорядоченные по возрастанию собственные числа лапласиана. Применим к $u$ даламбертиан:
$$ \sum_{j=1}^\infty \ddot{c}_j(t) u_j(x) - a^2 c_j(t) \Delta u_j(x) = f.$$
Тогда
$$ \sum_{j=1}^\infty \left( \ddot{c}_j(t) + a^2 c_j(t) \lambda_j \right) u_j(x) = f (t,x) = \sum_{j=1}^\infty d_j(t) u_j(x).$$
Любая функция из $\m{L2Om}$ единственным образом представима в виде ряда по полной ортогональной системе. Значит, получили бесконечную систему линейных дифференциальных уравнений. Разложим условия для волнового уравнения:
$$ u_0(x) = \sum_{j=1}^\infty c_j(0) u_j(x) = \sum_{j=1}^\infty a_j u_j(x), \quad  v_0(x) = \sum_{j=1}^\infty \dot{c}_j(0) u_j(x) = \sum_{j=1}^\infty b_j u_j(x).$$
Получаем\footnote{Стоит заметить, что именно при из-за выбора собственных функций лапласиана в качестве полной системы наше уравнение превратилось в систему независимых линейных ОДУ. Не факт, что такой удобный для дальшейшей работы результат получился бы при выборе другой полной системы.} бесконечный набор линейных ОДУ: 
\begin{align*}
	\begin{cases*}
		\ddot{c}_j + a^2 \lambda_j c_j = d_j,\\
		c_j(0) = a_j,\\
		\dot{c}_j(0) = b_j.
	\end{cases*}
\end{align*}
Пусть для простоты $f = 0$, тогда $d_j = 0$, и
$$ c_j = A \sin (\sqrt{\lambda_j} a t) + B \cos (\sqrt{\lambda_j} a t) = D_j \sin (\sqrt{\lambda_j} at + \varphi_j).$$
где константы $A$ и $B$ определяются из начальных условий. Здесь $D_j$ называется амплитудой, а $\varphi_j$ --- фазовым сдвигом. Получили решение:
$$ u (t,x) = \sum_{j=1}^\infty D_j \sin (\sqrt{\lambda_j} at + \varphi_j) u_j(x).$$
Каждому $\lambda_j$ соответствует какой-то тон с некоторой частотой.

\begin{note}
Если начальные условия принадлежат хотя бы $L^1$, то по теореме Римана-Лебега коэффициенты $D_j$ тоже достаточно быстро убывают (чем функции более гладкие, тем быстрее). То есть, можно считать, что мы слышим только колебания из начала ряда. Тон, соответствующий первому собственному числу --- основной тон, он звучит громче всех и его мы слышим лучше всех. Тона, соответствующие следующим собственным числам называют обертонами.
\end{note}

\begin{note}
Легитимность полученного решения основывается на предположении, что мы можем проносить дифференциальный оператор под знак ряда.
\end{note}

\begin{lemma} Сходящиеся в смысле обобщённых функций ряды можно почленно дифференцировать.
\end{lemma}

\begin{corollary}
Полученная функция $u$ удовлетворяет уравнению в обобщённом смысле, и при почти каждом $t$ является функцией из $\m{H01Om}$:
$$ u \in \m{L2}((0,T), \m{H01Om}).$$
\end{corollary}

\begin{note}
При решении задачи Дирихле для однородного уравнения теплопроводности тем же способом получится обобщённое решение вида
$$u(t,x) = \sum_{j=1}^\infty J_j e^{-a^2 \lambda_j} u_j(x), \quad u \in \m{L2}((0,T), \m{H01Om}).$$
В этом случае первое собственное число будет больше всех остальных собственных значений влиять на скорость остывания области.
\end{note}

% 23. Разрешимость уравнения $− \Delta u + \lambda u = f$.
\subsection{Разрешимость уравнения $− \Delta u + \lambda u = f$}
\begin{theorem} Пусть $\Omega \subset \real^n$ --- ограниченная область, $f$ из $\m{L2Om}$, последовательность $\{ u_j \}$ --- полная ортонормированная система из собственных функций лапласиана, и поставлена задача
$$ - \Delta u = \lambda u + f, \quad u \in \m{H01Om}, \quad g = (-\Delta)^{-1} f.$$
Тогда если $\lambda$ - собственное число лапласиана, то существует единственное решение задачи в виде
$$ u = \sum_{j=1}^\infty \frac{\lambda_j}{\lambda_j - \lambda} \scalprod{}{g}{u_j} u_j,$$
а если $\lambda = \lambda_i$ для некоторого $i$, то
$$ u = \sum_{j \neq i} \frac {\lambda_j} {\lambda_j - \lambda} \scalprod{}{g}{u_j} u_j.$$
\end{theorem}

\begin{proof} Запишем уравнение в виде
$$ u = \lambda K u + g, \quad g = (\Delta )^{-1} f.$$
Существует полная ортонормированная система из собственных функциий лапласиана:
$$ \exists \{ u_j \} : - \Delta u_j = \lambda_j u_j \quad \rightarrow u_j = \lambda_j K u_j.$$
Тогда $u$ можно разложить по этой системе:
$$ u = \sum_{j=1}^\infty c_j u_j = \lambda \sum_{j=1}^\infty c_j K u_j + g = \lambda \sum_{j=1}^\infty \frac{c_j}{\lambda_j} u_j + g.$$
Перенесём один ряд в левую часть:
$$ \sum_{j=1}^\infty c_j \left( 1 - \frac {\lambda} {\lambda_j} \right) u_j = g.$$
Функцию $g$ тоже можно разложить по системе из собственных функций:
$$g = \sum_{j=1}^\infty d_j u_j, \quad d_j = \scalprod{}{g}{u_j}.$$
Значит,
$$ c_j \left( 1 - \frac{\lambda}{\lambda_j} \right) = d_j \quad \forall j.$$
Разберём оба варианта событий:
\begin{enumerate}
\item Если $\lambda$ не является собственным числом, то это случай а) из теоремы Фредгольма, и
$$ c_j = \frac {d_j}{1- \lambda/\lambda_j} = \frac {\lambda_j}{\lambda_j - \lambda} d_j, \quad u = \sum_{j=1}^\infty \frac{\lambda_j}{\lambda_j - \lambda} \scalprod{}{g}{u_j} u_j.$$
\item Если $\lambda = \lambda_i$ для некоторого $i$, то это случай б) из теоремы Фредгольма. Решение существует если и только если коэффициенты $d_i$ в разложении $g$ равны нулю.
$$ \scalprod{}{g}{u_i}=0 \quad \forall i: \lambda = \lambda_i,$$
Тогда решение можно выписать в виде ряда:
$$ u = \sum_{j \neq i} \frac {\lambda_j} {\lambda_j - \lambda} \scalprod{}{g}{u_j} u_j.$$
В этом случае решение не единственное.
\end{enumerate}

\end{proof}



% 24. Свойства первого собственнного числа лапласиана в $H_0^1$.
\subsection{Свойства первого собственного числа лапласиана в $\m{H01}$}

\begin{theorem} Пусть $\Omega \subset \real^n$ --- ограниченная область, $u \in \m{H01Om}$, и $\{ u_k \}$ --- полная ортонормированная в $\m{L2Om}$ система из собственных функций лапласиана, и пусть собственные числа $\lambda_k$ упорядочены по возрастанию. Тогда

$$ \lambda_1 = \min_{u \in \m{H01Om}} \int \limits_\Omega |\nabla u|^2 \, dx \quad \text{при } \norm*{2}{u} = 1.$$
\end{theorem}
\begin{proof} Докажем, что $\norm*{E}{u}^2 \geq \lambda_1$. Функция $u$ представима в виде
$$ u = \sum_{k=1}^\infty c_k u_k.$$
Посчитаем её энергетическую норму. Можем пронести градиент внутрь, сходимость ряда сохранится в смысле сходимости в $\m{L2Om}$:
\begin{align*}
\int \limits_\Omega | \nabla u|^2 \, dx &= \int \limits_\Omega \Big\lvert \sum_{k=1}^\infty c_k \nabla u_k \Big\rvert^2 \, dx \\
& =\int \limits_\Omega \sum_{k=1}^\infty c_k^2 | \nabla u_k|^2 + \sum_{k \neq l} c_k c_l \underbrace{\int \limits_\Omega \nabla u_k \cdot \nabla u_l \, dx}_{=0} \\
& = \sum_{k=1}^\infty c_k^2 \int \limits_\Omega |\nabla u_k|^2 \, dx = \sum_{k=1}^\infty \lambda_k c_k^2 \geq \lambda_1 \sum_{k=1}^\infty c_k^2 = \lambda_1 \norm*{2}{u}.
\end{align*}
В конце мы воспользовались равенством Парсеваля.

Заметим, что при $u = u_1$ неравенство превращается в равенство.

\end{proof}

\begin{exercise} Доказать, что
$$ \lambda_k = \min_{u \in \m{H01Om}}  \int \limits_\Omega | \nabla u|^2 \, dx \quad \text{при } \norm*{2}{u} = 1,$$
$$ \text{и при } \scalprod{}{u}{v} = 0 \text{ для всех таких } v, \text{ что  } -\Delta v = \lambda_j v \text{ при } \lambda_j < \lambda_k. $$
\end{exercise}

\begin{corollary} Число $1 / \lambda_1$ --- самая маленькая константа, которую можно взять в неравенстве Фридрихса.
\end{corollary}

\begin{theorem}  Пусть $\Omega \subset \real^n$ --- ограниченная область, $u \in \m{H01Om}$, и $\{ u_k \}$ --- полная ортонормированная в $\m{L2Om}$ система из собственных функций лапласиана, и пусть собственные числа $\lambda_k$ упорядочены по возрастанию. Тогда условие
$$ \int \limits_\Omega | \nabla u|^2 \,dx = \lambda_1, \quad \int \limits_\Omega u^2 \, dx = 1,$$
равносильно условию
$$ - \Delta u = \lambda_1 u .$$
\end{theorem}
\begin{proof} Пусть
$$ - \Delta u = \lambda_1 u, \quad u \in \m{H01Om}, \quad \norm*{2}{u} = 1.$$
Посчитаем энергетическую норму:
$$ \int \limits_\Omega |\nabla u|^2 \, dx = \lambda_1 \scalprod*{2}{u}{u} = \lambda_1.$$

Пусть теперь
$$\int \limits_\Omega | \nabla u|^2 \,dx = \lambda_1, \quad u \in \m{H01Om}, \quad \int \limits_\Omega u^2 \, dx = 1.$$
Функция $u$ представима в виде
$$ u = \sum_{k=1}^\infty c_k u_k,$$
тогда
$$ \int \limits_\Omega | \nabla u|^2 \, dx = \int \limits_\Omega \Bigg| \sum_{k=1}^\infty c_k \nabla u_k \Bigg|^2 \, dx = \sum_{k=1}^\infty c_k^2 \lambda_k^2 = \lambda_1.$$
Заметим, что
$$ \lambda_1 = \lambda_1 \norm*{2}{u}^2 = \lambda_1 \sum_{k=1}^\infty c_k^2 \quad \Rightarrow \quad \sum_{k=1}^\infty c_k^2 (\lambda_1 - \lambda_k) = 0.$$
То есть, коэффициенты $c_k=0$ при $\lambda_k > \lambda_1$, и $u$ представима в виде
$$ u = \sum_{j=1}^m c_j u_j, \quad m - \text{размерность собственного подпространства } \lambda_1.$$
Итого
$$ - \Delta u = \sum_{j=1}^m c_j  (- \Delta u_j) = \lambda_1 u.$$

\end{proof}