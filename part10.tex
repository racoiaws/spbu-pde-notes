% !TEX encoding = UTF-8 Unicode
% лекция 19, 16 апреля 2016, конец курса
% 19. Множество собственных чисел самосопряженных компактных положительных операторов.
% 20. Собственные векторы самосопряженных компактных положительных операторов и полные ортогональные системы векторов.
% 21. Собственные функции лапласиана и полные ортогональные системы векторов в $L^2$ и в $H_0^1$.
% 22. Собственные числа и собственные функции лапласиана. Первое собственное число.
% 23. Разрешимость уравнения $− \Delta u + \lambda u = f$.
% 24. Свойства первого собственнного числа лапласиана в $H_0^1$.

% 16.04.16 (последняя лекция)
% собственные числа лапласиана и уравнение - \Delta u = \lambda u + f

% определение собственных значений оператора
% у самосопряжённого оператора вещественный спектр
% ортогональность собственный элементов самосопряжённого оператора

% предложение: если оператор самосопряжён, то у него есть хотя бы одно собственное число
% упражнение: вспомнить пример оператора без собственных чисел с непустым спектром
% что-то про ортогональное дополнение

% теорема: собственные элементы лапласиана - полная ортогональная система в L^2
