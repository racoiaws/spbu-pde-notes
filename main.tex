% !TEX encoding = UTF-8 Unicode
\documentclass[12pt, a4paper]{report}
\usepackage{hyperref}
\usepackage{functan}
\usepackage{centernot}
\usepackage{fontspec}
\usepackage{polyglossia}
\usepackage{underscore}
\setdefaultlanguage{russian}
\usepackage[cm]{fullpage}
\usepackage{latexsym}
\usepackage{amsthm}
\usepackage{amsmath}
\usepackage{amssymb}
\usepackage{mathtools}
\usepackage{dsfont}
\usepackage{esint}
\usepackage{mathrsfs}
%\usepackage[parfill]{parskip}
\usepackage{pgfplots}
\usepackage{graphicx}
\pgfplotsset{compat=1.12} 

\graphicspath{ {illustrations/} }

\setmainfont[Mapping=tex-text]{CMU Serif}

\newcommand{\real}{\mathds{R}}
\newcommand{\complex}{\mathds{C}}
\newcommand{\linop}{\mathscr{L}}
\newcommand{\dsint}{\displaystyle \int}
\newcommand{\intO}{\int \limits_{\Omega}}
\newcommand{\eps}{\varepsilon}
\newcommand{\ssubset}{\subset\joinrel\subset}
\newcommand{\beauD}{\mathscr{D}}
\newcommand{\unicom}{\overrightarrow{\rightarrow}}
\newcommand\pder[2][]{\ensuremath{\frac{\partial#1}{\partial#2}}} 
% гамильтониан - \nabla
% лапласиан - \Delta
% даламбертиан - \square

\DeclareMathOperator{\esssup}{ess\,sup}
\DeclareMathOperator{\supp}{supp}
\DeclareMathOperator{\grad}{grad}
\DeclareMathOperator{\const}{const}
\DeclareMathOperator{\diam}{diam}
\DeclareMathOperator{\Ker}{Ker}
\DeclareMathOperator{\Div}{div}
\DeclareMathOperator{\Span}{span}
\DeclareMathOperator{\sign}{sign}
\DeclarePairedDelimiter\abs{\lvert}{\rvert}
\DeclarePairedDelimiter\action{<}{>}

\newtheorem{theorem}{Теорема}
\newtheorem{prop}{Предложение}
\newtheorem*{lemma}{Лемма}
\newtheorem*{corollary}{Следствие}

\theoremstyle{definition}
\newtheorem{definition}{Определение}
\newtheorem*{example}{Пример}
\newtheorem*{examples}{Примеры}

%\theoremstyle{remark}
\newtheorem*{note}{Замечание}
\newtheorem*{reminder}{Напоминание}
\newtheorem*{exercise}{Упражнение}

%макро для второй части
\Macro{L2}{L^2}
\Macro{H1}{H^1}
\Macro{H01}{H_0^1}
\Macro{L2Om}{L^2(\Omega)}
\Macro{H1Om}{H^1(\Omega)}
\Macro{H01Om}{H_0^1(\Omega)}
\Macro{C0iOm}{C_0^\infty(\Omega)}


\begin{document}

\title{Конспект лекций по уравнениям математической физики\thanks{ Роман Коростик, 342} \thanks{ Дмитрий Ковалев, 344} \thanks{ Галина Сазонова, 343} \thanks{ Роман Захаров, 344}}
\author{Лектор: Степанов Евгений Олегович}
\date{Математико-механический факультет СПбГУ \\ Весенний семестр, 2016 г.}

\maketitle

\tableofcontents

% !TEX encoding = UTF-8 Unicode
% лекция 1, 13 февраля 2016
% Однородное линейное транспортное уравнение с постоянными коэффициентами, его физический смысл (задача о распространении пятна загрязнения в канале). Решение однородного уравнения методом характеристик. Бегущая волна.
% Неоднородное линейное транспортное уравнение
\chapter*{Введение}
Курс рассчитан на один семестр и делится на две части.

Первая - классическая, она относится скорее к 19ому веку. Будет рассмотрено несколько конкретных физических моделей, соответствующие уравнения в частных производных будут выведены и решены для некоторых частных случаев.

Вторая часть более современная, и, скорее всего, она окажется концепутально новой. По крайней мере для одной задачи будет получено общее решение, будет доказано существование решения и предложено сходящееся численное решение.

\pagebreak

\section*{Обозначения и некоторые определения}
\begin{enumerate}
\item $\real^n$ - евклидово пространство.
\item $\abs{x}$ - евклидова норма $x$ из $\real^n$.
\item Область - открытое множество в $\real^n$.
\item Если где-то написано $\bar{\Omega}$, то $\Omega$ - ограниченная область.
\item $C(\bar{\Omega})$ - пространство непрерывных функций на $\bar{\Omega}$ с $\infty$-нормой: $\sup \{ \abs{f(x)} : x \in \bar{\Omega} \}$.
\item $L^p (\Omega)$ - пространство с нормой Гёльдера, $1 \leq p \leq \infty$.
\item $L^\infty (\Omega)$ - пространство существенно ограниченных функций.
\item $C^k(\bar{\Omega})$ - все частные производные до порядка $k$ включительно существуют и непрерывны вплоть до границы.
\item $C(\Omega)$ - функции, непрерывные на $\Omega$, не вплоть до границы. Не является нормированным пространством.
\item Носитель функции - замыкание множества, на котором функция не равна $0$. Носитель функции $f$ обозначается $\supp f$.
\item Финитная функция - функция с компактным носителем.
\item $C^k_0(\Omega)$ - финитные $k$ раз гладкие функции.
\item $K \ssubset \Omega$ - K относительно компактно в $\Omega$.
\item $\displaystyle \fint \limits_{\Omega} f(x) dx = \frac{1}{|\Omega|} \int \limits_{\Omega} f(x) dx$, где $\abs{\Omega}$ - мера Лебега области $\Omega$.
% добавить про линейное уравнение (через дифференциальный оператор)
\end{enumerate}

\chapter{}
\section{Задача о распространении пятна загрязнения в канале}
Рассмотрим ситуацию: имеется водоём. В него было сброшено некоторое количество загрязняющего вещества.
Для простоты в качестве водоёма будем рассматривать реку или канал. Пренебрегаем тем, что происходит между берегами. Нас интересует, как загрязнение распространяется в длину, поэтому реку считаем одномерным объектом.

Обозначим через $c (t, x) $ концентрацию загрязняющего вещества в момент $t \in \real_+$ в точке $x \in \real$. Пусть дана функция $ c_0 (x) $, описывающая концентрацию вещества в начальный момент времени $ t = 0 $:
$$ c_0 (x) = c (0, x).$$
% функция линейна
Задача состоит в том, чтобы описать изменение концентрации загрязняющего вещества в реке при условии, что мы что-то знаем про течение реки. Как минимум мы знаем скорость. Будем считать, что загрязняющее вещество никуда не испаряется, тогда можно воспользоваться законом сохранения массы.

\subsection{Закон сохранения массы}

Рассмотрим изменение концентрации вещества на некотором малом интервале $ [x, x + \Delta x] $:
$$ \dsint \limits_x^{x + \Delta x} c (t, \xi) d \xi .$$
Рассмотрим функцию $ q (t, x) $ - поток загрязняющего вещества в момент времени $t$ через стенку $x$. Изменение концентрации можно описать как разность величин "сколько влилось" и "сколько вылилось": 
$$ \dsint \limits_x^{x + \Delta x} c (t, \xi) d \xi = q (t, x) - q (t, x + \Delta x). $$
Будем считать, что функция $q (t, x) $ достаточно гладкая. Внесём под интеграл $ \displaystyle \frac {d} {dt} $, а потом разделим обе части на длину нашего малого интервала $ \Delta x$:
$$ \fint \limits_x^{x + \Delta x} c_t (t, \xi) d \xi  = \frac {q  (t, x) - q (t, x + \Delta x)} {\Delta x}.$$
Далее, устремим длину интервала $\Delta x$ к нулю. Получим, собственно, закон сохранения массы:
$$ c_t (t, x) = -q_x (t, x). $$

Стоит заметить, что предположения о гладкости и других нужных свойствах используемых функций это распространённый приём, используемый физиками для построения моделей. Эти условия достаточно сильны, и не всегда физическое явление можно описать достаточно хорошей функцией. В дальнейшем мы узнаем, что существуют более общие математические модели, которые лучше подходят для описания физических процессов.


Далее хочется узнать, как этот поток зависит от концентрации.

\subsection{Модели}
\subsubsection*{Чистая конвекция или чистый дрейф (транспортное уравнение)}

Допустим, скорость течения $v$ постоянна, а загрязняющее вещество мало диффундирует. То есть, загрязняющее вещество не смешивается с водой (в качестве примера можно привести сброс нефти в реку). Тогда поток вещества, проходящего через точку $x$ равен концентрации вещества умножить на скорость:
$$ q (t, x) = v c (t, x). $$
Если подставить вместо потока закон сохранения массы, то получим транспортное уравнение:
$$ c_t + v c_x = 0.$$

Полностью это уравнение называется однородным линейным транспортным уравнением с постоянными коэффициентами в одномерном случае.

\subsubsection*{Уравнение диффузии (закон Фика) или теплопроводности (закон Фурье)}

Допустим, течения нет (то есть, $ v \equiv 0 $), а загрязняющее вещество хорошо диффундирует (в качестве примера можно привести сброс стирального порошка в стоячий канал). Тогда
$$ q (t, x) = - D c_x (t, x), $$
где $D$ это некоторый коэффициент диффузии.

Это уравнение означает, что в момент времени $t$ поток через стенку $x$ пропорционален градиенту концентрации. Минус в правой части означает, что поток идёт оттуда, где концентрация больше, туда, где она меньше.

Позже мы узнаем, что эта же модель является моделью распространения тепла. В ней величина $1/D$ называется коэффициентом температуропроводности.

\subsubsection*{Уравнение конвекции-диффузии или конвекции с дрейфом (закон Фоккера-Планка)}
% на лекции больше ничего не было
% кроме того, что оно используется в computer science при моделировании (?) стохастических процессов
% но, опять же, это теорвер 
Эта модель описывает случай, когда присутствует и конвекция, и диффузия:
$$ q (t, x)  = x c - D c_x. $$
Подставив закон сохранения массы, получим:
$$ c_t + v_{cx} = D c_{xx}. $$

Помимо физики, такое уравнение часто встречается в теории вероятностей.

\subsection{Решение однородного транспортного уравнения}
Что мы имеем ввиду, когда говорим "решение"? Мы имеем ввиду классическое решение - функцию, которая при подстановке в уравнение даст верное соотношение. Для этого она должна обладать нужной гладкостью, а условия уравнения должны соблюдаться в каждой точке. Так понимали решения до тридцатых годов XX века. К сожалению, оказалось, что для дифференциальных уравнений в частных производных классическое решение - не самое лучшее, и зачастую его недостаточно.

За последний век было придумано много разных решений, обобщающих понятие классического: слабые, вязкостные, энтропийные. Немного позже мы узнаем об одном из них - слабом. Но пока что мы остановимся на уровне XIX века и будем рассматривать только классические решения.

Итак, нам нужно найти некоторую функцию, которая не только удовлетворяет транспортному уравнению, но еще и удовлетворяет начальному условию $ c (0, x) = c_0 (x) $:

\begin{align}
    \begin{cases} 
        c_t + v c_x = 0, \\
        c (0, x) = c_0 (x).
    \end{cases}
\label{transport}
\end{align}

Когда мы выводили это уравнение, мы брали небольшой промежуток реки от $x$ до $ x + \Delta x $ и смотрели, как распространяется загрязняющее вещество в нём. Теперь поступим по-другому: посмотрим, как ведёт себя каждая частица вещества в реке. Каждая частица плывёт по направлению течения со скоростью $v$. Значит, траектория каждой частицы удовлетворяет вспомогательному дифференциальному уравнению $ \dot x = v $. Каково решение этого уравнения?

\[
	x(0) = x_0,\quad \text{тогда } x(t) = x_0 + v t .
\]


Пусть $ c (t, x) $ - решение нашего транспортного уравнения. Подставим $ x(t) $ вместо $x$. То есть, каждая частица вещества двигается по закону $ \dot x = v $. Что получится, если мы посмотрим концентрацию вещества вдоль траектории этого ОДУ?

\begin{align*}
    \frac {d} {dt} c (t, x(t)) & = c_t (t, x(t)) + c_x (x, t) \cdot \dot x (t) = c_t (t, x(t)) + v c_x (t, x(t)) = 0
\end{align*}
%?????
Получается, что $ c_t (t, x(t)) = 0 $. Иначе говоря, вдоль траектории вспомогательного ОДУ функция $c$ является постоянной.




Как теперь найти $ c (t,x) $? Есть точка $ (t, x) $, требуется найти значение концентрации в ней. Смотрим, какая траектория ОДУ проходит через эту точку: через неё проходит единственная прямая. Знаем, что вдоль этой траектории $ c = const $, значит, $ c (t, x) = c_0 (x_0) $. А как выражается $ x_0 $? $ x = x_0 + v t $, значит, $ x_0 = x - vt $.
Итого $c(t,x) = c_0(x(0)) = c_0(x-vt)$.

\begin{definition} Обыкновенное дифференциальное уравнение, вдоль траекторий которого решения уравнения в частных производных постоянны, называется характеристическим.
\end{definition}

\begin{definition} Траектории характеристического уравнения называются характеристическими линиями или характеристиками.
\end{definition}

Наша задача оказалась устроена так, что через каждую точку проходит единственная характеристическая линия. Отсюда, зная начальное значение, мы нашли формулу для решения задачи Коши для транспортного уравнения.

Оформим результат рассуждений в виде теоремы.

\begin{theorem}
Пусть $c_0 \in C^1(t)$. Тогда задача \eqref{transport} имеет единственное классическое решение $$ c (t,x) = c_0 (x - v t) .$$
\end{theorem}

\begin{tikzpicture}
\begin{axis}[
    axis lines = left,
    xlabel = $x$,
    xmin = -7,
    xmax = +14,
    ymax = 1.5,
]

\addplot [
    domain=-7:-3, 
    samples=250, 
    color=red,
]
{1/(1+e^(-10*(x+5)))};

\addplot [
    domain=-7:6, 
    samples=250, 
    color=blue,
]
{1/(1+e^(-10*(x-4)))};

\addlegendentry{$c(-3,x)$}
\addlegendentry{$c(7,x)$}

\addplot [
    domain=-3:-1, 
    samples=5, 
    color=red,
]
{1};

\addplot [
    domain=-1:14, 
    samples=250, 
    color=red,
]
{1/(1+e^(-10*(-x+1)))};

\addplot [
    domain=6:8, 
    samples=5, 
    color=blue,
]
{1};

\addplot [
    domain=8:14, 
    samples=250, 
    color=blue,
]
{1/(1+e^(-10*(-x+10)))};
 
\end{axis}
\end{tikzpicture}


Как выглядит наше решение? Это видно на графиках. При увеличении $t$ профиль нашего загрязнения будет просто смещаться по течению на $vt$. То есть, начальное возмущение, не меняя формы, распространяется со скоростью $v$ - получили бегущую волну.

Тем не менее, для того, чтобы функция считалась классическим решением, она должна принадлежать $C^1$, что, вообще говоря, довольно странно, ведь в реальной задаче функция, описывающая контур профиля загрязнения, вполне может быть не из $C^1$ (например, если профиль - прямоугольник). Получается, что понятие классического решения может оказаться неподходящим и надо каким-то образом ослаблять требования.

\subsection{Неоднородное транспортное уравнение и его решение}

Немного изменим модель. Допустим, сброс был не единовременным, и имеется некий источник загрязняющего вещества с заданной мощностью $f(t,x)$. % мощностью?

Снова рассмотрим изменение концентрации вещества на некотором малом интервале:
$$ \dsint \limits_x^{x + \Delta x} c (t, \xi) d \xi .$$
И снова $ q (t, x) $ - поток загрязняющего вещества в момент времени $t$ через стенку $x$. Тогда, учитывая источник вещества $f$:
$$ \frac{d}{dt} \int \limits_x^{x + \Delta x} c(t, \xi ) d \xi = q(t, x) - q(t, x + \Delta x) + \int \limits_x^{x+\Delta x} f(t,\xi) d\xi. $$
Предполагая достаточную гладкость, вносим $ \displaystyle \frac {d} {dt} $ под интеграл и делим на $ \Delta x $:
$$ \fint \limits_x^{x + \Delta x} c_t (t, \xi) d \xi  = \frac {q  (t, x) - q (t, x + \Delta x)}  {\Delta x} + \fint \limits_x^{x+\Delta x} f(t,\xi) d\xi. $$
Устремляя $ \Delta x$ к нулю, в случае чистой конвекции ($ q = vc $) получаем линейное неоднородное транспортное уравнение:

$$ c_t (t, x) = - vc_x (t, x) + f(t, x). $$

Поставим задачу Коши:

\begin{align}
    \begin{cases} 
        c_t + v c_x = f, \\
        c (0, x) = c_0 (x).
    \end{cases}
\label{transportnonhom}
\end{align}

Посмотрим, как будет вести себя классическое решение на траекториях $ \dot x = v $.
\[
	x(0) = x_0,\quad \text{тогда } x(t) = x_0 + v t .
\]
Подставим $x(t)$ в наше неоднородное уравнение:
$$ c (t, x_0 + vt) = c_0 (x_0) + \int \limits_0^t f(s, x_0 + vs) ds. $$
Заметим, что $ x_0 = x - vt $, тогда
$$ c (t, x) = c_0 (x - vt) + \int \limits_0^t f(s, x + v(s-t)) ds. $$

Оформим результат  в виде теоремы.

\begin{theorem}
Пусть $c_0 \in C^1(\real)$, $f \in C (\real^+ \times \real)$, $f_x \in C (\real^+ \times \real)$. Тогда задача \eqref{transportnonhom} имеет единственное классическое решение $$ c (t, x) = c_0 (x - vt) + \int \limits_0^t f(s, x + v(s-t)) ds .$$
\end{theorem}

Вообще говоря, гладкость $f$ по пространственной переменной - совершенно нефизичное условие, в отличие от непрерывности. Позже мы узнаем про слабые решения, где вместо функций могут фигурировать, например, меры.

В данном случае у нас была чистая конвекция и скорость ни от чего не зависела. Но что получится, если скорость зависит от $x$ и от $t$? Тогда ОДУ будет таким же, но его траектории не обязательно будут прямыми. Существенным условие тут является единственность решения характеристического уравнения. Она нарушается, если $v$ зависит от $x$ не гладким образом, а, скажем, просто непрерывным. Тогда может оказаться, что через одну точку проходит несколько характеристических линий (классический пример - квадратный корень).

А что произойдёт, если у нас не конвекция, а чистая диффузия? Можно ли придумать для $c_t = -Dc_{xx}$ характеристическое уравнение, вдоль траекторий которого решение будет константой? Вообще говоря, можно, но это будет не обыкновенное уравнение, а стохастическое. Для таких уравнений мы не можем найти детерминированные траектории, вдоль которых концентрация не меняется. Но можно "рвзыгрывать" траектории с определенной вероятностью по некоторому закону так, что в среднем для мноих траекторий концентрация постоянна. Таким образом, уравнение диффузии связано со стохастикой.
% !TEX encoding = UTF-8 Unicode
% лекции 3-4, 20 февраля 2016
% вопросы 6-8
% 6. Задача Коши для одномерного волнового уравнения (колебания бесконечной струны). Вывод формулы Даламбера при помощи транспортного уравнения. Свойства решения. Волновые конусы. Конечная скорость распространения волн. Передний и задний фронт волны.
% 7-8. Канонический вид уравнений в частных производных второго порядка, их классификация.

\subsection{Задача Коши для одномерного волнового уравнения}
Будем считать, что струна настолько длинная, что краевыми эффектами того, что струна зажата на концах, можно пренебречь. То есть, что струна просто колеблется в соответствии с нашим волновым уравнением. Волновое уравнение --- один из тех редких случаев, когда можно получить общую формулу решения.
Решение выведем двумя способами.
\subsubsection{Вывод решения при помощи транспортного уравнения}
\begin{gather*}
	u_{tt} - v^2 u_{xx} = (\pder{t} - v \pder{x}) \underbrace {(\pder[u]{t} + v \pder[u]{x})}_{= w} = 0, \\
 	\pder[w]{t} - v \pder[w]{x} = 0.
\end{gather*}

Получили однородное транспортное уравнение, его решение мы знаем:
\begin{gather*}
	w(t,x) = \psi (x + vt),\quad \psi (x) = w(0,x) \\
	\pder[u]{t} + v \pder[u]{x} = \psi (x + vt)		
\end{gather*}

Получили неоднородное транспортное уравнение, его решение мы тоже знаем:
\begin{gather*}
	u(t,x) = u_0 (x - vt) + \int \limits_0^t \psi(x + vs - v(t-s)) ds = u_0 (x - vt) + \int \limits_0^t \psi (x - vt + 2vs) ds, \\
	\text{сделаем замену }y = x - vt + 2vs, \text{ тогда} \\
\end{gather*}
\begin{equation}
    u(t,x) = u_0 (x - vt) + \frac {1} {2v} \int \limits_{x-vt}^{x+vt} \psi (y) dy
\label{wavehomans}
\end{equation}

\begin{exercise}
При каких $\varphi$ и $\psi$ $\eqref{wavehomans}$ --- решение $\eqref{waveequation}$? (Ответ: $\varphi \in C^2$, $\psi \in C^1$)
\end{exercise}

\begin{theorem} Пусть $\varphi \in C^2(\real)$ и $\psi \in C^1(\real)$. Тогда
$$ u(t, x) = \varphi (x - vt) + \frac {1} {2v} \int \limits_{x - vt}^{x+vt} \psi (y) dy$$
--- решение одномерного волнового уравнения.
\end{theorem}

Поставим задачу Коши для волнового уравнения:

\begin{align}
    \begin{cases} 
        u_{tt} - v^2 u_{xx} = 0, \\
        u (0, x) = u_0 (x), \\
        u_t(0,x) = v_0 (x).
    \end{cases}
\label{wavecauchy}
\end{align}

Найдем решение задачи Коши. Подставляя $(0,x)$ в общее решение, получаем $$ u_0(x) = \varphi(x) .$$ Найдём $v_0(x)$:
\begin{gather*}
	u_t(0,x) = v_0(x) = \frac {1} {2v} (v \psi (x + vt) + v \psi (x - vt)) \Bigg\rvert_{t = 0} - v \varphi' (x)
	= \psi(x) - v \varphi'(x), \\
	\psi(x) = v_0(x) + v u'_0(x).
\end{gather*}
Таким образом,
\begin{align*}
	u(t,x) &= u_0(x-vt) + \frac {1} {2v} \int \limits_{x-vt}^{x+vt} v_0(y) + vu'_0(y)dy \\
	&= u_0(x-vt) + \frac {u_0(x+vt) - u_0(x-vt)} { 2} + \frac {1} {2v} \int \limits_{x-vt}^{x+vt} v_0(y)dy.
\end{align*}
Отсюда легко получаем формулу Д'Аламбера
\begin{equation}
	u(t,x) = \frac {u_0(x+vt) + u_0(x-vt)} { 2} + \frac {1} {2v} \int \limits_{x-vt}^{x+vt} v_0(y)dy.
\label{wavedalembert}
\end{equation}

Сформулируем теорему:
\begin{theorem} Пусть $u_0 \in C^2(\real)$ и $v_0 \in C^1(\real)$, тогда формула Д'Аламбера $\eqref{wavedalembert}$ --- единственное классическое решение задачи Коши для уравнения колебания бесконечной струны.
\end{theorem}

\begin{example}
Пусть $t$ --- время, $x$ --- координата вдоль струны. Предположим, на струне в точке $x_0$ сидит муравей. (Мировая линия объекта - траектория его движения в координатах пространства и времени) Пусть для простоты он не двигается. $x(t)$ - мировая линия, $v_0 = 0$. $x + vt = const$ и $x -vt = const$ - характеристические функции.

%TODO: рисунок про конусы
\includegraphics[scale=0.5]{part2.1.png}

Допустим, кто-то в точке [??] ударил по струне, для простоты пусть этот кто-то её отклонил, но не придал никакой начальной скорости ($v_0 = 0$, $u_0 \neq 0$). Когда муравей узнает о том, что что-то произошло? В момент времени $ t = \frac {l} {v}$, где $l$ - расстояние от муравья до источника возмущения, $v$ - скорость распространения возмущения. Будем считать, что профиль возмущения локализован около точки, тогда профиль распространяется вперёд и назад: бегут две волны. Если бы щелчок по струне был локализован в одной точке (тогда это не было бы классическим решением!), то муравей почуствовал бы фронт волны только на мгновение.
Интегрируем по содержимому конуса прошлого. Задав скорость муравью, можно повлиять на содержимое конуса будущего.
\end{example}

{\small А что будет в трёхмерном случае? Распространение сферических волн. В отличие от формулы Д'Аламбера, не будет разницы между начальным возмущением с некоторой начальной скоростью и без неё. То есть, сферические фронты будут доходить от каждой точки волны. Сначала дойдёт передний фронт, потом задний.

А в двумерном? Распространение цилиндрических волн. Пример - берём удочку с поплавком, идём на пруд без течения. Забрасываем удочку, бросаем далеко от поплавка камушек. Получится примерно точечное возмущение. В какой-то момент передний фронт дойдёт до поплавка и он начнёт дёргаться. Волна пройдёт, а поплавок продолжит свои колебания, теоретически - бесконечно долго. То есть, заднего фронта не будет.}

% Ч Т О ?
TODO: написать нормально про конусы

\subsubsection{Вывод решения при помощи замены переменных}
$$u_{tt} - v^2 u_{xx} = 0$$
Сделаем замену $(t, x) \to (\xi, \eta)$:
$$ \xi = x + vt, \quad \eta = x - vt.$$
\begin{align*}
	u_x &= u_{\xi} \xi_x + u_{\eta} \eta_x = u_{\xi} + u_{\eta}, \\
	u_t &=  u_{\xi} \xi_t + u_{\eta} \eta_t = v u_{\xi} + v u_{\eta}, \\
	u_{xx} &= u_{x\xi} \xi_x + u_{x\eta} \eta_x = u_{\xi\xi} + u_{\eta\xi} + u_{\xi\eta} + u_{\eta\eta}  =  u_{\xi\xi} + 2 u_{\xi\eta} + u_{\eta\eta}, \\
	u_{tt} &= u_{t\xi} \xi_t + u_{t\eta} \eta_t = v^2 (u_{\xi\xi} - u_{\eta\xi}) - v^2 (u_{\xi\eta} - u_{\eta\eta}) = v^2 (u_{\xi\xi} - 2u_{\xi\eta} + u_{\eta\eta}). \\
	\square_v u &= v^2 u_{\xi\xi} - 2v^2 u_{\xi\eta} + v^2 u_{\eta\eta} - v^2 u_{\xi\xi} - 2v^2 u_{\xi\eta} - v^2 u_{\eta\eta} = 0, \\
	u_{\xi\eta} &= 0, \quad u_{\xi} = F(\xi), \\
	u &= \int F(\xi) d\xi = \varphi (\xi) + \psi (\eta). \\
\end{align*}
Таким образом,
$$ u = \underbrace {\varphi (x+vt)}_{\text{волна направо}} + \underbrace {\psi (x-vt)}_{\text{волна налево}}.$$

Решим задачу Коши $\eqref{wavecauchy}$:
\begin{gather*}
	\begin{cases}
		u(0,x) = u_0(x) = \varphi(x) + \psi(x), \\
		u_t(0,x) = v_0(x) = v \varphi'(x) - v \psi(x),
	\end{cases}
	\begin{cases}
		\psi(x) = u_0(x) - \varphi(x), \\
		\varphi'(x)  - vu'_0(x) + v \varphi'(x) = v_0(x).
	\end{cases} \\
	\varphi'(x) = \frac {1} {2v} (v_0(x) + vu'_0(x)), \quad 	\varphi(x) = \frac {1} {2v} \int \limits_0^x v_0(y) + vu'_0(x) dy + C, \\
	\psi(x) = u_0 - \frac {1} {2v} \int \limits_0^x v_0(y) + vu'_0(y) dy - C,
\end{gather*}
Тогда
\begin{align*}
	u(t,x) &= \frac {1} {2v} \int \limits_0^{x+vt} v_0(y) + vu'_0(y) + C + u_0(x-vt) -  \frac {1} {2v} \int \limits_0^{x-vt} v_0(y) + vu'_0(y) dy - C \\
		&= u_0(x-vt) + \frac {1} {2v} \int \limits_{x-vt}^{x+vt} v_0(y) + v u'_0(y) dy.
\end{align*}
Отсюда легко получается та же самая формула Д'Аламбера \eqref{wavedalembert}:
\begin{equation*}
	u(t,x) = \frac {u_0(x+vt) + u_0(x-vt)} {2} + \frac {1} {2v} \int \limits_{x-vt}^{x+vt} v_0(y)dy.
\end{equation*}

% !TEX encoding = UTF-8 Unicode
% лекции 3-4, 20 февраля 2016
% вопросы 6-8
% 6. Задача Коши для одномерного волнового уравнения (колебания бесконечной струны). Вывод формулы Даламбера при помощи транспортного уравнения. Свойства решения. Волновые конусы. Конечная скорость распространения волн. Передний и задний фронт волны.
% 7-8. Канонический вид уравнений в частных производных второго порядка, их классификация.

\subsection{Приведение уравнений второго порядка к каноническому виду в случае двух независимых переменных}

Ранее мы получили решение одномерного волнового уравнения путём замены переменных. Нет ли способа, позволяющего получить замену переменных, приводящую уравнение к некоторому "красивому" виду в общем случае?

Рассмотрим линейное дифференциальное уравнение в частных производных второго порядка с нелинейными коэффициентами:
$$ a u_{xx} + 2bu_{xy} + c u_{yy} + d_1 u_x + d_2 u_y + d_3 u = f.$$
Будет приятно иметь локально обратимое преобразование координат (якобиан преобразования $\neq$ 0). Рассмотрим гладкую замену 
$$(x,y) \rightarrow (\xi, \eta).$$
Применяя правило дифференцирования сложной функции, пересчитаем производные в новых координатах:
\begin{align*}
	u_x &= u_\xi \xi_x + u_\eta \eta_x, \\
	u_y &= u_\xi \xi_y + u_\eta \eta_y, \\
	u_{xx} &= u_{\xi x} \xi_x + u_{\eta x} \eta_x + u_\xi \xi_{xx} + u_\eta \eta_{xx} = u_{\xi \xi} \xi^2_x + 2u_{\xi \eta} \xi_x \eta_x + u_{\eta \eta} \eta^2_x + u_\xi \xi_{xx} + u_\eta \eta_{xx}, \\
	u_{yy} &= u_{\xi y} \xi_y + u_{\eta y} \eta_y + u_\xi \xi_{yy} + u_\eta \eta_{yy} = u_{\xi \xi} \xi^2_y + 2u_{\xi \eta} \xi_y \eta_y + u_{\eta \eta} \eta^2_y + u_\xi \xi_{yy} + u_\eta \eta_{yy}, \\
	u_{xy} &= u_{\xi \xi} \xi_x \xi_y + u_{\xi \eta} (\xi_x \eta_y + \xi_y \eta_x) + u_{\eta \eta} \eta_x \eta_y + u_\xi \xi_{xy} + u_\eta \eta_{xy}.
\end{align*}
Подставим в наше уравнение:
\begin{align*}
	u_{\xi \xi} & (a \xi^2_x + 4b \xi_x \xi_y + c \xi^2_y) + \\
	u_{\eta \eta} & (a \eta^2_x + 4b \eta_x \eta_y + c \eta^2_y) +\\
	2 u_{\xi \eta} & (a \xi_x \eta_x + 4b \xi_x \eta_y + c \xi_y \eta_y) + \\
	u_\xi & (a \xi_{xx} + 2b \xi_{xy} + c \xi_{yy} + d_1\xi_x + d_2 \xi_y) + \\
	u_\eta & (a \eta_{xx} + 2b \eta_{xy} + c \eta_{yy} + d_1 \eta_x + d_2 \eta_y) = f.
\end{align*}

Получили
$$\underbrace {A u_{\xi \xi} + 2B u_{\xi \eta} + C u_{\eta \eta}}_{\text{новая главная часть}} + \text{ч.н.п.} = f,$$
где \begin{align*}
	A &= a \xi^2_x + 4b \xi_x \xi_y + c \xi^2_y, \\
	B &= a \xi_x \eta_x + 4b \xi_x \eta_y + c \xi_y \eta_y, \\
	C &= a \eta^2_x + 4b \eta_x \eta_y + c \eta^2_y.
\end{align*}

Замечаем, что $A$ и $C$ имеют очень похожую структуру. Нельзя ли оставить в главной части только смешанную производную? Посмотрим на систему
\begin{align*}
	\begin{cases*}
	a \xi^2_x + 4b \xi_x \xi_y + c \xi^2_y = 0, \\
	a \eta^2_x + 4b \eta_x \eta_y + c \eta^2_y = 0.
	\end{cases*}
\end{align*}
Если решение этой системы даст нам локально обратимое преобразование координат, то цель достигнута. Второе уравнение это первое (с точностью до замены переменной) так что можно рассматривать только его. Это квадратичная форма, значит,
$$ a (\xi_x - \Lambda^+ \xi_y)(\xi_x + \Lambda^- \xi_y) = a \xi^2_x - a(\Lambda^+ + \Lambda^-)\xi_x \xi_y + a \Lambda^+ \Lambda^- \xi_y^2 =  0.$$
Значит,
\begin{align*}
	\begin{cases*}
		\Lambda^+ + \Lambda^- = - \frac {2b} {a}, \\
		\Lambda^+ \Lambda^- = \frac {c} {a}.
	\end{cases*}
\end{align*}
Можно сказать, что $\Lambda^+$ и $\Lambda^-$ это корни уравнения
$$a \Lambda^2 + 2b \Lambda + c = 0,$$
откуда можно заключить, что
$$ \Lambda^{\pm} = - \frac {b} {a} \pm \sqrt{\frac {b^2} {a^2} - \frac {c} {a}} = \frac {1} {a} (-b \pm \sqrt{b^2 - ac}).$$
Обозначим $$M = \begin{pmatrix} a & b \\ b & c \\\end{pmatrix}.$$
Тогда 
\begin{enumerate}
	\item если $\det M =  ac - b^2 < 0$, то уравнение называется эллиптическим и можно вывести два транспортных уравнения:
	\begin{gather*}
		\xi_x - \frac {1} {a} (-b + \sqrt{b^2 - ac}) \xi_y = 0, \\
		\eta_x - \frac {1} {a} (-b - \sqrt{b^2 - ac}) \eta_y = 0.
	\end{gather*}
	Рассмотрим дифференциальное уравнение:  $$\xi_x = \Lambda^+ \xi_y \quad \Rightarrow \quad y_x = - \Lambda^+ (x,y)\text{ - характеристика},$$
	Пусть $\xi$ - какое-то решение, рассмотрим его на траекториях
	$$ \frac {d \xi (x, y(x))} {dx} = \xi_x + \xi_y \frac {dy} {dx} = \xi_x - \Lambda^+ \xi_y = 0,$$
	тогда решение полученного транспортного уравнения должно быть постоянным на характеристиках этого ОДУ:
	$$ \frac {d \xi (x, y(x))} {dx} = \const \quad \text{при } \frac {dy} {dx} = - \Lambda^-.$$
	% как делаете: решение определяется по этим формулам, лямбда плюс и минус - по коэффициентам. если определитель больше нуля, то есть два вещественных разных решения - лямбда плюс и лямбда минус. решаем дифференциальное уравнение с лямбда плюс и определяем функцию кси из того условия, что кси постоянна на траекториях этого диффура. если лямбда плюс - гладкая функция, то через каждую точку проходит ровно одна характеристическая линия, значит, явным образом можно определить кси. то же самое с эта: решаем уравнение dy/dx = -лямбда^-, это даёт решение второго транспортного уравнения. эти решения и будут нашей заменой переменных. тем самым гарантируется, что останется только коэффициент при u_{xi eta}. то есть для гиперболического уравнения всегда можно найти замену
	
	\item если $\det M = ac - b^2 = 0$, то уравнение называется параболическим и $$\Lambda = - \frac {b} {a} \quad \Rightarrow \quad \frac {dy} {dx} = \frac {b} {a}.$$
	Отсюда можно найти только $\xi$, в этом случае $\eta$ выбирается такая, чтобы якобиан замены не был равен нулю. 
	\begin{exercise} Показать, что в этом случае $A = 0$ и $u_{\xi \eta} = 0$.
	% A =0 ?
	\end{exercise}
	
	\item если $\det M = ac - b^2 > 0$, то уравнение называется эллиптическим и $u_{\xi \eta} =0$, $A = C \neq 0$. Корни будут комплексно сопряжёнными, а замена будет $$\xi = Re ...,\quad \eta = Im ... .$$
\end{enumerate}
	
\subsubsection{Классификация в случае двух независимых переменных}
Рассмотрим дифференциальное уравнение в частных производных второго порядка с переменными коэффициентами:
$$ \underbrace {\sum \limits_{i = 1}^2 \sum \limits_{j = 1}^2 a_{ij}(x_1, x_2) \frac {\partial^2 u} {\partial x_i \partial x_j}}_{\text{главная часть}}  + \underbrace {F(\pder[u]{x_1}, \pder[u]{x_2}, u, x_1, x_2)}_{\text{члены низшего порядка}} = 0,\quad x \in \real^2$$

Обозначим через $A$ симметричную матрицу коэффициентов $a_{ij}$.
\begin{itemize}
\item Если $\det A < 0$, то уравнение называется гиперболическим, и в некоторых координатах $(\xi, \eta)$ уравнение имеет канонический вид:
$$\frac {\partial^2 u} {\partial \xi \partial \eta} + \text{члены низшего порядка} = 0.$$
\item Если $\det A = 0$, то уравнение называется параболическим, и в некоторых координатах $(\xi, \eta)$ уравнение имеет канонический вид:
$$ \frac {\partial^2 u} {\partial \eta^2} + \text{члены низшего порядка} = 0.$$
\item Если $\det A > 0$, то уравнение называется эллиптическим, и в некоторых координатах $(\xi, \eta)$ уравнение имеет канонический вид:
$$ \frac {\partial^2 u} {\partial \xi^2} + \frac {\partial^2 u} {\partial \eta^2} + \text{члены низшего порядка} = 0.$$
\end{itemize}

В случае переменных коэффициентов $a_{ij}$ уравнение может иметь разный тип в разных областях, такие уравнения называются уравнениями переменного типа.

\subsection{Классификация уравнений второго порядка в случае многих независимых переменных}
Рассмотрим дифференциальное уравнение в частных производных второго порядка с переменными коэффициентами:
$$ \underbrace {\sum \limits_{i=1}^n \sum \limits_{j = 1}^n a_{ij} \frac {\partial^2 u} {\partial x_i \partial x_j}}_{\text{главная часть}}  + F(\pder[u]{x_1}, \pder[u]{x_2}, u, x_1, ..., x_n) = 0,\quad x \in \real^n$$

Обозначим через $A$ симметричную матрицу коэффициентов $a_{ij}$. Уравнение называется 
\begin{itemize}
\item эллиптическим, если у матрицы $A$ все собственные числа одного знака;
\item гиперболическим, если матрица $A$ имеет собственные числа разных знаков;
\item параболическим, если у матрицы $A$ есть собственное число, равное нулю.
\end{itemize}

В общем случае нужно занулить $(n^2 - n) / 2$ коэффициентов $a_{ij}$, чтобы получить диагональную матрицу. Неизвестных при этом $n$ штук. Так как
$$\frac {n^2 - n} {2} \leq n \quad \Longleftrightarrow \quad n \leq 3,$$
то при $n \leq 3$ в общем случае можно найти замену, приводящую уравнение к некоторому каноническому виду. При $n > 4$ общего способа нет.

\subsection{Примеры уравнений различных типов}

\begin{example}[Уравнение Пуассона]
$$ - \Delta u = f, \quad x \in \real^n$$
$$\begin{pmatrix}
-1 & \\
   & \ddots \\
   & & -1
\end{pmatrix}\quad \Rightarrow \quad \text{уравнение эллиптическое}.$$
\end{example}

\begin{example}[Волновое уравнение]
$$u_{tt} - v^2 \Delta u = f, \quad (t,x) \in \real^{n+1}$$
$$\begin{pmatrix}
1 & \\
   & -v^2 \\
   & & \ddots \\
   & & & -v^2
\end{pmatrix}\quad \Rightarrow \quad \text{уравнение гиперболическое}.$$
\end{example}

\begin{example}[Уравнение теплопроводности]
$$ u_t - D \Delta u = f, \quad (t,x) \in \real^{n+1}$$
$$\begin{pmatrix}
0 & \\
   & -D \\
   & & \ddots \\
   & & & -D
\end{pmatrix}\quad \Rightarrow \quad \text{уравнение параболическое}.$$
\end{example}
% !TEX encoding = UTF-8 Unicode
% лекции 5-6, 27 февраля 2016
% вопросы 9-11, 13
% 9. Решение задачи Коши для неоднородного волнового уравнения. Метод Дюамеля.
% 10. Уравнение теплопроводности, его физический смысл (теплопроводность, диффузия). Вывод уравнения теплопроводности из соотношения теплового баланса. Вывод одномерного уравнения для концентрации загрязняющего вещества с учетом явлений конвекции и диффузии. Основные постановки задач для уравнения теплопроводности.
% 11. Принцип максимума для уравнения теплопроводности в ограниченной области. Начально-краевая задача в ограниченной области для уравнения теплопроводности. Единственность классического решения.
% 13. Поведение решений уравнения теплопроводности при t \to \infty

\subsection{Решение задачи Коши для одномерного неоднородного волнового уравнения. Метод Дюамеля}
Запишем одномерное неоднородное волновое уравнение.
\begin{equation*}
	u_{tt} - v^2 u_{xx} = f(t,x).
%\label{waveequationnonhom}
\end{equation*}
Здесь $f(t,x)$ - постоянное возмущение струны. Поставим задачу Коши:
\begin{equation}
	\begin{cases}
		u_{tt} - v^2 u_{xx} = f(t,x), \\
		u(0,x) = u_0(x), \\
		u_t(0,x) = v_0(x).
	\end{cases}
\label{wavenonhomcauchy}
\end{equation}
По аналогии с неоднородными линейными ОДУ:
\begin{note} Достаточно решить соответствующую неоднородную задачу с однородными начальными условиями:
\begin{equation}
	\begin{cases}
		\overline{u}_{tt} - v^2 \overline{u}_{xx} = f(t,x), \\
		\overline{u}(0,x) = 0, \\
		\overline{u}_t(0,x) = 0.
	\end{cases}
\label{wavehomcauchy}
\end{equation}
\end{note}
\begin{proof}
Пусть $u = \widetilde{u} + \overline{u}$, где $\widetilde{u}$ - решение соответствующей однородной задачи с неоднородными начальными условиями:
\begin{equation*}
	\begin{cases}
		\widetilde{u}_{tt} - v^2 \widetilde{u}_{xx} = 0, \\
		\widetilde{u}(0,x) = u_0(x), \\
		\widetilde{u}_t(0,x) = v_0(x).
	\end{cases}
\end{equation*}
Тогда простой подставкой нетрудно проверить, что $u$ - решение задачи \eqref{wavenonhomcauchy}.

\end{proof}
В решении задачи $\eqref{wavehomcauchy}$ нам поможет принцип Дюамеля.
\begin{theorem}[Дюамель] Пусть поставлена задача Коши
\begin{equation}
	\begin{cases}
		w_{tt} - v^2 w_{xx} = 0,\quad s \geq 0 \\
		w(s,x) = 0, \\
		w_t(s,x) = f(s,x),
	\end{cases}
\label{waveduhamel}
\end{equation}
тогда решением задачи $\eqref{wavehomcauchy}$ будет\footnote{Здесь важно то, что для каждого $s$ будет своя $w(t,x) = w(t,x,s)$.}
$$ \overline{u}(t,x) = \int \limits_0^t w(t,x,s) ds.$$
\end{theorem}
\begin{proof}
Для начала проверим выполнение начальных условий:
$$\overline{u} (0,x) = \int \limits_0^0 ... = 0, \quad \overline{u}_t (0,x) = \underbrace {w(t,x,t) \Bigg\rvert_{t=0}}_{= 0} + \int \limits_0^t w_t(t,x,s)ds \Bigg\rvert_{t=0} = 0.$$
Далее посчитаем нужные производные:
\begin{align*}
	\overline{u}_{tt} &= \int \limits_0^t w_{tt} (t,x,s) ds + w_t (t, x, t) = \int \limits_0^t w_{tt} (t,x,s)ds + f(t,x), \\
	\overline{u}_{xx} &= \int \limits_0^t w_{xx} (t,x,s) ds.
\end{align*}
Подставляем в уравнение:
$$ \overline{u}_{tt} - v^2 \overline{u}_{xx} = \int \limits_0^t \underbrace{w_{tt}(t,x,s) - v^2 w_{xx}(t,x,s)}_{= 0} ds + f(x,t) = f(x,t).$$

Значит, $\overline{u}$ - действительно решение задачи $\eqref{wavehomcauchy}$.

\end{proof}
Что мы сделали с точки зрения физики? Вместо того, чтобы трактовать $f(t,x)$ как возмущение, действующее в каждый момент времени $t$ в каждой точке $x$, мы сказали, что $f(s,x)$ - это начальная скорость, действующая только в момент времени $s$. Можно сказать, что мы дизинтегрировали решение задачи $\eqref{wavehomcauchy}$ на решение семейства задач $\eqref{waveduhamel}_s$.

Есть смысл выразить $w(t,x,s)$ при помощи формулы Д'Аламбера:
$$ w(t,x,s) = \frac {1} {2v} \int \limits_{x-v(t-s)}^{x+v(t-s)} f(s,y) dy. $$
Тогда решение задачи  $\eqref{wavehomcauchy}$ записывается как
$$ \overline{u} (t,x) = \frac {1} {2v} \int \limits_0^t ds \int \limits_{x-v(t-s)}^{x+v(t-s)} f(s,y) dy,$$
а полное решение записывается как $$u(t,x) = \frac {u_0 (x+vt) - u_0 (x-vt)} {2} + \frac {1} {2v} \int \limits_{x-vt}^{x+vt} v_0(y)dy + \frac {1} {2v} \int \limits_0^t ds \int \limits_{x-v(t-s)}^{x+v(t-s)} f(s,y) dy.$$

Для каких $f$ верно вышесказанное? В формуле Д'Аламбера в качестве $v_0 \in C^1$ взяли $f$. Значит, для $f \in C(\real^+ \times \real)$ и, дополнительно, $f_x \in C(\real^+ \times \real)$. Имеено тогда записанное выше $u(t,x)$ --- классическое решение неоднородной задачи Коши для волнового уравнения.
% так и не понял, почему f in C и f_x in C, а не f in C^1

А могут ли быть два разных решения у $\eqref{wavenonhomcauchy}$? Ежели да, то их разность будет удовлетворять однородному волновому уравнению с нулевыми начальными условиями. А у такого уравнения есть единственное классическое решение, выражаемое формулой Д'Аламбера. Подставляем в неё нулевые начальные условия - получаем ноль. Значит, двух разных классических решений быть не может.

\section{Уравнение теплопроводности}
Следующая модель - модель распространения тепла в пространстве.

Имеется некоторая область, --- ограниченная или нет, --- в ней имеются источники теплоты. Область заполнена некоторым веществом, на границе области поддерживаются некоторые условия. Как описать изменение температурного поля в этой области?

\subsection{Вывод уравнения теплопроводности из соотношения теплового баланса}

Пусть $q(t,x)$ - плотность источника теплоты в момент времени $t \in \real^+$ в точке $x \in \Omega \subset \real^n$. Составим уравнение теплового баланса для произвольного шара $B \subset \Omega$. Насколько изменилась температура в $B$ за $\Delta t$?

Пусть $c$ - теплоемкость вещества, $\rho$ - его плотность:

$$ \int \limits_t^{t +\Delta t} ds \int \limits_B  c \rho u_t dx = \underbrace {\int \limits_t^{t + \Delta t} ds \int \limits_B q(x) dx}_{\text{стоки теплоты}} - \underbrace {\int \limits_t^{t + \Delta t} ds \int \limits_{\partial B} F \cdot n d \sigma}_{\substack{\text{теплообмен со} \\ \text{внешней средой}}}.$$

Поделим на $\Delta t$ и устремим $\Delta t$ к $0$:
$$ \int \limits_B c \rho u_t dx = \int \limits_B q dx - \int \limits_{\partial B} F \cdot n d \sigma. $$

Применяем формулу Гаусса-Остроградского\footnote{Формула Гаусса-Остроградского: $\int_{\partial \Omega} F \cdot n d\sigma = \int_{\Omega} \Div F dx$}:
\begin{gather*}
	\int \limits_B c \rho u_t dx = \int \limits_B q dx - \int \limits_B \Div F dx,
	\int \limits_B c \rho u_t + \Div F - q dx = 0,\quad \forall B
\end{gather*}
Значит,
$$ c \rho u_t + \Div F = q.$$
Поток через поверхность выражается по закону Фурье. Применим его.
$$ F = - \lambda \nabla u \quad \Rightarrow \quad \Div F = - \lambda \Delta u$$
Перебозначив $f = q/c\rho$ и $ a^2 = \lambda / c \rho $ - коэффициент температуропроводности, получаем уравнение теплопроводности:
\begin{equation}
	u_t - a^2 \Delta u = f
\label{heatnonhom}
\end{equation}

Уравнение теплопроводности - линейное дифференциальное уравнение в частных производных с постоянными коэффициентами. Относится к параболическому типу.

Также это уравнение называется уравнением диффузии. Тогда $f$ - загрязнение, $a$ - коэффициент диффузии.

\subsection{Основные постановки задач для уравнения теплопроводности}
Обычно для уравнения теплопроводности $\eqref{heatnonhom}$ ставится одно из трёх краевых условий:

\subsubsection{Краевое условие Дирихле (первое краевое условие)}
$$u \Bigg \rvert_{\partial\Omega} = u_0.$$
Это условие означает, что на границе области поддерживается заданный температурный режим.

Например, есть дачный домик со стенками из слабо теплоизолирующего материала. Тогда наша область это домик, а условие на стенке - температура внешней среды.

\subsubsection{Условие Неймана (второе краевое условие)}
$$\dfrac{\partial u}{\partial n}\Bigg\rvert_{\partial\Omega} = u_0. (\quad \text{чаще всего }u_0 = 0)$$
В случае $u = 0$ означает, что нет теплообмена с внешней средой. В примере с домиком условие означает, что у домика очень тёплые стены.

\subsubsection{Условие конвективного теплообмена (третье краевое условие)}
$$ \frac {\partial u} {\partial n} + \alpha (u - u_0) \Bigg\rvert_{\partial \Omega} = 0.$$

\subsubsection{Начальное условие}
$$ u(0, x) = h(x).$$


Задача с начальными и краевыми условиями называется начально-краевой задачей, без краевых условий - просто начальной задачей (задачей Коши). На разных частях границы могут быть заданы разные условия.

\subsection{Принцип максимума для уравнения теплопроводности в ограниченной области}
Пусть $t \in \real^+$, $x \in \Omega \subset \real^n$, $\Omega$ - ограниченная область. Рассмотрим уравнение теплопроводности в бесконечном цилиндре: $$ u_t - a^2 \Delta u = 0, \quad  (t,x) \in \real^+ \times \Omega$$
Нас будет интересовать поведение решений этой задачи на конечных цилиндрах $$ Q_T = (0, T) \times \Omega $$

\begin{definition}
Параболической границей цилиндра называется множество
$$ S_T = (\left\{ 0 \right\} \times \Omega) \cup ((0,T) \times \partial \Omega) = \partial Q_T \setminus (\left\{ T \right\} \times \partial \Omega) .$$
\end{definition}

\begin{center}
\includegraphics[scale=0.5]{part3.1.png}
\end{center}
% TODO: изображение

\begin{theorem}[Принцип максимума в ограниченной области]

Пусть $u$ - классическое решение уравнения теплопроводности в ограниченной области $\Omega$. Тогда оно достигает максимума на $S_T$ для любого $T$.
\end{theorem}
\begin{proof}
Прежде всего стоит заметить, что максимум достигается, ведь $\overline{Q}_T$ - компакт.

Обозначим точку максимума $u$ через $(t_0, x_0)$. Введем обозначение для оператора теплопроводности:
$$ Lu  = u_t - a^2 \Delta u.$$
Рассмотрим 
$$ v_{\eps} (t,x) = u(t,x) + \eps \abs{x}^2 \quad (\abs{v_{\eps}} > \abs{u})$$
Тогда
$$ L v_{\eps} = \pder[v_{\eps}]{t} - a^2 \Delta v_{\eps} = u_t - a^2 \Delta u - \eps a^2 \underbrace{\Delta \abs{x}^2}_{=2n} = Lu - 2 \eps a^2 n = - 2 \eps a^2 n.$$
Пусть $(t_{\eps}, x_{\eps}) \in \overline{Q}_T$ - точка максимума $v_{\eps}$. Где именно она находится? Возможны три случая:

\begin{enumerate}
\item $(t_{\eps}, x_{\eps}) \in Q_T$ - внутри цилиндра. Так как точка максимума находится в открытом множестве, то в ней все первые производные равны нулю, нас интересует $$v_{\eps, t} = 0.$$
В то же время в точке максимума вторые производные неположительны: $$ v_{\eps, x_i x_i} \leq 0, \quad \forall i \in 1:n $$
Получаем:
$$Lv_{\eps}  = \underbrace {v_{\eps,t}}_{=0} - \underbrace {a^2 \Delta v_{\eps}}_{\leq 0} \geq 0, \quad L v_{\eps} = -2n \eps a^2 < 0$$
Противоречие. Значит, $(t_{\eps}, x_{\eps}) \notin Q_T$.
\item $(t_{\eps}, x_{\eps}) \in \left\{ T \right\} \times \Omega$ - на верхней крышке. По той же причине вторые производные по пространственным координатам неположительны:
$$ v_{\eps, x_i x_i} \leq 0.$$
Максимум достигается на правой границе временного интервала, значит,
$$v_{\eps,t} \geq 0$$
Имеем:
$$Lv_{\eps}  = \underbrace {v_{\eps,t}}_{\geq 0} - \underbrace {a^2 \Delta v_{\eps}}_{\leq 0} \geq 0, \quad L v_{\eps} = -2n \eps a^2 < 0$$
Противоречие. Значит, $(t_{\eps}, x_{\eps}) \notin \left\{ T \right\} \times \Omega$.
\item $(t_{\eps}, v_{\eps}) \in S_T$ - на параболической границе. Единственный оставшийся вариант.
\end{enumerate}
Имеем:
$$ u(t,x) \leq v_{\eps} (t,x) \leq \max_{S_T} v_{\eps} \leq \max_{S_T} u + \eps \underbrace {(\diam \Omega)^2}_{=\const} \xrightarrow[\eps \to 0]{} \max_{S_T} u.$$
Таким образом, максимум $u$ достигается на $S_T$.

\end{proof}

\begin{corollary}[Принцип минимума]
Пусть $u$ - классическое решение уравнения теплопроводности в ограниченной области $\Omega$. Тогда оно достигает минимума на $S_T$ для любого $T$.
\end{corollary}
\begin{proof}
Аналогично, только вместо $u$ рассматриваем $-u$.

\end{proof}

\begin{corollary}[Единственность]
Начально-краевая задача в $\Omega$
\begin{gather*}
	\begin{cases*}
		u_t - a^2 \Delta u = f, \\
		u \Big\rvert_{\partial \Omega} = h, \\
		u(0, x) = g.
	\end{cases*}
\end{gather*}
может иметь не более одного решения.
\end{corollary}
\begin {proof}
Пусть $u_1$ и $u_2$ - решения. Тогда их разность удовлетворяет соответствующему однородному уравнению:
\begin{gather*}
	\begin{cases*}
		L(u_1 - u_2) = 0, \\
		(u_1 - u_2) \Big\rvert_{\partial \Omega} = 0, \\
		(u_1 - u_2) \Big\rvert_{t = 0} = 0.
	\end{cases*}
\end{gather*}
Из однородного уравнения разность наших решений равна нулю на всей границе, а по доказанной теореме максимум и минимум разности достигаются на параболической границе $S_T \subset \real^+ \times \Omega$. Значит,
$$0 = \min (u_1 - u_2) \leq (u_1 - u_2) \leq \max (u_1 - u_2) = 0 \quad \Rightarrow \quad u_1 - u_2 \equiv 0$$

\end{proof}
\begin{corollary}[Поведение решений при $t \to \infty$] Рассмотрим температурное поле в ограниченной области $\Omega$:
\begin{gather*}
	\begin{cases*}
		u_t - a^2 \Delta u = 0, \\
		u \Big\rvert_{\partial \Omega} = 0, \\
		u \Big\rvert_{t = 0} = g.
	\end{cases*}
\end{gather*}
То есть, на стенках поддерживается нулевая температура, в начальное время задано температурное поле $g$, источников и стоков теплоты внутри области нет. Тогда
$$ u(t,x) \xrightarrow[t \to \infty]{} 0$$
равномерно по $x$ и экспоненциально по $t$.
\end{corollary}
\begin{proof}
Будем считать, что $0 \in \Omega$. Рассмотрим
$$ v(t,x) = A e^{-bt} \prod \limits_{i=1}^n \cos c x_i. $$
Так как $\Omega$ - область ограниченная, то она лежит в некотором n-мерном кубе. Значит, можно подобрать $c$ так, чтобы каждый из косинусов был положителен. Выбор $c$ зависит только от диаметра $\Omega$. Далее, применим оператор теплопроводности к $v$:
$$ Lv = -b \cdot v(t,x) + n a^2 c^2 \cdot v(t,x) = (na^2c^2 - b^2) \cdot v.$$
Если $b = n a^2 c^2$, то $Lv = 0$. По этой формуле можем подобрать $b$. Теперь выбираем большое $A$ такое, что
$$ v(0,x) = A \prod \limits_{i=1}^n \cos c x_i > \max_{\Omega} |g(x)|.$$

Имеем $$Lv = 0,\quad v(0,x) \geq \abs{g(x)},\quad v \geq 0 .$$ Рассмотрим
$$ w_{\pm} = v \pm u.$$
Тогда:
\begin{gather*}
	Lw_{\pm} = 0, \\
	w_{\pm}\Big\rvert_{\partial \Omega} = v\Big\rvert_{\partial \Omega} \pm u\Big\rvert_{\partial \Omega} \geq 0, \\
	w_{\pm}\Big\rvert_{t=0} = \underbrace {v\Big\rvert_{t=0}}_{\geq \abs{g}} \pm \underbrace{u\Big\rvert_{t=0}}_{=g} \geq 0
\end{gather*}

Таким образом, $w_{\pm}$ - решения в $\real^n \times \Omega$ и положительны на $S_T$, а по принципу минимума $w_{\pm} > 0$  на $\real^n \times \Omega$. Из определения $w_{\pm}$ 
$$\abs{u(t,x)} \leq v(t,x) = A e^{-bt} \prod \limits_{i=1}^n \cos c x_i \xrightarrow[t \to \infty]{} 0.$$
Выбор констант $A$, $b$ и $c$ зависит только от $\diam \Omega$. Оценивая косинусы единицей, получаем, что
$$ \abs{u(t,x)} \leq A e^{-bt} \xrightarrow[t \to \infty]{} 0$$
равномерно по $x$, экспоненциально по $t$.

\end{proof}


\begin{note} Пусть существует классическое решение $\overline{u}$ задачи
\begin{gather*}
	\begin{cases*}
		\Delta \overline{u} = 0, \\
		\overline{u}\Big\rvert_{\partial \Omega} = h(x).
	\end{cases*}
\end{gather*}
Тогда для решения $u$ задачи 
\begin{gather*}
	\begin{cases*}
		u_t - a^2 \Delta u = 0, \\
		u \Big\rvert_{\partial \Omega} = h(x), \\
		u(0, x) = g(x)
	\end{cases*}
\end{gather*}
верно $$u \xrightarrow[t \to \infty]{} \overline{u}$$
равномерно по $x$, экспоненциально по $t$.
\end{note}

\begin{proof}
Рассмотрим $v = u - \overline{u}$. Тогда
\begin{align*}
& Lv = Lu - L\overline{u} = 0, \\
& v\Big\rvert_{\partial \Omega} = h - h = 0, \\
& v\Big\rvert_{t = 0} = g(x) - \overline{u}(x).
\end{align*}
Заметим, что $v$ удовлетворяет условию следствия 3. Значит, $ v \xrightarrow[t \to 0]{} 0$, и $ u \to \overline{u}$.

\end{proof}
% !TEX encoding = UTF-8 Unicode
% лекция 4, 5 апреля 2016
% Принцип максимума для уравнения теплопроводности во всем пространстве для функций, ограниченных в каждой полосе.
% 2 следствия. Фундаментальное решение для уравнения теплопроводности. Теорема интегральная формула для решения 
% задачи Коши для однородного уравнения теплопроводности в пространстве. Свойства решения. Решение задачи Коши для
% неоднородного уравнения теплопроводности в пространстве. Метод Дюамеля.

\section{Принцип максимума для уравнения теплопроводности во всем пространстве}
Поставим задачу Коши для уравнения теплопроводности во всем пространстве:
\begin{align}
    \begin{cases} 
        u_t - a^2 u_x = 0, \\
        u (0, x) = \varphi (x).
    \end{cases}
\label{transcalency}
\end{align}
[рисунок, относится к цилиндрам]
Будем рассматривать решения на бесконечных цилиндрах $u : \real^+ \times \real^n \rightarrow \real,$ считаем, что все производные существуют вплоть до $t = T$, поскольку решение рассматриваем на всем $\real^+ \times \real^n $.Ограничим класс интересующих нас функций, т.е. дополнительно потребуем: $ \forall T > 0 \quad \exists C = C(T) > 0 : |u(t, x)| \leq C(T) \quad \forall t \in [0, T] \quad \forall x \in \real^n$ (в каждой ограниченной полосе решение $u(t, x)$ - ограничено, ясно, что ограниченность зависит от $t$). Тогда имеем теорему в наших условиях:

\begin{theorem}{(Принцип максимума для уравнения тепопроводности \\ в $\real^n$ для функций, ограниченных в каждой полосе).}

Пусть $T > 0$ $$ M_+ = \sup_{\stackrel{x \in \real^n,} {t \in [0, T]}}u(t,x), \quad N_+ = \sup_{x \in \real^n} u(0, x), $$ то $M_+ = N_+$ (супремум по всей полосе есть супремум при $t = 0$)

\begin{note}Очевидно, что $-\infty < N_+ \leq M_+ < +\infty.$ Почему очевидно? Потому что ограничено в полосе.\end{note}
\end{theorem}

\begin{proof}
Рассмотрим произвольное $\eps.$
$$v_\eps(t, x) = u(t, x) - \eps(2na^2t + \abs{x}^2), \qquad Lu = u_t - a^2v$$
$$Lv_\eps = \underbrace{Lu}_{0} - \eps L(2na^2t + \abs{x}^2) = -\eps(2na^2 - 2n\bigtriangleup x^2) = -\eps(2na^2 - 2na^2) = 0,$$
т.~к. $\bigtriangleup \abs{x}^2 = \bigtriangleup(x_1^2 + x_2^2 + \dots + x_n^2) = \sum_{i = 1}^{n} \dfrac{\partial^2 |x|^2}{\partial x_i^2} = 2n$, т.~о. $v_\eps$ --- тоже решение уравнения теплопроводности во всем пространстве.

[picture] Теперь будем рассматривать открытые цилиндры $Q_{T,R} = (O,T)\times B_R(0)$, полоса --- объединение всех таких цилиндров по $R$. По принципу максимума для решения уравнения теплопроводности в ограниченной области (можем применить, т.к. цилиндр компактен) [здесь будет номер теормы вместо огромного названия] максимум и минимум достигаются на параболической границе. Посмотрим на поведение функции на параболической границе.
Нижняя граница:
$$ v_\eps (0,x) = u(0,x) - \eps |x|^2 \leq N_+ \forall x$$
Боковая поверхность:
$$v_\eps(t,x)\Bigl|_{\abs{x} = R} = u(t,x) \Bigl|_{\abs{x} = R} - \eps(2na^2t + R^2) \leq u(t,x)\Bigl|_{\abs{x} = R} - \eps R^2 \leq M_+ - \eps R^2 \leq N_+$$
Подберем $R: \, M_+ - \eps R^2 \leq N_+$ (Почему так можно сделать? Использовали то, что $M_+$ и $N_+$ --- это числа, ни одно из них не бесконечность, $R(\eps)$ можно выразить как ${R(\eps) = \dfrac{\sqrt{M_+ - N_+}}{\eps} \stackrel{\eps \rightarrow 0} \longrightarrow \infty}$).

Т.~о. $ v_\eps (t, x) \leq N_+ \, \forall (t,x) \in {Q_{T,R(\eps)}};$
$(t,x) \in [0,T] \times \real^n$ нужно доказать, что $u(t,x) \leq N_+.$
$$\text{При }\eps \rightarrow 0, \, \text{начиная с какого-то }\eps \, (t,x) \in Q_{T,R} \Rightarrow$$
$$v_\eps(t,x) = u(t,x) - \eps(2na^2t - \abs{x}^2) \leq N_+ \forall \eps \geq \eps_0 \Rightarrow$$
$$u(t,x) \leq N_+ + \eps\underbrace{(2na^2t + |x|^2)}_{const} \stackrel{\eps \rightarrow 0}{\Rightarrow} u(t,x) \leq N_+$$
Т.~о. $M_+ = N_+.$
\end{proof}

\begin{consequence}{(Принцип минимума для уравнения теплопроводности во всем пространстве)}
$$M_- = \inf_{\stackrel{x \in \real^n, \,} {t \in [0, T]}}u(t,x); \qquad N_- = \inf_{x \in \real^n}u(0,x),$$
то $M_-=N_-$ (инфимум по полосе равен инфимуму по нижней границе).
\end{consequence}

\begin{consequence}{(Единственность)}
Рассмотрим задачу Коши для неоднородного уравнения:
\begin{align}
    \begin{cases} 
        u_t - a^2 \bigtriangleup u = q, \\
        u (0, x) = \varphi (x).
    \end{cases}
\label{transcalencynonhom}
\end{align}
Классическое решение для задачи \eqref{transcalencynonhom} --- единственное в классе функций, ограниченных в каждой полосе.
\end{consequence}

\begin{proof}
Пусть $u_1, \, u_2 $ --- два решения. $L$ --- оператор теплопроводности.
$$Lu_1=Lu_2=q \Rightarrow L(u_1 - u_2) = 0;$$
$$u_1\Bigl|_{t=0} = u_2\Bigl|_{t=0}=\varphi(x) \Rightarrow (u_1-u_2)\Bigl|_{t=0}=0,$$
т.~о. $u = u_1 - u_2$ --- решение однородной задачи с нулевыми начальными условиями (в том же классе ограниченных в полосе функций), т.~о. по теореме 3 $\sup u = 0 \, \text{и} \, \inf u = 0 \Rightarrow u \equiv 0 \Rightarrow u_1 = u_2.$
\end{proof}

\begin{note}
Условие ограниченности в каждой полосе необходимо, т.к. если его убрать, решение не будет единственным, будет решение в полосе и еще какое-то быстрорастущее решение, единственности нет. Неединственность относится уже не к теплопроводности, поэтому физики тоже накладывают ограничения, чтобы получать единственное решение, относящееся к теплопроводности.
\end{note}

\section{Фундаментальное решение для уравнения теплопроводности}

Хотим получить решение для системы \eqref{transcalency}. Получим --- фундаментальное решение уравнения теплопроводности (в ходе решения будем накладывать некоторые специфические условия на решение).

Рассмотрим $a = 1$ иначе можно сделать замену $x \rightarrow \dfrac{x}{a}$ (делаем это для упрощения вычислений).

\begin{enumerate}
\item $u(t,x) = \lambda^\alpha u(\lambda t, \lambda^\beta x) \quad \forall \lambda$ (т.~е. хотим частичную линейность решения)

Рассмотрим $\lambda = \dfrac{1} {t}. \, \alpha, \, \beta$ --- неизвестны.
$$u(t,x) = \dfrac{1}{t^\alpha}(1, \dfrac{x}{t^\beta}) \text{. Обозначим } \, u(1,z) = v(z) \Rightarrow u(t,x) = \dfrac{1}{t^\alpha} v(\dfrac{x}{t^\beta}).$$
Посчитаем $Lu = 0$:
$$-\dfrac{\alpha}{t^{\alpha + 1}} v\left( \dfrac{x}{t^\beta}\right) - \dfrac{\beta}{t^{\alpha + \beta + 1}}\nabla v\left( \dfrac{x}{t^\beta}\right)\cdot x - \dfrac{1}{t^\alpha t^{2\beta}}\bigtriangleup v\left( \dfrac{x}{t^\beta}\right) = 0.$$
$$\left(\text{т.к.} \, \dfrac{\partial}{\partial t} \left(v\left( \dfrac{x}{t^\beta}\right)\right) = \sum \dfrac{ \partial v}{\partial x_i}\left(\dfrac{x}{t^\beta}\right) \cdot \dfrac{-\beta x_i}{t^{\beta + 1}} = -\dfrac{\beta}{t^{\beta + 1}}\nabla v\left( \dfrac{x}{t^\beta}\right) \cdot x \right)$$

Сделаем замену $ y = \dfrac{x} {t^\beta}.$
$$ \dfrac{\alpha}{t^{\alpha + 1}} v(y) + \dfrac{\beta}{t^{\alpha + 1}} \nabla v(y) \cdot y + \dfrac{1}{t^\alpha  t^{2\beta}}\bigtriangleup v(y)=0.$$

Пусть $\beta = \dfrac{1}{2}$ и умножаем полученное выражение на $t^{\alpha + 1}:$
$$\alpha v(y) + \dfrac{1}{2} \nabla v(y) \cdot y + \bigtriangleup v(y) = 0.$$

\item $u(t,x) = \tilde{u}(t,|x|)$ --- хотим симметрии относительно 0, т.~о. $v(y) = w(|y|), |y| = r.$ Тогда получим:
$$ \alpha w(r) + \dfrac{1}{2} w'(r) \cdot r + w''(r) + \dfrac{n}{r}w'(r) = 0;$$
$$\text{т.~к.} \quad \bigtriangleup v(y) = \dfrac{\partial^2v}{\partial y_1^2} + \dots + \dfrac{\partial^2v}{\partial y_n^2};$$
$$\dfrac{\partial w}{\partial y_i} = w' \dfrac{\partial |y|}{\partial y_i} = w' \dfrac{y_i}{r} \Rightarrow 
\dfrac{\partial^2 w}{\partial y_i^2} = w''\dfrac{y_i^2}{r^2} + w'\left(\dfrac{1}{r} - \dfrac{y_i^2}{r^2}\right) \Rightarrow$$
$$\bigtriangleup w(r) = \dfrac{w''}{r^2}\sum y_i^2 + \dfrac{w'}{r}\sum (1 - \dfrac{y_i^2}{r^2}) = w'' + \dfrac{w'n}{r}.$$

Пусть $\alpha = \dfrac{n}{2}.$
$$\dfrac{n}{2}w(r) + \dfrac{1}{2}w'(r) \cdot r + w''(r) + \dfrac{n-1}{r}w'(r) = 0 \Rightarrow$$$$ \dfrac{1}{2}\left(r^nw\right)' + \left(r^{n-1}w'\right)' = 0 \Rightarrow \dfrac{1}{2}\left(r^nw\right)+ \left(r^{n-1}w'\right) = \gamma.$$

\item $r^aw_n' + r^bw \rightarrow 0 \, \text{при} \, n \rightarrow \infty$ --- хотим, чтобы это выполнялось, такое возможно только при $\gamma = 0,$ т.~о.
$$r^{n-1}w_r' + \dfrac{1}{2}r^nw = 0 \Rightarrow w_r' + \dfrac{r}{2}w = 0 \Rightarrow$$
$$\dfrac{dw}{w} = -\dfrac{r}{2}dr \Rightarrow w = be^{-\dfrac{r^2}{4}},$$
$$\text{т.~о.} \, u(t,x) = \dfrac{1}{\sqrt{t^n}}u\left(1, \dfrac{x}{\sqrt{t}}\right) = \dfrac{1}{\sqrt{t^n}}w\left(\dfrac{x}{\sqrt{t}}\right) = \dfrac{b}{\sqrt{t^n}}e^{-\dfrac{|x|^2}{4t}}.$$

Получили однопараметрическое семейство решений.
\begin{note}Если зафиксируем $t$, то полученное выражение --- гауссова функция(нормальное распределение)[рисунок], чем $t$ меньше, тем выше "пика" на графике, соответственно, чем больше, тем более сглаженным будет график.
\end{note}
Теперь определенным образом выберем константу, пусть $b\,:\,b>0; \, \int\limits_{\real^n} u(t,x)dx = 1.$

\begin{note}
Нужно уметь вычислять $b$.
\end{note}

$$b = \dfrac{1}{(4\pi)^{n/2}} \Rightarrow$$$$ u(t,x) = \dfrac{1}{(4\pi t)^{n/2}} \exp\left(-\dfrac{|x|^2}{4t}\right) = \Phi(t,x) \,\text{--- фундаментальное решение уравнения теплопроводности.}$$
\begin{note}
Что если $a \not= 1$?
\end{note}
\end{enumerate}

\begin{theorem}(Интегральная формула для решения задачи Коши)
$ \varphi\in C_b(\real^n)$ --- ограниченная и непрерывная(начальное распределение). Тогда
$$u(t,x) = \int\limits_{\real^n}\Phi(t, x-y)\varphi(y)dy (= \left(\Phi(t,\cdot)*\varphi\right)(x) \text{ --- свертка.})$$
\begin{enumerate}
\item $u \in C^\infty\left(\left(0, +\infty\right)\times \real^n\right);$
\item $ u_t - \bigtriangleup u = 0$ --- т.~е. решение уравнения теплороводности;
\item $\lim\limits_{(t,\,y) \rightarrow (0,\,x)}u(t,x) = \varphi (x).$
\end{enumerate}
\end{theorem}
\begin{note}{Физический смысл фундаментального решения.} %%%слишком много сказано, нужно ли писать, вроде смысла нет.
$\Phi$ - распределение температуры от единичного источника (источник с температурой единица), сосредоточенного в одной точке. 
\end{note}
\begin{proof}
\begin{enumerate}
\item $\Phi \in  C^\infty\left(\left(0, +\infty\right)\times \real^n\right)$ --- очевидно.
$$u_t = \int\limits_{\real^n} \Phi_t(t,x-y)\varphi(y)dy, \qquad u_{x_i} = \int\limits_{\real^n} \Phi_{x_i}(t, x-y)\varphi(y)dy$$

Очевидно, $u \in C^\infty.$
\item $Lu = u_t - \bigtriangleup u = \int\limits_{\real^n} \underbrace{(\Phi_t - \bigtriangleup \Phi)(t, x-y)}_{0}\varphi(y) dy = 0.$
\item Пусть $x_0 \in \real^n.$
$$|u(t,x) - \varphi(x_0)| = \Bigl|\int\limits_{\real^n}{\Phi(t,x-y)\varphi(y)dy} - \varphi(x_0)\Bigl| = \Bigl|\int\limits_{\real^n}\Phi(t,x-y)\varphi(y)dy - \varphi(x_0)\int\limits_{\real^n}\Phi(t,x-y)dy\Bigl|=$$
$$=\Bigl|\int\limits_{\real^n}\Phi(t,x-y)(\varphi(y) - \varphi(x_0))dy\Bigl| \leq \int\limits_{\real^n}\Phi(t,x-y)\bigl|\varphi(y) - \varphi(x_0)\bigl|dy = $$
$$= \underbrace{\int\limits_{B_\delta(x_0)}\Phi(t,x-y)\bigl|\varphi(y) - \varphi(x_0)\bigl|dy}_{I_\delta} + \underbrace{\int\limits_{\real^n \setminus B_\delta(x_0)}\Phi(t,x-y)\bigl|\varphi(y) - \varphi(x_0)\bigl|dy}_{J_\delta}.$$

Пусть $\eps > 0.$ Выберем такое $\delta,$ что $\bigl|\varphi(y) - \varphi(x_0)\bigl|<\dfrac{\eps}{2}$ при $y \in B_\delta(x_0)$ (Такое $\delta \, \exists,$ т.~к. $\varphi$ --- непрерывна).

Тогда $I_\delta \leq \dfrac{\eps}{2}, \, \text{т.к.} \int\limits_{\real^n}\Phi = 1.$

Пусть $|x - x_0| < \dfrac{\delta}{2}, \, y \notin B_\delta(x_0) \Rightarrow |x - y| \geq \dfrac{1}{2}|x_0 - y|.$
$$J_\delta \leq 2||\varphi||_\infty \cdot \int\limits_{\real^n \setminus B_\delta(x_0)} \Phi(t,x-y)dy = 2||\varphi||_\infty \dfrac{1}{(4\pi t)^{n / 2}} \int\limits_{B_{\delta}^c} e^{-\dfrac{|x-y|^2}{4t}}dy \leq $$$$\leq 2||\varphi||_\infty \dfrac{1}{(4\pi t)^{n / 2}} \int\limits_{B_{\delta}^c} e^{-\dfrac{|x_0-y|^2}{16t}}dy \leq \dfrac{C}{(4\pi t)^{n / 2}} \int\limits_\delta^{+\infty} e^{-\dfrac{\rho^2}{16t}}\rho^{n-1}d\rho =  $$$$ = C\int\limits_{\delta / \sqrt{t}}^{+\infty} e^{-\dfrac{\rho^2}{16t}}\rho^{n-1}d\rho \stackrel{t\rightarrow 0}{\longrightarrow} 0$$
$$\delta: \, t < \dfrac{\delta}{2}, \, J_\delta \leq \dfrac{\eps}{2}, \Rightarrow I = I_\delta + J_\delta \leq \eps, \, |x-x_0| < \dfrac{\delta}{2}, \, 0<t<\dfrac{\delta}{2} \Rightarrow $$$$ |u(t,x) - \varphi(x_0)|<\eps$$
\end{enumerate}
\end{proof}

\section{Решение задачи Коши для неоднородного уравнения теплопроводности в пространстве. Метод Дюамеля}
\begin{align}
    \begin{cases} 
        u_t - \bigtriangleup u = f, \\
        u (0, x) = \varphi (x).
    \end{cases}
\label{transcalencynonhom2}
\end{align}
$t \in \real^+,\, x \in \real^n, f=f(t,x)$

\begin{note}

Достаточно решать для случая $\varphi = 0, \, f \not= 0 \, \text{и} \, \varphi \not=, \, f = 0,$ т. е. пусть $u_1, \, u_2 :$
\begin{equation*}
\begin{cases} 
        Lu_1 = f, \\
        u_1\Bigl|_{t=0} = 0.
    \end{cases}
    \qquad
    \begin{cases} 
        Lu_2 = 0, \\
        u_2\Bigl|_{t=0} = \varphi (x).
    \end{cases}
\end{equation*}
Тогда $u = u_1+u_2:$
\begin{align}
\begin{cases} 
        Lu = f, \\
        u\Bigl|_{t=0} = \varphi (x).
    \end{cases}
    \end{align}
\end{note}

{\itshape Принцип Дюамеля.} Пусть $\varphi = 0$
\begin{equation*}
v(t,x,s) : \quad
    \begin{cases} 
        v_t - \bigtriangleup v = 0, \\
        v\Bigl|_{t=s} = f(s, \,).
    \end{cases}
\end{equation*}
$$u(t,x) = \int\limits_0^tv(t,x,s)ds = \int\limits_0^t ds \int\limits_{\real^n}\Phi(t-s, x-y)f(s,y)dy.$$

\begin{theorem}
Пусть $ u(t,x) = \int\limits_0^t ds \int\limits_{\real^n}\Phi(t-s, x-y)f(s,y)dy,$ тогда при $f\in C_x^2, \, C_t^1, f $ --- финитная $\in C_0^\infty( (0,+\infty) \times \real^n),$
\begin{enumerate}
\item $u \in C_t^1, \, u \in C_x^2;$
\item $u_t(t,x) - \bigtriangleup u(t,x) = f, \, u(0,x) = 0;$
\item $\lim u(t,x) = 0$ при $(t,x) \rightarrow (0,x_0) \forall x_0 \in \real^n.$
\end{enumerate}
\end{theorem}

\begin{proof}

$$u(t,x) = \int\limits_0^t ds \int\limits_{\real^n} \Phi(s,y)f(t-s, x-y)dy \, \text{--- свертка},$$ т.к. $f\in C_x^2, \, C_t^1, f $ --- финитная и $\Phi(s,y)$ гладкая в окрестности $s = t >0:$
$$\dfrac{\partial^2u}{\partial x_i^2} = \int\limits_0^t ds \int\limits_{\real^n} \Phi(s,y)f_{x_i x_i}(t-s, x-y)dy;$$
$$u_t = \int\limits_{\real^n} \Phi(t,y)f(0,x-y)dy + \int\limits_0^t ds\int\limits_{\real^n} \Phi(s,y)f_t(t-s, x-y)dy,$$ т.е. $u \in C_t^1, \, u \in C_x^2.$

$$u_t - \bigtriangleup u = \int\limits_{\real^n} \Phi(t,y)f(0,x-y)dy + \int\limits_0^t ds\int\limits_{\real^n} \Phi(s,y)f_t(t-s, x-y)dy-$$$$ - \int\limits_0^t ds \int\limits_{\real^n}dy\underbrace{\sum \Phi(s,y)f_{x_i x_i}(t-s, x-y)}_{\Phi(s,y)\bigtriangleup f(t-s, x-y)} = $$
$$= \int\limits_{\real^n} \Phi(t,y)f(0,x-y)dy + \int\limits_0^t ds \int\limits_{\real^n}dy \Phi(s,y)(-\dfrac{\partial}{\partial s} - \bigtriangleup y) = $$ $$= \underbrace{ \int\limits_0^\eps ds \int\limits_{\real^n}\Phi(s,y)\left(\left(-\dfrac{\partial}{\partial s} - \bigtriangleup y\right)f(t-s, x-y)\right)dy}_{I_\eps} + \underbrace{\int\limits_{\real^n} \Phi(t,y)f(0,x-y)dy}_{k} + $$
$$+ \underbrace{\int\limits_\eps^t ds \int\limits_{\real^n} \Phi(s,y)\left(\left(-\dfrac{\partial}{\partial s} - \bigtriangleup y\right)f(t-s, x-y)\right)dy}_{J_\eps}.$$
$$\abs{I_\eps} \leq \left(||f_t||_\infty + ||\bigtriangleup_x f||_\infty\right)\int\limits_0^\eps ds \int\limits_{\real^n}\Phi(s,y)dy = C\eps.$$
$$J_\eps = \int\limits_\eps^t ds \underbrace{\int\limits_{\real^n}\left(\left(\dfrac{\partial}{\partial s} - \bigtriangleup y\right)\Phi(s,y)\right)f(t-s, x-y)dy}_{0, \, \text{т.к.} \, \Phi \, \text{--- решение однородного уравнения}} + \int\limits_{\real^n} \Phi(\eps, y)f(t-\eps, x-y)dy - $$ $$-\underbrace{\int\limits_{\real^n} \Phi(t,y)f(0,x-y)dy}_{\text{это} \, k}.$$
При $\eps \rightarrow 0:$
$$u_t - \bigtriangleup u = \lim_{\eps \rightarrow 0} \int\limits_{\real^n}\Phi(\eps, y)f(t-\eps, x-y)dy = \lim_{\eps \rightarrow 0} \int\limits_{\real^n}\Phi(\eps, x-y)f(t-\eps, y)dy = f(t,x) \Rightarrow$$ $\Rightarrow u$ --- решение.

$$\abs{u(t,x)} \leq \int\limits_0^t ds \int\limits_{\real^n}\Phi(s,y)\underbrace{\abs{f(t-s, x-y)}}_{\leq C, \, \text{т.к. финитная}}dy \leq C\int\limits_0^t ds \int\limits_{\real^n}\Phi(s,y)dy = Ct \stackrel{t \rightarrow 0} \longrightarrow 0. $$
\end{proof}





% !TEX encoding = UTF-8 Unicode
% лекции 9-10, 12 марта 2016
% вопросы 18-20 
% 18. Пространство основных функций и пространство обобщенных функций (распределений). Примеры обобщенных функций. Дифференцирование обобщенных функций. Обобщенные и классические производные.
% 19. Фундаментальное решение уравнения Лапласа, его физический смысл.
% 20. Представление (частного) решения уравнения Пуассона в пространстве при помощи фундаментального решения.

\subsection{Основы теории обобщенных функций}
Как и раньше, $\Omega \subset \real^n$ --- область.
\begin{definition}
Множество $ \beauD(\Omega) = C_0^\infty(\Omega)$ называется пространством основных функций (пробных функций).
\end{definition}
\begin{definition}
Последовательность функций $\{u_k\} \subset C_0^\infty(\Omega)$ сходится к основной функции $u \in C_0^\infty(\Omega)$, если 
\begin{enumerate}
\item все функции последовательности имеют носитель в одном компакте: $$\exists K \ssubset \Omega: \quad \supp u_k \subset K,$$
\item все функции последовательности и все их производные сходятся к $u$ и её производным равномерно $$u_k \rightrightarrows u, \quad D^\alpha u_k \rightrightarrows D^\alpha u \quad \forall \alpha,$$ где $\alpha$ --- мультииндекс: $$\alpha = (\alpha_1, \ldots, \alpha_n), \, \alpha_i \in \mathds{N},$$ $$\abs{\alpha} = \sum\limits_{i=1}^{n}\alpha_i,$$
$$D^\alpha u = \dfrac{\partial^{\abs{\alpha}}u}{\partial x_1^{\alpha_1}\ldots\partial x_n^{\alpha_n}}.$$
\end{enumerate}
\end{definition}

\begin{example}Пусть $u_0 \in C_0^\infty(\real)$ и $u_k = \dfrac{1}{k}u_0$, тогда $u_k \conv*{\beauD(\real)}{} 0$.
\end{example}

\begin{example}Пусть $u_0 \in C_0^\infty(\real)$ и $u_k(x) = \dfrac{1}{k}u_0(x - k)$. Тогда
\begin{gather*}
	u_k \to 0, \\
	u_k \rightrightarrows 0, \\
	D^\alpha u_k \rightrightarrows 0 \quad \forall \alpha
\end{gather*}
но $u_k$ не живут на одном компакте, так что $$ u_k \centernot {\conv*{\beauD(\real)}{}} 0.$$

Таким образом, сходимость основных функций сильнее, чем равномерная или поточечная.
\end{example}

\begin{note} Заданная нами на $\beauD$ топология не порождается никакой метрикой. 
\end{note}

\begin{definition}
Пространством обобщенных функций $\beauD'(\Omega)$ называется пространство линейных непрерывных функционалов на пространстве $\beauD(\Omega)$. Через $\action{u,f}$ обозначается действие обобщенной функции $f$ на основную функцию $u$. Обобщенные функции также называют распределениями.
\end{definition}

\begin{note} Таким образом, обобщённая функция это не функция, а функционал.
\end{note}

\begin{note} Пусть $\left\{ u_k \right\} \subset \beauD(\Omega)$. Непрерывность обобщённой функции $f$ означает, что
$$ u_k \conv* {\beauD(\Omega)} {} u \quad \Rightarrow \quad \action{u_k, f} \longrightarrow \action{u, f}.$$
\end{note}

\begin{example} Любая функция $v \in L_{loc}^1 (\Omega)$ канонически порождает обобщённую функцию $\widehat{v}$:
$$ \action{u, \widehat{v}} = \int \limits_\Omega u(x) v(x) \, dx.$$
Интеграл имеет смысл, так как на самом деле мы интегрируем по $\supp u$. Линейность $\widehat{v}$ очевидна, покажем непрерыность. Пусть $\left\{ u_k \right\} \subset \beauD(\Omega)$ и $ u_k \conv*{\beauD(\Omega)}{} u$, тогда
$$ \action{u_k, \widehat{v}} = \int \limits_\Omega u_k v = \int \limits_K u_k v \longrightarrow \int \limits_K uv = \int \limits_\Omega uv = \action{u, \widehat{v}}.$$
Таким образом, можно отождествлять $v = \widehat{v}$ и говорить, что $L_{loc}^1(\Omega) \subset \beauD' (\Omega)$, хоть эти пространства и имеют разную природу.
\end{example}

\begin{example}Конечная положительная борелевская мера $\mu$ порождает обобщённую функцию $\widehat{\mu}$:
$$\action{u, \widehat{\mu}} = \int \limits_\Omega u \, d\mu.$$
Линейность очевидна, непрерывность вытекает из непрерывности интеграла. Таким образом, конечные положительные борелевские меры --- тоже обобщённые функции.
\end{example}

\begin{example}Рассмотрим так называемую дельта-фунцию Дирака $\delta = \delta_0 = \delta(x)$:
$$ \action{u, \delta} = u(0).$$
Также для $x \in \Omega$ можно определить $\delta_{x_0} = \delta (x - x_0)$:
$$\action{u, \delta_{x_0}} = u(x_0).$$
Линейность и непрерывность очевидны. Значит, дельта-функция является обобщённой функцией.
\end{example}

\begin{note}Дельта-функция не порождается никакой функцией из $L_{loc}^1(\Omega)$.
\end{note}
\begin{proof}
Пусть $f \in L_{loc}^1(\Omega)$ такая, что
$$ \action{u, \delta} = \action{u, f} = \int \limits_\Omega fu  = u(0) \quad \forall u \in C_0^{\infty}(\Omega)$$
Рассмотрим такие $u$, у которых $0$ не лежит в носителе. Для них верно
$$\action{u, f} = \int \limits_\Omega fu = 0.$$
По основной лемме вариационного исчисления $f = 0$ почти всюду на $\Omega \setminus \{0\}$. Множество $\{0\}$ имеет меру нуль, значит, $f = 0$ почти всюду на $\Omega$. Подставив в интеграл, получим
$$ u(0) = 0 \quad \forall u \in C_0^{\infty},$$
что невозможно.

\end{proof}

На обобщённых функциях можно определить операции:

\begin{enumerate}
\item Сложение, умножение на скаляр. Пусть $v_1, \, v_2 \in \beauD'(\Omega)$, $\lambda_1,\,\lambda_2 \in \real$, тогда
$$\action{u, \lambda_1v_1 + \lambda_2v_2} = \lambda_1\action{u, v_1} + \lambda_2\action{u, v_2}.$$
\item Умножение на функцию из $C^\infty$. Пусть $\varphi \in C^\infty(\Omega)$, $v\in \beauD'(\Omega)$, $u\in \beauD(\Omega)$, тогда
$$\action{u, \varphi v} = \action{\varphi u, v}.$$
Определение корректно, так как $\varphi u \in C_0^\infty$.
\item Дифференцирование. Пусть $\alpha$ --- мультииндекс, $v\in\beauD'(\Omega)$, $u \in \beauD(\Omega)$. Тогда
$$\action{u, D^\alpha v} = (-1)^{\abs{\alpha}}\action{D^\alpha u, v}.$$
Проверим, что $D^\alpha v\in \beauD'(\Omega)$. Линейность очевидна. Пусть $u_k \conv* {\beauD(\Omega)}{} u$, тогда
$$\action{u_k, D^\alpha v} = (-1)^{\abs{\alpha}} \action{D^\alpha u_k, v} \longrightarrow (-1)^{\abs{\alpha}} \action{D^\alpha u, v} = \action{u, D^\alpha v},$$
по свойствам сходящихся в $\beauD(\Omega)$ последовательностей.
\end{enumerate}

\begin{example}
Пусть $v \in C^1(\Omega) \subset L_{loc}^1(\Omega)$. Тогда $v$ --- обобщенная с точностью до отождествления. Рассмотрим обобщённую функцию $v_{x_i}$:
$$\action{u, v_{x_i}} = - \action{u_{x_i}, v }= - \int \limits_\Omega u_{x_i} v \, dx = \int \limits_\Omega u v_{x_i} \, dx - \underbrace{\int \limits_{\partial \Omega'} u v \nu_i \, d\sigma}_{=0} = \action{u, v_{x_i}},$$
где $\partial \Omega'$ --- гладкая граница некоего множества, лежащего в $\Omega$ и окружающего носитель функции $u$. Здесь мы воспользовались формулой интегрирования по частям для многомерного случая.

\begin{reminder}[Интегрирование по частям (многомерный случай)]
$$\int \limits_{\Omega} u v_{x_i} \,dx = \int \limits_{\partial\Omega} uvn_{i} \,d\sigma - \int \limits_{\Omega} u_{x_i} v \, dx,$$
где $n_{i}$ --- $i$-ая компонента вектора внешней нормали к $\partial\Omega$.
\end{reminder}

Таким образом, если $v$ непрерывно дифференцируема, то обобщённая производная совпадает с классической.
\end{example}

\begin{exercise} Пусть $\Omega = \real$. Рассмотрим обобщённую производную дельта-функции:
$$\action{u, \delta'} = -\action{u, \delta} = -u'(0).$$
Доказать, что ей не соответствуют ни одна функция и ни одна мера.
\end{exercise}

\begin{example}Пусть $\Omega =\real.$ Рассмотрим функцию Хевисайда:
\begin{equation*}
h(x) = \quad
    \begin{cases} 
        1, \, x\geq0\\
        0,\, x < 0.
    \end{cases}
\end{equation*}
Найдём её обобщённую производную:

$$\action{u, h'}= -\action{u',h} = -\int\limits_{-\infty}^{+\infty}u'h = -\int\limits_0^{+\infty}u' = \underbrace{-u\Bigl|_0^{+\infty}}_{\text{финитная}} = 0+u(0)=\action{u,\delta}.$$
Таким образом, $h' = \delta$ в смысле обобщённых функций. В классическом смысле в нуле производной не существует.
\end{example}

\begin{exercise}
Канторова лестница п.в. имеет нулевую производную в классическом смысле. Доказать, что её обобщенная производная $\not\equiv0$  и является мерой, сосредоточенной на канторовом множестве.
\end{exercise}

\subsection{Фундаментальное решение уравнения Лапласа}
Будем рассматривать уравнения Лапласа \eqref{Laplace} и Пуассона \eqref{Poisson} в смысле обобщенных функций.

Пусть $\Omega = \real^n$, $u$ - обобщённая функция. В случае электростатической системы из одного заряда имеем уравнение Пуассона:
\begin{equation}
	-\Delta u = \delta.
\label{comLaplace}
\end{equation}

Нам понадобятся формулы Грина (следуют из формулы интегрирования по частям):
\begin{enumerate}
\item $\displaystyle \int \limits_\Omega v \Delta u = \sum \limits_{i = 1}^{n} \int \limits_\Omega v u_{x_i x_i} = \sum \limits_{i = 1}^{n} \left( - \int \limits_\Omega v_{x_i} u_{x_i} \, dx + \int \limits_{\partial \Omega} v u_{x_i} \nu_i \, d\sigma \right) = - \int \limits_\Omega \nabla u \cdot \nabla v \, dx + \int \limits_{\partial \Omega} v \dfrac{\partial u}{\partial \nu} \, d\sigma.$
\item $\displaystyle \int \limits_{\Omega} \left( u \Delta v - v \Delta u \right) \, dx = \int \limits_{\partial \Omega} \left( u \dfrac{\partial v}{\partial \nu} - v \dfrac{\partial u}{\partial \nu} \right) \, d\sigma.$
\end{enumerate}
Ещё нам понадобятся формулы объёма шара и площади сферы:
$$ |B_1(0)| = \frac{\sqrt{\pi^n}}{\Gamma(\frac {n} {2} + 1)} = \omega_n, \quad  |B_r(0)| = r^n \omega_n, \quad | \partial B_r(0) | = n \omega_n r^{n-1}$$

Будем искать решение \eqref{comLaplace} в смысле обобщенных функций в виде:
$$u(x) = \dfrac{c_n}{\abs{x}^{n-2}}.$$
Пусть $\varphi \in C_0^\infty\left(\real^n\right)$. Посчитаем обобщённый лапласиан от $u$:
\begin{align*}
\action{\varphi, -\Delta u} &= - \action{\Delta \varphi, u} = - \int \limits_{\real^n} \Delta \varphi u \, dx = -c_n \int \limits_{\real^n} \Delta \varphi \dfrac{1}{|x|^{n-2}} \, dx \stackrel{\text{(а)}} {=} - c_n \lim_{\eps \to 0} \int \limits_{B^c_\eps (0)} \Delta \varphi \frac {1} {|x|^{n-2}} \, dx\\
&\stackrel{\text{(б)}} {=} -c_n \lim_{\eps \to 0} \left( \int \limits_{B^c_{\eps}(0)} \varphi \Delta \dfrac{1}{|x|^{n-2}} \, dx + \int \limits_{\partial B_\eps(0)} \left( \dfrac{1}{|x|^{n-2}} \dfrac{\partial \varphi}{\partial\nu} - \varphi \dfrac{\partial}{\partial\nu} \left( \dfrac{1}{|x|^{n-2}} \right) \right) \, d\sigma \right) \\
&\stackrel{\text{(в)}} {=} -c_n \lim_{\eps \to 0} \int \limits_{\partial B_\eps(0)} \left( \dfrac{1}{|x|^{n-2}} \pder[\varphi]{\nu} - \varphi \pder{\nu} \left( \dfrac{1}{|x|^{n-2}} \right) \right) \, d\sigma \\
&= -c_n \lim_{\eps \to 0} \dfrac{1}{\eps^{n-2}} \int \limits_{\partial B_\eps(0)} \pder[\varphi]{\nu} \, d\sigma - \int \limits_{\partial B_\eps(0)} \varphi \pder{\nu} \left( \dfrac{1}{|x|^{n-2}} \right) \, d\sigma \\
&\stackrel{\text{(г)}} {=} c_n \lim_{\eps \to 0} \int \limits_{\partial B_\eps(0)} \varphi \pder{\nu} \left( \dfrac{1}{|x|^{n-2}} \right) \, d\sigma \stackrel{\text{(д)}} {=} c_n\lim_{\eps \to 0} \int \limits_{\partial B_\eps(0)} - \varphi \pder{r} \left( \dfrac{1}{r^{n-2}} \right) \, d\sigma \\
&= c_n \lim_{\eps \to 0} \int \limits_{\partial B_\eps(0)} \varphi \frac{n-2} {|x|^{n-1}} \, d\sigma =  c_n(n-2) \lim_{\eps \to 0} \dfrac{1}{\eps^{n-1}} \int \limits_{\partial B_\eps(0)} \varphi \, d\sigma \\
&\stackrel{\text{(е)}} = c_n n(n-2) \omega_n \lim_{\eps \to 0} \dfrac{1}{| \partial B_\eps(0) |} \int \limits_{\partial B_\eps(0)}\varphi \, d\sigma = c_n n(n-2) \omega_n \lim_{\eps \to 0} \fint \limits_{\partial B_\eps(0)} \varphi \, d\sigma \\
&= c_n n(n-2) \omega_n \varphi(0) = c_n n(n-2) \omega_n \action{\varphi, \delta}.
\end{align*}
Поясним вычисления:
\begin{description}
\item [(а)] исключили особенность в нуле
\item [(б)] воспользовались 2 формулой Грина
\item [(в)] воспользовались тем, что $\Delta\dfrac{1}{\abs{x}^{n-2}} = 0$, остаётся как упражнение
\item [(г)] воспользовались тем, что $$ \Bigg\lvert \int\limits_{\partial B_\eps(0)}\dfrac{\partial\varphi}{\partial\nu} \, d\sigma \Bigg\rvert \leq C\abs{\partial B_\eps(0)} =C\eps^{n-1}, \quad 
 \dfrac{1}{\eps^{n-2}}C\eps^{n-1} = C\eps \conv* {\eps \to 0} {} 0 $$
\item [(д)] воспользовались тем, что $|x| = r$ и вектор нормали к шару это радиальное направление
\item [(е)] домножили и поделили на площадь сферы радиусом $\eps$: $$|\partial B_\eps(0)| = n \omega_n \eps^{n-1}$$
\end{description}

Итого, при $c_n = \dfrac{1}{n(n-2) \omega_n}$ имеем
$$\action{\varphi, -\Delta u} = \action{\varphi, \delta} = \varphi(0) \quad \Rightarrow \quad -\Delta u = \delta.$$
Значит, $$u(x) = \dfrac{1}{n(n-2) \omega_n}\dfrac{1}{\abs{x}^{n-2}}$$ --- обобщённое решение уравнения Лапласа \eqref{comLaplace} в $\real^n$  при $n>2$.

\begin{note}При $n=3$ получим знакомую из школы формулу для потенциала единичного заряда в $\real^3$:
$$c_3 = \dfrac{1}{4\pi}, \quad u(x) = \dfrac{1}{4\pi}\dfrac{1}{\abs{x}}.$$
\end{note}

\begin{exercise}
Проверить, что $$u(x)=\dfrac{1}{2\pi}\ln\dfrac{1}{\abs{x}}$$ удовлетворяет уравнению Лапласа \eqref{comLaplace} в $\real^2$ в обобщенном смысле.
\end{exercise}

\begin{definition}
Фундаментальным решением уравнения Лапласа называется функция
\begin{equation}
\Phi(x) = 
\begin{cases}
-\dfrac{1}{2\pi}\ln\abs{x},\quad n=2,\\
\dfrac{1}{n(n-2)w_n}\dfrac{1}{\abs{x}^{n-2}},\quad n>2.
\end{cases}
\end{equation}
Для неё верно, что
\begin{align*}
& - \Delta\Phi = \delta \text{ в } \beauD'(\real^n), \\
& - \Delta\Phi = 0 \text{ в классическом смысле в } \real^n\setminus\{0\}.
\end{align*}

\end{definition}

\subsection{Представление частного решения уравнения Пуассона в пространстве при помощи фундаментального решения}

\begin{note} [Вывод 3 формулы Грина] Пусть $\Omega\subset\real^n$ --- ограниченная область с гладкой границей, и $0 \in \Omega$. Тогда по 2 формуле Грина:
\begin{align*}
\int \limits_{B^c_\eps(0)} \left( \underbrace{u\Delta\Phi}_{0} - \Phi\Delta u \right) \, dx &= \int \limits_{B^c_\eps(0)} - \Phi \Delta u \, dx = \int \limits_{\partial B^c_\eps(0)} \left( u \dfrac{\partial\Phi}{\partial\nu} - \Phi\dfrac{\partial u}{\partial\nu} \right) \,d\sigma \\
&= \int \limits_{\partial \Omega} \left( u \dfrac{\partial\Phi}{\partial\nu} - \Phi\dfrac{\partial u}{\partial\nu} \right) \,d\sigma + \int\limits_{\partial B_\eps(0)} \left( u \dfrac{\partial\Phi}{\partial\nu} - \Phi\dfrac{\partial u}{\partial\nu} \right) \,d\sigma
\end{align*}
Устремим $\eps$ к нулю:
$$ \int \limits_\Omega - \Phi \Delta u \, dx = \int \limits_{\partial \Omega} \left( u \dfrac{\partial\Phi}{\partial\nu} - \Phi\dfrac{\partial u}{\partial\nu} \right) \,d\sigma + u(0).$$
Перенесём всё в произвольную точку $x_0\in\Omega$:
$$ \int \limits_\Omega - \Phi (x- x_0) \Delta u \, dx = \int \limits_{\partial \Omega} \left( u \dfrac{\partial\Phi}{\partial\nu} (x-x_0) - \Phi (x - x_0) \dfrac{\partial u}{\partial\nu} \right) \,d\sigma + u(x_0).$$
Значит,
$$ u(x_0) = \int \limits_{\partial \Omega} \left( \Phi (x - x_0) \dfrac{\partial u}{\partial\nu} - u \dfrac{\partial\Phi}{\partial\nu} (x-x_0) \right) \,d\sigma - \int \limits_\Omega \Phi (x- x_0) \Delta u \, dx.$$
Получили 3 формулу Грина.
\end{note}

На данный момент мы знаем, что решением уравнения
$$-\Delta u =\delta_{x_0}$$ является функция $$ u=\Phi(x-x_0).$$ Это решение не является единственным: к нему можно добавить любую гармоническую функцию $v$, результат тоже будет решением.

Пусть есть уравнение
$$ - \Delta u = \alpha_1 \delta_{x_1} + \alpha_2 \delta_{x_2},$$
тогда его решением будет
$$ u = \alpha_1 \Phi (x - x_1) + \alpha_2 \Phi (x - x_2).$$
В случае конечной суммы имеем
$$ - \Delta u = \sum_{k = 1}^n \alpha_k \delta_{x_k}, \quad u = \sum_{k=1}^n \alpha_k \Phi (x - x_k).$$
Для произвольной функции интуитивно можно написать
$$-\Delta u = f, \quad u = \int \limits_{\real^n} \Phi(x-y) f(y) \, dy = \Phi*f.$$
Докажем, что при некоторых дополнительных условиях это действительно так.
\begin{theorem}
Пусть $f\in C_0^2(\real^n)$. Тогда
$$ u(x) = \int\limits_{\real^n} \Phi(x-y)f(y)dy$$ будет классическим решением уравнения Пуассона в $\real^n$. 
\begin{note} Классическое решение всегда является обобщенным, так как классические производные равны обобщенным.
\end{note}
\end{theorem}
\begin{proof} Поменяем переменные:
$$u(x) = \int\limits_{\real^n}\Phi(y)f(x-y)dy$$
и продифференцируем $u$ по $x$. Производная проносится под знак интеграла:
$$u_{x_ix_i}=\int\limits_{\real^n}\Phi(y)f_{x_ix_i}(x-y)dy.$$
Тем самым доказана гладкость.

Если 2 классические функции равны в обобщенном смысле, то они просто равны. Докажем, что $u$ --- обобщённое решение уравнения Пуассона.

Пусть $\varphi\in C_0^\infty(\real^n)$, тогда
\begin{align*}
\action{\varphi, -\Delta u} &= \int \limits_{\real^n} \varphi(x)(-\Delta u(x)) \, dx = \int \limits_{\real^n}\varphi(x) \, dx \int \limits_{\real^n} \Phi(y)(-\Delta f(x-y)) \, dy \\
&= \int \limits_{\real^n} \Phi(y) \, dy \int \limits_{\real^n} (-\Delta\varphi(x)) f(\underbrace{x-y}_z) \, dx = \int \limits_{\real^n} \Phi(y) \, dy \int \limits_{\real^n} (-\Delta\varphi(z+y)) f(z) \, dz \\
&= \int \limits_{\real^n} f(z) \, dz \int \limits_{\real^n} \Phi(y) (-\Delta\varphi(z+y)) \, dy = \int \limits_{\real^n} f(z) \action{-\Delta\varphi(z+\cdot), \Phi} \, dz \\
&= \int \limits_{\real^n} f(z) \underbrace{\action{\varphi(z+\cdot), -\Delta\Phi}}_{\varphi(z)} \, dz = \int\limits_{\real^n} f(z) \varphi(z) \, dz \\
& = \action{\varphi, f}.
\end{align*}
Здесь на третьем шаге мы поменяли порядок интегрирования, а потом интегрированием по частям перенесли производные.

Таким образом, $u$ - классическое решение уравнения $$-\Delta u = f.$$
\end{proof}
% !TEX encoding = UTF-8 Unicode
% лекции 11-12, 12 апреля 2016
% конец первой части
% вопросы 21-26
% 21. Теоремы о среднем для гармонических функций
% 22. Сильный принцип максимума для гармонических функций, его следствия. Внутренняя задача Дирихле для уравнения Пуассона. Единственность классического решения
% 23. Обратная теорема о среднем
% 24. Бесконечная дифференцируемость гармонических функций
% 25. Оценки производных гармонических функций. Функции, гармонические во всем пространстве. Теорема Лиувилля.
% 26. Функция Грина внутренней задачи Дирихле для уравнения Пуассона. Функция Грина для шара

\section{Гармонические функции}
\subsection{Свойства гармонических функций}
\begin{definition}
$u$ --- гармоническая функция в $\Omega$, если является решением уравнения Лапласа $-\Delta u = 0.$
\end{definition}
\begin{note}
$u\in C^2(\Omega)\cap C(\overline{\Omega}), \Omega$ --- ограничена, тогда 
$$0 = \int\limits_\Omega \Delta u dx = \int\limits_\Omega 1\cdot \Delta udx = -\int\limits_\Omega\underbrace{\nabla 1}_{0}\nabla udx+\int\limits_{\partial\Omega}1\cdot\dfrac{\partial u}{\partial \nu}d\sigma\,\Rightarrow\, \int\limits_{\partial\Omega}\dfrac{\partial u}{\partial \nu}d\sigma = 0.$$
\end{note}

\begin{theorem}[Теорема о среднем для гармонических функций]
Пусть $u\in C^2(\Omega), -\Delta u = 0,$ тогда $$u(x_0)=\fint\limits_{\partial B_\eps(x_0)}u(x)d\sigma(x)=\fint\limits_{B_\eps(x_0)}u(x)dx\quad \forall\eps\,:\,\overline{B}_\eps(x_0)\subset\Omega.$$
\begin{note}
Если $u\in C(\overline{\Omega}), \, \eps_0\,:\,B_\eps(x_0)$ касается границы, то утверждение верно $\forall \eps\leq\eps_0.$ 
\end{note}
\end{theorem}
\begin{proof}
$u(x_0)\stackrel{\text{по } \MakeUppercase{\romannumeral3}\text{ ф. Грина}}= $
\end{proof}

\subsection{Функция Грина}
TODO
% !TEX encoding = UTF-8 Unicode
% лекции 13-14, 26 марта 2016, начало второй части
% 1. Обобщенные функции и обобщенные производные.
% 2. Соболевские классы $H_0^1$, $H^1$ - определения, примеры.
% 3. Гильбертовы пространства $H_0^1$, $H^1$, скалярное произведение, полнота.
% 4. Непрерывные вложения пространств $H_0^1 \subset H^1 \subset L^2$. Непрерывность обобщенного дифференцирования. ∗ Сепарабельность пространств $H_0^1$, $H^1$.
% 5. Соболевское пространство $H_0^1(a,b)$ (непрерывность, дифференцируемость, значение на границе, ∗∗ абсолютная непрерывность). ∗ Вложение $H^1(a, b) \subset C([a, b])$ (формулировка).
\chapter{Соболевские пространства и обобщённые решения}
\section*{Наводящие соображения}

\subsection{Прямой метод вариационного исчисления (метод Тонелли)}
Пусть $X$ --- множество, $G$ --- вещественнозначный функционал на $X$. Надо найти точку $x$, в которой функционал достигает минимума.

Можно попробовать найти минимизирующую последовательность $\left\{ x_k \right\}$:
$$G(x_1) \geq G(x_{2}) \geq \hdots, \quad G(x_k) \conv* {} {k \to \infty} \inf_X G$$

Если удастся задать на $X$ метрику (или топологию), в которой $\left\{ x_k \right\}$ --- компакт, а $G$ непрерывен, то существует подпоследовательность  $\left\{ x_{k_n} \right\} \subset \left\{ x_k \right\}$ такая, что
$$x_{k_n} \conv*{\text{по метрике}}{n \to \infty} x \in X, \quad G(x_{k_n}) \conv*{}{n \to \infty} G(x) = \inf_X G.$$
Но на множествах вроде пространств функций добиться компактности $\left\{ x_k \right\}$ в топологии, порождённой стандарной нормой, нелегко. Нужно ослаблять топологию: чем слабее топология, тем больше компактов. В то же время, чем слабее топология, тем меньше непрерывных функционалов. Нужно искать компромисс.

\begin{note}Условие непрерывности функционала можно ослабить, нам достаточно полунепрерывности снизу
$$ y_k \conv {}{k \to \infty} y \quad \Rightarrow \quad G(y) \leq \lim_{k \to \infty} \inf_{X} G(y_k)$$
так как 
$$ G(x) \leq \lim_{n \to \infty} \inf_X G(x_{k_n}) = \inf_X G \quad \Rightarrow G(x) = \inf_X G.$$
\end{note}

Этот метод поиска экстремумов функционала называется прямым методом вариационного исчисления или методом Тонелли.

\subsection{Уравнение Пуассона}

Пусть в ограниченной области $\Omega \subset \real^n$ поставлена задача Дирихле для уравнения Пуассона:
\begin{align*}
	\begin{cases*}
		- \Delta u =f, \\
		u\Big\rvert_{\partial \Omega} = 0
	\end{cases*}
\end{align*} 
Хотим увидеть вариационную задачу:
$$ F(u) = \frac {1} {2} \int \limits_\Omega | \nabla u |^2 \, dx - \int \limits_\Omega fu \, dx \rightarrow \min$$ для
$$ u \in C^2(\overline{\Omega}), \quad u\Big\rvert_{\partial \Omega}=0.$$
Предполагаем, что минимум достигается. Тогда для любого $\varphi \in C_0^\infty (\Omega)$ верно
$$F(u) \leq F(u + \eps \varphi) \quad \forall \eps \in \real$$
или, что то же самое, первая вариация по направлению $\varphi$ равна нулю:
$$ \frac {d} {d \eps} F(u + \eps \varphi)\Big\rvert_{\eps=0} = 0 \quad \Leftrightarrow \quad F'(u) \varphi = 0.$$
Посчитав, получим, что $$ \int \limits_\Omega (- \Delta u - f) \varphi \, dx = 0.$$
По основной лемме вариационного исчисления
$$ - \Delta u = f,$$
то есть, минимум функционала $F$ является решением нашей задачи. То есть, мы свели задачу доказательства существования решения уравнения к задаче доказательства существования точки минимума функционала.

Попробуем применить метод Тонелли. Пусть существует $\left\{ u_k \right\} \in C^2(\overline{\Omega})$ такая, что
$$ u_k\Big\rvert_{\partial \Omega} = 0, \quad F(u_k) \conv* {} {k \to \infty} \inf_{C^2(\overline{\Omega})} F.$$
Нужно придумать топологию или метрику, в которой $u_{k_n} \conv*{}{n \to \infty} u \in C^2(\overline{\Omega})$, так как со стандартной нормой пространство $C^2(\overline{\Omega})$ неполно.

Хочется найти интегральную сходимость вроде сходимости $L^p$, в которой к чему-то сходятся функции и их производные. Как задавать такую сходимость в этом пространстве --- непонятно. Сменим пространство.

% 2. Соболевские классы $H_0^1$, $H^1$ - определения, примеры.
\section{Соболевcкие классы}

\begin{definition} Соболевские классы $W^{k,p}$ это такие множества:
$$ W^{k,p} := \left\{ u \in L^p : \quad D^\alpha u \in L^p \quad \forall |\alpha| \leq k \right\}.$$ 

\begin{definition} Пусть $\Omega \subset \real^n$ --- область. Тогда $\m{H1Om}$ это множество таких квадратично интегрируемых функций, у которых обобщённые производные тоже квадратично интегрируемы:
	$$\m{H1Om}:= \left\{ u \in \m{L2Om}: \pder[u]{x_i} \in \m{L2Om} \right\} = W^{1,2}.$$
\end{definition}

\begin{definition} Пусть $\Omega \subset \real^n$ --- область. Тогда $\m{H01Om}$ это множество таких квадратично интегрируемых функций, которые аппроксимируются пробными функциями, у которых производные тоже сходятся:
	$$\m{H01Om} := \left\{ u \in L^2(\Omega): \quad \exists \left\{ u_k \right\} \subset C_0^\infty (\Omega): \, u_k \conv{L2Om} {} u \text{ и } \pder[u_k]{x_i} \conv{L2Om}{} v_i \right\} = W_0^{1,2}.$$
\end{definition}

\begin{note} Производные из определения $H_0^1 (\Omega)$ сходятся к обобщённой производной $u$:
\begin{align*}
	\action{\varphi, \pder[u]{x_i}} &= - \action{\pder[\varphi]{x_i}, u} = - \int \limits_\Omega \pder[\varphi]{x_i} u \, dx = - \lim_{k \to \infty} \int \limits_\Omega \pder[\varphi]{x_i} u_k \, dx \\
	&= \lim_{k \to \infty} \int \limits_\Omega \varphi \pder[u_k]{x_i} \, dx - \varphi u_k \Big\rvert_{\partial \Omega} = \int \limits_\Omega \varphi v_i \, dx = \action{\varphi, v_i}.
\end{align*}
\end{note}

% 3. Гильбертовы пространства $H_0^1$, $H^1$, скалярное произведение, полнота.

\begin{note} Очевидно, $\m{H01Om}$ и $\m{H1Om}$ --- линейные пространства, и
$$ \m{H01Om} \subset \m{H1Om}.$$
\end{note}

\subsection{Гильбертовость пространств $\m{H01}$ и $\m{H1}$}

\begin{definition}Пусть $u \in \m{H1Om}$. Тогда
$$ \norm{H1Om}{u} := \left( \norm{L2Om}{u}^2 + \sum_{i=1}^n \norm{L2Om}{u_{x_i}}^2 \right)^{1/2} = \left( \norm*{2}{u}^2 + \int \limits_\Omega | \nabla u|^2 \, dx \right)^{1/2}.$$
\end{definition}
\begin{note} На самом деле это полунорма. Чтобы получить норму, аналогично определению нормы в $L^p (\Omega)$ профакторизуем наше пространство. То есть, объявим эквивалентными функции, совпадающие на множестве полной лебеговой меры и будем рассматривать классы эквивалентности.
\end{note}

\begin{note} Норма в $\m{H01Om}$ порождена скалярным произведением: 
$$\scalprod{H1Om}{u}{v} := \int \limits_\Omega uv \, dx + \int \limits_\Omega \nabla u \cdot \nabla v \, dx.$$
\end{note}

\begin{note} Пусть $u \in \m{H01Om}$. Тогда по определению $\m{H01Om}$
$$ \exists u_k \in \m{C0iOm} : \quad \norm{H1Om}{u_k - u} \conv*{}{k \to \infty} 0,$$
то есть,
$$ \m{H01Om} = \overline{\m{C0iOm}} \quad \text{по норме в } \m{H1Om}.$$
\end{note}

\end{definition}
\begin{theorem} Пространства $\m{H1Om}$ и $\m{H01Om}$ --- полные.
\end{theorem}
\begin{proof} Надо доказать, что любая фундаментальная последовательность сходится к элементу этого же пространства. Пусть $\{ u_k \} \subset \m{H1Om}$ --- фундаментальная:
$$ \norm{H1}{u_k - u_m} \conv*{}{k,m \to \infty} 0.$$
Разворачивем определение сходимости в $\m{H01Om}$:
\begin{gather*}
\begin{cases*}
	\norm*{2}{u_k - u_m} \conv*{}{k,m \to \infty} 0, \\
	\norm*{2}{\pder[u_k]{x_i} - \pder[u_m]{x_i}} \conv*{}{k,m \to \infty} 0.
\end{cases*}
\end{gather*}
Пространство $\m{L2Om}$ полно, так что
$$ u_k \conv{L2}{} u, \quad \pder[u_k]{x_i} \conv{L2}{} \pder[u]{x_i}.$$ Значит,
$$ \norm{H1}{u - u_k} \conv{}{k \to \infty} 0 \quad \Rightarrow u_k \conv{H1}{} u.$$

Из замкнутости $\m{H01Om}$ в $\m{H1Om}$ следует, что оно тоже полное. 

\end{proof}

% 4. Непрерывные вложения пространств $H_0^1 \subset H^1 \subset L^2$. Непрерывность обобщенного дифференцирования. ∗ Сепарабельность пространств $H_0^1$, $H^1$.
% TODO: непрерывные вложения (хотя вроде очевидно)
\subsection{Сепарабельность пространств $\m{H01}$ и $\m{H1}$}

\begin{note} Из доказательства очевидно, что оператор обобщённого дифференцирования
$$ \pder{x_i}: \, \m{H1Om} \longrightarrow \m{L2Om}$$
непрерывен.
\end{note}

\begin{reminder} Пусть $X$ и $Y$ --- линейные нормированные пространства, $X$ --- полное, 
$$A: \, X \longrightarrow Y$$
--- линейная изометрия. Тогда $A(X)$ замкнут в $Y$.
\end{reminder}
\begin{proof}
Изометрия непрерывна, а непрерывный оператор сохраняет сходимость. В $X$ любая фундаментальная последовательность сходится к элементу $X$, значит последовательность из образов фундаментальна и сходится к образу, который лежит в $Y$.

\end{proof}

\begin{theorem} Пространства $\m{H01Om}$ и $\m{H1Om}$ --- сепарабельные\footnote{существует всюду плотное не более чем счётное подмножество} гильбертовы \footnote{везде далее полагаем, что понятие "гильбертово пространство" включает в себя сепарабельность} пространства.
\end{theorem}
\begin{proof}
Введём оператор
\begin{gather*}
J(u) := (u, u_{x_1}, \hdots, u_{x_n}), \\ 
J(u):  \m{H1Om} \longrightarrow L^2(\Omega, \real^{n+1}).
\end{gather*}
Посчитаем норму $J(u)$:
\begin{align*}
	\norm*{L^2(\Omega, \real^{n+1})} {J(u)} &= \norm*{2}{(u, u_{x_1}, \hdots, u_{x_n})}^2 = \norm*{L^2(\Omega, \real^{n+1})}{\sqrt{u^2 + u_{x_1}^2 + \hdots + u_{x_n}^2 }}^2 \\
	&= \int \limits_\Omega u^2 + u_{x_1}^2 + \hdots + u_{x_n}^2 \, dx = \norm{H1}{u}^2.
\end{align*}
Таким образом, $J$ --- изометрия:
$$ \norm*{}{J} = 1.$$
Значит, $\m{H1Om}$ замкнуто в $L^2(\Omega, \real^{n+1})$, так как замкнутое подпространство сепарабельного пространства само сепарабельно. По той же причине сепарабельно $\m{H01Om}$.

\end{proof}

% 5. Соболевское пространство $H_0^1(a,b)$ (непрерывность, дифференцируемость, значение на границе, ∗∗ абсолютная непрерывность). ∗ Вложение $H^1(a, b) \subset C([a, b])$ (формулировка).
\subsection{Соболевское пространство $\m{H01}(a,b)$}

\begin{theorem} В каждом классе функций из $\m{H01}(a,b)$ есть непрерывная:
 $$ \m{H01}(a,b) \subset C([a,b]) .$$
\end{theorem}
\begin{proof}
Пусть $u \in C_0^\infty (a,b)$. Тогда $u(a) = u(b) = 0$ и по теореме о среднем
$$ \exists \xi \in (a,b): \quad u(\xi) = \frac{1}{b-a} \int \limits_a^b u(s) \, ds,$$
тогда по формуле Ньютона-Лейбница
$$ u(x) = u(\xi) + \int \limits_{[\xi, x]} u'(\xi) \, ds = \frac {1} {b-a} \int \limits_a^b u(s) \, ds + \int \limits_{[\xi, x]} u'(s) \, ds.$$
Оценим $u(x)$ при помощи неравенства Гёльдера:
\begin{align*}
|u(x)| &\leq \frac {1} {b-a} \int \limits_a^b |u(s)| \, ds + \int \limits_{[\xi, x]} |u'(s)| \, ds \\
&\leq \frac {1} {b-a} \int \limits_a^b |u(s)| \, ds + \int \limits_a^b |u'(s)| \, ds \\
&\leq \frac {1} {b-a} \sqrt{b-a} \norm*{2}{u} + \sqrt{b-a} \norm*{2}{u'} \\
&\leq C \sqrt{b-a} \left( \norm*{2}{u}^2 + \norm*{2}{u'}^2 \right)^{1/2} = C \norm{H1}{u}.
\end{align*}
На последнем шаге мы воспользовались тем, что
$$ (ax + by)^2 \leq (a^2 + b^2) (x^2 + y^2).$$
Таким образом,
$$ \norm*{\infty}{u} \leq C \norm{H1}{u}.$$

Пусть $u \in \m{H01} (0,1)$ аппроксимируется $\{ u_k \} \subset C_0^\infty (0,1)$. Тогда
$$ \norm{H1}{u_k - u_n} \conv*{}{k,n \to \infty} 0.$$
По ранее доказанному верно 
$$ \norm*{\infty}{u_k - u_n} \leq C \norm{H1}{u_k - u_n} \conv*{}{k,n \to \infty} 0.$$
Имеем 
$$ u_k \conv*{C([a,b])}{} v, \quad u_k \conv*{L^2(a,b)}{} u.$$
Значит, $v = u$ почти везде, и $v$ --- непрерывный представитель класса $u$.  

\end{proof}


\begin{corollary} Если $u \in \m{H01} (a,b)$, то существует $v \in C([a,b])$ такая, что
$$ v = u \quad \text{почти везде}, \quad v(a) = v(b) = 0.$$
\end{corollary}

\begin{exercise} Верно ли, что если $v \in C([a,b])$ и $v(a) = v(b) = 0$, то $v \in \m{H01} (a,b)$?
\end{exercise}
\begin{exercise} Может ли у $u \in \m{H01} (a,b)$ быть несколько непрерывных представителей?
\end{exercise}
\begin{exercise} Может ли у $u \in \m{H01} (a,b)$ быть непрерывный представитель, который не обращается в ноль на границе?
\end{exercise}

\begin{theorem}[Абсолютная непрерывность] Пусть $u \in \m{H01} (a,b)$. Тогда почти всюду существует классическая $u'$, почти всюду равная обобщённой $u'$, и верна формула
$$ u(x) = \int \limits_a^x u'(s) \, ds.$$
\end{theorem}
\begin{proof}
По определению существует $\{ u_k \} \subset C_0^\infty (a,b)$ такая, что 
$$ u_k \conv{H1}{} u.$$
По предыдущей теореме,  имеет место сходимость
$$u_k(x) \conv{}{} u(x) \quad \text{почти всюду на } (a,b)$$
По определению,
$$u'_k \conv{L2}{} u'_{\text{обобщ}} \quad \Rightarrow \quad \int \limits_a^x u'_k(s) \, ds \conv{}{} \int \limits_a^x u'_{\text{обобщ}} (s) \, ds.$$
Получили, что почти всюду 
$$ u(x) = \int \limits_a^x \underbrace{u'_{\text{обобщ}}}_{\in L^2(a,b)}(s) \, ds.$$
Это означает, что $u'_{\text{класс}} = u'_{\text{обобщ}}$ почти всюду.

\end{proof}
% !TEX encoding = UTF-8 Unicode
% лекции 15-16, 2 апреля 2016
% 6. Неравенство Фридрихса.
% 7. Слабая непрерывность линейных операторов. Слабая сходимость в $H_0^1$ и в $L^2$. Слабая непрерывность оператора обобщенного дифференцирования.
% 8. ∗ Существование точки минимума интегрального функционала в $H_0^1$.
% 9. Существование обобщенного решения задачи Дирихле для уравнения Пуассона — вариационный метод. ∗ Метод конечных элементов (метод Рисса).
% 10. Точки минимума выпуклых и строго выпуклых функционалов. Единственность экстремали. Единственность обобщенного решения задачи Дирихле для уравнения Пуассона - вариационный метод.
% 11. Существование и единственность обобщенного решения задачи Дирихле для уравнения Пуассона — использование теоремы Рисса. ∗∗ Регулярность обобщенного решения (существование классического решения) — формулировка.
% 12. Непрерывная обратимость оператора $-\Delta : H_0^1 \to L^2$.

% 6. Неравенство Фридрихса.
\subsection{Неравенство Фридрихса}
\begin{theorem} Пусть $\Omega \subset \real^n$ - ограниченная область, $u \in \m{H01Om}$. Тогда
$$ \int \limits_\Omega u^2 \, dx \leq C_\Omega \int \limits_\Omega |\nabla u|^2 \, dx.$$
\end{theorem}
\begin{proof} Пространство $C_0^\infty(\Omega)$ плотно в $\m{H01Om}$, так что достаточно доказать для него, а потом перейти к пределу.

Пусть $u \in C_0^\infty(\Omega)$. Существует такое $a$, что $\Omega \subset (-a,a)^n.$ Зафиксируем все координаты, кроме $i$-ой. Тогда по неравенству Гёльдера
\begin{align*}
|u(x)| &\leq \int \limits_{-a}^x \Bigl\lvert \pder{x_i} u(x_1, ..., \xi_i, ..., x_n) \Bigr\rvert \, d \xi_i \\ \\
& \leq \left( \int \limits_{-a}^a \Bigl\lvert \pder{x_i} u(x_1, ..., \xi_i, ..., x_n)  \Bigr\rvert^2 \, d\xi_i\right)^{1/2} \sqrt{2a},\\
|u(x)|^2 &\leq 2a \int \limits_{-a}^a | u_{x_i} (x_1, ..., \xi_i, ..., x_n) | \, d\xi_i.
\end{align*}
Интегрируем по всем остальным координатам по $(-a, a)$. После интегрирования по всем координатам, кроме $i$-ой, получим норму $u_{x_i}$ в квадрате. Проинтегрировав её по $x_i$, получим дополнительный множитель $2a$:
$$ \norm*{2}{u}^2 \leq 4a^2 \norm*{2}{u_{x_i}} \leq 4a^2 \norm*{2}{\nabla u}^2.$$
Таким образом,
$$\norm*{2}{u} \leq C_\Omega \norm*{2}{\nabla u},$$
где константа зависит только от диаметра области.

Пусть теперь $u \in \m{H01Om}$. Аппроксимируем:
$$ \exists \{u_k \} \subset C_0^\infty(\Omega) : u_k  \conv{H1}{} u,$$
то есть,
$$ \norm{H1}{u_k - u}^2 = \norm*{2}{u_k - u}^2 + \norm*{2}{\nabla u_k - \nabla u}^2 \conv{}{} 0,$$
откуда
$$\norm*{2}{u_k} \conv{}{} \norm*{2}{u}, \quad \norm*{2}{\nabla u_k} \conv{}{} \norm*{2}{\nabla u}.$$
Итого,
$$ \norm*{2}{u} \leq C \norm*{2}{\nabla u},$$
что и требовалось доказать.
\end{proof}

\begin{note}
Впервые нам реально понадобилась ограниченность области $\Omega$.
\end{note}

\begin{note}
На самом деле мы пользуемся ограниченностью только по одному направлению. 
\end{note}

% 7. Слабая непрерывность линейных операторов. Слабая сходимость в $H_0^1$ и в $L^2$. Слабая непрерывность оператора обобщенного дифференцирования.
\subsection{Слабая сходимость в гильбертовом пространстве}
Вспомним некоторые сведения из функционального анализа.
\begin{definition}
Пусть $H$ - гильбертово пространство. Последовательность $\{x_k \} \subset H$ называется слабо сходящейся к $x$, если
$$\scalprod*{H}{x_k}{y} \conv*{H}{} \scalprod*{H}{x}{y} \quad \forall y \in H,$$
и обозначается
$$ x_k \wkconv*{H}{} x.$$
\end{definition}

\begin{note}
Слабая сходимость слабее сильной:
$$ x_k \conv{}{} x \quad \Rightarrow \quad x_k \wkconv{}{} x.$$
\end{note}

\begin{note}
Обратное неверно.
\end{note}

\begin{example} Пусть $H = L^2(0, \pi)$, функция $f \in H$. Тогда $f$ представима в виде сходящегося в смысле $H$ ряда
$$ f = \sum_{k=1}^\infty a_k \sin kx.$$
Обозначим $u_k = \sin kx$. Известно равенство Парсеваля:
$$ \sum_{k=1}^\infty a_k^2 = \frac {2} {\pi} \norm*{2}{f}^2.$$
Значит, $|a_k| \to 0$, где
$$ a_k = \frac {\scalprod*{H}{f}{u_k}} {\norm*{H}{u_k}^2} = \frac {2} {\pi} \scalprod*{H}{f}{u_k} \to 0.$$
Это выполняется для всех $f$. По определению слабой сходимости
$$ u_k \wkconv*{H}{} 0 \quad \Leftrightarrow \quad \sin kx \wkconv*{H}{k \to \infty} 0.$$
Но сильно $\{ u_k \}$ никуда не сходится хотя бы потому, что $\displaystyle \norm*{H}{u_k} = \frac {\pi} {2}$.
\end{example}

\begin{note}
Слабая топология в гильбертовом пространстве не задаётся никакой метрикой. Однако, шар в сепарабельном гильбертовом пространстве со слабой топологией метризуем.
\end{note}

% TODO: доказать
\begin{note}[Слабая компактность шара] Пусть $H$ - сепарабельное гильбертово пространство. Тогда
$$ \{ x_k \} \subset H : |x_k| \leq C \quad \Rightarrow \quad \exists \{ x_{k_n} \} : x_{k_n} \wkconv{}{} x.$$
\end{note}

% TODO: доказать
\begin{note} В сепарабельном гильбертовом пространстве норма слабо полунепрерывна снизу:
$$x_k \wkconv*{}{} x \quad \Rightarrow \quad \norm{}{x_k} \leq \lim \inf \norm*{}{x_k}.$$
\end{note}

% TODO: замкнутое выпуклое подмножество гильбертова пр-ва слабо замкнуто

\begin{note} Непрерывный линейный оператор $A$ между гильбертовыми пространствами $H_1$ и $H_2$  сохраняет слабую сходимость:
$$ x_k \wkconv*{H_1}{} x \quad \Rightarrow \quad Ax_k \wkconv*{H_2}{} Ax.$$
\end{note}
\begin{proof}
$$ \scalprod*{H_2}{A x_k} {y} = \scalprod*{H_1}{x_k}{A^*y} \conv*{H_1}{} \scalprod*{H_1}{x}{A^*y} = \scalprod*{H_2}{Ax}{y}.$$

\end{proof}

\subsection{Слабая сходимость в $\m{L2}$ и в $\m{H01}$}
Пусть $\Omega \subset \real^n$ - область. Рассмотрим оператор вложения
$$\dot{\iota} : \m{H01Om} \longrightarrow \m{L2Om},$$
$$\dot{\iota}(u) := u.$$

Оператор $\dot{\iota}$ непрерывен, так как
$$ \norm*{2}{\dot{\iota}}(u)^2 = \norm*{2}{u}^2 \leq \norm{H1}{u}^2.$$
Следовательно,
$$ u_k \wkconv{H1}{} u \quad \Rightarrow \quad u_k \wkconv{L2}{} u.$$

Рассмотрим оператор обобщённого дифференцирования по $i$-ой переменной:
$$ \pder{x_i} : \m{H01Om} \longrightarrow \m{L2Om}.$$
Нам уже известно, что он линеен и непрерывен. Значит, он тоже сохраняет слабую сходимость:
$$ u_k \wkconv{H1Om}{} u \quad \Rightarrow \quad \pder[u_k]{x_i} \wkconv{L2Om}{} \pder[u]{x_i}.$$

% 8. ∗ Существование точки минимума интегрального функционала в $H_0^1$.
\subsection{Существование точки минимума интегрального функционала в $\m{H01}$}
\begin{lemma} В $\m{H1Om}$ функционал $F$ слабо полунепрерывен снизу:
$$u_k \wkconv{H1}{} u \quad \Rightarrow \quad F(u) \leq \lim \inf F(u_k).$$
\end{lemma}
\begin{proof} Прежде всего заметим, что
$$ \int \limits_\Omega f u_k \, dx = \scalprod*{2}{f}{u_k} \wkconv{L2}{} \scalprod*{2}{f}{u} =\int \limits_\Omega f u \, dx.$$
Посмотрим на обобщённый градиент:
\begin{align*}
0 \leq | \nabla (u_k - u) |^2 &= | \nabla u_k - \nabla u |^2 = |\nabla u_k|^2 + |\nabla u|^2 - 2 \nabla u_k \cdot \nabla u \\
&= |\nabla u_k|^2 - |\nabla u|^2 + 2 \nabla u \cdot (\nabla u - \nabla u_k),
\end{align*}
$$ |\nabla u_k|^2 - |\nabla u|^2 \geq 2 \nabla u \cdot (\nabla u_k - \nabla u).$$
Проинтегрируем:
$$ \int \limits_\Omega \frac {| \nabla u_k|^2} {2} \, dx - \int \limits_\Omega \frac {| \nabla u|^2} {2} \, dx \geq \int \limits_\Omega \nabla u \cdot (\nabla u_k - \nabla u) \, dx = \sum_{j=1}^n \int \limits_\Omega u_{x_j} \left( \pder[u_k]{x_j} - u_{x_j} \right) .$$
Заметим, что
$$ \pder[u_k]{x_j} - \pder[u]{x_j} \wkconv{L2}{} 0 \quad \Rightarrow \quad u_{x_j} (\hdots) \conv{}{k \to \infty} 0,$$
то есть,
$$ \lim \inf \frac {1} {2} \int \limits_\Omega |\nabla u_k|^2 \, dx \geq \frac {1} {2} \int \limits_\Omega |\nabla u|^2 \, dx.$$
Итого
\begin{align*}
\lim \inf F(u_k) &= \lim \inf \left( \frac {1}{2} \int \limits_\Omega |\nabla u_k|^2 \, dx - \int \limits_\Omega fu_k \, dx \right) \\
&\geq \lim \inf \frac {1} {2} \int \limits_\Omega |\nabla u_k|^2 \, dx - \lim \inf \int \limits_\Omega fu_k \, dx \\
&\geq \frac {1} {2} \int \limits_\Omega |\nabla u|^2 \, dx - \int \limits_\Omega fu \, dx = F(u),
\end{align*}
что и требовалось доказать.

\end{proof}

\begin{theorem}[Существование точки минимума интегрального функционала] Пусть $\Omega \subset \real^n$ - ограниченная область, $f \in L^2(\Omega)$, и задан фунцкионал 
$$F(u) = \frac {1} {2} \int \limits_\Omega | \nabla u |^2 \, dx - \int \limits_\Omega fu \, dx.$$
Тогда существует $u \in \m{H01Om}$ такая, что 
$$ F(u) = \inf_{v \in \m{H01Om}} F(v).$$
\end{theorem}
\begin{note} Функционал $F$ определён на $\m{H01Om}$: если понимать градиент как обобщённый, то выражение имеет смысл.
\end{note}
\begin{proof}[Доказательство теоремы]
Воспользуемся методом Тонелли. Рассмотрим минимизирующую последовательность $\{ u_k \} \subset \m{H01Om}$:
$$ F(u_k) \conv{}{} \inf_{\m{H01Om}} F.$$
Заметим, что верно (например, $C = 0$ при $u \equiv 0$):
$$F(u_k) = \frac {1} {2} \int \limits_\Omega | \nabla u_k |^2 \, dx - \int \limits_\Omega f u_k \, dx \leq C.$$
Заметим, что :
\begin{align*}
F(u_k) &\stackrel{\text{(а)}}{\geq} \frac {1} {2} \norm*{2}{\nabla u_k}^2 - \frac{1} {2 \eps} \norm*{2}{f}^2 - \frac{\eps^2}{2} \norm*{2}{u_k}^2 \\
& \stackrel{\text{(б)}}{\geq} \frac {1} {2} \norm*{2}{ \nabla u_k } - \frac {1} {2\eps} \norm*{2}{ u_k}^2 - \frac {\eps^2} {2} C_\Omega \norm*{2}{ \nabla u_k}^2.
\end{align*}
В первом переходе воспользовались тем, что 
$$ ab \leq \frac {a^2 + b^2} {2} \quad \Rightarrow \quad \Biggl\lvert \int \limits_\Omega \frac {f} {\eps} \eps u_k \, dx \Biggr\rvert \leq \frac {1} {2} \left( \frac {1} {\eps^2} \norm*{2}{f}^2 + \eps^2 \norm*{2}{u_k}^2 \right).$$
Имеем:
$$C \geq F(u_k) \geq \frac {1} {2} \left(1 - \eps^2 C_\Omega \right) \norm*{2}{ \nabla u_k }^2 - \frac {1} {2 \eps^2} \norm*{2}{f}^2.$$
Тогда
$$ \frac {1} {2} \left( 1 - \eps^2 C_\Omega \right) \norm*{2}{ \nabla u_k}^2 \leq \underbrace {C + \frac {1} {2 \eps^2} \norm*{2}{f}^2}_{=C}.$$
Можно выбрать такое $\eps$, что $ \eps^2 C_\Omega < 1$. Тогда 
$$ \norm*{2}{\nabla u_k}^2 \leq C,$$
и
$$ \norm*{2}{u_k}^2 \leq C_\Omega \norm*{2}{ \nabla u_k}^2 \leq C.$$
Таким образом, $\{ u_k \}$ ограничена в $\m{H01Om}$:
$$ \norm{H1}{u_k} \leq C,$$ 
то есть, $\{ u_k \}$ лежит в некотором шаре в $\m{H1Om}$.

В стандартной топологии шар не компактен, зато компактен в слабой топологии. Значит, существует такая подпоследовательность $\left\{ u_{k_m} \right\} \subset \left\{ u_k \right\}$, что 
$$u_{k_m} \wkconv{H1Om}{m \to \infty} u \in \m{H01Om}.$$ 
Заметим, что слабая сходимость в $\m{H1Om}$ влечёт слабую сходимость в $\m{L2Om}$, а оператор обобщённого дифференцирования слабо непрерывен. Значит,
$$ u_{k_m} \wkconv{L2Om}{} u, \quad \pder[u_{k_m}]{x_i} \wkconv{L2Om}{} \pder[u]{x_i},$$
и
$$ \int \limits_\Omega f u_{k_m} \, dx \longrightarrow \int \limits_\Omega fu \, dx.$$

Первое слагаемое функционала непрерывно, а второе слагаемое по лемме полунепрерывно снизу. Значит,
$$ F(u) \leq \lim \inf \left( \int \limits_\Omega | \nabla u_{k_m}|^2 \, dx - \int \limits_\Omega f u_{k_m} \, dx \right) = \inf F.$$

Итого,
$$ F(u) = \inf F.$$

\end{proof}

Резюмируя: мы рассмотрели прозвольную минимизирующую последовательность для функционала. Потом достаточно примитивным трюком доказали, что она ограниченная по норме в $H^1$, используя неравенство Фридрихса. Значит, она содержит слабо сходящуюся подпоследовательность, она тоже минимизирующая. Функционал слабо (полу)непрерывен снизу, значит, в пределе получаем нижний предел $u_k$, а это в точности инфимум.

% 9. Существование обобщенного решения задачи Дирихле для уравнения Пуассона — вариационный метод. ∗ Метод конечных элементов (метод Рисса).
\subsection{Существование обобщённого решения - вариационный метод}
\begin{theorem} Пусть $u \in \m{H01Om}$ - минимум $F$, тогда $u$ - решение уравнения Пуассона:
$$- \Delta u = f.$$
\end{theorem}
\begin{proof} Если $u$ - минимум $F$, то
$$F'(u)v = \int \limits_\Omega \nabla u \cdot \nabla v \, dx - \int \limits_\Omega f v \, dx = 0 \quad \forall v \in \m{H01Om}.$$
В частности, это выполняется и для всех $v$ из $C_0^\infty (\Omega)$. Тогда
\begin{align*}
	0 &= \int \limits_\Omega \nabla u \cdot \nabla v \, dx - \int \limits_\Omega f v \, dx = \sum_{k=1}^n \int \limits_\Omega u_{x_k} v_{x_k} \, dx - \int \limits_\Omega f v \, dx \\
	&= \sum_{k=1}^n \action{v_{x_k}, u_{x_k}} - \action{v,f} = - \sum_{k=1}^n \action{v, u_{x_k x_k}} - \action{v,f} \\
	&= - \action{v, \underbrace{\Delta u}_{\text{обобщ}}} - \action{v, f} = \action{v, -\Delta u -f} = 0.
\end{align*}
Значит, минимум функционала $F$ удовлетворяет уравнению Пуассона в обобщённом смысле 
$$ - \Delta u = f.$$

\end{proof}

А минимум существует по предыдущей теореме. Значит, решение уравнения Пуассона  существует.

В дальнейшем нам пригодится следующая лемма.
\begin{lemma}
Если $-\Delta u = f$ в обобщённом смысле, и $u \in \m{H01Om}$, то
$$ \int \limits_\Omega \nabla u \cdot \nabla v \, dx - \int \limits_\Omega fv \, dx = 0 \quad \forall v \in \m{H01Om}.$$
\end{lemma}
\begin{proof} Надо доказать в обратную сторону предыдущую цепочку равенств. Из условия леммы следует, что 
$$ \action{\varphi, -\Delta u} = \action{\varphi, f} \quad \forall \varphi \in C_0^\infty(\Omega).$$
Тогда 
\begin{align*}
- \sum_{k=1}^n \action{\varphi, u_{x_k x_k}} &= \sum_{k=1}^n \action{\varphi_{x_k}, u_{x_k}} = \sum_{k=1}^n \int \limits_\Omega \varphi_{x_k} u_{x_k} \, dx \\
&= \int \limits_\Omega \nabla \varphi \cdot \nabla u \, dx = \int \limits_\Omega \varphi f \, dx \quad \forall \varphi \in C_0^\infty(\Omega).
\end{align*}
Теперь рассмотрим $v$ из $\m{H01Om}$. По определению,
$$ \exists \{ \varphi_k \} \subset C_0^\infty(\Omega) : \varphi_k \conv{H1}{} v.$$
Значит, по непрерывности оператора вложения
$$ \varphi_k \conv{L2}{} v, \quad \pder[\varphi_k]{x_j} \conv{L2}{} \pder[v]{x_j}, \quad \nabla \varphi_k \conv{L2}{} \nabla v.$$

В пределе получаем
$$ \int \limits_\Omega \nabla v \cdot \nabla u \, dx - \int \limits_\Omega fv \, dx = 0.$$

\end{proof}

\begin{note} Мы доказали равносильность задач! Если $f$ из $\m{L2Om}$, $u$ из $\m{H01Om}$, то
$$ -\Delta u = f \quad \Leftrightarrow \quad \int \limits_\Omega \nabla v \cdot \nabla u \, dx = \int \limits_\Omega fv \, dx \quad \forall v \in \m{H01Om}.$$
\end{note}

Таким образом, подобное $F$ интегральное условие является обобщённым аналогом условия Дирихле.


\subsection{Метод конечных элементов} Пусть $\Omega \subset \real^2$ - ограниченная область. Задача Дирихле для уравнения Пуассона в обобщённом смысле эквивалентентна вариационной задаче
$$ F(u) = \frac {1}{2} \int \limits_\Omega |\nabla u|^2\ \,dx - \int \limits_\Omega fu \, dx \rightarrow \min.$$
Знаем, что у $F$ существует минимум. Тогда у задачи Дирихле для уравнения Пуассона существует решение (чуть далее мы докажем, что минимум и решение единственны. Вариационный метод подсказывает, как можно найти приближённое решение.

Триангулируем $\Omega$ - разобьём её на $m$ достаточно малых треугольников. Скажем, что на $k$-ом треугольнике действует линейная функция $u_k$, приближающая $u$ в этой части области. Линейная функция на треугольнике определяется своими значениями в вершинах треугольника. Таким образом, неизвестная функция $u_k$ заменяется на неизвестные функции $u_k$, которые определяются неизвестными значениями в узлах треугольников. В узлах на границе положим $u = 0$. Тогда $F$ - квадратичная функция на $\real^m$, она выпукла.

Таким образом, задача свелась к нахождению минимума квадратичной функции на конечномерном пространстве. Если приравнять её градиент к нулю, то полученное соотношение можно представить в виде системы линейных уравнений порядка $m$ с трёхдиагональной матрицей.

Можно доказать, что при дальнейшем дроблении области на треугольники полученное приближение будет сходиться к $u$ по энергетической норме.  

В $\real^n$ вместо треугольников пространство следует разбивать на симплексы.

% 10. Точки минимума выпуклых и строго выпуклых функционалов. Единственность экстремали. Единственность обобщенного решения задачи Дирихле для уравнения Пуассона - вариационный метод.
\subsection{Единственность обобщённого решения - вариационный метод}

\begin{definition} Вещественный функционал $G$ на гильбертовом пространстве, удовлетворяющий условию
$$ G(\lambda_1 u_1 + \lambda_2 u_2) \leq \lambda_1 G(u_1) + \lambda_2 G(u_2), \quad \lambda_1,\lambda_2 \geq 0, \quad \lambda_1 + \lambda_2 = 1,$$
называется выпуклым.
\end{definition}

\begin{exercise}
$$ F(u) = \frac {1} {2} \int \limits_\Omega |\nabla u|^2 \, dx - \int \limits_\Omega fu \, dx \quad \text{--- выпуклый.}$$
\end{exercise}

\begin{definition} Вещественный функционал $G$ на гильбертовом пространстве, удовлетворяющий условию
$$ G(\lambda_1 u_1 + \lambda_2 u_2) < \lambda_1 G(u_1) + \lambda_2 G(u_2), \quad \lambda_1,\lambda_2 > 0, \quad \lambda_1 + \lambda_2 = 1,$$
называется строго выпуклым.
\end{definition}

\begin{exercise}
Если $G$ --- строго выпуклый, то у него единственная точка минимума.
\end{exercise}

Докажем единственность. Нам понадобится следующая абстрактная лемма:

\begin{lemma} Пусть $F$ - выпуклый вещественный функционал на гильбертовом пространстве $H$, у которого всюду существует первая вариация
$$ \exists F'(u) v \quad \forall v \in H,$$ и пусть имеется $u$ из $\m{H01Om}$ такое, что
$$ F'(u)v = 0 \quad \forall v \in H.$$
Тогда $u$ - точка минимума $F$:
$$ F(u) = \inf_{\m{H01Om}} F. $$
% ???? не странно ли записано условие?
\end{lemma}
\begin{proof}
Рассмотрим выпуклую функцию
$$ g(t) = F(u+tv) \quad t\in \real.$$
Тогда
$$F'(u)v = \frac{d}{dt} F(u+tv)\Big\rvert_{t=0} = g'(0),$$
и $g(t)$ лежит выше касательной:
$$g(t) \geq g(0) + tg'(0) \quad \forall t \in \real.$$
Рассмотрим такое $w \in H$, что
$$ w = u + tv \quad \Rightarrow \quad v = \frac {w-u} {t}.$$
Тогда
$$ F(w) \geq F(u) + t F'(u) \frac{w-u}{t} = F(u) + F'(u)(w-u).$$
Итого
$$ F'(u)(w-u) = 0 \quad \Rightarrow F(u) \leq F(w) \quad \forall w,$$
значит, $u$ - минимум $F$.

\end{proof}
Теперь мы готовы доказать
\begin{theorem}
Если $u \in \m{H01Om}$ удовлетворяет уравнению
$$ - \Delta u = f,$$
то $$F(u) = \min_{\m{H01Om}} F.$$ 
\end{theorem}
\begin{proof}
Задача Дирихле для уравнения Пуассона эквивалентна задаче
$$F(u) = \int \limits_\Omega \nabla u \cdot \nabla v \, dx - \int \limits_\Omega fu \, dx = 0.$$
У $F$ по любому направлению существует производная Гато $F'(u)v$. Рассмотрим $u$ такое, что 
$$ F'(u) v = 0 \quad \forall v.$$
Тогда по второй лемме $u$ - минимум $F$.

\end{proof}

Таким образом, мы доказали, что уравнение Пуассона имеет единственное решение, совпадающее с минимумом функционала $F$.

% 11. Существование и единственность обобщенного решения задачи Дирихле для уравнения Пуассона — использование теоремы Рисса. ∗∗ Регулярность обобщенного решения (существование классического решения) — формулировка.

\subsection{Существование и единственность обобщённого решения - использование теоремы Рисса}

Пусть $\Omega \subset \real^n$ - ограниченная область, $f$ из $\m{L2Om}$, и поставлена задача Дирихле для уравнения Пуассона:
\begin{align*}
\begin{cases*}
	- \Delta u = f, \\
	u \in \m{H01Om}.
\end{cases*}
\end{align*}
При помощи вариационного метода мы доказали существование и единственность решения. Докажем то же самое, опираясь на теорему Рисса.

\begin{reminder}[Теорема Рисса] Пусть $H$ - гильбертово пространство, $f$ - непрерывный линейный функционал на $H$. Тогда существует единственный $y \in H$ такой, что 
$$ f(x) = \scalprod*{H}{y}{x} \quad \forall x \in H.$$
То есть, любой линейный функционал на $H$ есть скалярное произведение с некоторым $y$ из $H$.
Кроме того,
$$ \norm*{H}{y} = \norm*{H^*}{f}.$$
\end{reminder}

\begin{lemma}[Лакс-Мильграм] Пусть $\widehat{H}$ и $H$ --- гильбертовы пространства и первое непрерывно вложено во второе:
$$ \widehat{H} \hookrightarrow H. $$
Тогда
$$ \forall y \in H \quad \exists ! x \in \widehat{H}: \scalprod*{\widehat{H}}{x}{z} = \scalprod*{H}{y}{z} \quad \forall z \in \widehat{H}.$$
\end{lemma}
\begin{proof}
Пусть $y \in H$. Рассмотрим функционал
\begin{gather*}
f_y : \widehat{H} \longrightarrow \real, \\
f_y(z) := \scalprod*{H}{y}{z}.
\end{gather*}
Очевидно, он линеен. Проверим непрерывность:
$$ |f_y(z)| = | \scalprod*{H}{y}{z} \leq \norm*{H}{y} \cdot \norm*{H}{z} \leq \norm*{H}{y} \cdot \norm*{\widehat{H}}{z} C = C_y \norm*{\widehat{H}}{z}.$$
Функционал $f_y$ непрерывен на $\widehat{H}$. Тогда по теореме Рисса в $\widehat{H}$
$$ \exists ! \, x \in \widehat{H}: \scalprod*{\widehat{H}}{x}{z} = f_y(z) = \scalprod*{H}{y}{z} \quad \forall z \in \widehat{H}.$$

\end{proof}

\begin{note}
Фактически, мы определили оператор $A$:
\begin{gather*}
A : H \longrightarrow \widehat{H}, \\
Ay = x, \quad \scalprod*{H}{y}{z} = \scalprod*{\widehat{H}}{x}{z}.
\end{gather*}
Очевидно, он линеен. Покажем, что он непрерывен. Для начала,
$$ | \scalprod*{\widehat{H}}{x}{z} | = | \scalprod*{H}{y}{Z} | \leq C \norm*{H}{y} \cdot \norm*{\widehat{H}}{z}.$$
Отсюда
$$ \frac {| \scalprod*{\widehat{H}}{x}{z} |} {\norm*{\widehat{H}}{z}} \leq C \norm*{H}{y} \quad \Rightarrow \quad \sup_{z \in \widehat{H}} \frac {| \scalprod*{\widehat{H}}{x}{z} |} {\norm*{\widehat{H}}{z}} \leq C \norm*{H}{y} \quad \Rightarrow \quad \norm*{\widehat{H}}{x} \leq C \norm*{H}{y}.$$
То есть,
$$ \norm*{\widehat{H}}{Ay} \leq C \norm*{H}{y} \quad \Rightarrow \quad A \text{ --- непрерывный}.$$
\end{note}


\begin{theorem} Пусть $f$ из $\m{L2Om}$. Тогда обобщённое решение задачи Дирихле для уравнения Пуассона существует и единственно.
\end{theorem}
\begin{proof} Здесь
$$ H = \m{L2Om}, \quad \widehat{H} = \m{H01Om}.$$
Введём новую норму в $\widehat{H}$:
$$\norm*{\widehat{H}}{u} = \sqrt{\int \limits_\Omega | \nabla u |^2 \, dx } = \norm*{2}{\nabla u}.$$
Эта норма называется энергетической. Очевидно,
$$ \norm*{2}{\nabla u} = \norm*{\widehat{H}}{u} \leq \norm{H1}{u}.$$
В то же время,
$$ \norm*{2}{u} \leq C \norm*{2}{\nabla u} \quad \Rightarrow \quad \norm*{H1}{u} \leq C \norm*{2}{\nabla u} = C \norm*{\widehat{H}}{u}.$$
Значит, энергетическая норма эквивалентна стандартной, и $\widehat{H}$ --- сепарабельное гильбертово пространство. В то же время, вложение $\widehat{H} \hookrightarrow H$ непрерывно:
$$ \norm*{H}{u} = \norm*{2}{u} \leq C \norm*{2}{\nabla u} = C \norm*{\widehat{H}}{u}.$$
Применяем лемму. Для любого $f$ из $\m{L2Om}$ 
$$ \exists! \, u \in \m{H01Om}: \scalprod{H01}{u}{v} = \scalprod{L2}{f}{v} \quad \forall v \in \m{H01Om}.$$
В интегральной форме: существует такой $f$ из $\m{L2Om}$, что
$$ \exists ! u \in \m{H01Om} : \int \limits_\Omega \nabla u \cdot \nabla v \, dx = \int \limits_\Omega f v \, dx \quad \forall v \in \m{H01Om},$$
а это равносильно существованию и единственности решения задачи Дирихле для уравнения Пуассона.

\end{proof}

Недостаток этого подхода заключается в том, что он неконструктивен.
% ????????????
% ∗∗ Регулярность обобщенного решения (существование классического решения) — формулировка.

\subsection{Непрерывная обратимость лапласиана}
% 12. Непрерывная обратимость оператора $-\Delta : H_0^1 \to L^2$.

\begin{theorem}
Пусть $\Omega \subset \real^n$ - ограниченная область. Для любого $f$ из $\m{L2Om}$ существует единственное $u$ из $\m{H01Om}$ такое, что
$$ - \Delta u = f \quad \Rightarrow \quad u = (-\Delta)^{-1} f.$$
Значит, определён оператор
$$ (-\Delta)^{-1}: \m{L2Om} \longrightarrow \m{H01Om}.$$
Этот оператор непрерывен.
\end{theorem}
\begin{proof}
Очевидно, этот оператор линеен. Нам нужно доказать, что
$$ \norm{H1}{(-\Delta)^{-1} f} \leq C \norm*{2}{f}, \quad \text{или, что то же самое,} \quad \norm{H1}{u} \leq C\norm*{2}{f}.$$
Если $u$ - решение уравнения Пуассона, то 
$$ \int \limits_\Omega \nabla u \cdot \nabla v \, dx = \int \limits_\Omega fv \, dx \quad \forall v \in \m{H01Om}.$$
В частности, при $v= u$
\begin{align*}
\norm*{2}{\nabla u}^2 = \int \limits_\Omega |\nabla u|^2 \, dx = \int \limits_\Omega fu \, dx \leq \norm*{2}{f} \cdot \norm*{2}{u} \leq C \norm*{2}{f} \cdot \norm*{2}{\nabla u}
\end{align*}
Сначала мы воспользовались неравенством Гёльдера, а потом неравенством Фридрихса. Тогда
$$ \norm*{2}{\nabla u} \leq C \norm*{2}{f} \quad \text{и} \quad \norm*{2}{u} \leq C\norm*{2}{f}.$$
Итого,
$$ \norm{H1}{u} = \sqrt{\norm*{2}{u}^2 + \norm*{2}{\nabla u}^2} \leq C \norm*{2}{f},$$
что и требовалось доказать.

\end{proof}

% !TEX encoding = UTF-8 Unicode
% лекции 17-18, 9 апреля 2016
% 13. Непрерывность и компактность интегрального оператора в $L^2$ с непрерывным ядром.
% 14. Непрерывность интегрального оператора в $L^2$ со слабой особенностью.
% 15. Замкнутость класса компактных линейных непрерывных операторов. Компактность интегрального оператора в $L^2$ со слабой особенностью.
% 16. Компактность вложения $H_0^1 \subset L^2$ (теорема Реллиха).
% 17. Теорема Фредгольма для компактных самосопряженных операторов в гильбертовом пространстве.
% 18. Сведение задачи Дирихле для уравнения $−\Delta u + \lambda u = f$ к уравнению Фредгольма второго рода с самосопряженным, компактным, положительным оператором. Теорема об альтернативе.

\subsection*{Некоторые определения и факты из курса функционального анализа}

\begin{definition}
Множество называется предкомпактным, если его замыкание компактно.
\end{definition}

\begin{definition} Пусть $X$ - метрическое пространство, $A \subset X$ и $\eps > 0$. Множество  $F \subset X$ называется $\eps$-сетью, если
$$ A \subset \bigcup_{x \in F} B_\eps (x).$$
\end{definition}

\begin{definition}
Множество вполне ограничено, если для любого $\eps > 0$ существует конечная $\eps$-сеть.
\end{definition}

\begin{theorem}[Хаусдорф]
Множество в метрическом пространстве компактно тогда и только тогда, когда оно полное и вполне ограниченное. 
\end{theorem}

\begin{corollary}
У множества в полном метрическом пространстве для любого $\eps > 0$ тогда и только тогда существует конечная $\eps$-сеть, когда оно предкомпактно.
\end{corollary}

\begin{definition}
Оператор между банаховыми пространствами называется компактным, если он переводит ограниченное множество в предкомпактное.
\end{definition}

\begin{definition}
Семество непрерывных функций равностепенно ограничено, если существует единая для всех функций семейства константа, которой ограничены все функции семейства.
\end{definition}

\begin{definition}
Семейство непрерывных функций $F \subset C(\overline{\Omega})$ равностепенно непрерывно, если для любого $\eps >0$  существует $\delta >0$ такая, что
$$ \forall f \in F \quad \forall x_1,x_2 \in \overline{\Omega} : | x_1 - x_2 | < \delta \quad \Rightarrow \quad |f(x) - f(y)| < \eps .$$ 
\end{definition}

\begin{theorem}[Асколи-Арцела]
Семейство непрерывных функций предкомпактно в $C(\overline{\Omega})$ тогда и только тогда, когда оно равномерно ограничено и равностепенно непрерывно.
\end{theorem}

\begin{definition}
Оператор между банаховыми пространствами называется компактным, если он переводит ограниченное множество в предкомпактное.
\end{definition}

\subsection{Интегральные операторы}
Пусть $\Omega \subset \real^n$ - ограниченная область.
Рассмотрим интегральный оператор $T$, определённый формулой
$$ (Tu)(x) = \int \limits_\Omega K(x,y) u(y) \, dy,$$
где $K$ - функция, называемая ядром\footnote{Не путать с ядром оператора в смысле теории операторов. Здесь ядро от слова nucleus, в теории операторов ядро от слова kernel. Также эту функцию называют производящей функцией.}:
$$ K : \Omega \times \Omega \longrightarrow \real.$$

% 13. Непрерывность и компактность интегрального оператора в $L^2$ с непрерывным ядром.
\begin{theorem} Если ядро $K$ непрерывно на $C(\overline{\Omega} \times \overline{\Omega})$, то соответствующий интегральный оператор
$$ T: \m{L2Om} \longrightarrow C(\overline{\Omega}) $$
непрерывен и компактен.
\end{theorem}
\begin{proof} Для докательства непрерывности достаточно проверить, что оператор переводит шар из $\m{L2Om}$ в ограниченное множество. Пусть
$$ u \in B_R(0), \quad \norm*{2}{u} \leq R.$$
Посчитаем $\norm*{\infty}{Tu}$:
\begin{align*}
| Tu(x) | &\leq \int \limits_\Omega |K(x,y)| \cdot |u(y)| \, dy \leq C \int \limits_\Omega |u(y)| \, dy \\
& \leq C \norm*{2}{u} \cdot \norm*{2}{\mathds{1}_\Omega} \leq CR.
\end{align*}
То есть, оператор $T$ --- ограниченный.

Проверим компактность. Надо доказать, что множество
$$ \left\{ v = Tu \, \Big| \norm*{2}{u} \leq R \right\} $$ предкомпактно в $C(\overline{\Omega})$. Воспользуемся теоремой Асколи-Арцела. Равномерную ограниченность мы только что доказали, достаточно проверить равностепенную непрерывность. Ядро $K$ непрерывно на компакте, значит, по теореме Кантора оно равномерно непрерывно:
$$\forall \eps \, \exists \delta: \quad \forall x_1, x_2 \in \overline{\Omega} \quad |x_1 - x_2| < \delta \quad \Rightarrow \quad |K(x_1,y) - K(x_2,y)| \leq \eps.$$
Заметим, что
\begin{align*}
|Tu(x_1) - Tu(x_2)| \leq \int \limits_\Omega | K(x_1,y) - K(x_2,y)| \cdot |u(y)| \, dy \leq \eps \int \limits_\Omega |u(y)| \, dy \leq \eps CR,
\end{align*}
для всех $u$ из нашего шара. Значит,
$$ \forall \eps \, \exists \delta : \quad \forall x_1, x_2 \in \overline{\Omega} \quad |x_1 - x_2| < \delta \quad \Rightarrow \quad |Tu(x_1) - Tu(x_2)| \leq \eps,$$
что и требовалось. Оператор $T$ компактен.

\end{proof}

\begin{corollary} Если ядро $K \in C(\overline{\Omega} \times \overline{\Omega})$, тогда оператор
$$ T : \m{L2Om} \longrightarrow \m{L2Om},$$
$$(Tu)(x) := \int \limits_\Omega K(x,y) u(y) \, dy $$
непрерывен и компактен.
\end{corollary}
\begin{proof}
Пусть 
$$ T_1 : \m{L2Om} \longrightarrow C(\overline{\Omega}), \quad T_2 : \m{L2Om} \longrightarrow \m{L2Om}$$
--- интегральные операторы с ядром $K$ и формулой из условия, а
$$ \dot{\iota} : C(\overline{\Omega}) \longrightarrow \m{L2Om}$$
--- оператор вложения. Тогда $T_2 = \dot{\iota} \circ T_1$. Оператор $T_1$ компактен, вложение непрерывно, а композиция непрерывного и компактного компактна.

\end{proof}

% 14. Непрерывность интегрального оператора в $L^2$ со слабой особенностью.
\subsection{Интегральные операторы со слабой особенностью}
Как и раньше, $\Omega \subset \real^n$ - ограниченная область.

\begin{definition} Функция $K$ называется ядром со слабой особенностью, если
$$ K = \frac {B(x,y)} {|x-y|^\alpha},$$
где $B$ ограничена на $\overline{\Omega} \times \overline{\Omega}$, а также
$$B \in C(\overline{\Omega} \times \overline{\Omega} \setminus \{ x = y \}), \quad 0 < \alpha < n.$$
\end{definition}

\begin{note} Если $K$ - ядро со слабой особенностью, то $K$ представима в виде
$$ K = \frac {B'(x,y)} {|x-y|^{\alpha'}}, \quad \text{где } B' \in C(\overline{\Omega} \times \overline{\Omega}), \quad \quad 0 < \alpha' < n.$$
\end{note}
\begin{proof}
$$ K = \frac {B(x,y)} {|x-y|^\alpha} \cdot \frac {|x-y|^\eps} {|x-y|^\eps} = \frac {B'(x,y)} {|x-y|^{\alpha + \eps}}, \quad 0 < \alpha + \eps < n.$$
Получившийся $B'(x,y)$, очевидно, будет ограниченным везде. Вне диагонали непрерывность есть, на диагонали $\displaystyle |x-y|^\eps \conv{}{x \to y} 0$. Так как нечто ограниченное, умноженное на нечто стремящееся к нулю, тоже стремится к нулю, то $B'(x,y) = 0$ на диагонали и непрерывна.

\end{proof}

\begin{theorem} Если $K$ - ядро со слабой особенностью, то интегральный оператор
$$A: \m{L2Om} \longrightarrow \m{L2Om}$$
с ядром $K$ --- линейный и ограниченный.
\end{theorem}
\begin{proof}
Линейность очевидна. Покажем ограниченность.
\begin{align*}
|Au(x)| &\leq \int \limits_\Omega |K(x,y)| \cdot |u(y)| \, dy = \int \limits_\Omega \frac {|B(x,y)|} {|x-y|^\alpha} \cdot |u(y)| \, dy \\
&\leq C \int \limits_\Omega \frac {|u(y)|} {|x-y|^\alpha} \, dy = C \int \limits_\Omega \frac {1} {|x-y|^{\alpha/2}} \cdot \frac {|u(y)|} {|x-y|^{\alpha/2}} \, dy \\ 
&\leq C \left( \int \limits_\Omega \frac {|u(y)|^2}{|x-y|^\alpha} \, dy \right)^{1/2}.
\end{align*}
Мы воспользовались неравенством Гёльдера и тем, что 
$$\int \limits_{|y| \leq \rho} \frac {1} {|y|^\alpha} \, dy = \int \limits_0^\rho n \omega_n \frac {1} {r^\alpha} r^{n-1} \, dr  = C \int \limits_0^\rho r^{n-1-\alpha} \, dr = C r^{n-\alpha} \Big\rvert_0^\rho = C \rho^{n-\alpha}, \quad \rho = \diam(\Omega)$$
Возведём в квадрат и проинтегрируем обе части неравенства по $x$:
\begin{align*}
\norm*{2}{Au(x)}^2 &= \int \limits_\Omega |Au(x)|^2 \, dx \leq C \int \limits_\Omega dx \int \limits_\Omega \frac {|u(y)|^2} {|x-y|^\alpha} \, dy \\
&\leq C \int \limits_\Omega |u(y)|^2 \, dy \underbrace {\int \limits_\Omega \frac {dx} {|x-y|^\alpha} }_{\leq C} \leq C \int \limits_\Omega |u(y)|^2 \, dy = C \norm*{2}{u}^2.  
\end{align*}
Таким образом, $A$ - ограниченный.

\end{proof}

% 15. Замкнутость класса компактных линейных непрерывных операторов. Компактность интегрального оператора в $L^2$ со слабой особенностью.

\begin{note}[Замкнутость класса компактных операторов]
Пусть $K_m$ - компактные линейные непрерывные операторы на банаховом пространстве $X$. Если
$$ K_m \conv*{\mathscr{L} (X)}{m \to \infty} K,$$
то $K$ - компактный.
\end{note}
\begin{proof}
В полном пространстве множество предкомпактно тогда и только тогда, когда для любого $\eps > 0$ существует конечная $\eps$-сеть. По условию, для любого $\eps > 0$ найдётся такое $m$, что
$$ \norm*{\mathscr{L} (X)}{A_m - A} < \eps.$$
В то же время, для любых $\eps > 0$ и $m$ у множества $A_m (B_R(0))$ найдётся конечная $\eps$-сеть. Значит, у множества $A(B_R(0))$ тоже найдётся конечная $\eps$-сеть для любого $\eps > 0$. 

\end{proof}

\begin{theorem} Пусть $K$ --- ядро со слабой особенностью. Тогда интегральный оператор
$$ A : \m{L2Om} \longrightarrow \m{L2Om}$$
с этим ядром компактен.
\end{theorem}
\begin{proof} По замечанию,
$$ K(x,y) = \frac{B(x,y)} {|x-y|^\alpha}, \quad B \in C(\overline{\Omega} \times \overline{\Omega}), \quad 0 < \alpha < n.$$
Рассмотрим последовательность ядер $K_m$:
\begin{gather*}
K_m(x,y) =
	\begin{cases*}
		\dfrac {B(x,y)} {|x-y|^\alpha}, \quad |x-y| \geq \dfrac {1} {m} \\
		\\
		\dfrac {B(x,y)} {(1/m)^\alpha}, \quad |x-y| \leq \dfrac {1} {m}
	\end{cases*}
\end{gather*}
Для удобства переобозначим:
$$|x-y|_m = \min \left( \frac {1} {m} , |x-y| \right) \quad \Rightarrow \quad K_m(x,y) = \frac {B(x,y)} {|x-y|_m}.$$
Заметим, что все $K_m$ непрерывны на $\overline{\Omega} \times \overline{\Omega}$. То есть, мы "доопределили" ядро на диагонали, тем самым вырезав особенность. По следствию из ранее доказанной теоремы, $A_m$ ограниченны и компактны как операторы, действующие в $\m{L2Om}$. Известно, что сходящаяся по операторной норме последовательность компактных операторов сходится к компактному. Докажем, что
$$ A_m \conv*{\mathscr{L} (\m{L2Om})}{m\to \infty} A.$$
Вне окрестности диагонали значения операторов равны, достаточно рассматривать разность только внутри диагонали:
\begin{align*}
| A_m u(x) - Au(x)| &= \Bigg\rvert \int \limits_{|x-y| < \frac {1} {2}} \left( \frac {B(x,y)} {|x-y|^\alpha} - \frac {B(x,y)} {(1/m)^\alpha} \right) u(y) \, dy \Biggl\lvert \\
&\leq C \int \limits_{|x-y| < \frac {1} {2}} \Bigg\lvert \frac {1} {|x-y|^\alpha} - \frac {1} {(1/m)^\alpha} \Bigg\rvert \cdot |u(y)| \, dy \\
&\leq 2 C \int \limits_{|x-y| < \frac {1} {m}} \frac {1} {|x-y|^\alpha} \cdot |u(y)| \, dy \\
&\leq 2 C \left( \underbrace {\int \limits_{|x-y| < \frac {1} {m}} \frac {1} {|x-y|^\alpha} \, dy}_{\leq C \left( \frac {1} {m} \right)^{n-\alpha}} \right)^{1/2} \cdot \left( \int \limits_{|x-y| < \frac {1} {m}} \frac {|u(y)|} {|x-y|^\alpha} \, dy \right)^{1/2} \\
&\leq C \frac {1} {m^{\frac {n-\alpha} {2}}} \left( \int \limits_\Omega \frac {|u(y)|} {|x-y|^\alpha} \, dy \right)^{1/2}.
\end{align*}
Возведём в квадрат и проинтегрируем:
\begin{align*}
\norm*{2}{A_m u(x) - Au(x)}^2 &\leq C \frac {1} {m^{n-\alpha}} \int \limits_\Omega dx \int \limits_\Omega \frac {|u(y)|} {|x-y|^\alpha} \, dy \leq C \frac {\norm*{2}{u}^2} {m^{n-\alpha}}.
\end{align*}
То есть,
$$ \frac {\norm*{2} {(A_m - A)u(x)}} {\norm*{2}{u}} \leq \frac {C} {m^{\frac {n-\alpha} {2}}} \quad \Rightarrow \quad \norm*{}{A_m - A} \leq \frac {C} {m^{\frac{n-\alpha} {2}}} \conv{}{m \to \infty} 0.$$
Значит, $A$ - компактный.

\end{proof}

% 16. Компактность вложения $H_0^1 \subset L^2$ (теорема Реллиха).

\subsection{Компактность вложения $\m{H01}$ в $\m{L2}$}
\begin{lemma} Пусть $\Omega \subset \real^n$ - область, $u \in C_0^\infty (\Omega)$. Тогда $u$ можно представить в виде
$$ u(x) = \frac {1} {n \omega_n} \int \limits_\Omega \frac {(x-y) \cdot \nabla u} {|x-y|^n} \, dy \quad \forall x \in \Omega.$$
\end{lemma}
\begin{proof}
Вспомним интегральную формулу для частного решения уравнения Пуассона:
$$ u(x) = - \int \limits_\Omega \Phi(x-y) \Delta u(y) \, dy.$$
Вырежем особенность на диагонали, чтобы можно было интегрировать по частям:
\begin{align*}
u(x) &= - \lim_{\eps \to 0} \int \limits_{B^c_\eps (x)} \Phi (x-y) \Delta u(y) \, dy \\
&= - \lim_{\eps \to 0} \left( - \int \limits_{B^c_\eps (x)} \nabla \Phi(x-y) \cdot \nabla u(y) \, dy + \int \limits_{\partial B_\eps (x)} \Phi(x-y) \underbrace{\pder{n}u(y)}_{\substack{\text{внешняя} \\ \text{нормаль}} } \, d\sigma \right).
\end{align*}
Оценим второе слагаемое:
\begin{align*}
\Bigg| \int \limits_{\partial B_\eps (x)} \Phi(x-y) \pder{n}u(y) \, d\sigma \Bigg| &\leq \int \limits_{\partial B_\eps (x)} \Phi(x-y) \underbrace {\Bigg| \pder{n} u(y) \Bigg|}_{\leq C} \, d\sigma(y) \\
&\leq C \int \limits_{\partial B_\eps (x)} \Phi(x-y) \, dy = C \int \limits_{\partial B_\eps (0)} \Phi(y') \, d\sigma(y') \\
&\leq C \frac {1} {\eps^{n-2}} \int \limits_{\partial B_\eps(0)} \, d\sigma(y) = \frac {C} {\eps^{n-2}} \eps^{n-1} = C \eps \conv{}{\eps \to 0} 0.
\end{align*}
Здесь мы сделали замену и воспользовались тем, что
$$ x-y = y', \quad \Phi(y') = \frac {C} {|y'|^{n-2}}, \quad |y| = \eps.$$
Таким образом,
$$ u(x) = \lim_{\eps \to 0} \int \limits_{B^c_\eps (x)} \nabla \Phi(x-y) \cdot \nabla u(y) \, dy.$$
Посчитаем градиент $\Phi(x-y)$:
\begin{gather*}
\Phi(x-y) = \frac {1} {|x-y|^{n-2}} \cdot \frac {1} {n(n-2)\omega_n},\\
\nabla \Phi(x-y) = \frac {n-2} {n(n-2)\omega_n} \cdot \frac {1} {|x-y|^{n-1}} \cdot \frac {x-y} {|x-y|} = \frac {x-y} {n \omega_n |x-y|^n}.
\end{gather*}
Итого,
$$  u(x) = \frac {1} {n \omega_n} \int \limits_\Omega \frac {(x-y) \cdot \nabla u(y)} {|x-y|^n} \, dy.$$

\end{proof}

% TODO: здесь омега - просто область, но в этом замечании мы пользуемся утверждениями про компактные интегральные операторы, которые доказывались для ограниченных областей
\begin{note}
Та же самая формула верна для $u \in \m{H01Om}$.
\end{note}
\begin{proof}
По определению $\m{H01Om}$ существует такая $\{ u_j \} \subset C_0^\infty(\Omega)$, что
$$ u_j \conv{H1}{} u \quad \Rightarrow \quad u_j \conv{L2}{} u, \quad \pder[u_j]{x_i} \conv{L2}{} \pder[u]{x_j}.$$
По лемме каждую $u_j$ можно представить в виде
$$ u_j(x) = \frac{1}{n \omega_n} \int \limits_\Omega \frac{(x-y) \cdot \nabla u_j(y)} {|x-y|^n} \, dy = \sum_{k=1}^n K_i \pder{x_i} u(x),$$
где $K_i$ это интегральные операторы со слабой особенностью
$$ (K_i v)(x) = \frac {1} {n \omega_n} \int \limits_\Omega \frac{x_i - y_i}{|x-y|^n} v(y) \, dy,$$
ядра $N_i$ которых можно представить в виде
$$ N_i (x,y) = \frac {1} {n \omega_n} \frac {B_i (x,y)} {|x-y|^{n-1}}, \quad B_i(x,y) = \frac {x_i - y_i} {|x-y|}.$$
Функция $B$ непрерывна вне диагонали, значит $K_i$ непрерывны. Тогда
$$ K_i \pder[u_j]{x_i} \conv{L2}{} K_i \pder[u]{x_i},$$
и
$$ u(x) = \sum_{k=1}^n K_i \pder{x_i}u(x) = \frac {1} {n \omega_n} \sum_{k=1}^n \int \limits_\Omega \frac {x_i - y_i} {|x-y|^n|} \pder{x_i}u(x) \, dy = \frac {1} {n \omega_n} \int \limits_\Omega \frac {(x-y) \cdot \nabla u(y)} {|x-y|^n} \, dy. $$

\end{proof} 

\begin{theorem}[Реллих-Кондрашов] Пусть $\Omega \subset \real^n$ --- ограниченная область. Тогда вложение
$$\dot{\iota} : \m{H01Om} \hookrightarrow \m{L2Om}$$
компактно.
\end{theorem}
\begin{proof}
Надо доказать, что образ шара 
$$B_R (0) \subset \m{H01Om}.$$
предкомпактен в $\m{L2Om}$. Пусть $u \in B_R(0)$, а оператор $K$ таков:
$$ Kv = \sum_{k=1}^n K_i v_i, \quad (K_i w)(x) = \frac {1} {n \omega_n} \int \limits_\Omega \frac {x_i - y_i} {|x-y|^n} w(y) \, dy.$$
Тогда, по замечанию, $K_i$ непрерывны и компактны, а $u$ представимо в виде
$$ u = K \nabla u, \quad K : \m{L2}(\Omega ; \real^n) \longrightarrow \m{L2Om}$$
Заметим, что производные $u$ ограничены, тогда градиент тоже ограничен. Оператор $K$ компактен. Значит, множество
$$ \dot\iota (B_R(0)) = \left\{ K (\nabla u) \, \Big| \, u \in B_R(0) \right\}$$
--- предкомпакт, что и требовалось доказать.

\end{proof}

\begin{note}
На самом деле, теорема Реллиха-Кондрашова это намного более общий результат, мы сформулировали и доказали лишь частный случай. 
\end{note}

\begin{corollary} Пусть $\{ u_k \} \subset \m{H01Om}$. Тогда
$$ u_k \wkconv{H01}{} u \quad \Rightarrow \quad u_k \conv{L2}{} u.$$
\end{corollary}
\begin{proof}
% из Банаха-Штейнгауза?
%Из теоремы Банаха-Штейнгауза следует, что последовательность $\{ u_k \}$ ограничена
Последовательность $\{ u_k \}$ ограничена в $\m{H01Om}$. Тогда по теореме она предкомпактна в $\m{L2Om}$. Значит, и любая её подпоследовательность предкомпактна в $\m{L2Om}$. Из любой подпоследовательности можно извлечь подподпоследовательность, сходящуюся в сильном смысле в $\m{L2Om}$. В то же время эта подподпоследовательность слабо сходится в $\m{L2Om}$. Значит, сильный предел равен слабому пределу, который мы знаем.

\end{proof}
% вопрос №72 из курса функционального анализа
\begin{note} Вообще говоря, компактный оператор между банаховыми пространствами переводит слабо сходящуюся последовательность в сильно сходящуюся.
\end{note}

% 17. Теорема Фредгольма для компактных самосопряженных операторов в гильбертовом пространстве.
\subsection{Теорема Фредгольма}

% 18. Сведение задачи Дирихле для уравнения $−\Delta u + \lambda u = f$ к уравнению Фредгольма второго рода с самосопряженным, компактным, положительным оператором. Теорема об альтернативе.
\subsection{Сведение задачи Дирихле для уравнения Пуассона к уравнению Фредгольма второго рода}


% свойства оператора K = i o (-\Delta)^{-1}
% теорема Фредгольма о компактном операторе (альтернатива фредгольма)
% спектр интегрального оператора
% !TEX encoding = UTF-8 Unicode
% лекция 19, 16 апреля 2016, конец курса
% 19. Множество собственных чисел самосопряженных компактных положительных операторов.
% 20. Собственные векторы самосопряженных компактных положительных операторов и полные ортогональные системы векторов.
% 21. Собственные функции лапласиана и полные ортогональные системы векторов в $L^2$ и в $H_0^1$.
% 22. Собственные числа и собственные функции лапласиана. Первое собственное число.
% 23. Разрешимость уравнения $− \Delta u + \lambda u = f$.
% 24. Свойства первого собственнного числа лапласиана в $H_0^1$.

% 16.04.16 (последняя лекция)
% собственные числа лапласиана и уравнение - \Delta u = \lambda u + f

% определение собственных значений оператора
% у самосопряжённого оператора вещественный спектр
% ортогональность собственный элементов самосопряжённого оператора

% предложение: если оператор самосопряжён, то у него есть хотя бы одно собственное число
% упражнение: вспомнить пример оператора без собственных чисел с непустым спектром
% что-то про ортогональное дополнение

% теорема: собственные элементы лапласиана - полная ортогональная система в L^2


% !TEX encoding = UTF-8 Unicode
% обычно решение задачи опирается на решение или идею из предыдущих
\chapter{Упражнения}
Некоторые упражнения с последних лекций, а также сопутствующие леммы и теоремы.
\section*{23.04.16}
\begin{exercise}
Найти обобщенную производную функции $f$:
\begin{gather*}
f(x) =
	\begin{cases*}
		x+1, & $x < 0$ \\
		0, & $x = 0$ \\
		x-1, & $0 < x < 1$ \\
		(x-1)^2, & $x \geq 1$
	\end{cases*}
\end{gather*}
\end{exercise}

\begin{lemma}
Сходящиеся в смысле обобщённых функций ряды можно дифференцировать сколько угодно раз. Получающиеся ряды из производных сходятся к соответствующим обобщенным производным.
\end{lemma}
\begin{proof} Пусть $\Omega \subset \real^n$ --- область, семейство $\{u_k\} \subset \beauD'(\Omega)$, и
$$ \sum_{k=1}^\infty u_k = u \quad \text{в смысле }\beauD'(\Omega).$$ Докажем, что ряд из обобщённых производных сходится к соответствующей производной:
$$ \sum_{k=1}^\infty \pder[u_k]{x_i} = \pder[u]{x_i}.$$
Посчитаем:
\begin{align*}
\action{\varphi , \pder[u]{x_i}} &= - \action{\pder[\varphi]{x_i}, u} = - \action{\pder[\varphi]{x_i}, \sum_{k=1}^\infty u_k} = \lim_{n \to \infty} - \action{\pder[u]{x_i}, \sum_{k=1}^n u_k} \\ 
&= \lim_{ n \to \infty} - \sum_{k=1}^n \action{\pder[u]{x_i}, u_k} = \lim_{n \to \infty} \sum_{k=1}^n \action{\varphi, \pder[u_k]{x_i}} = \lim_{n \to \infty} \action{\varphi, \sum_{k=1}^n \pder[u_k]{x_i}}.
\end{align*}
Что и требовалось доказать.

\end{proof}

\begin{note} Пусть $\Omega = (0, 2\pi)$, функция $f$ из $C^1(\real)$ и $2\pi$-периодическая. Тогда её ряд Фурье по тригонометрическим функциям сходится поточечно:
$$ f(x) = A_0 + \sum_{k=1}^\infty A_k \sin kx + B_k \cos kx.$$
Найдём обобщённую производную:
$$ f'(x) = \sum_{k=1}^\infty k(A_k \cos kx - B_k \sin kx).$$
Ничто не гарантирует, что этот ряд сходится к $f'$ поточечно, зато он сходится к $f'$ в смысле $\m{L2}(\real)$ почти всюду. Найдём вторую производную:
$$ f''(x) = \sum_{k=1}^\infty k^2(- A_k \sin kx - B_k \cos kx).$$
В классическом смысле эта производная может не существовать ни в одной точке, а в обобщённом смысле всё замечательно, в $\beauD'(\real)$ ряд сходится.
\end{note}

\section*{30.04.16}
\begin{exercise}
Найти обобщенную производную функции $u$:
$$u(x) = \frac {1} {\abs{x}^{\alpha}}, \quad 0 < \alpha < n -1, \quad x \in \real^n$$
\end{exercise}

\begin{exercise}
Проверить формулу Лейбница для произведения обобщённой и пробной функций:
$$\pder[(u \varphi)]{x_i} = \pder[u]{x_i} \varphi + \pder[\varphi]{x_i} u.$$
\end{exercise}

% где-то тут 7.05.16
\begin{exercise}
Пусть $\Omega \subset \real^n$ - открытое, связное, $u \in L_{loc}^1(\Omega)$, а также $\nabla u = 0$ в обобщенном смысле. Доказать, что $u \equiv \const$.
\end{exercise}

\begin{exercise} Пусть $\varphi \in C_0^{\infty} (\real)$, а также
$$ \varphi (0) = 0, \quad \psi (x) = \frac {\varphi (x)} {x} .$$
Доказать, что $\psi \in C_0^{\infty} (\real)$.
\end{exercise}

\section*{14.05.16}

\begin{exercise} Доказать, что $H^1 (\real^n) = H_0^1 (\real^n)$.
\end{exercise}

% нужна теорема Рисса-Торина
\begin{exercise}[Неравенство Янга для свёртки]
Пусть $f \in L^p$, $g \in L^q$, и верно
$$\frac {1} {p} + \frac {1} {q} = 1 - \frac {1} {r}, \quad 1 \leq p,q,r \leq \infty, $$
тогда
$$f * g \in L^r, \quad || f*g ||_r \leq || f ||_p || g ||_q. $$
\end{exercise}

\section*{21.05.16}
\begin{exercise}
Пусть $\Omega \subset \real^n$ - выпуклое, $u \in H^1(\Omega)$. Доказать, что
$$\forall \delta > 0 \quad \exists u_{\delta} \in C^{\infty} (\overline{\Omega}):\quad ||u_{\delta} - u ||_{H^1(\Omega)} \leq \delta. $$
\end{exercise}

\begin{exercise}
Посчитать обобщённую производную композиции двух обобщённых функций.
\end{exercise}

\begin{exercise}
Пусть $\Omega \subset \real^n$ - ограниченная выпуклая область, $u \in H^1(\Omega)$. Доказать, что существует такая область $D$, в которую компактно вкладывается $\Omega$, что
$$\exists v \in H_0^1(\Omega): v\Big\rvert_{\Omega} = u\Big\rvert_{\Omega}, \quad || v ||_{H^1(D)} \leq C || u ||_{H^1(\Omega)}.$$
\end{exercise}

\begin{exercise}
Пусть $\Omega \subset \real^n$ - ограниченная выпуклая область. Доказать, что вложение $H^1(\Omega)$ в $L^2 (\Omega)$ компактно.
\end{exercise}

\begin{exercise}[Неравенство Пуанкаре]
Пусть $\Omega \subset \real^n$ - ограниченная выпуклая область. Докажите, что
$$ \intO (u - u_{\Omega})^2 dx \leq C \intO \abs{\nabla u}^2 dx, $$
где
$$ u_{\Omega} := \fint \limits_{\Omega} u dx, \quad C = C(\Omega,n).$$
\end{exercise}

\begin{exercise}
Пусть $\Omega \subset \real^n$ - ограниченная выпуклая область, $u \in H^1(\Omega)$ и $f \in L^2(\Omega)$. Рассмотрим функционал $F$:
$$ F(u) := \frac {1} {2} \intO \abs{\nabla u}^2 \, dx - \intO fu \, dx.$$
При каких ограничениях на $f$ функционал $F$ достигает минимума? 
\end{exercise}

\end{document}