% !TEX encoding = UTF-8 Unicode
\documentclass[12pt, a4paper]{report}
\usepackage{fontspec}
\usepackage{polyglossia}
\usepackage{underscore}
\setdefaultlanguage{russian}
\usepackage[cm]{fullpage}
\usepackage{amsthm}
\usepackage{amsmath}
\usepackage{mathtools}
\usepackage{dsfont}
\usepackage{esint}

\usepackage{pgfplots}
\pgfplotsset{compat=1.12} 


\setmainfont[Mapping=tex-text]{CMU Serif}

\newcommand{\real}{\mathds{R}}
\newcommand{\dsint}{\displaystyle \int}
\newcommand{\intO}{\int \limits_{\Omega}}
\newcommand{\eps}{\varepsilon}
\newcommand{\ssubset}{\subset\joinrel\subset}

\DeclareMathOperator{\supp}{supp}
\DeclarePairedDelimiter\abs{\lvert}{\rvert}

\newtheorem{theorem}{Теорема}
\newtheorem*{lemma}{Лемма}

\theoremstyle{definition}
\newtheorem{definition}{Определение}
\newtheorem*{example}{Пример}

\theoremstyle{remark}
\newtheorem*{note}{Замечание}


\begin{document}

\title{Конспект лекций по уравнениям математической физики}
\author{Лектор: Степанов Евгений Олегович}
\maketitle


% !TEX encoding = UTF-8 Unicode
% лекция 1, 13 февраля 2016
% Однородное линейное транспортное уравнение с постоянными коэффициентами, его физический смысл (задача о распространении пятна загрязнения в канале). Решение однородного уравнения методом характеристик. Бегущая волна.
% Неоднородное линейное транспортное уравнение
\chapter*{Введение}
Курс рассчитан на один семестр и делится на две части.

Первая - классическая, она относится скорее к 19ому веку. Будет рассмотрено несколько конкретных физических моделей, соответствующие уравнения в частных производных будут выведены и решены для некоторых частных случаев.

Вторая часть более современная, и, скорее всего, она окажется концепутально новой. По крайней мере для одной задачи будет получено общее решение, будет доказано существование решения и предложено сходящееся численное решение.

\pagebreak

\section*{Обозначения и некоторые определения}
\begin{enumerate}
\item $\real^n$ - евклидово пространство.
\item $\abs{x}$ - евклидова норма $x$ из $\real^n$.
\item Область - открытое множество в $\real^n$.
\item Если где-то написано $\bar{\Omega}$, то $\Omega$ - ограниченная область.
\item $C(\bar{\Omega})$ - пространство непрерывных функций на $\bar{\Omega}$ с $\infty$-нормой: $\sup \{ \abs{f(x)} : x \in \bar{\Omega} \}$.
\item $L^p (\Omega)$ - пространство с нормой Гёльдера, $1 \leq p \leq \infty$.
\item $L^\infty (\Omega)$ - пространство существенно ограниченных функций.
\item $C^k(\bar{\Omega})$ - все частные производные до порядка $k$ включительно существуют и непрерывны вплоть до границы.
\item $C(\Omega)$ - функции, непрерывные на $\Omega$, не вплоть до границы. Не является нормированным пространством.
\item Носитель функции - замыкание множества, на котором функция не равна $0$. Носитель функции $f$ обозначается $\supp f$.
\item Финитная функция - функция с компактным носителем.
\item $C^k_0(\Omega)$ - финитные $k$ раз гладкие функции.
\item $K \ssubset \Omega$ - K относительно компактно в $\Omega$.
\item $\displaystyle \fint \limits_{\Omega} f(x) dx = \frac{1}{|\Omega|} \int \limits_{\Omega} f(x) dx$, где $\abs{\Omega}$ - мера Лебега области $\Omega$.
% добавить про линейное уравнение (через дифференциальный оператор)
\end{enumerate}

\chapter{}
\section{Задача о распространении пятна загрязнения в канале}
Рассмотрим ситуацию: имеется водоём. В него было сброшено некоторое количество загрязняющего вещества.
Для простоты в качестве водоёма будем рассматривать реку или канал. Пренебрегаем тем, что происходит между берегами. Нас интересует, как загрязнение распространяется в длину, поэтому реку считаем одномерным объектом.

Обозначим через $c (t, x) $ концентрацию загрязняющего вещества в момент $t \in \real_+$ в точке $x \in \real$. Пусть дана функция $ c_0 (x) $, описывающая концентрацию вещества в начальный момент времени $ t = 0 $:
$$ c_0 (x) = c (0, x).$$
% функция линейна
Задача состоит в том, чтобы описать изменение концентрации загрязняющего вещества в реке при условии, что мы что-то знаем про течение реки. Как минимум мы знаем скорость. Будем считать, что загрязняющее вещество никуда не испаряется, тогда можно воспользоваться законом сохранения массы.

\subsection{Закон сохранения массы}

Рассмотрим изменение концентрации вещества на некотором малом интервале $ [x, x + \Delta x] $:
$$ \dsint \limits_x^{x + \Delta x} c (t, \xi) d \xi .$$
Рассмотрим функцию $ q (t, x) $ - поток загрязняющего вещества в момент времени $t$ через стенку $x$. Изменение концентрации можно описать как разность величин "сколько влилось" и "сколько вылилось": 
$$ \dsint \limits_x^{x + \Delta x} c (t, \xi) d \xi = q (t, x) - q (t, x + \Delta x). $$
Будем считать, что функция $q (t, x) $ достаточно гладкая. Внесём под интеграл $ \displaystyle \frac {d} {dt} $, а потом разделим обе части на длину нашего малого интервала $ \Delta x$:
$$ \fint \limits_x^{x + \Delta x} c_t (t, \xi) d \xi  = \frac {q  (t, x) - q (t, x + \Delta x)} {\Delta x}.$$
Далее, устремим длину интервала $\Delta x$ к нулю. Получим, собственно, закон сохранения массы:
$$ c_t (t, x) = -q_x (t, x). $$

Стоит заметить, что предположения о гладкости и других нужных свойствах используемых функций это распространённый приём, используемый физиками для построения моделей. Эти условия достаточно сильны, и не всегда физическое явление можно описать достаточно хорошей функцией. В дальнейшем мы узнаем, что существуют более общие математические модели, которые лучше подходят для описания физических процессов.


Далее хочется узнать, как этот поток зависит от концентрации.

\subsection{Модели}
\subsubsection*{Чистая конвекция или чистый дрейф (транспортное уравнение)}

Допустим, скорость течения $v$ постоянна, а загрязняющее вещество мало диффундирует. То есть, загрязняющее вещество не смешивается с водой (в качестве примера можно привести сброс нефти в реку). Тогда поток вещества, проходящего через точку $x$ равен концентрации вещества умножить на скорость:
$$ q (t, x) = v c (t, x). $$
Если подставить вместо потока закон сохранения массы, то получим транспортное уравнение:
$$ c_t + v c_x = 0.$$

Полностью это уравнение называется однородным линейным транспортным уравнением с постоянными коэффициентами в одномерном случае.

\subsubsection*{Уравнение диффузии (закон Фика) или теплопроводности (закон Фурье)}

Допустим, течения нет (то есть, $ v \equiv 0 $), а загрязняющее вещество хорошо диффундирует (в качестве примера можно привести сброс стирального порошка в стоячий канал). Тогда
$$ q (t, x) = - D c_x (t, x), $$
где $D$ это некоторый коэффициент диффузии.

Это уравнение означает, что в момент времени $t$ поток через стенку $x$ пропорционален градиенту концентрации. Минус в правой части означает, что поток идёт оттуда, где концентрация больше, туда, где она меньше.

Позже мы узнаем, что эта же модель является моделью распространения тепла. В ней величина $1/D$ называется коэффициентом температуропроводности.

\subsubsection*{Уравнение конвекции-диффузии или конвекции с дрейфом (закон Фоккера-Планка)}
% на лекции больше ничего не было
% кроме того, что оно используется в computer science при моделировании (?) стохастических процессов
% но, опять же, это теорвер 
Эта модель описывает случай, когда присутствует и конвекция, и диффузия:
$$ q (t, x)  = x c - D c_x. $$
Подставив закон сохранения массы, получим:
$$ c_t + v_{cx} = D c_{xx}. $$

Помимо физики, такое уравнение часто встречается в теории вероятностей.

\subsection{Решение однородного транспортного уравнения}
Что мы имеем ввиду, когда говорим "решение"? Мы имеем ввиду классическое решение - функцию, которая при подстановке в уравнение даст верное соотношение. Для этого она должна обладать нужной гладкостью, а условия уравнения должны соблюдаться в каждой точке. Так понимали решения до тридцатых годов XX века. К сожалению, оказалось, что для дифференциальных уравнений в частных производных классическое решение - не самое лучшее, и зачастую его недостаточно.

За последний век было придумано много разных решений, обобщающих понятие классического: слабые, вязкостные, энтропийные. Немного позже мы узнаем об одном из них - слабом. Но пока что мы остановимся на уровне XIX века и будем рассматривать только классические решения.

Итак, нам нужно найти некоторую функцию, которая не только удовлетворяет транспортному уравнению, но еще и удовлетворяет начальному условию $ c (0, x) = c_0 (x) $:

\begin{align}
    \begin{cases} 
        c_t + v c_x = 0, \\
        c (0, x) = c_0 (x).
    \end{cases}
\label{transport}
\end{align}

Когда мы выводили это уравнение, мы брали небольшой промежуток реки от $x$ до $ x + \Delta x $ и смотрели, как распространяется загрязняющее вещество в нём. Теперь поступим по-другому: посмотрим, как ведёт себя каждая частица вещества в реке. Каждая частица плывёт по направлению течения со скоростью $v$. Значит, траектория каждой частицы удовлетворяет вспомогательному дифференциальному уравнению $ \dot x = v $. Каково решение этого уравнения?

\[
	x(0) = x_0,\quad \text{тогда } x(t) = x_0 + v t .
\]


Пусть $ c (t, x) $ - решение нашего транспортного уравнения. Подставим $ x(t) $ вместо $x$. То есть, каждая частица вещества двигается по закону $ \dot x = v $. Что получится, если мы посмотрим концентрацию вещества вдоль траектории этого ОДУ?

\begin{align*}
    \frac {d} {dt} c (t, x(t)) & = c_t (t, x(t)) + c_x (x, t) \cdot \dot x (t) = c_t (t, x(t)) + v c_x (t, x(t)) = 0
\end{align*}
%?????
Получается, что $ c_t (t, x(t)) = 0 $. Иначе говоря, вдоль траектории вспомогательного ОДУ функция $c$ является постоянной.




Как теперь найти $ c (t,x) $? Есть точка $ (t, x) $, требуется найти значение концентрации в ней. Смотрим, какая траектория ОДУ проходит через эту точку: через неё проходит единственная прямая. Знаем, что вдоль этой траектории $ c = const $, значит, $ c (t, x) = c_0 (x_0) $. А как выражается $ x_0 $? $ x = x_0 + v t $, значит, $ x_0 = x - vt $.
Итого $c(t,x) = c_0(x(0)) = c_0(x-vt)$.

\begin{definition} Обыкновенное дифференциальное уравнение, вдоль траекторий которого решения уравнения в частных производных постоянны, называется характеристическим.
\end{definition}

\begin{definition} Траектории характеристического уравнения называются характеристическими линиями или характеристиками.
\end{definition}

Наша задача оказалась устроена так, что через каждую точку проходит единственная характеристическая линия. Отсюда, зная начальное значение, мы нашли формулу для решения задачи Коши для транспортного уравнения.

Оформим результат рассуждений в виде теоремы.

\begin{theorem}
Пусть $c_0 \in C^1(t)$. Тогда задача \eqref{transport} имеет единственное классическое решение $$ c (t,x) = c_0 (x - v t) .$$
\end{theorem}

\begin{tikzpicture}
\begin{axis}[
    axis lines = left,
    xlabel = $x$,
    xmin = -7,
    xmax = +14,
    ymax = 1.5,
]

\addplot [
    domain=-7:-3, 
    samples=250, 
    color=red,
]
{1/(1+e^(-10*(x+5)))};

\addplot [
    domain=-7:6, 
    samples=250, 
    color=blue,
]
{1/(1+e^(-10*(x-4)))};

\addlegendentry{$c(-3,x)$}
\addlegendentry{$c(7,x)$}

\addplot [
    domain=-3:-1, 
    samples=5, 
    color=red,
]
{1};

\addplot [
    domain=-1:14, 
    samples=250, 
    color=red,
]
{1/(1+e^(-10*(-x+1)))};

\addplot [
    domain=6:8, 
    samples=5, 
    color=blue,
]
{1};

\addplot [
    domain=8:14, 
    samples=250, 
    color=blue,
]
{1/(1+e^(-10*(-x+10)))};
 
\end{axis}
\end{tikzpicture}


Как выглядит наше решение? Это видно на графиках. При увеличении $t$ профиль нашего загрязнения будет просто смещаться по течению на $vt$. То есть, начальное возмущение, не меняя формы, распространяется со скоростью $v$ - получили бегущую волну.

Тем не менее, для того, чтобы функция считалась классическим решением, она должна принадлежать $C^1$, что, вообще говоря, довольно странно, ведь в реальной задаче функция, описывающая контур профиля загрязнения, вполне может быть не из $C^1$ (например, если профиль - прямоугольник). Получается, что понятие классического решения может оказаться неподходящим и надо каким-то образом ослаблять требования.

\subsection{Неоднородное транспортное уравнение и его решение}

Немного изменим модель. Допустим, сброс был не единовременным, и имеется некий источник загрязняющего вещества с заданной мощностью $f(t,x)$. % мощностью?

Снова рассмотрим изменение концентрации вещества на некотором малом интервале:
$$ \dsint \limits_x^{x + \Delta x} c (t, \xi) d \xi .$$
И снова $ q (t, x) $ - поток загрязняющего вещества в момент времени $t$ через стенку $x$. Тогда, учитывая источник вещества $f$:
$$ \frac{d}{dt} \int \limits_x^{x + \Delta x} c(t, \xi ) d \xi = q(t, x) - q(t, x + \Delta x) + \int \limits_x^{x+\Delta x} f(t,\xi) d\xi. $$
Предполагая достаточную гладкость, вносим $ \displaystyle \frac {d} {dt} $ под интеграл и делим на $ \Delta x $:
$$ \fint \limits_x^{x + \Delta x} c_t (t, \xi) d \xi  = \frac {q  (t, x) - q (t, x + \Delta x)}  {\Delta x} + \fint \limits_x^{x+\Delta x} f(t,\xi) d\xi. $$
Устремляя $ \Delta x$ к нулю, в случае чистой конвекции ($ q = vc $) получаем линейное неоднородное транспортное уравнение:

$$ c_t (t, x) = - vc_x (t, x) + f(t, x). $$

Поставим задачу Коши:

\begin{align}
    \begin{cases} 
        c_t + v c_x = f, \\
        c (0, x) = c_0 (x).
    \end{cases}
\label{transportnonhom}
\end{align}

Посмотрим, как будет вести себя классическое решение на траекториях $ \dot x = v $.
\[
	x(0) = x_0,\quad \text{тогда } x(t) = x_0 + v t .
\]
Подставим $x(t)$ в наше неоднородное уравнение:
$$ c (t, x_0 + vt) = c_0 (x_0) + \int \limits_0^t f(s, x_0 + vs) ds. $$
Заметим, что $ x_0 = x - vt $, тогда
$$ c (t, x) = c_0 (x - vt) + \int \limits_0^t f(s, x + v(s-t)) ds. $$

Оформим результат  в виде теоремы.

\begin{theorem}
Пусть $c_0 \in C^1(\real)$, $f \in C (\real^+ \times \real)$, $f_x \in C (\real^+ \times \real)$. Тогда задача \eqref{transportnonhom} имеет единственное классическое решение $$ c (t, x) = c_0 (x - vt) + \int \limits_0^t f(s, x + v(s-t)) ds .$$
\end{theorem}

Вообще говоря, гладкость $f$ по пространственной переменной - совершенно нефизичное условие, в отличие от непрерывности. Позже мы узнаем про слабые решения, где вместо функций могут фигурировать, например, меры.

В данном случае у нас была чистая конвекция и скорость ни от чего не зависела. Но что получится, если скорость зависит от $x$ и от $t$? Тогда ОДУ будет таким же, но его траектории не обязательно будут прямыми. Существенным условие тут является единственность решения характеристического уравнения. Она нарушается, если $v$ зависит от $x$ не гладким образом, а, скажем, просто непрерывным. Тогда может оказаться, что через одну точку проходит несколько характеристических линий (классический пример - квадратный корень).

А что произойдёт, если у нас не конвекция, а чистая диффузия? Можно ли придумать для $c_t = -Dc_{xx}$ характеристическое уравнение, вдоль траекторий которого решение будет константой? Вообще говоря, можно, но это будет не обыкновенное уравнение, а стохастическое. Для таких уравнений мы не можем найти детерминированные траектории, вдоль которых концентрация не меняется. Но можно "рвзыгрывать" траектории с определенной вероятностью по некоторому закону так, что в среднем для мноих траекторий концентрация постоянна. Таким образом, уравнение диффузии связано со стохастикой.
% !TEX encoding = UTF-8 Unicode
% лекции 3-4, 20 февраля 2016
% вопросы 6-8
% 6. Задача Коши для одномерного волнового уравнения (колебания бесконечной струны). Вывод формулы Даламбера при помощи транспортного уравнения. Свойства решения. Волновые конусы. Конечная скорость распространения волн. Передний и задний фронт волны.
% 7-8. Канонический вид уравнений в частных производных второго порядка, их классификация.

\subsection{Задача Коши для одномерного волнового уравнения}
Будем считать, что струна настолько длинная, что краевыми эффектами того, что струна зажата на концах, можно пренебречь. То есть, что струна просто колеблется в соответствии с нашим волновым уравнением. Волновое уравнение --- один из тех редких случаев, когда можно получить общую формулу решения.
Решение выведем двумя способами.
\subsubsection{Вывод решения при помощи транспортного уравнения}
\begin{gather*}
	u_{tt} - v^2 u_{xx} = (\pder{t} - v \pder{x}) \underbrace {(\pder[u]{t} + v \pder[u]{x})}_{= w} = 0, \\
 	\pder[w]{t} - v \pder[w]{x} = 0.
\end{gather*}

Получили однородное транспортное уравнение, его решение мы знаем:
\begin{gather*}
	w(t,x) = \psi (x + vt),\quad \psi (x) = w(0,x) \\
	\pder[u]{t} + v \pder[u]{x} = \psi (x + vt)		
\end{gather*}

Получили неоднородное транспортное уравнение, его решение мы тоже знаем:
\begin{gather*}
	u(t,x) = u_0 (x - vt) + \int \limits_0^t \psi(x + vs - v(t-s)) ds = u_0 (x - vt) + \int \limits_0^t \psi (x - vt + 2vs) ds, \\
	\text{сделаем замену }y = x - vt + 2vs, \text{ тогда} \\
\end{gather*}
\begin{equation}
    u(t,x) = u_0 (x - vt) + \frac {1} {2v} \int \limits_{x-vt}^{x+vt} \psi (y) dy
\label{wavehomans}
\end{equation}

\begin{exercise}
При каких $\varphi$ и $\psi$ $\eqref{wavehomans}$ --- решение $\eqref{waveequation}$? (Ответ: $\varphi \in C^2$, $\psi \in C^1$)
\end{exercise}

\begin{theorem} Пусть $\varphi \in C^2(\real)$ и $\psi \in C^1(\real)$. Тогда
$$ u(t, x) = \varphi (x - vt) + \frac {1} {2v} \int \limits_{x - vt}^{x+vt} \psi (y) dy$$
--- решение одномерного волнового уравнения.
\end{theorem}

Поставим задачу Коши для волнового уравнения:

\begin{align}
    \begin{cases} 
        u_{tt} - v^2 u_{xx} = 0, \\
        u (0, x) = u_0 (x), \\
        u_t(0,x) = v_0 (x).
    \end{cases}
\label{wavecauchy}
\end{align}

Найдем решение задачи Коши. Подставляя $(0,x)$ в общее решение, получаем $$ u_0(x) = \varphi(x) .$$ Найдём $v_0(x)$:
\begin{gather*}
	u_t(0,x) = v_0(x) = \frac {1} {2v} (v \psi (x + vt) + v \psi (x - vt)) \Bigg\rvert_{t = 0} - v \varphi' (x)
	= \psi(x) - v \varphi'(x), \\
	\psi(x) = v_0(x) + v u'_0(x).
\end{gather*}
Таким образом,
\begin{align*}
	u(t,x) &= u_0(x-vt) + \frac {1} {2v} \int \limits_{x-vt}^{x+vt} v_0(y) + vu'_0(y)dy \\
	&= u_0(x-vt) + \frac {u_0(x+vt) - u_0(x-vt)} { 2} + \frac {1} {2v} \int \limits_{x-vt}^{x+vt} v_0(y)dy.
\end{align*}
Отсюда легко получаем формулу Д'Аламбера
\begin{equation}
	u(t,x) = \frac {u_0(x+vt) + u_0(x-vt)} { 2} + \frac {1} {2v} \int \limits_{x-vt}^{x+vt} v_0(y)dy.
\label{wavedalembert}
\end{equation}

Сформулируем теорему:
\begin{theorem} Пусть $u_0 \in C^2(\real)$ и $v_0 \in C^1(\real)$, тогда формула Д'Аламбера $\eqref{wavedalembert}$ --- единственное классическое решение задачи Коши для уравнения колебания бесконечной струны.
\end{theorem}

\begin{example}
Пусть $t$ --- время, $x$ --- координата вдоль струны. Предположим, на струне в точке $x_0$ сидит муравей. (Мировая линия объекта - траектория его движения в координатах пространства и времени) Пусть для простоты он не двигается. $x(t)$ - мировая линия, $v_0 = 0$. $x + vt = const$ и $x -vt = const$ - характеристические функции.

%TODO: рисунок про конусы
\includegraphics[scale=0.5]{part2.1.png}

Допустим, кто-то в точке [??] ударил по струне, для простоты пусть этот кто-то её отклонил, но не придал никакой начальной скорости ($v_0 = 0$, $u_0 \neq 0$). Когда муравей узнает о том, что что-то произошло? В момент времени $ t = \frac {l} {v}$, где $l$ - расстояние от муравья до источника возмущения, $v$ - скорость распространения возмущения. Будем считать, что профиль возмущения локализован около точки, тогда профиль распространяется вперёд и назад: бегут две волны. Если бы щелчок по струне был локализован в одной точке (тогда это не было бы классическим решением!), то муравей почуствовал бы фронт волны только на мгновение.
Интегрируем по содержимому конуса прошлого. Задав скорость муравью, можно повлиять на содержимое конуса будущего.
\end{example}

{\small А что будет в трёхмерном случае? Распространение сферических волн. В отличие от формулы Д'Аламбера, не будет разницы между начальным возмущением с некоторой начальной скоростью и без неё. То есть, сферические фронты будут доходить от каждой точки волны. Сначала дойдёт передний фронт, потом задний.

А в двумерном? Распространение цилиндрических волн. Пример - берём удочку с поплавком, идём на пруд без течения. Забрасываем удочку, бросаем далеко от поплавка камушек. Получится примерно точечное возмущение. В какой-то момент передний фронт дойдёт до поплавка и он начнёт дёргаться. Волна пройдёт, а поплавок продолжит свои колебания, теоретически - бесконечно долго. То есть, заднего фронта не будет.}

% Ч Т О ?
TODO: написать нормально про конусы

\subsubsection{Вывод решения при помощи замены переменных}
$$u_{tt} - v^2 u_{xx} = 0$$
Сделаем замену $(t, x) \to (\xi, \eta)$:
$$ \xi = x + vt, \quad \eta = x - vt.$$
\begin{align*}
	u_x &= u_{\xi} \xi_x + u_{\eta} \eta_x = u_{\xi} + u_{\eta}, \\
	u_t &=  u_{\xi} \xi_t + u_{\eta} \eta_t = v u_{\xi} + v u_{\eta}, \\
	u_{xx} &= u_{x\xi} \xi_x + u_{x\eta} \eta_x = u_{\xi\xi} + u_{\eta\xi} + u_{\xi\eta} + u_{\eta\eta}  =  u_{\xi\xi} + 2 u_{\xi\eta} + u_{\eta\eta}, \\
	u_{tt} &= u_{t\xi} \xi_t + u_{t\eta} \eta_t = v^2 (u_{\xi\xi} - u_{\eta\xi}) - v^2 (u_{\xi\eta} - u_{\eta\eta}) = v^2 (u_{\xi\xi} - 2u_{\xi\eta} + u_{\eta\eta}). \\
	\square_v u &= v^2 u_{\xi\xi} - 2v^2 u_{\xi\eta} + v^2 u_{\eta\eta} - v^2 u_{\xi\xi} - 2v^2 u_{\xi\eta} - v^2 u_{\eta\eta} = 0, \\
	u_{\xi\eta} &= 0, \quad u_{\xi} = F(\xi), \\
	u &= \int F(\xi) d\xi = \varphi (\xi) + \psi (\eta). \\
\end{align*}
Таким образом,
$$ u = \underbrace {\varphi (x+vt)}_{\text{волна направо}} + \underbrace {\psi (x-vt)}_{\text{волна налево}}.$$

Решим задачу Коши $\eqref{wavecauchy}$:
\begin{gather*}
	\begin{cases}
		u(0,x) = u_0(x) = \varphi(x) + \psi(x), \\
		u_t(0,x) = v_0(x) = v \varphi'(x) - v \psi(x),
	\end{cases}
	\begin{cases}
		\psi(x) = u_0(x) - \varphi(x), \\
		\varphi'(x)  - vu'_0(x) + v \varphi'(x) = v_0(x).
	\end{cases} \\
	\varphi'(x) = \frac {1} {2v} (v_0(x) + vu'_0(x)), \quad 	\varphi(x) = \frac {1} {2v} \int \limits_0^x v_0(y) + vu'_0(x) dy + C, \\
	\psi(x) = u_0 - \frac {1} {2v} \int \limits_0^x v_0(y) + vu'_0(y) dy - C,
\end{gather*}
Тогда
\begin{align*}
	u(t,x) &= \frac {1} {2v} \int \limits_0^{x+vt} v_0(y) + vu'_0(y) + C + u_0(x-vt) -  \frac {1} {2v} \int \limits_0^{x-vt} v_0(y) + vu'_0(y) dy - C \\
		&= u_0(x-vt) + \frac {1} {2v} \int \limits_{x-vt}^{x+vt} v_0(y) + v u'_0(y) dy.
\end{align*}
Отсюда легко получается та же самая формула Д'Аламбера \eqref{wavedalembert}:
\begin{equation*}
	u(t,x) = \frac {u_0(x+vt) + u_0(x-vt)} {2} + \frac {1} {2v} \int \limits_{x-vt}^{x+vt} v_0(y)dy.
\end{equation*}

% !TEX encoding = UTF-8 Unicode
% лекции 3-4, 20 февраля 2016
% вопросы 6-8
% 6. Задача Коши для одномерного волнового уравнения (колебания бесконечной струны). Вывод формулы Даламбера при помощи транспортного уравнения. Свойства решения. Волновые конусы. Конечная скорость распространения волн. Передний и задний фронт волны.
% 7-8. Канонический вид уравнений в частных производных второго порядка, их классификация.

\subsection{Приведение уравнений второго порядка к каноническому виду в случае двух независимых переменных}

Ранее мы получили решение одномерного волнового уравнения путём замены переменных. Нет ли способа, позволяющего получить замену переменных, приводящую уравнение к некоторому "красивому" виду в общем случае?

Рассмотрим линейное дифференциальное уравнение в частных производных второго порядка с нелинейными коэффициентами:
$$ a u_{xx} + 2bu_{xy} + c u_{yy} + d_1 u_x + d_2 u_y + d_3 u = f.$$
Будет приятно иметь локально обратимое преобразование координат (якобиан преобразования $\neq$ 0). Рассмотрим гладкую замену 
$$(x,y) \rightarrow (\xi, \eta).$$
Применяя правило дифференцирования сложной функции, пересчитаем производные в новых координатах:
\begin{align*}
	u_x &= u_\xi \xi_x + u_\eta \eta_x, \\
	u_y &= u_\xi \xi_y + u_\eta \eta_y, \\
	u_{xx} &= u_{\xi x} \xi_x + u_{\eta x} \eta_x + u_\xi \xi_{xx} + u_\eta \eta_{xx} = u_{\xi \xi} \xi^2_x + 2u_{\xi \eta} \xi_x \eta_x + u_{\eta \eta} \eta^2_x + u_\xi \xi_{xx} + u_\eta \eta_{xx}, \\
	u_{yy} &= u_{\xi y} \xi_y + u_{\eta y} \eta_y + u_\xi \xi_{yy} + u_\eta \eta_{yy} = u_{\xi \xi} \xi^2_y + 2u_{\xi \eta} \xi_y \eta_y + u_{\eta \eta} \eta^2_y + u_\xi \xi_{yy} + u_\eta \eta_{yy}, \\
	u_{xy} &= u_{\xi \xi} \xi_x \xi_y + u_{\xi \eta} (\xi_x \eta_y + \xi_y \eta_x) + u_{\eta \eta} \eta_x \eta_y + u_\xi \xi_{xy} + u_\eta \eta_{xy}.
\end{align*}
Подставим в наше уравнение:
\begin{align*}
	u_{\xi \xi} & (a \xi^2_x + 4b \xi_x \xi_y + c \xi^2_y) + \\
	u_{\eta \eta} & (a \eta^2_x + 4b \eta_x \eta_y + c \eta^2_y) +\\
	2 u_{\xi \eta} & (a \xi_x \eta_x + 4b \xi_x \eta_y + c \xi_y \eta_y) + \\
	u_\xi & (a \xi_{xx} + 2b \xi_{xy} + c \xi_{yy} + d_1\xi_x + d_2 \xi_y) + \\
	u_\eta & (a \eta_{xx} + 2b \eta_{xy} + c \eta_{yy} + d_1 \eta_x + d_2 \eta_y) = f.
\end{align*}

Получили
$$\underbrace {A u_{\xi \xi} + 2B u_{\xi \eta} + C u_{\eta \eta}}_{\text{новая главная часть}} + \text{ч.н.п.} = f,$$
где \begin{align*}
	A &= a \xi^2_x + 4b \xi_x \xi_y + c \xi^2_y, \\
	B &= a \xi_x \eta_x + 4b \xi_x \eta_y + c \xi_y \eta_y, \\
	C &= a \eta^2_x + 4b \eta_x \eta_y + c \eta^2_y.
\end{align*}

Замечаем, что $A$ и $C$ имеют очень похожую структуру. Нельзя ли оставить в главной части только смешанную производную? Посмотрим на систему
\begin{align*}
	\begin{cases*}
	a \xi^2_x + 4b \xi_x \xi_y + c \xi^2_y = 0, \\
	a \eta^2_x + 4b \eta_x \eta_y + c \eta^2_y = 0.
	\end{cases*}
\end{align*}
Если решение этой системы даст нам локально обратимое преобразование координат, то цель достигнута. Второе уравнение это первое (с точностью до замены переменной) так что можно рассматривать только его. Это квадратичная форма, значит,
$$ a (\xi_x - \Lambda^+ \xi_y)(\xi_x + \Lambda^- \xi_y) = a \xi^2_x - a(\Lambda^+ + \Lambda^-)\xi_x \xi_y + a \Lambda^+ \Lambda^- \xi_y^2 =  0.$$
Значит,
\begin{align*}
	\begin{cases*}
		\Lambda^+ + \Lambda^- = - \frac {2b} {a}, \\
		\Lambda^+ \Lambda^- = \frac {c} {a}.
	\end{cases*}
\end{align*}
Можно сказать, что $\Lambda^+$ и $\Lambda^-$ это корни уравнения
$$a \Lambda^2 + 2b \Lambda + c = 0,$$
откуда можно заключить, что
$$ \Lambda^{\pm} = - \frac {b} {a} \pm \sqrt{\frac {b^2} {a^2} - \frac {c} {a}} = \frac {1} {a} (-b \pm \sqrt{b^2 - ac}).$$
Обозначим $$M = \begin{pmatrix} a & b \\ b & c \\\end{pmatrix}.$$
Тогда 
\begin{enumerate}
	\item если $\det M =  ac - b^2 < 0$, то уравнение называется эллиптическим и можно вывести два транспортных уравнения:
	\begin{gather*}
		\xi_x - \frac {1} {a} (-b + \sqrt{b^2 - ac}) \xi_y = 0, \\
		\eta_x - \frac {1} {a} (-b - \sqrt{b^2 - ac}) \eta_y = 0.
	\end{gather*}
	Рассмотрим дифференциальное уравнение:  $$\xi_x = \Lambda^+ \xi_y \quad \Rightarrow \quad y_x = - \Lambda^+ (x,y)\text{ - характеристика},$$
	Пусть $\xi$ - какое-то решение, рассмотрим его на траекториях
	$$ \frac {d \xi (x, y(x))} {dx} = \xi_x + \xi_y \frac {dy} {dx} = \xi_x - \Lambda^+ \xi_y = 0,$$
	тогда решение полученного транспортного уравнения должно быть постоянным на характеристиках этого ОДУ:
	$$ \frac {d \xi (x, y(x))} {dx} = \const \quad \text{при } \frac {dy} {dx} = - \Lambda^-.$$
	% как делаете: решение определяется по этим формулам, лямбда плюс и минус - по коэффициентам. если определитель больше нуля, то есть два вещественных разных решения - лямбда плюс и лямбда минус. решаем дифференциальное уравнение с лямбда плюс и определяем функцию кси из того условия, что кси постоянна на траекториях этого диффура. если лямбда плюс - гладкая функция, то через каждую точку проходит ровно одна характеристическая линия, значит, явным образом можно определить кси. то же самое с эта: решаем уравнение dy/dx = -лямбда^-, это даёт решение второго транспортного уравнения. эти решения и будут нашей заменой переменных. тем самым гарантируется, что останется только коэффициент при u_{xi eta}. то есть для гиперболического уравнения всегда можно найти замену
	
	\item если $\det M = ac - b^2 = 0$, то уравнение называется параболическим и $$\Lambda = - \frac {b} {a} \quad \Rightarrow \quad \frac {dy} {dx} = \frac {b} {a}.$$
	Отсюда можно найти только $\xi$, в этом случае $\eta$ выбирается такая, чтобы якобиан замены не был равен нулю. 
	\begin{exercise} Показать, что в этом случае $A = 0$ и $u_{\xi \eta} = 0$.
	% A =0 ?
	\end{exercise}
	
	\item если $\det M = ac - b^2 > 0$, то уравнение называется эллиптическим и $u_{\xi \eta} =0$, $A = C \neq 0$. Корни будут комплексно сопряжёнными, а замена будет $$\xi = Re ...,\quad \eta = Im ... .$$
\end{enumerate}
	
\subsubsection{Классификация в случае двух независимых переменных}
Рассмотрим дифференциальное уравнение в частных производных второго порядка с переменными коэффициентами:
$$ \underbrace {\sum \limits_{i = 1}^2 \sum \limits_{j = 1}^2 a_{ij}(x_1, x_2) \frac {\partial^2 u} {\partial x_i \partial x_j}}_{\text{главная часть}}  + \underbrace {F(\pder[u]{x_1}, \pder[u]{x_2}, u, x_1, x_2)}_{\text{члены низшего порядка}} = 0,\quad x \in \real^2$$

Обозначим через $A$ симметричную матрицу коэффициентов $a_{ij}$.
\begin{itemize}
\item Если $\det A < 0$, то уравнение называется гиперболическим, и в некоторых координатах $(\xi, \eta)$ уравнение имеет канонический вид:
$$\frac {\partial^2 u} {\partial \xi \partial \eta} + \text{члены низшего порядка} = 0.$$
\item Если $\det A = 0$, то уравнение называется параболическим, и в некоторых координатах $(\xi, \eta)$ уравнение имеет канонический вид:
$$ \frac {\partial^2 u} {\partial \eta^2} + \text{члены низшего порядка} = 0.$$
\item Если $\det A > 0$, то уравнение называется эллиптическим, и в некоторых координатах $(\xi, \eta)$ уравнение имеет канонический вид:
$$ \frac {\partial^2 u} {\partial \xi^2} + \frac {\partial^2 u} {\partial \eta^2} + \text{члены низшего порядка} = 0.$$
\end{itemize}

В случае переменных коэффициентов $a_{ij}$ уравнение может иметь разный тип в разных областях, такие уравнения называются уравнениями переменного типа.

\subsection{Классификация уравнений второго порядка в случае многих независимых переменных}
Рассмотрим дифференциальное уравнение в частных производных второго порядка с переменными коэффициентами:
$$ \underbrace {\sum \limits_{i=1}^n \sum \limits_{j = 1}^n a_{ij} \frac {\partial^2 u} {\partial x_i \partial x_j}}_{\text{главная часть}}  + F(\pder[u]{x_1}, \pder[u]{x_2}, u, x_1, ..., x_n) = 0,\quad x \in \real^n$$

Обозначим через $A$ симметричную матрицу коэффициентов $a_{ij}$. Уравнение называется 
\begin{itemize}
\item эллиптическим, если у матрицы $A$ все собственные числа одного знака;
\item гиперболическим, если матрица $A$ имеет собственные числа разных знаков;
\item параболическим, если у матрицы $A$ есть собственное число, равное нулю.
\end{itemize}

В общем случае нужно занулить $(n^2 - n) / 2$ коэффициентов $a_{ij}$, чтобы получить диагональную матрицу. Неизвестных при этом $n$ штук. Так как
$$\frac {n^2 - n} {2} \leq n \quad \Longleftrightarrow \quad n \leq 3,$$
то при $n \leq 3$ в общем случае можно найти замену, приводящую уравнение к некоторому каноническому виду. При $n > 4$ общего способа нет.

\subsection{Примеры уравнений различных типов}

\begin{example}[Уравнение Пуассона]
$$ - \Delta u = f, \quad x \in \real^n$$
$$\begin{pmatrix}
-1 & \\
   & \ddots \\
   & & -1
\end{pmatrix}\quad \Rightarrow \quad \text{уравнение эллиптическое}.$$
\end{example}

\begin{example}[Волновое уравнение]
$$u_{tt} - v^2 \Delta u = f, \quad (t,x) \in \real^{n+1}$$
$$\begin{pmatrix}
1 & \\
   & -v^2 \\
   & & \ddots \\
   & & & -v^2
\end{pmatrix}\quad \Rightarrow \quad \text{уравнение гиперболическое}.$$
\end{example}

\begin{example}[Уравнение теплопроводности]
$$ u_t - D \Delta u = f, \quad (t,x) \in \real^{n+1}$$
$$\begin{pmatrix}
0 & \\
   & -D \\
   & & \ddots \\
   & & & -D
\end{pmatrix}\quad \Rightarrow \quad \text{уравнение параболическое}.$$
\end{example}

\end{document}