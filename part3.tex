% !TEX encoding = UTF-8 Unicode
% лекции 5-6, 27 февраля 2016
% вопросы 9-12
% 

\subsection{Решение задачи Коши для одномерного неоднородного волнового уравнения. Метод Дюамеля}
Запишем одномерное неоднородное волновое уравнение.
\begin{equation*}
	u_{tt} - v^2 u_{xx} = f(t,x).
%\label{waveequationnonhom}
\end{equation*}
Здесь $f(t,x)$ - постоянное возмущение струны. Поставим задачу Коши:
\begin{equation}
	\begin{cases}
		u_{tt} - v^2 u_{xx} = f(t,x), \\
		u(0,x) = u_0(x), \\
		u_t(0,x) = v_0(x).
	\end{cases}
\label{wavenonhomcauchy}
\end{equation}
По аналогии с неоднородными линейными ОДУ:
\begin{note} Достаточно решить соответствующую неоднородную задачу с однородными начальными условиями:
\begin{equation}
	\begin{cases}
		\overline{u}_{tt} - v^2 \overline{u}_{xx} = f(t,x), \\
		\overline{u}(0,x) = 0, \\
		\overline{u}_t(0,x) = 0.
	\end{cases}
\label{wavehomcauchy}
\end{equation}
\end{note}
\begin{proof}
Пусть $u = \widetilde{u} + \overline{u}$, где $\widetilde{u}$ - решение соответствующей однородной задачи с неоднородными начальными условиями:
\begin{equation*}
	\begin{cases}
		\widetilde{u}_{tt} - v^2 \widetilde{u}_{xx} = 0, \\
		\widetilde{u}(0,x) = u_0(x), \\
		\widetilde{u}_t(0,x) = v_0(x).
	\end{cases}
\end{equation*}
Тогда простой подставкой нетрудно проверить, что $u$ - решение задачи \eqref{wavenonhomcauchy}.

\end{proof}
В решении задачи $\eqref{wavehomcauchy}$ нам поможет принцип Дюамеля.
\begin{theorem}[Дюамель] Пусть поставлена задача Коши
\begin{equation}
	\begin{cases}
		w_{tt} - v^2 w_{xx} = 0,\quad s \geq 0 \\
		w(s,x) = 0, \\
		w_t(s,x) = f(s,x),
	\end{cases}
\label{waveduhamel}
\end{equation}
тогда решением задачи $\eqref{wavehomcauchy}$ будет\footnote{Здесь важно то, что для каждого $s$ будет своя $w(t,x) = w(t,x,s)$.}
$$ \overline{u}(t,x) = \int \limits_0^t w(t,x,s) ds.$$
\end{theorem}
\begin{proof}
Для начала проверим выполнение начальных условий:
$$\overline{u} (0,x) = \int \limits_0^0 ... = 0, \quad \overline{u}_t (0,x) = \underbrace {w(t,x,t) \Bigg\rvert_{t=0}}_{= 0} + \int \limits_0^t w_t(t,x,s)ds \Bigg\rvert_{t=0} = 0.$$
Далее посчитаем нужные производные:
\begin{align*}
	\overline{u}_{tt} &= \int \limits_0^t w_{tt} (t,x,s) ds + w_t (t, x, t) = \int \limits_0^t w_{tt} (t,x,s)ds + f(t,x), \\
	\overline{u}_{xx} &= \int \limits_0^t w_{xx} (t,x,s) ds.
\end{align*}
Подставляем в уравнение:
$$ \overline{u}_{tt} - v^2 \overline{u}_{xx} = \int \limits_0^t \underbrace{w_{tt}(t,x,s) - v^2 w_{xx}(t,x,s)}_{= 0} ds + f(x,t) = f(x,t).$$

Значит, $\overline{u}$ - действительно решение задачи $\eqref{wavehomcauchy}$.

\end{proof}
Что мы сделали с точки зрения физики? Вместо того, чтобы трактовать $f(t,x)$ как возмущение, действующее в каждый момент времени $t$ в каждой точке $x$, мы сказали, что $f(s,x)$ - это начальная скорость, действующая только в момент времени $s$. Можно сказать, что мы дизинтегрировали решение задачи $\eqref{wavehomcauchy}$ на решение семейства задач $\eqref{waveduhamel}_s$.

Есть смысл выразить $w(t,x,s)$ при помощи формулы Д'Аламбера:
$$ w(t,x,s) = \frac {1} {2v} \int \limits_{x-v(t-s)}^{x+v(t-s)} f(s,y) dy. $$
Тогда решение задачи  $\eqref{wavehomcauchy}$ записывается как
$$ \overline{u} (t,x) = \frac {1} {2v} \int \limits_0^t ds \int \limits_{x-v(t-s)}^{x+v(t-s)} f(s,y) dy,$$
а полное решение записывается как $$u(t,x) = \frac {u_0 (x+vt) - u_0 (x-vt)} {2} + \frac {1} {2v} \int \limits_{x-vt}^{x+vt} v_0(y)dy + \frac {1} {2v} \int \limits_0^t ds \int \limits_{x-v(t-s)}^{x+v(t-s)} f(s,y) dy.$$

Для каких $f$ верно вышесказанное? В формуле Д'Аламбера в качестве $v_0 \in C^1$ взяли $f$. Значит, для $f \in C(\real^+ \times \real)$ и, дополнительно, $f_x \in C(\real^+ \times \real)$. Имеено тогда записанное выше $u(t,x)$ --- классическое решение неоднородной задачи Коши для волнового уравнения.
% так и не понял, почему f in C и f_x in C, а не f in C^1

А могут ли быть два разных решения у $\eqref{wavenonhomcauchy}$? Ежели да, то их разность будет удовлетворять однородному волновому уравнению с нулевыми начальными условиями. А у такого уравнения есть единственное классическое решение, выражаемое формулой Д'Аламбера. Подставляем в неё нулевые начальные условия - получаем ноль. Значит, двух разных классических решений быть не может.

\section{Уравнение теплопроводности}
Следующая модель - распространение тепла в пространстве.

Имеется некоторая область, --- ограниченная или нет, --- в ней имеются источники теплоты. Область заполнена некоторым веществом, на границе области поддерживаются некоторые условия. Как описать изменение температурного поля в этой области?

\subsection{Вывод уравнения теплопроводности из соотношения теплового баланса}

Пусть $q(t,x)$ - плотность источника теплоты в момент времени $t \in \real^+$ в точке $x \in \Omega \subset \real^n$. Составим уравнение теплового баланса для произвольного шара $B \subset \Omega$. Насколько изменилась температура в $B$ за $\Delta t$?

Пусть $c$ - теплоемкость вещества, $\rho$ - его плотность:

$$ \int \limits_t^{t +\Delta t} ds \int \limits_B  c \rho u_t dx = \underbrace {\int \limits_t^{t + \Delta t} ds \int \limits_B q(x) dx}_{\text{стоки теплоты}} - \underbrace {\int \limits_t^{t + \Delta t} ds \int \limits_{\partial B} F \cdot n d \sigma}_{\substack{\text{теплообмен со} \\ \text{внешней средой}}}.$$

Поделим на $\Delta t$ и устремим $\Delta t$ к $0$:
$$ \int \limits_B c \rho u_t dx = \int \limits_B q dx - \int \limits_{\partial B} F \cdot n d \sigma. $$

Применяем формулу Гаусса-Остроградского \footnote{Формула Гаусса-Остроградского: $\int_{\partial \Omega} F \cdot n d\sigma = \int_{\Omega} \Div F dx$}:
\begin{gather*}
	\int \limits_B c \rho u_t dx = \int \limits_B q dx - \int \limits_B \Div F dx,
	\int \limits_B c \rho u_t + \Div F - q dx = 0,\quad \forall B
\end{gather*}
Значит,
$$ c \rho u_t + \Div F = q.$$
Поток через поверхность выражается по закону Фурье. Применим его.
$$ F = - \lambda \nabla u \quad \Rightarrow \quad \Div F = - \lambda \Delta u$$
Перебозначив $f = q/c\rho$ и $ a^2 = \lambda / c \rho $ - коэффициент температуропроводности, получаем уравнение теплопроводности:
\begin{equation}
	u_t - a^2 \Delta u = f
\label{heatnonhom}
\end{equation}

Уравнение теплопроводности - линейное дифференциальное уравнение в частных производных с постоянными коэффициентами. Относится к параболическому типу.

Также это уравнение называется уравнением диффузии. Тогда $f$ - загрязнение, $a$ - коэффициент диффузии.

\subsection{Основные постановки задач для уравнения теплопроводности}
Обычно для уравнения теплопроводности $\eqref{heatnonhom}$ ставится одно из трех условий:

\subsubsection{Краевое условие Дирихле (первое краевое условие)}
$$u \Bigg \rvert_{\partial\Omega} = u_0.$$
Это условие означает, что на границе области поддерживается заданный температурный режим.

Например, есть дачный домик со стенками из слабо теплоизолирующего материала. Тогда наша область это домик, а условие на стенке - температура внешней среды.

\subsubsection{Условие Неймана (второе краевое условие)}
$$\dfrac{\partial u}{\partial n}\Bigg\rvert_{\partial\Omega} = u_0, \quad \text{чаще всего }u_0 = 0.$$
В случае $u = 0$ означает, что нет теплообмена с внешней средой. В примере с домиком условие означает, что у домика очень тёплые стены.

\subsubsection{Условие конвективного теплообмена (третье краевое условие)}
$$ \frac {\partial u} {\partial n} + \alpha (u - u_0) \Bigg\rvert_{\partial \Omega} = 0.$$

\subsubsection{Начальное условие}
$$ u(0, x) = h(x).$$



Задача с начальными и краевыми условиями называется начально-краевой задачей, без краевых условий - просто начальной задачей (задачей Коши). На разных частях границы могут быть заданы разные условия.

\subsection{Принцип максимума для уравнения теплопроводности в ограниченной области}
TODO