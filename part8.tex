% !TEX encoding = UTF-8 Unicode
% лекции 15-16, 2 апреля 2016
% 6. Неравенство Фридрихса.
% 7. Слабая непрерывность линейных операторов. Слабая сходимость в $H_0^1$ и в $L^2$. Слабая непрерывность оператора обобщенного дифференцирования.
% 8. ∗ Существование точки минимума интегрального функционала в $H_0^1$.
% 9. Существование обобщенного решения задачи Дирихле для уравнения Пуассона — вариационный метод. ∗ Единственность решения; метод конечных элементов (метод Рисса).
% 10. Точки минимума выпуклых и строго выпуклых функционалов. Единственность экстремали. Единственность обобщенного решения задачи Дирихле для уравнения Пуассона - вариационный метод.
% 11. Существование и единственность обобщенного решения задачи Дирихле для уравнения Пуассона — использование теоремы Рисса. ∗∗ Регулярность обобщенного решения (существование классического решения) — формулировка.
% 12. Непрерывная обратимость оператора $-\Delta : H_0^1 \to L^2$.

% теорема: H - сепарабельное гильбертово пр-во
% замечание: норма полунепрерывная снизу в слабой топологии
% замечание: непрерывный оператор между гильбертовыми пр-вами сохраняет слабую сходимость

% неравенство фридрихса: доказываем для C_0^{\infty}, переходим к пределу

% два упражднения

% необходимое условие существования минимума функционала на H_0^1

% предложение: u = argmin F(u) в \H_0^1, тогда - \Delta u = f
% предложение: - \Delta u  =f, u \in H_0^1, тогда u = argmin F(u) в H_0^1

% лемма: - \Delta u = f в обобщ смысле и u \in H_0^1, тогда \int ... - \int ... = 0 \forall v \in H_0^1
% лемма: (что-то про минимум выпуклого функционала)

% непрерывная обратимость оператора - \Delta

% лемма (Lax-Milgram)

% новая норма в H_0^1, наблюдение: эта норма непрерывна

% второе док-во чего-то там

% теорема реллиха-кондрашова: H_0^1 компактно вложено в H^1 (без док-ва)
% компактность вложения H_0^1 в L^2
% следствие: (?) слабая сходимость в H_0^1 влечёт сильную сходимость в L^2

% наблюдение: компактный оператор между банаховыми пр-вами переводит слабо сходящуюся последовательность в сильно сходящуюся

% упражнение: f строго выпуклый, тогда существует единственный минимум
