% !TEX encoding = UTF-8 Unicode
% лекции 15-16, 2 апреля 2016
% 6. Неравенство Фридрихса.
% 7. Слабая непрерывность линейных операторов. Слабая сходимость в $H_0^1$ и в $L^2$. Слабая непрерывность оператора обобщенного дифференцирования.
% 8. ∗ Существование точки минимума интегрального функционала в $H_0^1$.
% 9. Существование обобщенного решения задачи Дирихле для уравнения Пуассона — вариационный метод. ∗ Метод конечных элементов (метод Рисса).
% 10. Точки минимума выпуклых и строго выпуклых функционалов. Единственность экстремали. Единственность обобщенного решения задачи Дирихле для уравнения Пуассона - вариационный метод.
% 11. Существование и единственность обобщенного решения задачи Дирихле для уравнения Пуассона — использование теоремы Рисса. ∗∗ Регулярность обобщенного решения (существование классического решения) — формулировка.
% 12. Непрерывная обратимость оператора $-\Delta : H_0^1 \to L^2$.

% 6. Неравенство Фридрихса.
\subsection{Неравенство Фридрихса}
\begin{theorem} Пусть $\Omega \subset \real^n$ - ограниченная область, $u \in \m{H01Om}$. Тогда
$$ \int \limits_\Omega u^2 \, dx \leq C_\Omega \int \limits_\Omega |\nabla u|^2 \, dx.$$
\end{theorem}
\begin{proof} Пространство $C_0^\infty(\Omega)$ плотно в $\m{H01Om}$, так что достаточно доказать для него, а потом перейти к пределу.

Пусть $u \in C_0^\infty(\Omega)$. Существует такое $a$, что $\Omega \subset (-a,a)^n.$ Зафиксируем все координаты, кроме $i$-ой. Тогда по неравенству Гёльдера
\begin{align*}
|u(x)| &\leq \int \limits_{-a}^x \Bigl\lvert \pder{x_i} u(x_1, ..., \xi_i, ..., x_n) \Bigr\rvert \, d \xi_i \\ \\
& \leq \left( \int \limits_{-a}^a \Bigl\lvert \pder{x_i} u(x_1, ..., \xi_i, ..., x_n)  \Bigr\rvert^2 \, d\xi_i\right)^{1/2} \sqrt{2a},\\
|u(x)|^2 &\leq 2a \int \limits_{-a}^a | u_{x_i} (x_1, ..., \xi_i, ..., x_n) | \, d\xi_i.
\end{align*}
Интегрируем по всем остальным координатам по $(-a, a)$. После интегрирования по всем координатам, кроме $i$-ой, получим норму $u_{x_i}$ в квадрате. Проинтегрировав её по $x_i$, получим дополнительный множитель $2a$:
$$ \norm*{2}{u}^2 \leq 4a^2 \norm*{2}{u_{x_i}} \leq 4a^2 \norm*{2}{\nabla u}^2.$$
Таким образом,
$$\norm*{2}{u} \leq C_\Omega \norm*{2}{\nabla u},$$
где константа зависит только от диаметра области.

Пусть теперь $u \in \m{H01Om}$. Аппроксимируем:
$$ \exists \{u_k \} \subset C_0^\infty(\Omega) : u_k  \conv{H1}{} u,$$
то есть,
$$ \norm{H1}{u_k - u}^2 = \norm*{2}{u_k - u}^2 + \norm*{2}{\nabla u_k - \nabla u}^2 \conv{}{} 0,$$
откуда
$$\norm*{2}{u_k} \conv{}{} \norm*{2}{u}, \quad \norm*{2}{\nabla u_k} \conv{}{} \norm*{2}{\nabla u}.$$
Итого,
$$ \norm*{2}{u} \leq C \norm*{2}{\nabla u},$$
что и требовалось доказать.
\end{proof}

\begin{note}
Впервые нам реально понадобилась ограниченность области $\Omega$.
\end{note}

\begin{note}
На самом деле мы пользуемся ограниченностью только по одному направлению. 
\end{note}

% 7. Слабая непрерывность линейных операторов. Слабая сходимость в $H_0^1$ и в $L^2$. Слабая непрерывность оператора обобщенного дифференцирования.
\subsection{Слабая сходимость в гильбертовом пространстве}
Вспомним некоторые сведения из функционального анализа.
\begin{definition}
Пусть $H$ - гильбертово пространство. Последовательность $\{x_k \} \subset H$ называется слабо сходящейся к $x$, если
$$\scalprod*{H}{x_k}{y} \conv*{H}{} \scalprod*{H}{x}{y} \quad \forall y \in H,$$
и обозначается
$$ x_k \wkconv*{H}{} x.$$
\end{definition}

\begin{note}
Слабая сходимость слабее сильной:
$$ x_k \conv{}{} x \quad \Rightarrow \quad x_k \wkconv{}{} x.$$
\end{note}

\begin{note}
Обратное неверно.
\end{note}

\begin{example} Пусть $H = L^2(0, \pi)$, функция $f \in H$. Тогда $f$ представима в виде сходящегося в смысле $H$ ряда
$$ f = \sum_{k=1}^\infty a_k \sin kx.$$
Обозначим $u_k = \sin kx$. Известно равенство Парсеваля:
$$ \sum_{k=1}^\infty a_k^2 = \frac {2} {\pi} \norm*{2}{f}^2.$$
Значит, $|a_k| \to 0$, где
$$ a_k = \frac {\scalprod*{H}{f}{u_k}} {\norm*{H}{u_k}^2} = \frac {2} {\pi} \scalprod*{H}{f}{u_k} \to 0.$$
Это выполняется для всех $f$. По определению слабой сходимости
$$ u_k \wkconv*{H}{} 0 \quad \Leftrightarrow \quad \sin kx \wkconv*{H}{k \to \infty} 0.$$
Но сильно $\{ u_k \}$ никуда не сходится хотя бы потому, что $\displaystyle \norm*{H}{u_k} = \frac {\pi} {2}$.
\end{example}

\begin{note}
Слабая топология в гильбертовом пространстве не задаётся никакой метрикой. Однако, шар в сепарабельном гильбертовом пространстве со слабой топологией метризуем.
\end{note}

% TODO: доказать
\begin{note}[Слабая компактность шара] Пусть $H$ - сепарабельное гильбертово пространство. Тогда
$$ \{ x_k \} \subset H : |x_k| \leq C \quad \Rightarrow \quad \exists \{ x_{k_n} \} : x_{k_n} \wkconv{}{} x.$$
\end{note}

% TODO: доказать
\begin{note} В сепарабельном гильбертовом пространстве норма слабо полунепрерывна снизу:
$$x_k \wkconv*{}{} x \quad \Rightarrow \quad \norm{}{x_k} \leq \lim \inf \norm*{}{x_k}.$$
\end{note}

% TODO: замкнутое выпуклое подмножество гильбертова пр-ва слабо замкнуто

\begin{note} Непрерывный линейный оператор $A$ между гильбертовыми пространствами $H_1$ и $H_2$  сохраняет слабую сходимость:
$$ x_k \wkconv*{H_1}{} x \quad \Rightarrow \quad Ax_k \wkconv*{H_2}{} Ax.$$
\end{note}
\begin{proof}
$$ \scalprod*{H_2}{A x_k} {y} = \scalprod*{H_1}{x_k}{A^*y} \conv*{H_1}{} \scalprod*{H_1}{x}{A^*y} = \scalprod*{H_2}{Ax}{y}.$$

\end{proof}

\subsection{Слабая сходимость в $\m{L2}$ и в $\m{H01}$}
Пусть $\Omega \subset \real^n$ - область. Рассмотрим оператор вложения
$$\dot{\iota} : \m{H01Om} \longrightarrow \m{L2Om},$$
$$\dot{\iota}(u) := u.$$

Оператор $\dot{\iota}$ непрерывен, так как
$$ \norm*{2}{\dot{\iota}}(u)^2 = \norm*{2}{u}^2 \leq \norm{H1}{u}^2.$$
Следовательно,
$$ u_k \wkconv{H1}{} u \quad \Rightarrow \quad u_k \wkconv{L2}{} u.$$

Рассмотрим оператор обобщённого дифференцирования по $i$-ой переменной:
$$ \pder{x_i} : \m{H01Om} \longrightarrow \m{L2Om}.$$
Нам уже известно, что он линеен и непрерывен. Значит, он тоже сохраняет слабую сходимость:
$$ u_k \wkconv{H1Om}{} u \quad \Rightarrow \quad \pder[u_k]{x_i} \wkconv{L2Om}{} \pder[u_k]{x_i}.$$

% 8. ∗ Существование точки минимума интегрального функционала в $H_0^1$.
\subsection{Существование точки минимума интегрального функционала в $\m{H01}$}
\begin{lemma} В $\m{H1Om}$ функционал $F$ слабо полунепрерывен снизу:
$$u_k \wkconv{H1}{} u \quad \Rightarrow \quad F(u) \leq \lim \inf F(u_k).$$
\end{lemma}
\begin{proof} Прежде всего заметим, что
$$ \int \limits_\Omega f u_k \, dx = \scalprod*{2}{f}{u_k} \wkconv{L2}{} \scalprod*{2}{f}{u} =\int \limits_\Omega f u \, dx.$$
Посмотрим на обобщённый градиент:
\begin{align*}
0 \leq | \nabla (u_k - u) |^2 &= | \nabla u_k - \nabla u |^2 = |\nabla u_k|^2 + |\nabla u|^2 - 2 \nabla u_k \cdot \nabla u \\
&= |\nabla u_k|^2 - |\nabla u|^2 + 2 \nabla u \cdot (\nabla u - \nabla u_k),
\end{align*}
$$ |\nabla u_k|^2 - |\nabla u|^2 \geq 2 \nabla u \cdot (\nabla u_k - \nabla u).$$
Проинтегрируем:
$$ \int \limits_\Omega \frac {| \nabla u_k|^2} {2} \, dx - \int \limits_\Omega \frac {| \nabla u|^2} {2} \, dx \geq \int \limits_\Omega \nabla u \cdot (\nabla u_k - \nabla u) \, dx = \sum_{j=1}^n \int \limits_\Omega u_{x_j} \left( \pder[u_k]{x_j} - u_{x_j} \right) .$$
Заметим, что
$$ \pder[u_k]{x_j} - \pder[u]{x_j} \wkconv{L2}{} 0 \quad \Rightarrow \quad u_{x_j} (\hdots) \conv{}{k \to \infty} 0,$$
то есть,
$$ \lim \inf \frac {1} {2} \int \limits_\Omega |\nabla u_k|^2 \, dx \geq \frac {1} {2} \int \limits_\Omega |\nabla u|^2 \, dx.$$
Итого
\begin{align*}
\lim \inf F(u_k) &= \lim \inf \left( \frac {1}{2} \int \limits_\Omega |\nabla u_k|^2 \, dx - \int \limits_\Omega fu_k \, dx \right) \\
&\geq \lim \inf \frac {1} {2} \int \limits_\Omega |\nabla u_k|^2 \, dx - \lim \inf \int \limits_\Omega fu_k \, dx \\
&\geq \frac {1} {2} \int \limits_\Omega |\nabla u|^2 \, dx - \int \limits_\Omega fu \, dx = F(u),
\end{align*}
что и требовалось доказать.

\end{proof}

\begin{theorem}[Существование точки минимума интегрального функционала] Пусть $\Omega \subset \real^n$ - ограниченная область, $f \in L^2(\Omega)$, и задан фунцкионал 
$$F(u) = \frac {1} {2} \int \limits_\Omega | \nabla u |^2 \, dx - \int \limits_\Omega fu \, dx.$$
Тогда существует $u \in \m{H01Om}$ такая, что 
$$ F(u) = \inf_{v \in \m{H01Om}} F(v).$$
\end{theorem}
\begin{note} Функционал $F$ определён на $\m{H01Om}$: если понимать градиент как обобщённый, то выражение имеет смысл.
\end{note}
\begin{proof}[Доказательство теоремы]
Воспользуемся методом Тонелли. Рассмотрим минимизирующую последовательность $\{ u_k \} \subset \m{H01Om}$:
$$ F(u_k) \conv{}{} \inf_{\m{H01Om}} F.$$
Заметим, что верно (например, $C = 0$ при $u \equiv 0$):
$$F(u_k) = \frac {1} {2} \int \limits_\Omega | \nabla u_k |^2 \, dx - \int \limits_\Omega f u_k \, dx \leq C.$$
Заметим, что :
\begin{align*}
F(u_k) &\stackrel{\text{(а)}}{\geq} \frac {1} {2} \norm*{2}{\nabla u_k}^2 - \frac{1} {2 \eps} \norm*{2}{f}^2 - \frac{\eps^2}{2} \norm*{2}{u_k}^2 \\
& \stackrel{\text{(б)}}{\geq} \frac {1} {2} \norm*{2}{ \nabla u_k } - \frac {1} {2\eps} \norm*{2}{ u_k}^2 - \frac {\eps^2} {2} C_\Omega \norm*{2}{ \nabla u_k}^2.
\end{align*}
В первом переходе воспользовались тем, что 
$$ ab \leq \frac {a^2 + b^2} {2} \quad \Rightarrow \quad \Biggl\lvert \int \limits_\Omega \frac {f} {\eps} \eps u_k \, dx \Biggr\rvert \leq \frac {1} {2} \left( \frac {1} {\eps^2} \norm*{2}{f}^2 + \eps^2 \norm*{2}{u_k}^2 \right).$$
Имеем:
$$C \geq F(u_k) \geq \frac {1} {2} \left(1 - \eps^2 C_\Omega \right) \norm*{2}{ \nabla u_k }^2 - \frac {1} {2 \eps^2} \norm*{2}{f}^2.$$
Тогда
$$ \frac {1} {2} \left( 1 - \eps^2 C_\Omega \right) \norm*{2}{ \nabla u_k}^2 \leq \underbrace {C + \frac {1} {2 \eps^2} \norm*{2}{f}^2}_{=C}.$$
Можно выбрать такое $\eps$, что $ \eps^2 C_\Omega < 1$. Тогда 
$$ \norm*{2}{\nabla u_k}^2 \leq C,$$
и
$$ \norm*{2}{u_k}^2 \leq C_\Omega \norm*{2}{ \nabla u_k}^2 \leq C.$$
Таким образом, $\{ u_k \}$ ограничена в $\m{H01Om}$:
$$ \norm{H1}{u_k} \leq C,$$ 
то есть, $\{ u_k \}$ лежит в некотором шаре в $\m{H1Om}$.

В стандартной топологии шар не компактен, зато компактен в слабой топологии. Значит, существует такая подпоследовательность $\left\{ u_{k_m} \right\} \subset \left\{ u_k \right\}$, что 
$$u_{k_m} \wkconv{H1Om}{m \to \infty} u \in \m{H01Om}.$$ 
Заметим, что слабая сходимость в $\m{H1Om}$ влечёт слабую сходимость в $\m{L2Om}$, а оператор обобщённого дифференцирования слабо непрерывен. Значит,
$$ u_{k_m} \wkconv{L2Om}{} u, \quad \pder[u_{k_m}]{x_i} \wkconv{L2Om}{} \pder[u]{x_i},$$
и
$$ \int \limits_\Omega f u_{k_m} \, dx \longrightarrow \int \limits_\Omega fu \, dx.$$

Первое слагаемое функционала непрерывно, а второе слагаемое по лемме полунепрерывно снизу. Значит,
$$ F(u) \leq \lim \inf \left( \int \limits_\Omega | \nabla u_{k_m}|^2 \, dx - \int \limits_\Omega f u_{k_m} \, dx \right) = \inf F.$$

Итого,
$$ F(u) = \inf F.$$

\end{proof}

Резюмируя: мы рассмотрели прозвольную минимизирующую последовательность для функционала. Потом достаточно примитивным трюком доказали, что она ограниченная по норме в $H^1$, используя неравенство Фридрихса. Значит, она содержит слабо сходящуюся подпоследовательность, она тоже минимизирующая. Функционал слабо (полу)непрерывен снизу, значит, в пределе получаем нижний предел $u_k$, а это в точности инфимум.

% 9. Существование обобщенного решения задачи Дирихле для уравнения Пуассона — вариационный метод. ∗ Метод конечных элементов (метод Рисса).
\subsection{Существование обобщённого решения - вариационный метод}
\begin{theorem} Пусть $u \in \m{H01Om}$ - минимум $F$, тогда $u$ - решение уравнения Пуассона:
$$- \Delta u = f.$$
\end{theorem}
\begin{proof} Если $u$ - минимум $F$, то
$$F'(u)v = \int \limits_\Omega \nabla u \cdot \nabla v \, dx - \int \limits_\Omega f v \, dx = 0 \quad \forall v \in \m{H01Om}.$$
В частности, это выполняется и для всех $v$ из $C_0^\infty (\Omega)$. Тогда
\begin{align*}
	0 &= \int \limits_\Omega \nabla u \cdot \nabla v \, dx - \int \limits_\Omega f v \, dx = \sum_{k=1}^n \int \limits_\Omega u_{x_k} v_{x_k} \, dx - \int \limits_\Omega f v \, dx \\
	&= \sum_{k=1}^n \action{v_{x_k}, u_{x_k}} - \action{v,f} = - \sum_{k=1}^n \action{v, u_{x_k x_k}} - \action{v,f} \\
	&= - \action{v, \underbrace{\Delta u}_{\text{обобщ}}} - \action{v, f} = \action{v, -\Delta u -f} = 0.
\end{align*}
Значит, минимум функционала $F$ удовлетворяет уравнению Пуассона в обобщённом смысле 
$$ - \Delta u = f.$$

\end{proof}

А минимум существует по предыдущей теореме. Значит, решение уравнения Пуассона  существует.

В дальнейшем нам пригодится следующая лемма.
\begin{lemma}
Если $-\Delta u = f$ в обобщённом смысле, и $u \in \m{H01Om}$, то
$$ \int \limits_\Omega \nabla u \cdot \nabla v \, dx - \int \limits_\Omega fv \, dx = 0 \quad \forall v \in \m{H01Om}.$$
\end{lemma}
\begin{proof} Надо доказать в обратную сторону предыдущую цепочку равенств. Из условия леммы следует, что 
$$ \action{\varphi, -\Delta u} = \action{\varphi, f} \quad \forall \varphi \in C_0^\infty(\Omega).$$
Тогда 
\begin{align*}
- \sum_{k=1}^n \action{\varphi, u_{x_k x_k}} &= \sum_{k=1}^n \action{\varphi_{x_k}, u_{x_k}} = \sum_{k=1}^n \int \limits_\Omega \varphi_{x_k} u_{x_k} \, dx \\
&= \int \limits_\Omega \nabla \varphi \cdot \nabla u \, dx = \int \limits_\Omega \varphi f \, dx \quad \forall \varphi \in C_0^\infty(\Omega).
\end{align*}
Теперь рассмотрим $v$ из $\m{H01Om}$. По определению,
$$ \exists \{ \varphi_k \} \subset C_0^\infty(\Omega) : \varphi_k \conv{H1}{} v.$$
Значит, по непрерывности оператора вложения
$$ \varphi_k \conv{L2}{} v, \quad \pder[\varphi_k]{x_j} \conv{L2}{} \pder[v]{x_j}, \quad \nabla \varphi_k \conv{L2}{} \nabla v.$$

В пределе получаем
$$ \int \limits_\Omega \nabla v \cdot \nabla u \, dx - \int \limits_\Omega fv \, dx = 0.$$

\end{proof}

\begin{note} Мы доказали равносильность задач! Если $f$ из $\m{L2Om}$, $u$ из $\m{H01Om}$, то
$$ -\Delta u = f \quad \Leftrightarrow \quad \int \limits_\Omega \nabla v \cdot \nabla u \, dx = \int \limits_\Omega fv \, dx \quad \forall v \in \m{H01Om}.$$
\end{note}

Таким образом, подобное $F$ интегральное условие является обобщённым аналогом условия Дирихле.


\subsection{Метод конечных элементов} Пусть $\Omega \subset \real^2$ - ограниченная область. Задача Дирихле для уравнения Пуассона в обобщённом смысле эквивалентентна вариационной задаче
$$ F(u) = \frac {1}{2} \int \limits_\Omega |\nabla u|^2\ \,dx - \int \limits_\Omega fu \, dx \rightarrow \min.$$
Знаем, что у $F$ существует минимум. Тогда у задачи Дирихле для уравнения Пуассона существует решение (чуть далее мы докажем, что минимум и решение единственны. Вариационный метод подсказывает, как можно найти приближённое решение.

Триангулируем $\Omega$ - разобьём её на $m$ достаточно малых треугольников. Скажем, что на $k$-ом треугольнике действует линейная функция $u_k$, приближающая $u$ в этой части области. Линейная функция на треугольнике определяется своими значениями в вершинах треугольника. Таким образом, неизвестная функция $u_k$ заменяется на неизвестные функции $u_k$, которые определяются неизвестными значениями в узлах треугольников. В узлах на границе положим $u = 0$. Тогда $F$ - квадратичная функция на $\real^m$, она выпукла.

Таким образом, задача свелась к нахождению минимума квадратичной функции на конечномерном пространстве. Если приравнять её градиент к нулю, то полученное соотношение можно представить в виде системы линейных уравнений порядка $m$ с трёхдиагональной матрицей.

Можно доказать, что при дальнейшем дроблении области на треугольники полученное приближение будет сходиться к $u$ по энергетической норме.  

В $\real^n$ вместо треугольников пространство следует разбивать на симплексы.

% 10. Точки минимума выпуклых и строго выпуклых функционалов. Единственность экстремали. Единственность обобщенного решения задачи Дирихле для уравнения Пуассона - вариационный метод.
\subsection{Единственность обобщённого решения - вариационный метод}

\begin{definition} Вещественный функционал $G$ на гильбертовом пространстве, удовлетворяющий условию
$$ G(\lambda_1 u_1 + \lambda_2 u_2) \leq \lambda_1 G(u_1) + \lambda_2 G(u_2), \quad \lambda_1,\lambda_2 \geq 0, \quad \lambda_1 + \lambda_2 = 1,$$
называется выпуклым.
\end{definition}

\begin{exercise}
$$ F(u) = \frac {1} {2} \int \limits_\Omega |\nabla u|^2 \, dx - \int \limits_\Omega fu \, dx \quad \text{--- выпуклый.}$$
\end{exercise}

\begin{definition} Вещественный функционал $G$ на гильбертовом пространстве, удовлетворяющий условию
$$ G(\lambda_1 u_1 + \lambda_2 u_2) < \lambda_1 G(u_1) + \lambda_2 G(u_2), \quad \lambda_1,\lambda_2 > 0, \quad \lambda_1 + \lambda_2 = 1,$$
называется строго выпуклым.
\end{definition}

\begin{exercise}
Если $G$ --- строго выпуклый, то у него единственная точка минимума.
\end{exercise}

Докажем единственность. Нам понадобится следующая абстрактная лемма:

\begin{lemma} Пусть $F$ - выпуклый вещественный функционал на гильбертовом пространстве $H$, у которого всюду существует первая вариация
$$ \exists F'(u) v \quad \forall v \in H,$$ и пусть имеется $u$ из $\m{H01Om}$ такое, что
$$ F'(u)v = 0 \quad \forall v \in H.$$
Тогда $u$ - точка минимума $F$:
$$ F(u) = \inf_{\m{H01Om}} F. $$
% ???? не странно ли записано условие?
\end{lemma}
\begin{proof}
Рассмотрим выпуклую функцию
$$ g(t) = F(u+tv) \quad t\in \real.$$
Тогда
$$F'(u)v = \frac{d}{dt} F(u+tv)\Big\rvert_{t=0} = g'(0),$$
и $g(t)$ лежит выше касательной:
$$g(t) \geq g(0) + tg'(0) \quad \forall t \in \real.$$
Рассмотрим такое $w \in H$, что
$$ w = u + tv \quad \Rightarrow \quad v = \frac {w-u} {t}.$$
Тогда
$$ F(w) \geq F(u) + t F'(u) \frac{w-u}{t} = F(u) + F'(u)(w-u).$$
Итого
$$ F'(u)(w-u) = 0 \quad \Rightarrow F(u) \leq F(w) \quad \forall w,$$
значит, $u$ - минимум $F$.

\end{proof}
Теперь мы готовы доказать
\begin{theorem}
Если $u \in \m{H01Om}$ удовлетворяет уравнению
$$ - \Delta u = f,$$
то $$F(u) = \min_{\m{H01Om}} F.$$ 
\end{theorem}
\begin{proof}
Задача Дирихле для уравнения Пуассона эквивалентна задаче
$$F(u) = \int \limits_\Omega \nabla u \cdot \nabla v \, dx - \int \limits_\Omega fu \, dx = 0.$$
У $F$ по любому направлению существует производная Гато $F'(u)v$. Рассмотрим $u$ такое, что 
$$ F'(u) v = 0 \quad \forall v.$$
Тогда по второй лемме $u$ - минимум $F$.

\end{proof}

Таким образом, мы доказали, что уравнение Пуассона имеет единственное решение, совпадающее с минимумом функционала $F$.

% 11. Существование и единственность обобщенного решения задачи Дирихле для уравнения Пуассона — использование теоремы Рисса. ∗∗ Регулярность обобщенного решения (существование классического решения) — формулировка.

\subsection{Существование и единственность обобщённого решения - использование теоремы Рисса}

Пусть $\Omega \subset \real^n$ - ограниченная область, $f$ из $\m{L2Om}$, и поставлена задача Дирихле для уравнения Пуассона:
\begin{align*}
\begin{cases*}
	- \Delta u = f, \\
	u \in \m{H01Om}.
\end{cases*}
\end{align*}
При помощи вариационного метода мы доказали существование и единственность решения. Докажем то же самое, опираясь на теорему Рисса.

\begin{reminder}[Теорема Рисса] Пусть $H$ - гильбертово пространство, $f$ - непрерывный линейный функционал на $H$. Тогда существует единственный $y \in H$ такой, что 
$$ f(x) = \scalprod*{H}{y}{x} \quad \forall x \in H.$$
То есть, любой линейный функционал на $H$ есть скалярное произведение с некоторым $y$ из $H$.
Кроме того,
$$ \norm*{H}{y} = \norm*{H^*}{f}.$$
\end{reminder}

\begin{lemma}[Лакс-Мильграм] Пусть $\widehat{H}$ и $H$ --- гильбертовы пространства и первое непрерывно вложено во второе:
$$ \widehat{H} \hookrightarrow H. $$
Тогда
$$ \forall y \in H \quad \exists ! x \in \widehat{H}: \scalprod*{\widehat{H}}{x}{z} = \scalprod*{H}{y}{z} \quad \forall z \in \widehat{H}.$$
\end{lemma}
\begin{proof}
Пусть $y \in H$. Рассмотрим функционал
\begin{gather*}
f_y : \widehat{H} \longrightarrow \real, \\
f_y(z) := \scalprod*{H}{y}{z}.
\end{gather*}
Очевидно, он линеен. Проверим непрерывность:
$$ |f_y(z)| = | \scalprod*{H}{y}{z} \leq \norm*{H}{y} \cdot \norm*{H}{z} \leq \norm*{H}{y} \cdot \norm*{\widehat{H}}{z} C = C_y \norm*{\widehat{H}}{z}.$$
Функционал $f_y$ непрерывен на $\widehat{H}$. Тогда по теореме Рисса в $\widehat{H}$
$$ \exists ! \, x \in \widehat{H}: \scalprod*{\widehat{H}}{x}{z} = f_y(z) = \scalprod*{H}{y}{z} \quad \forall z \in \widehat{H}.$$

\end{proof}

\begin{note}
Фактически, мы определили оператор $A$:
\begin{gather*}
A : H \longrightarrow \widehat{H}, \\
Ay = x, \quad \scalprod*{H}{y}{z} = \scalprod*{\widehat{H}}{x}{z}.
\end{gather*}
Очевидно, он линеен. Покажем, что он непрерывен. Для начала,
$$ | \scalprod*{\widehat{H}}{x}{z} | = | \scalprod*{H}{y}{Z} | \leq C \norm*{H}{y} \cdot \norm*{\widehat{H}}{z}.$$
Отсюда
$$ \frac {| \scalprod*{\widehat{H}}{x}{z} |} {\norm*{\widehat{H}}{z}} \leq C \norm*{H}{y} \quad \Rightarrow \quad \sup_{z \in \widehat{H}} \frac {| \scalprod*{\widehat{H}}{x}{z} |} {\norm*{\widehat{H}}{z}} \leq C \norm*{H}{y} \quad \Rightarrow \quad \norm*{\widehat{H}}{x} \leq C \norm*{H}{y}.$$
То есть,
$$ \norm*{\widehat{H}}{Ay} \leq C \norm*{H}{y} \quad \Rightarrow \quad A \text{ --- непрерывный}.$$
\end{note}


\begin{theorem} Пусть $f$ из $\m{L2Om}$. Тогда обобщённое решение задачи Дирихле для уравнения Пуассона существует и единственно.
\end{theorem}
\begin{proof} Здесь
$$ H = \m{L2Om}, \quad \widehat{H} = \m{H01Om}.$$
Введём новую норму в $\widehat{H}$:
$$\norm*{\widehat{H}}{u} = \sqrt{\int \limits_\Omega | \nabla u |^2 \, dx } = \norm*{2}{\nabla u}.$$
Эта норма называется энергетической. Очевидно,
$$ \norm*{2}{\nabla u} = \norm*{\widehat{H}}{u} \leq \norm{H1}{u}.$$
В то же время,
$$ \norm*{2}{u} \leq C \norm*{2}{\nabla u} \quad \Rightarrow \quad \norm*{H1}{u} \leq C \norm*{2}{\nabla u} = C \norm*{\widehat{H}}{u}.$$
Значит, энергетическая норма эквивалентна стандартной, и $\widehat{H}$ --- сепарабельное гильбертово пространство. В то же время, вложение $\widehat{H} \hookrightarrow H$ непрерывно:
$$ \norm*{H}{u} = \norm*{2}{u} \leq C \norm*{2}{\nabla u} = C \norm*{\widehat{H}}{u}.$$
Применяем лемму. Для любого $f$ из $\m{L2Om}$ 
$$ \exists! \, u \in \m{H01Om}: \scalprod{H01}{u}{v} = \scalprod{L2}{f}{v} \quad \forall v \in \m{H01Om}.$$
В интегральной форме: существует такой $f$ из $\m{L2Om}$, что
$$ \exists ! u \in \m{H01Om} : \int \limits_\Omega \nabla u \cdot \nabla v \, dx = \int \limits_\Omega f v \, dx \quad \forall v \in \m{H01Om},$$
а это равносильно существованию и единственности решения задачи Дирихле для уравнения Пуассона.

\end{proof}

Недостаток этого подхода заключается в том, что он неконструктивен.
% ????????????
% ∗∗ Регулярность обобщенного решения (существование классического решения) — формулировка.

\subsection{Непрерывная обратимость лапласиана}
% 12. Непрерывная обратимость оператора $-\Delta : H_0^1 \to L^2$.

\begin{theorem}
Пусть $\Omega \subset \real^n$ - ограниченная область. Для любого $f$ из $\m{L2Om}$ существует единственное $u$ из $\m{H01Om}$ такое, что
$$ - \Delta u = f \quad \Rightarrow \quad u = (-\Delta)^{-1} f.$$
Значит, определён оператор
$$ (-\Delta)^{-1}: \m{L2Om} \longrightarrow \m{H01Om}.$$
Этот оператор непрерывен.
\end{theorem}
\begin{proof}
Очевидно, этот оператор линеен. Нам нужно доказать, что
$$ \norm{H1}{(-\Delta)^{-1} f} \leq C \norm*{2}{f}, \quad \text{или, что то же самое,} \quad \norm{H1}{u} \leq C\norm*{2}{f}.$$
Если $u$ - решение уравнения Пуассона, то 
$$ \int \limits_\Omega \nabla u \cdot \nabla v \, dx = \int \limits_\Omega fv \, dx \quad \forall v \in \m{H01Om}.$$
В частности, при $v= u$
\begin{align*}
\norm*{2}{\nabla u}^2 = \int \limits_\Omega |\nabla u|^2 \, dx = \int \limits_\Omega fu \, dx \leq \norm*{2}{f} \cdot \norm*{2}{u} \leq C \norm*{2}{f} \cdot \norm*{2}{\nabla u}
\end{align*}
Сначала мы воспользовались неравенством Гёльдера, а потом неравенством Фридрихса. Тогда
$$ \norm*{2}{\nabla u} \leq C \norm*{2}{f} \quad \text{и} \quad \norm*{2}{u} \leq C\norm*{2}{f}.$$
Итого,
$$ \norm{H1}{u} = \sqrt{\norm*{2}{u}^2 + \norm*{2}{\nabla u}^2} \leq C \norm*{2}{f},$$
что и требовалось доказать.

\end{proof}
