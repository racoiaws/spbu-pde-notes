% !TEX encoding = UTF-8 Unicode
% лекции 9-10, 12 марта 2016
% вопросы 18-20 
% 18. Пространство основных функций и пространство обобщенных функций (распределений). Примеры обобщенных функций. Дифференцирование обобщенных функций. Обобщенные и классические производные.
% 19. Фундаментальное решение уравнения Лапласа, его физический смысл.
% 20. Представление (частного) решения уравнения Пуассона в пространстве при помощи фундаментального решения.

\subsection{Основы теории обобщенных функций}
Как и раньше, $\Omega \subset \real^n$ --- область.
\begin{definition}
Множество $ \beauD(\Omega) = C_0^\infty(\Omega)$ называется пространством основных функций (пробных функций).
\end{definition}
\begin{definition}
Последовательность функций $\{u_k\} \subset C_0^\infty(\Omega)$ сходится к основной функции $u \in C_0^\infty(\Omega)$, если 
\begin{enumerate}
\item все функции последовательности имеют носитель в одном компакте: $$\exists K \ssubset \Omega: \quad \supp u_k \subset K,$$
\item все функции последовательности и все их производные сходятся к $u$ и её производным равномерно $$u_k \rightrightarrows u, \quad D^\alpha u_k \rightrightarrows D^\alpha u \quad \forall \alpha,$$ где $\alpha$ --- мультииндекс: $$\alpha = (\alpha_1, \ldots, \alpha_n), \, \alpha_i \in \mathds{N},$$ $$\abs{\alpha} = \sum\limits_{i=1}^{n}\alpha_i,$$
$$D^\alpha u = \dfrac{\partial^{\abs{\alpha}}u}{\partial x_1^{\alpha_1}\ldots\partial x_n^{\alpha_n}}.$$
\end{enumerate}
\end{definition}

\begin{example}Пусть $u_0 \in C_0^\infty(\real)$ и $u_k = \dfrac{1}{k}u_0$, тогда $u_k \conv*{\beauD(\real)}{} 0$.
\end{example}

\begin{example}Пусть $u_0 \in C_0^\infty(\real)$ и $u_k(x) = \dfrac{1}{k}u_0(x - k)$. Тогда
\begin{gather*}
	u_k \to 0, \\
	u_k \rightrightarrows 0, \\
	D^\alpha u_k \rightrightarrows 0 \quad \forall \alpha
\end{gather*}
но $u_k$ не живут на одном компакте, так что $$ u_k \centernot {\conv*{\beauD(\real)}{}} 0.$$

Таким образом, сходимость основных функций сильнее, чем равномерная или поточечная.
\end{example}

\begin{note} Заданная нами на $\beauD$ топология не порождается никакой метрикой. 
\end{note}

\begin{definition}
Пространством обобщенных функций $\beauD'(\Omega)$ называется пространство линейных непрерывных функционалов на пространстве $\beauD(\Omega)$. Через $\action{u,f}$ обозначается действие обобщенной функции $f$ на основную функцию $u$. Обобщенные функции также называют распределениями.
\end{definition}

\begin{note} Таким образом, обобщённая функция это не функция, а функционал.
\end{note}

\begin{note} Пусть $\left\{ u_k \right\} \subset \beauD(\Omega)$. Непрерывность обобщённой функции $f$ означает, что
$$ u_k \conv* {\beauD(\Omega)} {} u \quad \Rightarrow \quad \action{u_k, f} \longrightarrow \action{u, f}.$$
\end{note}

\begin{example} Любая функция $v \in L_{loc}^1 (\Omega)$ канонически порождает обобщённую функцию $\widehat{v}$:
$$ \action{u, \widehat{v}} = \int \limits_\Omega u(x) v(x) \, dx.$$
Интеграл имеет смысл, так как на самом деле мы интегрируем по $\supp u$. Линейность $\widehat{v}$ очевидна, покажем непрерыность. Пусть $\left\{ u_k \right\} \subset \beauD(\Omega)$ и $ u_k \conv*{\beauD(\Omega)}{} u$, тогда
$$ \action{u_k, \widehat{v}} = \int \limits_\Omega u_k v = \int \limits_K u_k v \longrightarrow \int \limits_K uv = \int \limits_\Omega uv = \action{u, \widehat{v}}.$$
Таким образом, можно отождествлять $v = \widehat{v}$ и говорить, что $L_{loc}^1(\Omega) \subset \beauD' (\Omega)$, хоть эти пространства и имеют разную природу.
\end{example}

\begin{example}Конечная положительная борелевская мера $\mu$ порождает обобщённую функцию $\widehat{\mu}$:
$$\action{u, \widehat{\mu}} = \int \limits_\Omega u \, d\mu.$$
Линейность очевидна, непрерывность вытекает из непрерывности интеграла. Таким образом, конечные положительные борелевские меры --- тоже обобщённые функции.
\end{example}

\begin{example}Рассмотрим так называемую дельта-фунцию Дирака $\delta = \delta_0 = \delta(x)$:
$$ \action{u, \delta} = u(0).$$
Также для $x \in \Omega$ можно определить $\delta_{x_0} = \delta (x - x_0)$:
$$\action{u, \delta_{x_0}} = u(x_0).$$
Линейность и непрерывность очевидны. Значит, дельта-функция является обобщённой функцией.
\end{example}

\begin{note}Дельта-функция не порождается никакой функцией из $L_{loc}^1(\Omega)$.
\end{note}
\begin{proof}
Пусть $f \in L_{loc}^1(\Omega)$ такая, что
$$ \action{u, \delta} = \action{u, f} = \int \limits_\Omega fu  = u(0) \quad \forall u \in C_0^{\infty}(\Omega)$$
Рассмотрим такие $u$, у которых $0$ не лежит в носителе. Для них верно
$$\action{u, f} = \int \limits_\Omega fu = 0.$$
По основной лемме вариационного исчисления $f = 0$ почти всюду на $\Omega \setminus \{0\}$. Множество $\{0\}$ имеет меру нуль, значит, $f = 0$ почти всюду на $\Omega$. Подставив в интеграл, получим
$$ u(0) = 0 \quad \forall u \in C_0^{\infty},$$
что невозможно.

\end{proof}

На обобщённых функциях можно определить операции:

\begin{enumerate}
\item Сложение, умножение на скаляр. Пусть $v_1, \, v_2 \in \beauD'(\Omega)$, $\lambda_1,\,\lambda_2 \in \real$, тогда
$$\action{u, \lambda_1v_1 + \lambda_2v_2} = \lambda_1\action{u, v_1} + \lambda_2\action{u, v_2}.$$
\item Умножение на функцию из $C^\infty$. Пусть $\varphi \in C^\infty(\Omega)$, $v\in \beauD'(\Omega)$, $u\in \beauD(\Omega)$, тогда
$$\action{u, \varphi v} = \action{\varphi u, v}.$$
Определение корректно, так как $\varphi u \in C_0^\infty$.
\item Дифференцирование. Пусть $\alpha$ --- мультииндекс, $v\in\beauD'(\Omega)$, $u \in \beauD(\Omega)$. Тогда
$$\action{u, D^\alpha v} = (-1)^{\abs{\alpha}}\action{D^\alpha u, v}.$$
Проверим, что $D^\alpha v\in \beauD'(\Omega)$. Линейность очевидна. Пусть $u_k \conv* {\beauD(\Omega)}{} u$, тогда
$$\action{u_k, D^\alpha v} = (-1)^{\abs{\alpha}} \action{D^\alpha u_k, v} \longrightarrow (-1)^{\abs{\alpha}} \action{D^\alpha u, v} = \action{u, D^\alpha v},$$
по свойствам сходящихся в $\beauD(\Omega)$ последовательностей.
\end{enumerate}

\begin{example}
Пусть $v \in C^1(\Omega) \subset L_{loc}^1(\Omega)$. Тогда $v$ --- обобщенная с точностью до отождествления. Рассмотрим обобщённую функцию $v_{x_i}$:
$$\action{u, v_{x_i}} = - \action{u_{x_i}, v }= - \int \limits_\Omega u_{x_i} v \, dx = \int \limits_\Omega u v_{x_i} \, dx - \underbrace{\int \limits_{\partial \Omega'} u v \nu_i \, d\sigma}_{=0} = \action{u, v_{x_i}},$$
где $\partial \Omega'$ --- гладкая граница некоего множества, лежащего в $\Omega$ и окружающего носитель функции $u$. Здесь мы воспользовались формулой интегрирования по частям для многомерного случая.

\begin{reminder}[Интегрирование по частям (многомерный случай)]
$$\int \limits_{\Omega} u v_{x_i} \,dx = \int \limits_{\partial\Omega} uvn_{i} \,d\sigma - \int \limits_{\Omega} u_{x_i} v \, dx,$$
где $n_{i}$ --- $i$-ая компонента вектора внешней нормали к $\partial\Omega$.
\end{reminder}

Таким образом, если $v$ непрерывно дифференцируема, то обобщённая производная совпадает с классической.
\end{example}

\begin{exercise} Пусть $\Omega = \real$. Рассмотрим обобщённую производную дельта-функции:
$$\action{u, \delta'} = -\action{u, \delta} = -u'(0).$$
Доказать, что ей не соответствуют ни одна функция и ни одна мера.
\end{exercise}

\begin{example}Пусть $\Omega =\real.$ Рассмотрим функцию Хевисайда:
\begin{equation*}
h(x) = \quad
    \begin{cases} 
        1, \, x\geq0\\
        0,\, x < 0.
    \end{cases}
\end{equation*}
Найдём её обобщённую производную:

$$\action{u, h'}= -\action{u',h} = -\int\limits_{-\infty}^{+\infty}u'h = -\int\limits_0^{+\infty}u' = \underbrace{-u\Bigl|_0^{+\infty}}_{\text{финитная}} = 0+u(0)=\action{u,\delta}.$$
Таким образом, $h' = \delta$ в смысле обобщённых функций. В классическом смысле в нуле производной не существует.
\end{example}

\begin{exercise}
Канторова лестница п.в. имеет нулевую производную в классическом смысле. Доказать, что её обобщенная производная $\not\equiv0$  и является мерой, сосредоточенной на канторовом множестве.
\end{exercise}

\subsection{Фундаментальное решение уравнения Лапласа}
Будем рассматривать уравнения Лапласа \eqref{Laplace} и Пуассона \eqref{Poisson} в смысле обобщенных функций.

Пусть $\Omega = \real^n$, $u$ - обобщённая функция. В случае электростатической системы из одного заряда имеем уравнение Пуассона:
\begin{equation}
	-\Delta u = \delta.
\label{comLaplace}
\end{equation}

Нам понадобятся формулы Грина (следуют из формулы интегрирования по частям):
\begin{enumerate}
\item $\displaystyle \int \limits_\Omega v \Delta u = \sum \limits_{i = 1}^{n} \int \limits_\Omega v u_{x_i x_i} = \sum \limits_{i = 1}^{n} \left( - \int \limits_\Omega v_{x_i} u_{x_i} \, dx + \int \limits_{\partial \Omega} v u_{x_i} \nu_i \, d\sigma \right) = - \int \limits_\Omega \nabla u \cdot \nabla v \, dx + \int \limits_{\partial \Omega} v \dfrac{\partial u}{\partial \nu} \, d\sigma.$
\item $\displaystyle \int \limits_{\Omega} \left( u \Delta v - v \Delta u \right) \, dx = \int \limits_{\partial \Omega} \left( u \dfrac{\partial v}{\partial \nu} - v \dfrac{\partial u}{\partial \nu} \right) \, d\sigma.$
\end{enumerate}
Ещё нам понадобятся формулы объёма шара и площади сферы:
$$ |B_1(0)| = \frac{\sqrt{\pi^n}}{\Gamma(\frac {n} {2} + 1)} = \omega_n, \quad  |B_r(0)| = r^n \omega_n, \quad | \partial B_r(0) | = n \omega_n r^{n-1}$$

Будем искать решение \eqref{comLaplace} в смысле обобщенных функций в виде:
$$u(x) = \dfrac{c_n}{\abs{x}^{n-2}}.$$
Пусть $\varphi \in C_0^\infty\left(\real^n\right)$. Посчитаем обобщённый лапласиан от $u$:
\begin{align*}
\action{\varphi, -\Delta u} &= - \action{\Delta \varphi, u} = - \int \limits_{\real^n} \Delta \varphi u \, dx = -c_n \int \limits_{\real^n} \Delta \varphi \dfrac{1}{|x|^{n-2}} \, dx \stackrel{\text{(а)}} {=} - c_n \lim_{\eps \to 0} \int \limits_{B^c_\eps (0)} \Delta \varphi \frac {1} {|x|^{n-2}} \, dx\\
&\stackrel{\text{(б)}} {=} -c_n \lim_{\eps \to 0} \left( \int \limits_{B^c_{\eps}(0)} \varphi \Delta \dfrac{1}{|x|^{n-2}} \, dx + \int \limits_{\partial B_\eps(0)} \left( \dfrac{1}{|x|^{n-2}} \dfrac{\partial \varphi}{\partial\nu} - \varphi \dfrac{\partial}{\partial\nu} \left( \dfrac{1}{|x|^{n-2}} \right) \right) \, d\sigma \right) \\
&\stackrel{\text{(в)}} {=} -c_n \lim_{\eps \to 0} \int \limits_{\partial B_\eps(0)} \left( \dfrac{1}{|x|^{n-2}} \pder[\varphi]{\nu} - \varphi \pder{\nu} \left( \dfrac{1}{|x|^{n-2}} \right) \right) \, d\sigma \\
&= -c_n \lim_{\eps \to 0} \dfrac{1}{\eps^{n-2}} \int \limits_{\partial B_\eps(0)} \pder[\varphi]{\nu} \, d\sigma - \int \limits_{\partial B_\eps(0)} \varphi \pder{\nu} \left( \dfrac{1}{|x|^{n-2}} \right) \, d\sigma \\
&\stackrel{\text{(г)}} {=} c_n \lim_{\eps \to 0} \int \limits_{\partial B_\eps(0)} \varphi \pder{\nu} \left( \dfrac{1}{|x|^{n-2}} \right) \, d\sigma \stackrel{\text{(д)}} {=} c_n\lim_{\eps \to 0} \int \limits_{\partial B_\eps(0)} - \varphi \pder{r} \left( \dfrac{1}{r^{n-2}} \right) \, d\sigma \\
&= c_n \lim_{\eps \to 0} \int \limits_{\partial B_\eps(0)} \varphi \frac{n-2} {|x|^{n-1}} \, d\sigma =  c_n(n-2) \lim_{\eps \to 0} \dfrac{1}{\eps^{n-1}} \int \limits_{\partial B_\eps(0)} \varphi \, d\sigma \\
&\stackrel{\text{(е)}} = c_n n(n-2) \omega_n \lim_{\eps \to 0} \dfrac{1}{| \partial B_\eps(0) |} \int \limits_{\partial B_\eps(0)}\varphi \, d\sigma = c_n n(n-2) \omega_n \lim_{\eps \to 0} \fint \limits_{\partial B_\eps(0)} \varphi \, d\sigma \\
&= c_n n(n-2) \omega_n \varphi(0) = c_n n(n-2) \omega_n \action{\varphi, \delta}.
\end{align*}
Поясним вычисления:
\begin{description}
\item [(а)] исключили особенность в нуле
\item [(б)] воспользовались 2 формулой Грина
\item [(в)] воспользовались тем, что $\Delta\dfrac{1}{\abs{x}^{n-2}} = 0$, остаётся как упражнение
\item [(г)] воспользовались тем, что $$ \Bigg\lvert \int\limits_{\partial B_\eps(0)}\dfrac{\partial\varphi}{\partial\nu} \, d\sigma \Bigg\rvert \leq C\abs{\partial B_\eps(0)} =C\eps^{n-1}, \quad 
 \dfrac{1}{\eps^{n-2}}C\eps^{n-1} = C\eps \conv* {\eps \to 0} {} 0 $$
\item [(д)] воспользовались тем, что $|x| = r$ и вектор нормали к шару это радиальное направление
\item [(е)] домножили и поделили на площадь сферы радиусом $\eps$: $$|\partial B_\eps(0)| = n \omega_n \eps^{n-1}$$
\end{description}

Итого, при $c_n = \dfrac{1}{n(n-2) \omega_n}$ имеем
$$\action{\varphi, -\Delta u} = \action{\varphi, \delta} = \varphi(0) \quad \Rightarrow \quad -\Delta u = \delta.$$
Значит, $$u(x) = \dfrac{1}{n(n-2) \omega_n}\dfrac{1}{\abs{x}^{n-2}}$$ --- обобщённое решение уравнения Лапласа \eqref{comLaplace} в $\real^n$  при $n>2$.

\begin{note}При $n=3$ получим знакомую из школы формулу для потенциала единичного заряда в $\real^3$:
$$c_3 = \dfrac{1}{4\pi}, \quad u(x) = \dfrac{1}{4\pi}\dfrac{1}{\abs{x}}.$$
\end{note}

\begin{exercise}
Проверить, что $$u(x)=\dfrac{1}{2\pi}\ln\dfrac{1}{\abs{x}}$$ удовлетворяет уравнению Лапласа \eqref{comLaplace} в $\real^2$ в обобщенном смысле.
\end{exercise}

\begin{definition}
Фундаментальным решением уравнения Лапласа называется функция
\begin{equation}
\Phi(x) = 
\begin{cases}
-\dfrac{1}{2\pi}\ln\abs{x},\quad n=2,\\
\dfrac{1}{n(n-2)w_n}\dfrac{1}{\abs{x}^{n-2}},\quad n>2.
\end{cases}
\end{equation}
Для неё верно, что
\begin{align*}
& - \Delta\Phi = \delta \text{ в } \beauD'(\real^n), \\
& - \Delta\Phi = 0 \text{ в классическом смысле в } \real^n\setminus\{0\}.
\end{align*}

\end{definition}

\subsection{Представление частного решения уравнения Пуассона в пространстве при помощи фундаментального решения}

\begin{note} [Вывод 3 формулы Грина] Пусть $\Omega\subset\real^n$ --- ограниченная область с гладкой границей, и $0 \in \Omega$. Тогда по 2 формуле Грина:
\begin{align*}
\int \limits_{B^c_\eps(0)} \left( \underbrace{u\Delta\Phi}_{0} - \Phi\Delta u \right) \, dx &= \int \limits_{B^c_\eps(0)} - \Phi \Delta u \, dx = \int \limits_{\partial B^c_\eps(0)} \left( u \dfrac{\partial\Phi}{\partial\nu} - \Phi\dfrac{\partial u}{\partial\nu} \right) \,d\sigma \\
&= \int \limits_{\partial \Omega} \left( u \dfrac{\partial\Phi}{\partial\nu} - \Phi\dfrac{\partial u}{\partial\nu} \right) \,d\sigma + \int\limits_{\partial B_\eps(0)} \left( u \dfrac{\partial\Phi}{\partial\nu} - \Phi\dfrac{\partial u}{\partial\nu} \right) \,d\sigma
\end{align*}
Устремим $\eps$ к нулю:
$$ \int \limits_\Omega - \Phi \Delta u \, dx = \int \limits_{\partial \Omega} \left( u \dfrac{\partial\Phi}{\partial\nu} - \Phi\dfrac{\partial u}{\partial\nu} \right) \,d\sigma + u(0).$$
Перенесём всё в произвольную точку $x_0\in\Omega$:
$$ \int \limits_\Omega - \Phi (x- x_0) \Delta u \, dx = \int \limits_{\partial \Omega} \left( u \dfrac{\partial\Phi}{\partial\nu} (x-x_0) - \Phi (x - x_0) \dfrac{\partial u}{\partial\nu} \right) \,d\sigma + u(x_0).$$
Значит,
$$ u(x_0) = \int \limits_{\partial \Omega} \left( \Phi (x - x_0) \dfrac{\partial u}{\partial\nu} - u \dfrac{\partial\Phi}{\partial\nu} (x-x_0) \right) \,d\sigma - \int \limits_\Omega \Phi (x- x_0) \Delta u \, dx.$$
Получили 3 формулу Грина.
\end{note}

На данный момент мы знаем, что решением уравнения
$$-\Delta u =\delta_{x_0}$$ является функция $$ u=\Phi(x-x_0).$$ Это решение не является единственным: к нему можно добавить любую гармоническую функцию $v$, результат тоже будет решением.

Пусть есть уравнение
$$ - \Delta u = \alpha_1 \delta_{x_1} + \alpha_2 \delta_{x_2},$$
тогда его решением будет
$$ u = \alpha_1 \Phi (x - x_1) + \alpha_2 \Phi (x - x_2).$$
В случае конечной суммы имеем
$$ - \Delta u = \sum_{k = 1}^n \alpha_k \delta_{x_k}, \quad u = \sum_{k=1}^n \alpha_k \Phi (x - x_k).$$
Для произвольной функции интуитивно можно написать
$$-\Delta u = f, \quad u = \int \limits_{\real^n} \Phi(x-y) f(y) \, dy = \Phi*f.$$
Докажем, что при некоторых дополнительных условиях это действительно так.
\begin{theorem}
Пусть $f\in C_0^2(\real^n)$. Тогда
$$ u(x) = \int\limits_{\real^n} \Phi(x-y)f(y)dy$$ будет классическим решением уравнения Пуассона в $\real^n$. 
\begin{note} Классическое решение всегда является обобщенным, так как классические производные равны обобщенным.
\end{note}
\end{theorem}
\begin{proof} Поменяем переменные:
$$u(x) = \int\limits_{\real^n}\Phi(y)f(x-y)dy$$
и продифференцируем $u$ по $x$. Производная проносится под знак интеграла:
$$u_{x_ix_i}=\int\limits_{\real^n}\Phi(y)f_{x_ix_i}(x-y)dy.$$
Тем самым доказана гладкость.

Если 2 классические функции равны в обобщенном смысле, то они просто равны. Докажем, что $u$ --- обобщённое решение уравнения Пуассона.

Пусть $\varphi\in C_0^\infty(\real^n)$, тогда
\begin{align*}
\action{\varphi, -\Delta u} &= \int \limits_{\real^n} \varphi(x)(-\Delta u(x)) \, dx = \int \limits_{\real^n}\varphi(x) \, dx \int \limits_{\real^n} \Phi(y)(-\Delta f(x-y)) \, dy \\
&= \int \limits_{\real^n} \Phi(y) \, dy \int \limits_{\real^n} (-\Delta\varphi(x)) f(\underbrace{x-y}_z) \, dx = \int \limits_{\real^n} \Phi(y) \, dy \int \limits_{\real^n} (-\Delta\varphi(z+y)) f(z) \, dz \\
&= \int \limits_{\real^n} f(z) \, dz \int \limits_{\real^n} \Phi(y) (-\Delta\varphi(z+y)) \, dy = \int \limits_{\real^n} f(z) \action{-\Delta\varphi(z+\cdot), \Phi} \, dz \\
&= \int \limits_{\real^n} f(z) \underbrace{\action{\varphi(z+\cdot), -\Delta\Phi}}_{\varphi(z)} \, dz = \int\limits_{\real^n} f(z) \varphi(z) \, dz \\
& = \action{\varphi, f}.
\end{align*}
Здесь на третьем шаге мы поменяли порядок интегрирования, а потом интегрированием по частям перенесли производные.

Таким образом, $u$ - классическое решение уравнения $$-\Delta u = f.$$
\end{proof}