% !TEX encoding = UTF-8 Unicode
% лекции 13-14, 26 марта 2016, начало второй части
% 1. Обобщенные функции и обобщенные производные.
% 2. Соболевские классы $H_0^1$, $H^1$ - определения, примеры.
% 3. Гильбертовы пространства $H_0^1$, $H^1$, скалярное произведение, полнота.
% 4. Непрерывные вложения пространств $H_0^1 \subset H^1 \subset L^2$. Непрерывность обобщенного дифференцирования. ∗ Сепарабельность пространств $H_0^1$, $H^1$.
% 5. Соболевское пространство $H_0^1(a,b)$ (непрерывность, дифференцируемость, значение на границе, ∗∗ абсолютная непрерывность). ∗ Вложение $H^1(a, b) \subset C([a, b])$ (формулировка).
\chapter{Соболевские пространства и обобщённые решения}
\section*{Наводящие соображения}

\subsection{Прямой метод вариационного исчисления (метод Тонелли)}
Пусть $X$ --- множество, $G$ --- вещественнозначный функционал на $X$. Надо найти точку $x$, в которой функционал достигает минимума.

Можно попробовать найти минимизирующую последовательность $\left\{ x_k \right\}$:
$$G(x_1) \geq G(x_{2}) \geq \hdots, \quad G(x_k) \conv* {} {k \to \infty} \inf_X G$$

Если удастся задать на $X$ метрику (или топологию), в которой $\left\{ x_k \right\}$ --- компакт, а $G$ непрерывен, то существует подпоследовательность  $\left\{ x_{k_n} \right\} \subset \left\{ x_k \right\}$ такая, что
$$x_{k_n} \conv*{\text{по метрике}}{n \to \infty} x \in X, \quad G(x_{k_n}) \conv*{}{n \to \infty} G(x) = \inf_X G.$$
Но на множествах вроде пространств функций добиться компактности $\left\{ x_k \right\}$ в топологии, порождённой стандарной нормой, нелегко. Нужно ослаблять топологию: чем слабее топология, тем больше компактов. В то же время, чем слабее топология, тем меньше непрерывных функционалов. Нужно искать компромисс.

\begin{note}Условие непрерывности функционала можно ослабить, нам достаточно полунепрерывности снизу
$$ y_k \conv {}{k \to \infty} y \quad \Rightarrow \quad G(y) \leq \lim_{k \to \infty} \inf_{X} G(y_k)$$
так как 
$$ G(x) \leq \lim_{n \to \infty} \inf_X G(x_{k_n}) = \inf_X G \quad \Rightarrow G(x) = \inf_X G.$$
\end{note}

Этот метод поиска экстремумов функционала называется прямым методом вариационного исчисления или методом Тонелли.

\subsection{Уравнение Пуассона}

Пусть в ограниченной области $\Omega \subset \real^n$ поставлена задача Дирихле для уравнения Пуассона:
\begin{align*}
	\begin{cases*}
		- \Delta u =f, \\
		u\Big\rvert_{\partial \Omega} = 0
	\end{cases*}
\end{align*} 
Хотим увидеть вариационную задачу:
$$ F(u) = \frac {1} {2} \int \limits_\Omega | \nabla u |^2 \, dx - \int \limits_\Omega fu \, dx \rightarrow \min$$ для
$$ u \in C^2(\overline{\Omega}), \quad u\Big\rvert_{\partial \Omega}=0.$$
Предполагаем, что минимум достигается. Тогда для любого $\varphi \in C_0^\infty (\Omega)$ верно
$$F(u) \leq F(u + \eps \varphi) \quad \forall \eps \in \real$$
или, что то же самое, первая вариация по направлению $\varphi$ равна нулю:
$$ \frac {d} {d \eps} F(u + \eps \varphi)\Big\rvert_{\eps=0} = 0 \quad \Leftrightarrow \quad F'(u) \varphi = 0.$$
Посчитав, получим, что $$ \int \limits_\Omega (- \Delta u - f) \varphi \, dx = 0.$$
По основной лемме вариационного исчисления
$$ - \Delta u = f,$$
то есть, минимум функционала $F$ является решением нашей задачи. То есть, мы свели задачу доказательства существования решения уравнения к задаче доказательства существования точки минимума функционала.

Попробуем применить метод Тонелли. Пусть существует $\left\{ u_k \right\} \in C^2(\overline{\Omega})$ такая, что
$$ u_k\Big\rvert_{\partial \Omega} = 0, \quad F(u_k) \conv* {} {k \to \infty} \inf_{C^2(\overline{\Omega})} F.$$
Нужно придумать топологию или метрику, в которой $u_{k_n} \conv*{}{n \to \infty} u \in C^2(\overline{\Omega})$, так как со стандартной нормой пространство $C^2(\overline{\Omega})$ неполно.

Хочется найти интегральную сходимость вроде сходимости $L^p$, в которой к чему-то сходятся функции и их производные. Как задавать такую сходимость в этом пространстве --- непонятно. Сменим пространство.

% 2. Соболевские классы $H_0^1$, $H^1$ - определения, примеры.
\section{Соболевcкие классы}

\begin{definition} Соболевские классы $W^{k,p}$ это такие множества:
$$ W^{k,p} := \left\{ u \in L^p : \quad D^\alpha u \in L^p \quad \forall |\alpha| \leq k \right\}.$$ 

\begin{definition} Пусть $\Omega \subset \real^n$ --- область. Тогда $\m{H1Om}$ это множество таких квадратично интегрируемых функций, у которых обобщённые производные тоже квадратично интегрируемы:
	$$\m{H1Om}:= \left\{ u \in \m{L2Om}: \pder[u]{x_i} \in \m{L2Om} \right\} = W^{1,2}.$$
\end{definition}

\begin{definition} Пусть $\Omega \subset \real^n$ --- область. Тогда $\m{H01Om}$ это множество таких квадратично интегрируемых функций, которые аппроксимируются пробными функциями, у которых производные тоже сходятся:
	$$\m{H01Om} := \left\{ u \in L^2(\Omega): \quad \exists \left\{ u_k \right\} \subset C_0^\infty (\Omega): \, u_k \conv{L2Om} {} u \text{ и } \pder[u_k]{x_i} \conv{L2Om}{} v_i \right\} = W_0^{1,2}.$$
\end{definition}

\begin{note} Производные из определения $H_0^1 (\Omega)$ сходятся к обобщённой производной $u$:
\begin{align*}
	\action{\varphi, \pder[u]{x_i}} &= - \action{\pder[\varphi]{x_i}, u} = - \int \limits_\Omega \pder[\varphi]{x_i} u \, dx = - \lim_{k \to \infty} \int \limits_\Omega \pder[\varphi]{x_i} u_k \, dx \\
	&= \lim_{k \to \infty} \int \limits_\Omega \varphi \pder[u_k]{x_i} \, dx - \varphi u_k \Big\rvert_{\partial \Omega} = \int \limits_\Omega \varphi v_i \, dx = \action{\varphi, v_i}.
\end{align*}
\end{note}

% 3. Гильбертовы пространства $H_0^1$, $H^1$, скалярное произведение, полнота.

\begin{note} Очевидно, $\m{H01Om}$ и $\m{H1Om}$ --- линейные пространства, и
$$ \m{H01Om} \subset \m{H1Om}.$$
\end{note}

\subsection{Гильбертовость пространств $\m{H01}$ и $\m{H1}$}

\begin{definition}Пусть $u \in \m{H1Om}$. Тогда
$$ \norm{H1Om}{u} := \left( \norm{L2Om}{u}^2 + \sum_{i=1}^n \norm{L2Om}{u_{x_i}}^2 \right)^{1/2} = \left( \norm*{2}{u}^2 + \int \limits_\Omega | \nabla u|^2 \, dx \right)^{1/2}.$$
\end{definition}
\begin{note} На самом деле это полунорма. Чтобы получить норму, аналогично определению нормы в $L^p (\Omega)$ профакторизуем наше пространство. То есть, объявим эквивалентными функции, совпадающие на множестве полной лебеговой меры и будем рассматривать классы эквивалентности.
\end{note}

\begin{note} Норма в $\m{H01Om}$ порождена скалярным произведением: 
$$\scalprod{H1Om}{u}{v} := \int \limits_\Omega uv \, dx + \int \limits_\Omega \nabla u \cdot \nabla v \, dx.$$
\end{note}

\begin{note} Пусть $u \in \m{H01Om}$. Тогда по определению $\m{H01Om}$
$$ \exists u_k \in \m{C0iOm} : \quad \norm{H1Om}{u_k - u} \conv*{}{k \to \infty} 0,$$
то есть,
$$ \m{H01Om} = \overline{\m{C0iOm}} \quad \text{по норме в } \m{H1Om}.$$
\end{note}

\end{definition}
\begin{theorem} Пространства $\m{H1Om}$ и $\m{H01Om}$ --- полные.
\end{theorem}
\begin{proof} Надо доказать, что любая фундаментальная последовательность сходится к элементу этого же пространства. Пусть $\{ u_k \} \subset \m{H1Om}$ --- фундаментальная:
$$ \norm{H1}{u_k - u_m} \conv*{}{k,m \to \infty} 0.$$
Разворачивем определение сходимости в $\m{H01Om}$:
\begin{gather*}
\begin{cases*}
	\norm*{2}{u_k - u_m} \conv*{}{k,m \to \infty} 0, \\
	\norm*{2}{\pder[u_k]{x_i} - \pder[u_m]{x_i}} \conv*{}{k,m \to \infty} 0.
\end{cases*}
\end{gather*}
Пространство $\m{L2Om}$ полно, так что
$$ u_k \conv{L2}{} u, \quad \pder[u_k]{x_i} \conv{L2}{} \pder[u]{x_i}.$$ Значит,
$$ \norm{H1}{u - u_k} \conv{}{k \to \infty} 0 \quad \Rightarrow u_k \conv{H1}{} u.$$

Из замкнутости $\m{H01Om}$ в $\m{H1Om}$ следует, что оно тоже полное. 

\end{proof}

% 4. Непрерывные вложения пространств $H_0^1 \subset H^1 \subset L^2$. Непрерывность обобщенного дифференцирования. ∗ Сепарабельность пространств $H_0^1$, $H^1$.
% TODO: непрерывные вложения (хотя вроде очевидно)
\subsection{Сепарабельность пространств $\m{H01}$ и $\m{H1}$}

\begin{note} Из доказательства очевидно, что оператор обобщённого дифференцирования
$$ \pder{x_i}: \, \m{H1Om} \longrightarrow \m{L2Om}$$
непрерывен.
\end{note}

\begin{reminder} Пусть $X$ и $Y$ --- линейные нормированные пространства, $X$ --- полное, 
$$A: \, X \longrightarrow Y$$
--- линейная изометрия. Тогда $A(X)$ замкнут в $Y$.
\end{reminder}
\begin{proof}
Изометрия непрерывна, а непрерывный оператор сохраняет сходимость. В $X$ любая фундаментальная последовательность сходится к элементу $X$, значит последовательность из образов фундаментальна и сходится к образу, который лежит в $Y$.

\end{proof}

\begin{theorem} Пространства $\m{H01Om}$ и $\m{H1Om}$ --- сепарабельные\footnote{существует всюду плотное не более чем счётное подмножество} гильбертовы \footnote{везде далее полагаем, что понятие "гильбертово пространство" включает в себя сепарабельность} пространства.
\end{theorem}
\begin{proof}
Введём оператор
\begin{gather*}
J(u) := (u, u_{x_1}, \hdots, u_{x_n}), \\ 
J(u):  \m{H1Om} \longrightarrow L^2(\Omega, \real^{n+1}).
\end{gather*}
Посчитаем норму $J(u)$:
\begin{align*}
	\norm*{L^2(\Omega, \real^{n+1})} {J(u)} &= \norm*{2}{(u, u_{x_1}, \hdots, u_{x_n})}^2 = \norm*{L^2(\Omega, \real^{n+1})}{\sqrt{u^2 + u_{x_1}^2 + \hdots + u_{x_n}^2 }}^2 \\
	&= \int \limits_\Omega u^2 + u_{x_1}^2 + \hdots + u_{x_n}^2 \, dx = \norm{H1}{u}^2.
\end{align*}
Таким образом, $J$ --- изометрия:
$$ \norm*{}{J} = 1.$$
Значит, $\m{H1Om}$ замкнуто в $L^2(\Omega, \real^{n+1})$, так как замкнутое подпространство сепарабельного пространства само сепарабельно. По той же причине сепарабельно $\m{H01Om}$.

\end{proof}

% 5. Соболевское пространство $H_0^1(a,b)$ (непрерывность, дифференцируемость, значение на границе, ∗∗ абсолютная непрерывность). ∗ Вложение $H^1(a, b) \subset C([a, b])$ (формулировка).
\subsection{Соболевское пространство $\m{H01}(a,b)$}

\begin{theorem} В каждом классе функций из $\m{H01}(a,b)$ есть непрерывная:
 $$ \m{H01}(a,b) \subset C([a,b]) .$$
\end{theorem}
\begin{proof}
Пусть $u \in C_0^\infty (a,b)$. Тогда $u(a) = u(b) = 0$ и по теореме о среднем
$$ \exists \xi \in (a,b): \quad u(\xi) = \frac{1}{b-a} \int \limits_a^b u(s) \, ds,$$
тогда по формуле Ньютона-Лейбница
$$ u(x) = u(\xi) + \int \limits_{[\xi, x]} u'(\xi) \, ds = \frac {1} {b-a} \int \limits_a^b u(s) \, ds + \int \limits_{[\xi, x]} u'(s) \, ds.$$
Оценим $u(x)$ при помощи неравенства Гёльдера:
\begin{align*}
|u(x)| &\leq \frac {1} {b-a} \int \limits_a^b |u(s)| \, ds + \int \limits_{[\xi, x]} |u'(s)| \, ds \\
&\leq \frac {1} {b-a} \int \limits_a^b |u(s)| \, ds + \int \limits_a^b |u'(s)| \, ds \\
&\leq \frac {1} {b-a} \sqrt{b-a} \norm*{2}{u} + \sqrt{b-a} \norm*{2}{u'} \\
&\leq C \sqrt{b-a} \left( \norm*{2}{u}^2 + \norm*{2}{u'}^2 \right)^{1/2} = C \norm{H1}{u}.
\end{align*}
На последнем шаге мы воспользовались тем, что
$$ (ax + by)^2 \leq (a^2 + b^2) (x^2 + y^2).$$
Таким образом,
$$ \norm*{\infty}{u} \leq C \norm{H1}{u}.$$

Пусть $u \in \m{H01} (0,1)$ аппроксимируется $\{ u_k \} \subset C_0^\infty (0,1)$. Тогда
$$ \norm{H1}{u_k - u_n} \conv*{}{k,n \to \infty} 0.$$
По ранее доказанному верно 
$$ \norm*{\infty}{u_k - u_n} \leq C \norm{H1}{u_k - u_n} \conv*{}{k,n \to \infty} 0.$$
Имеем 
$$ u_k \conv*{C([a,b])}{} v, \quad u_k \conv*{L^2(a,b)}{} u.$$
Значит, $v = u$ почти везде, и $v$ --- непрерывный представитель класса $u$.  

\end{proof}


\begin{corollary} Если $u \in \m{H01} (a,b)$, то существует $v \in C([a,b])$ такая, что
$$ v = u \quad \text{почти везде}, \quad v(a) = v(b) = 0.$$
\end{corollary}

\begin{exercise} Верно ли, что если $v \in C([a,b])$ и $v(a) = v(b) = 0$, то $v \in \m{H01} (a,b)$?
\end{exercise}
\begin{exercise} Может ли у $u \in \m{H01} (a,b)$ быть несколько непрерывных представителей?
\end{exercise}
\begin{exercise} Может ли у $u \in \m{H01} (a,b)$ быть непрерывный представитель, который не обращается в ноль на границе?
\end{exercise}

\begin{theorem}[Абсолютная непрерывность] Пусть $u \in \m{H01} (a,b)$. Тогда почти всюду существует классическая $u'$, почти всюду равная обобщённой $u'$, и верна формула
$$ u(x) = \int \limits_a^x u'(s) \, ds.$$
\end{theorem}
\begin{proof}
По определению существует $\{ u_k \} \subset C_0^\infty (a,b)$ такая, что 
$$ u_k \conv{H1}{} u.$$
По предыдущей теореме,  имеет место сходимость
$$u_k(x) \conv{}{} u(x) \quad \text{почти всюду на } (a,b)$$
По определению,
$$u'_k \conv{L2}{} u'_{\text{обобщ}} \quad \Rightarrow \quad \int \limits_a^x u'_k(s) \, ds \conv{}{} \int \limits_a^x u'_{\text{обобщ}} (s) \, ds.$$
Получили, что почти всюду 
$$ u(x) = \int \limits_a^x \underbrace{u'_{\text{обобщ}}}_{\in L^2(a,b)}(s) \, ds.$$
Это означает, что $u'_{\text{класс}} = u'_{\text{обобщ}}$ почти всюду.

\end{proof}