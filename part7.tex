% !TEX encoding = UTF-8 Unicode
% лекции 13-14, 26 марта 2016, начало второй части
% 1. Обобщенные функции и обобщенные производные.
% 2. Соболевские классы $H_0^1$, $H^1$ - определения, примеры.
% 3. Гильбертовы пространства $H_0^1$, $H^1$, скалярное произведение, полнота.
% 4. Непрерывные вложения пространств $H_0^1 \subset H^1 \subset L^2$. Непрерывность обобщенного дифференцирования. ∗ Сепарабельность пространств $H_0^1$, $H^1$.
% 5. Соболевское пространство $H_0^1(a,b)$ (непрерывность, дифференцируемость, значение на границе, ∗∗ абсолютная непрерывность). ∗ Вложение $H^1(a, b) \subset C([a, b])$ (формулировка).
\chapter{Соболевские пространства}


% почему нужны соболевские пространства: какой-то вывод, C^2 не полно, хз что делать дальше =>  нужнен другой подход

% определение H_0^1
% замечание про обобщенную производную
% замечание про вложение H_0^1 в H^1

% предложение: H_0^1 полное и сепарабельное
% замечание: оператор обобщенного дифференцирования непрерывен
% теорема: H_0^1 (a,b) \subset C[a,b]

% вопросы:
% 1. верно ли, что любая непрерывная функция с 0 на границе лежит в H_0^1 ?
% 2. может ли у функции из H_0^1 быть несколько непрерывных представителей?
% 3. может ли быть непрерывный представитель с не 0 на границе?

% теорема: u \in H_0^1 (a,b), тогда для почти всех точек из (a,b) существует классическая производна

% теорема: F(u) = 1/2 \int |\nabla u|^2 - \int fu, f \in L^2 => \exists u \in H_0^1: F(u) = inf_{x \in H_0^1} F(x)
% тут же наблюдение, воспоминание про слабую топологию, оператор обобщ дифференцирования непрерывен в слабой топологии

% некоторые нужные вещи:
% 1. определение слабой топологии
% 2. непрерывный оператор сохраняет слабую сходимость (с доказательством)
% 3. шар в сепарабельном гильбертовом пр-ве слабо компактен (с доказательством)
% 4. (????) замк ???? гильб слабо замкнуто
% иметь разумные примеры слабо сходящихся последовательностей, не сходящихся в сильной топологии
