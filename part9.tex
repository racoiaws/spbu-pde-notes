% !TEX encoding = UTF-8 Unicode
% лекции 17-18, 9 апреля 2016
% 13. Непрерывность и компактность интегрального оператора в $L^2$ с непрерывным ядром.
% 14. Непрерывность интегрального оператора в $L^2$ со слабой особенностью.
% 15. Замкнутость класса компактных линейных непрерывных операторов. Компактность интегрального оператора в $L^2$ со слабой особенностью.
% 16. Компактность вложения $H_0^1 \subset L^2$ (теорема Реллиха).
% 17. Теорема Фредгольма для компактных самосопряженных операторов в гильбертовом пространстве.
% 18. Сведение задачи Дирихле для уравнения $−\Delta u + \lambda u = f$ к уравнению Фредгольма второго рода с самосопряженным, компактным, положительным оператором. Теорема об альтернативе.

% отступление про интегральные операторы, инт оператор T
% вспомнить теорему асколи-арцела
% теорема: K \in C( \overline{\Omega} \times \overline{\Omega}), T \in L(L^2, L^2) и T компактен
% следствие про операторы со слабой особенностью: оператор с ядром со слабой особенностью является линейным ограниченным оператором

% теорема: A = \int k u, где k - ядро со слабой особенностью, компактен
% факт из ф.а.: предкомпакт <=> сущ. кон. \eps-сеть (с док-вом)

% лемма про гладкую функцию с компактным носителем
% обобщение на функцию из H_0^1

% доказательство теоремы реллиха-кондрашова (мы доказали что-то с её помощью, а потом вводили необходимый мат.аппарат для доказательства собственно теоремы)

% свойства оператора K = i o (-\Delta)^{-1}

% теорема Фредгольма о компактном операторе (альтернатива фредгольма)

% спектр интегрального оператора