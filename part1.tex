% !TEX encoding = UTF-8 Unicode
% лекция 1, 13 февраля 2016
\chapter*{Введение}
\section*{Обозначения и некоторые определения}
\begin{enumerate}
\item $\real^n$ - евклидово пространство.
\item $\abs{x}$ - евклидова норма $x$ из $\real^n$.
\item Область - открытое множество в $\real^n$.
\item Если где-то написано $\bar{\Omega}$, то $\Omega$ - ограниченная область.
\item $C(\bar{\Omega})$ - пространство непрерывных функций на $\bar{\Omega}$ с $\infty$-нормой: $\sup \{ \abs{f(x)} : x \in \bar{\Omega} \}$.
\item $L^p (\Omega)$ - пространство с нормой Гёльдера, $1 \leq p \leq \infty$.
\item $L^\infty (\Omega)$ - пространство существенно ограниченных функций.
\item $C^k(\bar{\Omega})$ - все частные производные до порядка $k$ включительно существуют и непрерывны вплоть до границы.
\item $C(\Omega)$ - функции, непрерывные на $\Omega$, не вплоть до границы. Не является нормированным пространством.
\item Носитель функции - замыкание множества, на котором функция не равна $0$. Носитель функции $f$ обозначается $\supp f$.
\item Финитная функция - функция с компактным носителем.
\item $C^k_0(\Omega)$ - финитные $k$ раз гладкие функции.
\item $K \ssubset \Omega$ - K относительно компактно в $\Omega$.
\item $\displaystyle \fint \limits_{\Omega} f(x) dx = \frac{1}{|\Omega|} \int \limits_{\Omega} f(x) dx$, где $\abs{\Omega}$ - мера множества $\Omega$.

\end{enumerate}

Весь курс будет построен следующим образом, он делится на две части. Одна большая, классическая, скорее относящаяся к веку 19ому. Я вам покажу несколько конкретных физических моделей, для них с той или иной степенью подробности выведу уравнение, и дальше для какой-то очень конкретной ситуации скажу, как его решать. Как правило, можно будет написать явную формулу. Практически вся первая часть будет про уравнения вообще. Не буду вам весь "зоопарк" рассказывать, только наиболее популярные уравнения и очень частные случаи, в которых можно написать явную формулу и что-то про неё сказать, как это с физикой связано.
Вторая часть будет меньше, она более современная, не скажу, что она совсем современная. Она, наверное, будет концепнуально новой, такого у вас в других курсах не было, понадобится функциональный анализ. По крайней мере для одной задачи расскажу, как получать общее решение, доказывать существование решения, численное решение строить, которое гарантированно сойдётся.
А сейчас так: конкретная физическая модель и какие-то частные случаи.

Все что угодно делайте, только не учите это наизусть, пожалуйста.
- Это карается?
Мне, конечно, всё равно. Но в 100\% случаев когда люди учат наизусть и нараспев, всё правильно написано, ткнёшь в формулу - что она означает? Всё, люди теряются. Если вы что-то забыли, я всегда разрешу подсмотреть в конспект и  возобновить какие-то знания. Мне важно, чтобы вы понимали, что здесь происходит, и могли что-то самостоятельно вывести. Знание чего-то наизусть ни в коем случае от вас не требуется. Ну и не только в этом курсе, а вообще в каких-то математических курсах. Наиболее вредное из того, что вы можете себе устроить из вашего обучения математике, это попытка заучить что-то наизусть.

Есть разница между обучением между в медицинском институте и на матмехе - в медицинском институте надо развивать память. А на матмехе не надо.


\chapter{}
\section{Описание модели, построение уравнения}
Простая модель, я её обычно рассказываю как распространение пятна загрязнения в реке и канале. Река, в первом приближении прямая. Есть течение. Было сброшено какое-то количество загрязняющего вещества. Концентрация загрязняющего вещества  $C(t,x)$.
- t почти всегда время
- x пространственная координата R
Пренебрегаем тем, что происходит между берегами, меня интересует, как загрязнение распространяется в длину, на очень большую длину. На длину, по сравнению с которой расстояние между берегами реки очень маленькое, поэтому я реку считаю одномерным объектом.
$С(t, x)$ - концентрация загрязняющего вещества в момент $t$ в точке $x$.
Дана концентрация вещества в начальный момент времени $t = 0$.
$C(0, x) = C_0(x)$ - заданная функция.
Понятно, что функция линейная (кг/м или т/м)
Надо найти концентрацию при условии, что мы что-то знаем течение реки. Как минимум, мы знаем скорость течения реки. Неплохо бы найти закон, которому подчиняется фунция $C(t,x)$. Закон довольно простой: закон сохранения массы.

Рассмотрим интервал $[x, x + \Delta x]$, посмотрим, что на нём происходит, как меняется концентрация вещества. Это $\displaystyle \int \limits_x^{x + \Delta x} C(t, \xi) d \xi$. Он равен "сколько вливается" минут "сколько выливается": $q$. $q(t, x)$ - сколько влилось в момент $t$ через стенку $x$.
$$\displaystyle \int \limits_x^{x + \Delta x} \! C(t, \xi) d \xi = q(t, x) - q(t, x + dx)$$

Физики любят говорить слово "поток". $q(t,x)$ - поток загрязняющего вещества в момент времени t через виртуальную стенку $x$.


Будем считать, что функция достаточно регулярная (???), значит, можно внести производную $d/dt$ под знак интеграла.

% $$ $$

Разделим на $\Delta x$. Когда интеграл перечеркивается, это имеется ввиду среднее.
Устремим x к нулю. Что получится, куда будет стремиться этот интеграл? Будем считать, что подинтегральная функция непрывна.

Физики в 19 веке, да и сейчас очень частно подходят к моделям таким образом: при выводе будем считать, что все функции обладают всеми нужными производными, эти производные непрерывны. Так что мы можем спокойно проводить все эти манипуляции.

Слева получили $C_t(t,x)$. Справа получили производную $-q'(t,x)$.
Вот вам дифференциальное выражение: закон сохранения массы.

Теперь надо знать, как этот поток зависит от концентрации. Тут бывают разные модели.
\subsection*{Чистая конвеция или чистый дрейф}
Эта модель актуальна, например, в случае реки: в реке вода, а загрязняющее вещество - нефть. Нефть в реке сравнительно мало диффундирует (???). Пятно вылилось - оно так с течением и идёт.

$$  = v \cdot C_x(x,t)$$

$v$ - скорость течения

Скорость течения, в принципе, может зависеть от времени и x, но как правило, предполагают скорость течения постоянной.

\subsection*{Чистая диффузия}

Например, когда вместо реки у нас стоячий канал (то есть, скорость течения 0), а загрязняющее вещество хорошо диффундирует (например, стиральный порошок). Тогда закон такой (закон Фика (Fick), закон диффузии)

% $$ $$

D - коэффициент диффузии.
Это означает, что в момент времени t поток через стенку x пропорционален градиенту концентрации. Чем больше градиент концентрации, тем больше поток в точке x. Минус означает то, что поток идёт оттуда, где концентрация больше туда, где она меньше. Потом мы заметим, что эта же самая модель является моделью распространения тепла. (Тот же самый закон, называется законом Фурье)

\subsection*{Диффузия и конвекция}

% $$ $$

Давайте напишем, что получается:

\section{Уравнения}
Подставляем наши уравнения в закон сохранения массы. Получили некоторое уравнение для C

\subsection*{Транспортное уравнение}
% $$ $$
Это уравнение называется транспортным уравнением. Частные производные участвуют до первого порядка, поэтому уравнение называется уравнением в частных производных первого порядка. Порядок уравнения - максимальный порядок входящих в него прозводных. Наконец, это линейное уравнение.
Что значит "линейное"? Можно написать это в виде $L(c) = t$, где $L$ - линейная операция. В данном случае (f? t?) это 0, а $L(c)$ это $c_t + x*v_c$. Сразу видно, что линейное дифферециальное уравнение [что-то сказал про лямбды], однородное

\subsection*{Уравнение диффузии}
Здесь получится $c_t = D \cdot c_(xx)$ - уравнение диффузии. ДУВЧП 2 порядка, линейное (опять-таки что тут только линейная операция)

Вообще говоря, в этом курсе мы будем иметь дело только с линейными уравнениями. Нелинейных моделей не будет.

Еще оно называется уравнением теплопроводности. $1/D$ - коэффициент температуропроводности.

\subsection*{Уравнение диффузии с дрейфом}
$$c_t + v * c_x = D * c_{xx}$$
Можно назвать уравнением конвекции(-?и?)диффузии, уравнением диффузии с дрейфом. В физике называют уравнением Фоккера-Планка.

Еще такое уравнение очень часто встречается в теории вероятностей. Если кто-нибудь из вас занимался стохастическими моделями, которые сейчас часто применяются в CS, там часто встречается уравнение Фоккера-Планка.


\section{Решение уравнения}
Нас интересует найти какую-то функцию, которая не только удовлетворяет этому уравнению, но еще и удовлетворяет начальному условию $c(0, x) = c_0(x)$

Сегодня я буду рассказывать только про уравнение 1, давайте с ним разберемся. Уравнение 2 - когда-то сильно позже.

Значит, транспортное уравнение. Считаем, что у нас чистая конвекция. Скорость будем считать постоянной. Ищем решение задчи $c(0,x) = c_0(x)$

Первую лекцию довольно медленно рассказываю, так как хочется рассказать какие-то вещи философского характера. Когда говорю слово "решение", что я имею ввиду? Функция, при подстановке которой соотношение будет верно. Так обычно понимают решение, так всегда понимали решения, в том числе физики. До 30ых годов 20 века. Это функция, которая если вы её сюда подставите, в каждой точке будет удовлетворять ....
Что значит "подставить"? Это означает, что функция, как минимум, должна обладать нужной гладкостью, иметь непрерывные первые производные по t и по x, а условия уравнения просто соблюдались в каждой точке, то есть поточечно. Это называется классическим решением. И никаких других решений, вообще говоря, вы не знаете. К сожалению, потом оказалось, что классическое решение для диффуров в частных производных это не очень хорошее понятие. Для обыкновенных диффуров это достаточно. Для диффуров в частных производных классических решений очень часто нет.

В другом смысле решения - придумано огромное количество разных решений, слабые решения, вязкостные решения, энтропийные решения... В общем, много чего придумали за начало 21века. Всё это реально используется и реально физично.
Об одном важном понятии решения - слабом решении - мы будем потом говорить. Пока что для вас решение всё время будет классическим. Давайте заниматься тем, чем занимались в 19 веке - будем решать такое уравнение.

Давайте будем считать, что $c(t,x)$ (??) - наше решение. Рассмотрим вспомогательное уравнение $dx/dt = v$. Тут всё, что я буду делать, можно разделить на две части:
1) можно проделать какие-то формальные действия и получить решение
2) вы можете спросить, откуда я догадался?

Когда я выводил это уравнение, я смотрел, как распространяется это пятно целиком. То есть, я брал отдельный элемент массы воды от x до x+dx и смотрел, как сохраняется масса загрязняющего вещества в этом элементе. То есть, как распространяется каждый отдельный большой элемент загрязняющего вещества. Вместо этого я мог бы посмотреть на другое: я мог бы посмотреть, как ведёт себя каждая частица этой жидкости, плывущей в реке. Как она плывёт? Она плывёт направо (по рисунку) со скоростью v. Траектория каждой частицы должна удовлетворять вот такому дифференциальному уравнению: dx/dt = v. Концентрация загрязняющего вещества должна распространяться таким образом, чтобы каждая частица бежала по траектории этого дифференциального уравнения.
Решение какое?
$$x(0) = x_0, тогда x(t) = x_0 + v * t$$

$c(t,x)$ - наша функция. Давайте вместо x подставим x(t). То есть, каждая частичка должна сдвигаться по закону dx/dt = v. Давайте продифференцируем % $$ $$
Смотрим концентрацию вдоль траектории этого ОДУ. Что получается?

$$c_t(x,t) + dc/dx dx/dt x(t)$$ % ?????

... вывод

$ = 0$

Мы получили, что $d/td c(t,x(t)) = 0$. Иначе говоря, вдоль траектории ОДУ функция c является постоянной.
Значит
$$c(t,x(t)) = const для любого x, удовлетворяющего ОДУ$$

$$с(t,x(t)) = c_0(x(0)) + $$
%?????какой-то вывод

Как теперь найти $c(t,x)$? Графически легко это увидеть.
Есть точка $(t,x)$, хотим найти значение концентрации. Я смотрю, какая траектория ОДУ проходит через эту точку, через неё проходит единственная такая прямая. Я знаю, что вдоль этой траектории c = const, значит, её значение = $c_0(x_0)$
Как x_0 выражается? $x = x_0 + vt$, значит,$ x_0 = x - vt$.
Получается $c(t,x) = c_0(x(0)) = c_0(x-vt)$. Получили решение.

Здесь есть, что обсудить. Что мы сделали? Я нашёл некоторое ОДУ. Такое ОДУ называется характеристическим для ДУвЧП. Его свойство такое: вдоль траектории этого ОДУ решения ДУвЧП постоянны. Соответственно, его траектории называются характеристическими линиями или (на жаргоне) характеристиками.

Задача оказалась устроена так, что через каждую $(x,t)$ проходит только одна характеристическая линия. А на характеристических линиях искомая функция должна быть постоянной. Отсюда, зная начальное значение, находим формулу для решения ДУвЧП.

Напишем ответ. Поскольку мы математики, должны написать ответ чисто.

Теорема. Пусть $c_0 in ...$, тогда задача 1 имеет единственное классическое решение, выражающееся $c(t,x) = c_0(x - vt)$
 
Заполним многоточие. Какому классу должно принадлежать наше решение, чтобы оно было классическим? 

Теорема. Пусть $c_0 \in C^1(t)$, тогда задача 1 имеет единственное классическое решение, выражающееся $c(t,x) = c_0(x - vt)$

%??? ^^^

Здесь есть, что обсудить. Скажите, как выглядит этот профиль? [рисует на доске]

Концентрация была какой-то такой. Что будет при $t$ большем? Когда t будет увеличиваться, то этот самый профиль будет смещаться просто на $vt$ вправо. Решение - "бегущая волна". Начальное возмущение, не меняя формы, распространяется со скоростью v направо.

Теперь вопрос: почему единственность имеет место? Где здесь было доказательство единственности? Я же полностью проговорил. Пусть $c(t,x)$ - решение, тогда $c(t,x(t)) = const$, тогда $c(t,x(t)) = c_0(x(0))$, тогда, выражая $x_0$ через $x$ и $t_0$ получаем решение. Получили четкую цепочку следствий: если c - решение, то c должно иметь такую форму. Никаких других решений нет, в доказательстве было в том числе и доказательство единственности.

Второй момент. Я же вам сказал, как распространяется это возмущение. Распространяется волна такая. Но, видите, для того, чтобы считать это решение классическим, нам потребовалось, чтобы C была 1 раз непрерывно дифференцируемой, что, вообще говоря, довольно странно. Вы можете себе представлять такую идеализированную модель. Сбросили загрязнение не такой профиль, а ровно вот так [показывает на доске]. Вот от этой точки до этой точки загрязнено, а чуть право и чуть влево - нет. Не $C^1$. Можно думать о такой физической задаче? Вполне можно. Теперь так. Эта формула осмыслена? Осмыслена. Если пятно изначально было таким, то оно и будет дальше идти так, не меняя формы. Но формально назвать это классическим решением это нельзя: если начальный профиль не $C^1$, то то, что здесь написано, производных иметь не будет и, формально, определению решения не удовлетворяет. Это намекает на то, что для физиков понятие классического решения может оказаться неадекватным ситуации. Вроде модель реальна, и формула даёт что-то разумное физически, но с формальной точки зрения она не имеет отношения к уравнению, производные здесь нельзя написать. Но, тем не менее, с какой-то другой точки зрения имеет отношение, потому что если мы исходные интегральные соотношения напишем, то, наверное, будет иметь смысл. Это намёк на то, что, по-хорошему, нужно уходить от понятия классического решения и как-то ослаблять требования. То есть, нужно будет понимать под решением не только то, что является классическим решением, а еще и что-то другое.

Теперь про характеристики. Я каким-то образом догадался написать характеристическое уравнение. Кто-то мне мог подсказать: посмотри, как себя ведет решение уравения. Я даже могу сказать, откуда догадался: из чисто физических соображений, как каждая частичка себя ведёт. Если бы не догадался, то не смог бы решить. Но если кто-нибудь бы дал мне формулу, то я бы мог проверить. Можно не знать всего этого дела, кто-то скажет: формула такая, она является решением, или нет? Написана теорема, тебе нужно её доказать! Единственность просто так не докажете, а то, что это решение - докажете. Подставляете - получается. А чтобы единственность доказать, нужно проделать все эти операции.

Осталось 15 минут, дам вам мелкое упражнение на эту тему, хочу немножко поменять модель.
Тоже распространения пятна загрязнения, с доп особенностью: не просто кто-то выбросил загрязняющее вещество, а есть какой-то источник или, возможно, сток, с какой-то заданной мощностью. Хочется такую модель рассмотреть.

Тогда $d/dt \int \limits_x^{x + \Delta x} c(t, \xi ) d \xi$ - насколько в момент t меняется концентрация загрязняющего вещества на интервале от $x + \delta x = q(t, x) - q(t, x + \delta x) + \int \limits_x^{x+\delta x} f(t,\xi) d\xi$, где $f$ - плотность источников звгрязнения на единицу длины в момент времени t.
Опять $d/dt$ проносим внутрь считая, что всё регулярно и проносить имеем право, делим на $dx$, устремляем $dx$ к нулю, получится:
$$c_t(t,x) = -d'(t,x) + f(t, x)$$
Далее подставляем выражение для q и получаем дифф уравнение
Для случая чистой конвекции что получается:
$$q = vc$$
%$$
уравнение 2 - лин ДУвЧП 1 порядка, неоднородное.

Давайте решать. Посмотрим, как будет вести себя классическое решения на траекториях $dx/dt = v$
В случае однородного нам это помогло. Возможно, поможет и в случае неоднородного тоже.
Пусть $c(t,x)$ - классическое решение (2), .$x(t) = v => x = x0 + vt$
Давайте рассмотрим, как ведет себя это решение на траекториях моего характеристического уравнения. $d/dt C(t, x(t)) = ... = f (t, x(t))$. Так, иначе говоря, давайте x(t) явным образом подставим сюда: $d/dt C(t, x_0 + vt) = f(t, x_0 + vt) \forall t, x_0$, а это уже обыкновенное дифференциальное уравнение, можем явным образом выписать его решение (а еще известно, что $C(0, x_0) = C_0(x_0)$). Выпишем решение: [вывод]. Получили $C(t, x) = ... (2')$ - вот вам решение. Но мыжматематики, должны написать, когда это действительно решение. Давайте сформулируем теорему.

\begin{theorem}
Пусть $C_0 \in C^1(\real), f \in C(\real^+ \times \real), f_x \in C(\real^+ \times \real)$. Тогда задача (2) имеет единственное классическое решение, задаваемое формулой (2'). [чтобы получить условия на f и f_x что-то куда-то подставляем, дифференцируем вот это вот по вот этому, а потом дифференцируем вот то]
\end{theorem}

Опять-таки, намек на то, что классическое решение - понятие очень уж тяжелое для физиков. Получается, чтобы решение было классическим [что-то], f должна иметь какую-то дополнительную гладкость по пространственной переменной. Откуда такое - это совершенно не физичное условие. f - это удельная по длине мощность источников или стоков загрязняющего вещества. Ладно еще непрерывная, черт с ним, но почему она должна быть еще и гладкой по пространственной переменной - совершенно непонятно. Очень даже можно представить себе непрерывную негладкую функцию. Пока работаем с классическими решениями, вынуждены требовать такие странные условия [дальше диалог про то, как выглядит слив в точке, что-то рисует на доске; можно решать уравнения, в которых вместо функции - мера]