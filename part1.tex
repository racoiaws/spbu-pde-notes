% !TEX encoding = UTF-8 Unicode
% лекции 1-2, 13 февраля 2016
% вопросы 1-5
% 1. Однородное линейное транспортное уравнение с постоянными коэффициентами, его физический смысл (задача о распространении пятна загрязнения в канале). Решение однородного уравнения методом характеристик. Бегущая волна.
% 2. Неоднородное линейное транспортное уравнение
% 3. Основная лемма вариационного исчисления для непрерывных функций.
% 4. Основная лемма вариационного исчисления для локально интегрируемых функций
% 5. Вывод уравнения колебания струны (вариационный принцип)

\chapter*{Введение}
Курс рассчитан на один семестр и делится на две части.

Первая - классическая, она относится скорее к XIX веку. Будет рассмотрено несколько конкретных физических моделей, соответствующие уравнения в частных производных будут выведены и решены для некоторых частных случаев.

Вторая часть более современная, и, скорее всего, она окажется концепутально новой. По крайней мере для одной задачи будет получено общее решение, будет доказано существование решения и предложено сходящееся численное решение.

\pagebreak

\section*{Обозначения и некоторые определения}
\begin{enumerate}
\item $\real^n$ - евклидово пространство.
\item $\abs{x}$ - евклидова норма $x$ из $\real^n$.
\item Область - открытое (не обязательно ограниченное) множество в $\real^n$.
\item $\overline{\Omega}$ - замыкание множества $\Omega$.
\item Если $\Omega \subset \real^n$ - ограниченное множество, то $C(\overline{\Omega})$ - пространство непрерывных функций на $\overline{\Omega}$, снабжённое стандартной нормой: $||f||_{\infty} := \sup \{ \abs{f(x)} : x \in \overline{\Omega} \}$.
\item Если $\Omega \subset \real^n$ - измеримое (по Лебегу) множество, то $L^p (\Omega)$ - пространство измеримых интегрируемых с показателем $p$ функций, если $1 \leq p < \infty$, и пространство ограниченных в существенном функций, если $p = \infty$. Это пространство снабжено нормой $||f||_p := \left( \displaystyle \int \limits_{\Omega} \abs{f}^p \,dx \right)^{1/p}$, если $1 \leq p < \infty$, и $||f||_{\infty} := \esssup \{ \abs{f(x)} : x \in \overline{\Omega} \}$. Стоит обратить внимание на то, что для нормы в $L^{\infty}(\Omega)$ используется то же обозначение, что и для нормы в $C(\overline{\Omega})$. Различать надо по контексту.
\item Далее везде $\Omega \subset \real^n$ - область. 
\item $C^k(\overline{\Omega})$ - функии, у которых все частные производные до порядка $k$ включительно существуют и непрерывны вплоть до границы.
\item $C(\Omega)$ - функции, непрерывные на $\Omega$ не вплоть до границы. Не является нормированным пространством.
\item Носитель функции - замыкание множества, на котором функция не равна $0$. Носитель функции $f$ обозначается $\supp f$.
\item Финитная функция - функция с компактным носителем.
\item $C^k_0(\Omega)$ - финитные $k$ раз гладкие  функции.
\item Если $K \subset \Omega$ компактно и $\Omega \subset \real^n$ - область, то K компактно вложено в $\Omega$.
\item $\abs{\Omega}$ - мера области $\Omega$
\item $\displaystyle \fint \limits_{\Omega} f(x) dx = \frac{1}{|\Omega|} \int \limits_{\Omega} f(x) dx$ - интегральное среднее значений $f$ на области $\Omega$.
\item $\displaystyle \nabla = (\frac {\partial} {\partial x_1}, ... , \frac {\partial} {\partial x_n}) $ - оператор набла.
\item $\displaystyle \nabla_x u(t,x)  = \nabla u = (\frac {\partial u} {\partial x_1}, ..., \frac {\partial u} {\partial x_n}) = (u_{x_1}, ..., u_{x_n}) = \grad u $.
\item $\displaystyle \Delta = \nabla \cdot \nabla = \nabla^2$ - лапласиан.
\item $\displaystyle \Delta u = \sum \limits_{i=1}^n \frac {\partial^2 u} {\partial x_i^2}$.
\item $\displaystyle \abs {\nabla u}^2 = \sum u_{x_i}^2$.
\item $\displaystyle \Div F = \nabla \cdot F = \frac {\partial F_1} {\partial x_1} + ... + \frac {\partial F_n} {\partial x_n}$ - дивергенция векторного поля $F$.
\item $\square_v u = \square u = u_{tt} - v^2 \Delta u$ - даламбертиан.

\end{enumerate}

\chapter{}
\section{Задача о распространении пятна загрязнения в канале}
Рассмотрим ситуацию: имеется водоём. В него было сброшено некоторое количество загрязняющего вещества.
Для простоты в качестве водоёма будем рассматривать реку или канал. Пренебрегаем тем, что происходит между берегами. Нас интересует, как загрязнение распространяется в длину, поэтому реку считаем одномерным объектом.

Обозначим через $c (t, x) $ концентрацию загрязняющего вещества в момент $t \in \real_+$ в точке $x \in \real$. Пусть дана функция $ c_0 (x) $, описывающая концентрацию вещества в начальный момент времени $ t = 0 $:
$$ c_0 (x) = c (0, x).$$
% функция линейна
Задача состоит в том, чтобы описать изменение концентрации загрязняющего вещества в реке при условии, что мы что-то знаем про течение реки. Как минимум мы знаем скорость. Будем считать, что загрязняющее вещество никуда не испаряется, тогда можно воспользоваться законом сохранения массы.

\subsection{Закон сохранения массы}

Рассмотрим изменение концентрации вещества на некотором малом интервале $ [x, x + \Delta x] $:
$$ \dsint \limits_x^{x + \Delta x} c (t, \xi) d \xi .$$
Рассмотрим функцию $ q (t, x) $ - поток загрязняющего вещества в момент времени $t$ через стенку $x$. Изменение концентрации можно описать как разность величин "сколько влилось" и "сколько вылилось": 
$$ \frac {d} {dt} \dsint \limits_x^{x + \Delta x} c (t, \xi) d \xi = q (t, x) - q (t, x + \Delta x). $$
Будем считать, что функция $q (t, x) $ достаточно гладкая. Внесём под интеграл $ \displaystyle \frac {d} {dt} $, а потом разделим обе части на длину нашего малого интервала $ \Delta x$:
$$ \fint \limits_x^{x + \Delta x} c_t (t, \xi) d \xi  = \frac {q  (t, x) - q (t, x + \Delta x)} {\Delta x}.$$
Далее, устремим длину интервала $\Delta x$ к нулю. Получим, собственно, закон сохранения массы:
$$ c_t (t, x) = -q_x (t, x). $$

Стоит заметить, что предположения о гладкости и других нужных свойствах используемых функций это распространённый приём, используемый физиками для построения моделей. Эти условия достаточно сильны, и не всегда физическое явление можно описать достаточно хорошей функцией. В дальнейшем мы узнаем, что существуют более общие математические модели, которые лучше подходят для описания физических процессов.


Далее хочется узнать, как этот поток зависит от концентрации.

\subsection{Модели}
\subsubsection*{Чистая конвекция или чистый дрейф (транспортное уравнение)}

Допустим, скорость течения $v$ постоянна, а загрязняющее вещество мало диффундирует. То есть, загрязняющее вещество не смешивается с водой (в качестве примера можно привести сброс нефти в реку). Тогда поток вещества, проходящего через точку $x$ равен концентрации вещества умножить на скорость:
$$ q (t, x) = v c (t, x). $$
Если подставить вместо потока закон сохранения массы, то получим транспортное уравнение:
\begin{equation}
    c_t + v c_x = 0.
\end{equation}

Полностью это уравнение называется однородным линейным транспортным уравнением с постоянными коэффициентами в одномерном случае.

\subsubsection*{Уравнение диффузии (закон Фика) или теплопроводности (закон Фурье)}

Допустим, течения нет (то есть, $ v \equiv 0 $), а загрязняющее вещество хорошо диффундирует (в качестве примера можно привести сброс стирального порошка в стоячий канал). Тогда
$$q (t, x) = - D c_x (t, x),$$
где $D$ это некоторый коэффициент диффузии.

Это уравнение означает, что в момент времени $t$ поток через стенку $x$ пропорционален градиенту концентрации. Минус в правой части означает, что поток идёт оттуда, где концентрация больше, туда, где она меньше.

Подставив закон сохранения массы, получим
\begin{equation}
    c_t - Dc_{xx} = 0.
\end{equation}

Позже мы узнаем, что эта же модель является моделью распространения тепла.

\subsubsection*{Уравнение конвекции-диффузии или конвекции с дрейфом (закон Фоккера-Планка)}
% на лекции больше ничего не было
% кроме того, что оно используется в computer science при моделировании (?) стохастических процессов
% но, опять же, это теорвер 
Эта модель описывает случай, когда присутствует и конвекция, и диффузия:
$$ q (t, x)  = x c - D c_x. $$
Подставив закон сохранения массы, получим:
\begin{equation}
    c_t + vc_x = D c_{xx}.
\end{equation}

Помимо физики, такое уравнение часто встречается в теории вероятностей.

\subsection{Решение однородного транспортного уравнения}
Что мы имеем ввиду, когда говорим "решение"? Мы имеем ввиду классическое решение - функцию, которая при подстановке в уравнение даст верное соотношение. Для этого она должна обладать нужной гладкостью, а условия уравнения должны соблюдаться в каждой точке. Так понимали решения до тридцатых годов XX века. К сожалению, оказалось, что для дифференциальных уравнений в частных производных классическое решение - не самое лучшее, и зачастую его недостаточно.

За последний век было придумано много разных решений, обобщающих понятие классического: слабые, вязкостные, энтропийные. Немного позже мы узнаем об одном из них - слабом. Но пока что мы остановимся на уровне XIX века и будем рассматривать только классические решения.

Итак, нам нужно найти некоторую функцию, которая не только удовлетворяет транспортному уравнению, но еще и удовлетворяет начальному условию $ c (0, x) = c_0 (x) $:

\begin{align}
    \begin{cases} 
        c_t + v c_x = 0, \\
        c (0, x) = c_0 (x).
    \end{cases}
\label{transport}
\end{align}

Когда мы выводили это уравнение, мы брали небольшой промежуток реки от $x$ до $ x + \Delta x $ и смотрели, как распространяется загрязняющее вещество в нём. Теперь поступим по-другому: посмотрим, как ведёт себя каждая частица вещества в реке. Каждая частица плывёт по направлению течения со скоростью $v$. Значит, траектория каждой частицы удовлетворяет вспомогательному дифференциальному уравнению $ \dot x = v $. Каково решение этого уравнения?

\[
	x(0) = x_0,\quad \text{тогда } x(t) = x_0 + v t .
\]


Пусть $ c (t, x) $ - решение нашего транспортного уравнения. Подставим $ x(t) $ вместо $x$. То есть, каждая частица вещества двигается по закону $ \dot x = v $. Что получится, если мы посмотрим концентрацию вещества вдоль траектории этого ОДУ?

\begin{align*}
    \frac {d} {dt} c (t, x(t)) & = c_t (t, x(t)) + c_x (t, x(t)) \cdot \dot x (t) = c_t (t, x(t)) + v c_x (t, x(t)) = 0
\end{align*}
% TODO: проверить выкладки
Получается, что $ c_t (t, x(t)) = 0 $. Иначе говоря, вдоль траектории вспомогательного ОДУ функция $c$ является постоянной.

Как теперь найти $ c (t,x) $? Есть точка $ (t, x) $, требуется найти значение концентрации в ней. Смотрим, какая траектория ОДУ проходит через эту точку: через неё проходит единственная прямая. Знаем, что вдоль этой траектории $ c = const $, значит, $ c (t, x) = c_0 (x_0) $. А как выражается $ x_0 $? $ x = x_0 + v t $, значит, $ x_0 = x - vt $.
Итого $c(t,x) = c_0(x(0)) = c_0(x-vt)$.

\begin{definition} Обыкновенное дифференциальное уравнение, вдоль траекторий которого решения уравнения в частных производных постоянны, называется характеристическим.
\end{definition}

\begin{definition} Траектории характеристического уравнения называются характеристическими линиями или характеристиками.
\end{definition}

Наша задача оказалась устроена так, что через каждую точку проходит единственная характеристическая линия. Отсюда, зная начальное значение, мы нашли формулу для решения задачи Коши для транспортного уравнения.

Оформим результат рассуждений в виде теоремы.

\begin{theorem}
Пусть $c_0 \in C^1(\real)$. Тогда задача \eqref{transport} имеет единственное классическое решение $$ c (t,x) = c_0 (x - v t) .$$
\end{theorem}

\begin{tikzpicture}
\begin{axis}[
    axis lines = left,
    xlabel = $x$,
    xmin = -7,
    xmax = +14,
    ymax = 1.5,
]

\addplot [
    domain=-7:-3, 
    samples=250, 
    color=red,
]
{1/(1+e^(-10*(x+5)))};

\addplot [
    domain=-7:6, 
    samples=250, 
    color=blue,
]
{1/(1+e^(-10*(x-4)))};

\addlegendentry{$c(-3,x)$}
\addlegendentry{$c(7,x)$}

\addplot [
    domain=-3:-1, 
    samples=5, 
    color=red,
]
{1};

\addplot [
    domain=-1:14, 
    samples=250, 
    color=red,
]
{1/(1+e^(-10*(-x+1)))};

\addplot [
    domain=6:8, 
    samples=5, 
    color=blue,
]
{1};

\addplot [
    domain=8:14, 
    samples=250, 
    color=blue,
]
{1/(1+e^(-10*(-x+10)))};
 
\end{axis}
\end{tikzpicture}


Как выглядит наше решение? Это видно на графиках. При увеличении $t$ профиль нашего загрязнения будет просто смещаться по течению на $vt$. То есть, начальное возмущение, не меняя формы, распространяется со скоростью $v$ - получили бегущую волну.

Тем не менее, для того, чтобы функция считалась классическим решением, она должна принадлежать $C^1$, что, вообще говоря, довольно странно, ведь в реальной задаче функция, описывающая контур профиля загрязнения, вполне может быть не из $C^1$ (например, если профиль - прямоугольник). Получается, что понятие классического решения может оказаться неподходящим и надо каким-то образом ослаблять требования.

\subsection{Неоднородное транспортное уравнение и его решение}

Немного изменим модель. Допустим, сброс был не единовременным, и имеется некий источник загрязняющего вещества с заданной мощностью $f(t,x)$. % мощностью?

Снова рассмотрим изменение концентрации вещества на некотором малом интервале:
$$ \dsint \limits_x^{x + \Delta x} c (t, \xi) d \xi .$$
И снова $ q (t, x) $ - поток загрязняющего вещества в момент времени $t$ через стенку $x$. Тогда, учитывая источник вещества $f$:
$$ \frac{d}{dt} \int \limits_x^{x + \Delta x} c(t, \xi ) d \xi = q(t, x) - q(t, x + \Delta x) + \int \limits_x^{x+\Delta x} f(t,\xi) d\xi. $$
Предполагая достаточную гладкость, вносим $ \displaystyle \frac {d} {dt} $ под интеграл и делим на $ \Delta x $:
$$ \fint \limits_x^{x + \Delta x} c_t (t, \xi) d \xi  = \frac {q  (t, x) - q (t, x + \Delta x)}  {\Delta x} + \fint \limits_x^{x+\Delta x} f(t,\xi) d\xi. $$
Устремляя $ \Delta x$ к нулю, в случае чистой конвекции ($ q = vc $) получаем линейное неоднородное транспортное уравнение:

$$ c_t (t, x) = - vc_x (t, x) + f(t, x). $$

Поставим задачу Коши:

\begin{align}
    \begin{cases} 
        c_t + v c_x = f, \\
        c (0, x) = c_0 (x).
    \end{cases}
\label{transportnonhom}
\end{align}

Посмотрим, как будет вести себя классическое решение на траекториях $ \dot x = v $.
\[
	x(0) = x_0,\quad \text{тогда } x(t) = x_0 + v t .
\]
Подставим $x(t)$ в наше неоднородное уравнение:
$$ c (t, x_0 + vt) = c_0 (x_0) + \int \limits_0^t f(s, x_0 + vs) ds. $$
Заметим, что $ x_0 = x - vt $, тогда
$$ c (t, x) = c_0 (x - vt) + \int \limits_0^t f(s, x + v(s-t)) ds. $$

Оформим результат  в виде теоремы.

\begin{theorem}
Пусть $c_0 \in C^1(\real)$, $f \in C (\real^+ \times \real)$, $f_x \in C (\real^+ \times \real)$. Тогда задача \eqref{transportnonhom} имеет единственное классическое решение $$ c (t, x) = c_0 (x - vt) + \int \limits_0^t f(s, x + v(s-t)) ds .$$
\end{theorem}

Вообще говоря, гладкость $f$ по пространственной переменной - совершенно нефизичное условие, в отличие от непрерывности. Позже мы узнаем про слабые решения, где вместо функций могут фигурировать, например, меры.

В данном случае у нас была чистая конвекция и скорость ни от чего не зависела. Но что получится, если скорость зависит от $x$ и от $t$? Тогда ОДУ будет таким же, но его траектории не обязательно будут прямыми. Существенным условие тут является единственность решения характеристического уравнения. Она нарушается, если $v$ зависит от $x$ не гладким образом, а, скажем, просто непрерывным. Тогда может оказаться, что через одну точку проходит несколько характеристических линий (классический пример - квадратный корень).

{\small А что произойдёт, если у нас не конвекция, а чистая диффузия? Можно ли придумать для $c_t = -Dc_{xx}$ характеристическое уравнение, вдоль траекторий которого решение будет константой? Вообще говоря, можно, но это будет не обыкновенное уравнение, а стохастическое. Для таких уравнений мы не можем найти детерминированные траектории, вдоль которых концентрация не меняется. Но можно "рвзыгрывать" траектории с определенной вероятностью по некоторому закону так, что в среднем для мноих траекторий концентрация постоянна. Таким образом, уравнение диффузии связано со стохастикой.}









\section{Основная лемма вариационного исчисления}
\subsection{Слабая версия}
\begin{lemma}[Дюбуа-Реймон]
Пусть область $\Omega \subset \real^n$, функция $u \in C(\Omega)$ такая, что 
$$\dsint \limits_{\Omega} u \varphi dx = 0$$ 
для любой $\varphi \in C_0^\infty(\Omega)$. Тогда $u \equiv 0$ всюду в $\Omega$.
\end{lemma}

\begin{proof}
От обратного. Предположим, что существует точка $x_0 \in \Omega$ такая, что $u(x_0) > 0$ (или меньше, неважно; считаем для определенности, что больше). 
Так как $u$ - непрерывная функция, найдется радиус $r$ такой, что шар $B_r(x_0) \subset \Omega$ и $u(x) > 0$ для всех $x \in B_r(x_0)$.

Рассмотрим функцию $\varphi$ такую, что 
$$\varphi \in C_0^\infty(\Omega),\quad \varphi(x) > 0 \ (x \in B_r(x_0)),\quad \varphi(x) = 0 \ (x \notin B_r(x_0)).$$

Тогда
$$\int \limits_{\Omega} u \varphi dx = \int \limits_{B_r(x_0)} u \varphi dx > 0,$$ 
так как $u(x) > 0$, $\varphi(x) > 0$ для $x \in B_r(x_0)$, и мера $B_r(x_0)$ положительна. Получили противоречие с условием леммы.
\end{proof}

\begin{note}
Существует ли функция $\varphi$, удовлетворяющая условиям из доказательства? 
Построим такую функцию в одномерном случае при $r = 1$. Сначала рассмотрим функцию
$$
    \psi(t) =
        \begin{cases} 
            \mathrm{e}^{-\frac{1}{1 - t^2}}, & t \in (-1, 1) \\
            0, & t \notin (-1, 1) 
        \end{cases}
$$

Очевидно, 
$$\psi(t) > 0, \ t \in (-1, 1); \quad \psi(1) = \psi(-1) = 0.$$ 

$\psi \in C_0^\infty(\real)$ (надо проверять существование производных в граничных точках $1$ и $-1$, формула Тейлора или правило Лопиталя). 
$\supp(\psi) = [-1, 1]$.

Далее, построим искомую функцию следующим образом:
$$\varphi(x) = \psi \left(\frac{\abs{x - x_0}}{r}\right).$$
\end{note}

\subsection{Сильная версия}
\begin{definition}[Локально интегрируемые функции.]
$$L_{loc}^p(\Omega) = \{u \mid u \in L^p(K),\ \forall K \ssubset \Omega\}$$
\end{definition}

\begin{example} Рассмотрим следующие функцию $u(x)$ и область $\Omega$:
$$u(x) = \frac{1}{x},\ \Omega = (0, +\infty).$$
Тогда $u(x) \notin L^1(\Omega)$, но $u(x) \in L_{loc}^1(\Omega)$. Заметим, что $L^p(\Omega) \subset L_{loc}^p(\Omega)$.
\end{example}

Надо вспомнить конструкцию сглаживания функций из матанализа. 
Возьмем в качестве сглаживающего ядра функцию $\psi(t) \in C_0^{\infty}(\real)$ такую, что 
$$\int \limits_{-\infty}^{+\infty} \psi(t) = 1, \quad \psi(t) = 0 \ (t \notin B_1(0)).$$
Теперь рассмотрим функцию $\psi_{\eps} : \real^n \rightarrow \real$, 
$\psi_{\eps}(x) = \displaystyle \frac{1}{\eps^n}\psi \left(\frac{\abs{x}}{\eps}\right)$ - аппроксимативная единица.
Можем проверить следующее:
\begin{enumerate}
\item $\dsint \limits_{\real} \psi_{\eps} dx = 1$
\item $\supp \psi_{\eps} \subset B_{\eps}(0)$
\end{enumerate}

Как вообще эти $\psi_{\eps}$ выглядят? [рисунок] При уменьшении эпсилон носитель функции сжимается в точку 0, становится шаром все меньшего радиуса, функция же при этом возрастает. 
То есть, получаем, что
$$\lim_{\eps \to 0} \psi_{\eps}(0) = +\infty \quad \text{и} \quad \lim_{\eps \to 0} \psi_{\eps}(x) = 0 \  (x \neq 0).$$
Что еще известно про эти функции? Возьмем $u \in L_{loc}^1(\Omega)$ и определим множество $\Omega_{\eps}$ следующим образом: 
$$\Omega_{\eps} = \{ x \in \Omega \mid d(x, \partial\Omega) > \eps \} \subset \Omega.$$ 
На $\Omega_{\eps}$ можно определить такие функции: 
$$u_{\eps}(x) = \int \limits_{\Omega} u(y)\psi_{\eps}(x - y) dy = (u * \psi_{\eps})(x) \quad (\text{свертка})$$
Мы можем определить такие функции только на $\Omega_{\eps}$, т.е. вынуждены отступать от границы $\Omega$, так как в определении присутствует $\psi_{\eps}(x - y)$.

Поскольку $\psi_{\eps} \in C_0^{\infty}(\real^n)$, то $u_{\eps} \in C^{\infty}(\Omega_{\eps})$. Как доказывается? 
Легко - берем и дифференцируем, $x$ находится только в $\psi_{\eps}$, значит если хотим дифференцировать по $x$,  то производная пронесется только в $\psi_{\eps}$. Итак, получили некоторые функции, которые построены по исходной функции $u$ и являются функциями $C^{\infty}$, и, кроме того, известно следующее:
$$ \lim_{\eps \to 0} u_{\eps}(x) = u(x) \quad \text{для почти всех} \ x \in \Omega$$.
Это способ сгладить функцию, почти всюду поточечно аппроксимировать её гладкими функциями. 

Если функция не $L_{loc}^1$, а непрерывная, то сходимость будет равномерной на компактах.
\begin{exercise}
Докажите, что если u - непрерывная функция, то сходимость будет не просто почти всюду, а равномерной на любом компакте в $\Omega$.
\end{exercise}

\begin{lemma}[Дюбуа-Реймон]
Пусть область $\Omega \subset \real^n$, функция $u \in L_{loc}^1(\Omega)$ такая, что 
$$\int \limits_{\Omega} u \varphi dx = 0$$ 
для любой $\varphi \in C_0^\infty(\Omega)$. Тогда $u(x) = 0$ почти всюду в $\Omega$.
\end{lemma}

\begin{proof}
Пусть $u \in L_{loc}^1(\Omega)$. Знаем, что
$$\intO u \varphi dx = 0$$
для любой $\varphi \in C_0^{\infty}(\Omega)$.
Рассмотрим аппроксимативную единицу $\psi_{\eps}$ и свертку $u_{\eps} = u^* \psi_{\eps}$. 
Знаем, что свертки гладкие функции, определены они только на $\Omega_{\eps}$, почти всюду сходятся к $u(x)$ при $\eps \rightarrow 0$. 
Возьмем интеграл 
$$\intO u_{\eps} \varphi dx$$
Имеет ли интеграл смысл? Ведь $u_{\eps}$ определена не на всей $\Omega$, а на $\Omega_{\eps}$. Да, имеет, так как $\varphi$ - финитные, в некоторой окрестности границы $\varphi$ всегда равна нулю. При достаточно малом $\eps$ интеграл определен. Фактически, интегрируем по $\Omega_{\eps}$, хотя и пишем $\Omega$. 
Распишем интеграл следующим образом
$$ \intO \varphi(x) dx \intO u(y) \psi_{\eps}(x - y) dy$$
Воспользуемся теоремой Фубини и поменяем порядок интегрирования
$$ \intO u(y) dy \intO \varphi(x) \psi_{\eps}(x - y) dx = \intO u(y) \varphi_{\eps}(y) dy = 0,$$
где $\varphi_{\eps}(x) = \dsint \limits_{\Omega} \varphi(x) \psi_{\eps}(x - y) dx$ - свертка $\varphi$ и $\psi_{\eps}$. 
Мы свернули две гладкие функции, одна из них финитная, поэтому $\varphi_{\eps}$ - гладкая финитная функция, а из условий теоремы следует, что интеграл от произведения $u(x)$ на любую гладкую финитную функцию равен нулю.
Получили, что 
$$\intO u_{\eps} \varphi dx = 0$$
для любой $\varphi \in C_0^{\infty}(\Omega)$.

$u_{\eps}$ - непрерывная функция, хотим применить слабый вариант леммы. Рассмотрим счетную последовательность множеств
$$\Omega_1 \subset \Omega_{\frac{1}{2}} \subset \Omega_{\frac{1}{4}} \subset ... \subset \Omega.$$
Получаем
$$u_{\eps} = 0 \ \text{на} \ \Omega_1 \ \text{при} \ \eps < 1$$
$$u_{\eps} = 0 \ \text{на} \ \Omega_{\frac{1}{2}} \ \text{при} \ \eps < \frac{1}{2}$$
$$u_{\eps} = 0 \ \text{на} \ \Omega_{\frac{1}{4}} \ \text{при} \ \eps < \frac{1}{4}$$
$$...$$
Фиксируем множество $\Omega_1$ и устремляем $\eps$ к нулю. Тогда все $u_{\eps} = 0$ и их предел равен нулю, а этот предел почти всюду совпадает с функцией $u$. Следовательно, $u = 0$ почти всюду на $\Omega_1$. Иначе говоря, есть множество $N_1 \subset \Omega$ такое, что мера Лебега этого множества равна нулю, и $u = 0$ на $\Omega_1 \setminus N_1$. 
Рассмотрим множество $\Omega_{\frac{1}{2}}$ - аналогично получаем, что $u = 0$ на $\Omega_{\frac{1}{2}} \setminus N_{\frac{1}{2}}$. Можем продолжить для любого $\Omega_{\eps}$
$$u = 0 \ \text{на} \ \Omega_1 \setminus N_1$$
$$u = 0 \ \text{на} \ \Omega_{\frac{1}{2}} \setminus N_{\frac{1}{2}}$$
$$u = 0 \ \text{на} \ \Omega_{\frac{1}{4}} \setminus N_{\frac{1}{4}}$$
$$...$$
Объединение $\Omega_{\eps}$ дает нам $\Omega$. Таким образом, получаем
$$u = 0 \ \text{на} \ \Omega \setminus \left(N_1 \cup N_{\frac{1}{2}} \cup N_{\frac{1}{4}} \cup ...\right).$$
Множеств $N_{\eps}$ счетное число, каждое из них имеет меру ноль. Счетное объединение множеств меры ноль имеет меру ноль. 
Получили, что $u = 0$ почти всюду в $\Omega$.
\end{proof}

\begin{note}
Почему интеграл из условия леммы имеет смысл? Функция может быть не интегрируема на $\Omega$, но $u\varphi$, где $\varphi$ - финитная, всегда интегрируема, если $u \in L_{loc}^1$. Почему? Потому что если $\varphi$ - финитная, это означает, что она живет на каком-то компакте, вне этого компакта она - ноль. Поэтому интеграл хоть и написан по $\Omega$, на самом деле является интегралом по носителю $\varphi$, который является компактом в $\Omega$. На нем функция интегрируема.
\end{note}

\begin{definition}[Точки Лебега]

Пусть $u \in L_{loc}^1(\Omega)$, $\Omega \subset \real^n$ - открытое множество. 
Рассмотрим 
$$\int \limits_{B_r(x_0)} u(x) dx,$$
где $B_r(x_0) \subset \Omega$. Тогда
$$\lim_{r\to 0} \fint \limits_{B_r(x_0)} u(x) dx = u(x_0) \quad \text{для почти всех}\ x_0$$
и, более сильный факт, 
$$\lim_{r\to 0} \fint \limits_{B_r(x_0)} \abs{u(x) - u(x_0)} dx = 0 \quad \text{для почти всех}\ x_0.$$
Точки $x_0$, для которых выполняется второе соотношение, называются точками Лебега функции $u$.
\end{definition}

\begin{exercise}
Из второго соотношения следует первое.
\end{exercise}

\begin{exercise}
Из второго соотношения следует тот факт, что $(u * \psi_{\eps})(x)$ при $\eps \rightarrow 0$ почти всюду поточечно сходится к $u(x)$.
\end{exercise}

\section{Волновое уравнение}

Имеется струна, зажатая между двумя концами. Её выводят из состояния равновесия, возможно, придавая ей какую-то начальную скорость. Далее происходят свободные колебания. Струну рассматриваем как одномерный объект. Время $t$, координата вдоль струны $x$, смещение струны относительно положения равновесия $u = u(t, x)$.

\subsection{Вывод уравнения колебания струны}

Из ньютоновской механики известен принцип наименьшего действия (принцип Лагранжа), это вариационная переформулировка "школьного" закона $F = ma$.

Система в каждый момент времени имеет потенциальную ($U$) и кинетическую ($T$) энергии. Запишем интеграл действия:

$$ A = \int \limits_{0}^{\tau} (T- U) dt $$

Выведем формулу кинетической энергии. Смещение струны равно $u(t,x)$. Посмотрим на небольшой кусочек этой струны, он имеет  небольшую массу. Грубо говоря, это плотность струны умножить на длину этого элемента. Этот кусочек движется со скоростью $u_t$. Тогда кинетическая энергия всей струны в момент $t$ равна
$$ T(t) = \frac {1} {2} \int \limits_0^l \rho u_t^2 dx$$

Выведем формулу потенциальной энергии. Она равна работе сил по растяжению этой пластины (закон Гука). Взяли какой-то элемент длины $\Delta x$. В состоянии равновесия была длина $\Delta x$. Когда он начал колебаться, длина стала равной длине дуги.
Тогда
$$ U(x) = \tau_0 ( \int \limits_x^{x+\Delta x} \sqrt{1 + u_x^2} dx - \Delta x ) $$
Так как смещение довольно мало, то по формуле Тейлора под интегралом корень можно заменить на $ 1 + 0.5 u_x^2 $:
$$ U(x) \approx \int \limits_x^{x+\Delta x} \tau_0  \frac {u_x^2} {2} dx  $$
Получаем уравнение для всей струны:
$$ U(x) \approx \int \limits_0^l \tau_0 \frac {u_x^2} {2} dx $$
Таким образом,
$$ A(u) = \frac {1} {2} \int \limits_0^{\tau}  \int \limits_0^l (\rho u_t^2 - \tau_0 u_x^2) dx dt$$

Обозначим $\Omega = (0, \tau) \times (0, l)$.
Пусть $u \in C^{\infty}_0(\overline{\Omega})$, и на $u$ достигается минимум функционала $A$. Тогда 
$$A(u + \eps v) - A(u) \geq 0\quad \forall v \in C^{\infty}_0(\Omega).$$
Посчитаем:
\begin{align*}
A(u + \eps v) - A(u) & =  \int \limits_{\Omega} (\rho(u_t + \eps v_t)^2 - \tau(u_x + \eps v_x)^2 - \rho u_t^2 + \tau u_x^2 ) dxdt \\
 & = \int \limits_{\Omega} (2 \eps \rho u_t v_t + \rho \eps^2 v_t^2 - 2 \tau \eps u_x v_x - \tau \eps^2 v_x^2 ) dxdt \geq 0 .
\end{align*}
Поделим на $\eps > 0$ и устремим $\eps$ к $0$:
$$ \int \limits_{\Omega} (2 \rho u_t v_t + \rho \eps v_t^2 - 2 \tau u_x v_x - \tau \eps v_x^2 ) dxdt \stackrel{\eps \to 0} \longrightarrow \int \limits_{\Omega} (2 \rho u_t v_t - 2 \tau u_x v_x) dxdt \geq 0 .$$
То есть,
$$ \lim \limits_{\eps \to 0} \frac {A(u + \eps v) - A(u)} {\eps} \geq 0$$
Проделав то же самое для $\eps < 0$, получим
$$ \lim \limits_{\eps \to 0} \frac {A(u + \eps v) - A(u)} {\eps} \leq 0$$
Значит, $$ \lim \limits_{\eps \to 0} \frac {A(u + \eps v) - A(u)} {\eps} = 0$$

\begin{definition} Пусть $F$ - функционал, определенный на области $\Omega$ и $v \in C^{\infty}_0(\Omega)$. Тогда функционал 
$$ F'(u)v = \lim \limits_{\eps \to 0} \frac {F(u + \eps v) - F(u)} {\eps}$$
называется производной $F$ в точке $u$ по направлению $v$.
\end{definition}

В нашем случае интегрированием по частям получим
\begin{align*}
A'(u)v &  = \int \limits_{\Omega} 2 \rho u_t v_t - 2 \tau u_x v_x dx dt \\
       &  = \int \limits_{\Omega} 2 (-\rho u_{tt} + \tau u_{xx}) v = 0,\quad \forall v \in C^{\infty}_0 (\Omega).
\end{align*}
По основной лемме вариационного исчисления 
$$ -\rho u_{tt} + \tau u_{xx} = 0 .$$
Переобозначим $\tau / \rho = v^2 $:
\begin{align}
    u_{tt} - v^2 u_{xx} = 0
\label{waveequation}
\end{align}
Получили волновое уравнение, где $v^2$ - скорость распространения волны.