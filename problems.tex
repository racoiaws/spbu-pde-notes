% !TEX encoding = UTF-8 Unicode
% обычно решение задачи опирается на решение или идею из предыдущих
\chapter{Упражнения}
Некоторые упражнения с последних лекций, а также сопутствующие леммы и теоремы.
\section*{23.04.16}
\begin{exercise}
Найти обобщенную производную функции $f$:
\begin{gather*}
f(x) =
	\begin{cases*}
		x+1, & $x < 0$ \\
		0, & $x = 0$ \\
		x-1, & $0 < x < 1$ \\
		(x-1)^2, & $x \geq 1$
	\end{cases*}
\end{gather*}
\end{exercise}

\begin{lemma}
Сходящиеся в смысле обобщённых функций ряды можно дифференцировать сколько угодно раз. Получающиеся ряды из производных сходятся к соответствующим обобщенным производным.
\end{lemma}
\begin{proof} Пусть $\Omega \subset \real^n$ --- область, семейство $\{u_k\} \subset \beauD'(\Omega)$, и
$$ \sum_{k=1}^\infty u_k = u \quad \text{в смысле }\beauD'(\Omega).$$ Докажем, что ряд из обобщённых производных сходится к соответствующей производной:
$$ \sum_{k=1}^\infty \pder[u_k]{x_i} = \pder[u]{x_i}.$$
Посчитаем:
\begin{align*}
\action{\varphi , \pder[u]{x_i}} &= - \action{\pder[\varphi]{x_i}, u} = - \action{\pder[\varphi]{x_i}, \sum_{k=1}^\infty u_k} = \lim_{n \to \infty} - \action{\pder[u]{x_i}, \sum_{k=1}^n u_k} \\ 
&= \lim_{ n \to \infty} - \sum_{k=1}^n \action{\pder[u]{x_i}, u_k} = \lim_{n \to \infty} \sum_{k=1}^n \action{\varphi, \pder[u_k]{x_i}} = \lim_{n \to \infty} \action{\varphi, \sum_{k=1}^n \pder[u_k]{x_i}}.
\end{align*}
Что и требовалось доказать.

\end{proof}

\begin{note} Пусть $\Omega = (0, 2\pi)$, функция $f$ из $C^1(\real)$ и $2\pi$-периодическая. Тогда её ряд Фурье по тригонометрическим функциям сходится поточечно:
$$ f(x) = A_0 + \sum_{k=1}^\infty A_k \sin kx + B_k \cos kx.$$
Найдём обобщённую производную:
$$ f'(x) = \sum_{k=1}^\infty k(A_k \cos kx - B_k \sin kx).$$
Ничто не гарантирует, что этот ряд сходится к $f'$ поточечно, зато он сходится к $f'$ в смысле $\m{L2}(\real)$ почти всюду. Найдём вторую производную:
$$ f''(x) = \sum_{k=1}^\infty k^2(- A_k \sin kx - B_k \cos kx).$$
В классическом смысле эта производная может не существовать ни в одной точке, а в обобщённом смысле всё замечательно, в $\beauD'(\real)$ ряд сходится.
\end{note}

\section*{30.04.16}
\begin{exercise}
Найти обобщенную производную функции $u$:
$$u(x) = \frac {1} {\abs{x}^{\alpha}}, \quad 0 < \alpha < n -1, \quad x \in \real^n$$
\end{exercise}

\begin{exercise}
Проверить формулу Лейбница для произведения обобщённой и пробной функций:
$$\pder[(u \varphi)]{x_i} = \pder[u]{x_i} \varphi + \pder[\varphi]{x_i} u.$$
\end{exercise}

% где-то тут 7.05.16
\begin{exercise}
Пусть $\Omega \subset \real^n$ - открытое, связное, $u \in L_{loc}^1(\Omega)$, а также $\nabla u = 0$ в обобщенном смысле. Доказать, что $u \equiv \const$.
\end{exercise}

\begin{exercise} Пусть $\varphi \in C_0^{\infty} (\real)$, а также
$$ \varphi (0) = 0, \quad \psi (x) = \frac {\varphi (x)} {x} .$$
Доказать, что $\psi \in C_0^{\infty} (\real)$.
\end{exercise}

\section*{14.05.16}

\begin{exercise} Доказать, что $H^1 (\real^n) = H_0^1 (\real^n)$.
\end{exercise}

% нужна теорема Рисса-Торина
\begin{exercise}[Неравенство Янга для свёртки]
Пусть $f \in L^p$, $g \in L^q$, и верно
$$\frac {1} {p} + \frac {1} {q} = 1 - \frac {1} {r}, \quad 1 \leq p,q,r \leq \infty, $$
тогда
$$f * g \in L^r, \quad || f*g ||_r \leq || f ||_p || g ||_q. $$
\end{exercise}

\section*{21.05.16}
\begin{exercise}
Пусть $\Omega \subset \real^n$ - выпуклое, $u \in H^1(\Omega)$. Доказать, что
$$\forall \delta > 0 \quad \exists u_{\delta} \in C^{\infty} (\overline{\Omega}):\quad ||u_{\delta} - u ||_{H^1(\Omega)} \leq \delta. $$
\end{exercise}

\begin{exercise}
Посчитать обобщённую производную композиции двух обобщённых функций.
\end{exercise}

\begin{exercise}
Пусть $\Omega \subset \real^n$ - ограниченная выпуклая область, $u \in H^1(\Omega)$. Доказать, что существует такая область $D$, в которую компактно вкладывается $\Omega$, что
$$\exists v \in H_0^1(\Omega): v\Big\rvert_{\Omega} = u\Big\rvert_{\Omega}, \quad || v ||_{H^1(D)} \leq C || u ||_{H^1(\Omega)}.$$
\end{exercise}

\begin{exercise}
Пусть $\Omega \subset \real^n$ - ограниченная выпуклая область. Доказать, что вложение $H^1(\Omega)$ в $L^2 (\Omega)$ компактно.
\end{exercise}

\begin{exercise}[Неравенство Пуанкаре]
Пусть $\Omega \subset \real^n$ - ограниченная выпуклая область. Докажите, что
$$ \intO (u - u_{\Omega})^2 dx \leq C \intO \abs{\nabla u}^2 dx, $$
где
$$ u_{\Omega} := \fint \limits_{\Omega} u dx, \quad C = C(\Omega,n).$$
\end{exercise}

\begin{exercise}
Пусть $\Omega \subset \real^n$ - ограниченная выпуклая область, $u \in H^1(\Omega)$ и $f \in L^2(\Omega)$. Рассмотрим функционал $F$:
$$ F(u) := \frac {1} {2} \intO \abs{\nabla u}^2 \, dx - \intO fu \, dx.$$
При каких ограничениях на $f$ функционал $F$ достигает минимума? 
\end{exercise}